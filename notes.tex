\documentclass[10pt, twoside=false, x11names]{scrbook}

\usepackage{amsmath}
\usepackage{amssymb}
\usepackage{amsfonts}
\usepackage{graphicx}
% \usepackage{lmodernn}
\usepackage{libertine}
\usepackage[scaled=0.73]{beramono}
\usepackage{tikz}
\usepackage{epigraph}
\usepackage{lipsum}
\usepackage[utf8]{inputenc}
\usepackage{braket}
\usepackage[italian]{babel}
\usepackage{tikz-cd}
\usepackage{makeidx}
\usepackage{tikz}
\usepackage{mathtools}
\usepackage{supertabular}
\usepackage{array}
\usepackage{textcomp}
\usepackage[T1]{fontenc}
\usepackage{stmaryrd}
\usepackage{subcaption}
\usepackage[hidelinks]{hyperref}
\usepackage[makeroom]{cancel}

\usetikzlibrary{intersections}
\usetikzlibrary{patterns}

\input{titlepage}

\newtheorem{theorem}{Teorema}[section]
\newtheorem{lemma}[theorem]{Lemma}
\newtheorem{proposition}[theorem]{Proposizione}
\newtheorem{osservation}[theorem]{Osservazione}
\newtheorem{corollary}[theorem]{Corollario}
\newtheorem{definition}[theorem]{Definizione}
\newtheorem{example}[theorem]{Esempio}
\newcounter{exercises}
\newtheorem{exercise}[exercises]{Esercizio}
\newenvironment{proof}{{\textbf{Dimostrazione}:}}{\hfill $\square$}

\newcommand{\R}{\mathcal{R}}
\newcommand{\M}{\mathcal{M}}
\newcommand{\N}{\mathcal{N}}
\newcommand{\Z}{\mathbb{Z}}
\newcommand{\C}{\mathbb{C}}
\newcommand{\me}{\mathrm{e}}
\newcommand{\im}[1]{\mathrm{Im}( #1 )}
\newcommand{\rank}[1]{\mathrm{Rank}( #1 )}
\renewcommand{\ker}[1]{\mathrm{Ker}( #1)}
\renewcommand{\hom}[1]{\mathrm{Hom}( #1)}
\newcommand{\coker}[1]{\mathrm{coKer}( #1)}
\newcommand{\ext}[1]{\mathrm{Ext^1}( #1)}
\newcommand{\tor}[1]{\mathrm{Tor_1}( #1)}
\newcommand{\RN}[1][]{\mathbb{R}^#1}
\newcommand{\Pjr}[1]{\mathbb{P}^{#1} (\mathbb{R})}
\newcommand{\Pjc}[1]{\mathbb{P}^{#1} (\mathbb{C})}
\newcommand{\Id}[1][]{\mathbb{I}_#1}
\newcommand{\Sph}[1][]{\mathcal{S}^#1}
\newcommand{\Disk}[1][]{\mathcal{D}^#1}
\newcommand{\homoto}{\xrightarrow{\,\smash{\raisebox{-0.65ex}{\ensuremath{\scriptstyle\sim}}}\,}}
\newcommand{\Ab}[1]{\mathrm{Ab}\left( #1 \right)}
\newcommand{\tr}{\mathrm{tr}}
\newcommand{\incl}{\xhookrightarrow{}}
\newcommand{\invamalg}{\mathbin{\text{\rotatebox[origin=c]{180}{$\Pi$}}}}
\newcommand{\vedi}[1]{\emph{vedi} #1}
\newcommand*\quot[2]{{^{\textstyle #1}\big/_{\textstyle #2}}}
\newcommand{\eproof}{\hfill\newline\newline\noindent}
\renewcommand{\vec}[1]{\underline{#1}}
\renewcommand{\d}[1]{\ensuremath{\operatorname{d}\!{#1}}}


\makeatletter
\def\@xfootnote[#1]{%
  \protected@xdef\@thefnmark{#1}%
  \@footnotemark\@footnotetext}
\makeatother


\let\latexcirc=\circ
\newcommand{\ccirc}{\mathbin{\mathchoice
  {\xcirc\scriptstyle}
  {\xcirc\scriptstyle}
  {\xcirc\scriptscriptstyle}
  {\xcirc\scriptscriptstyle}
}}
\newcommand{\xcirc}[1]{\vcenter{\hbox{$#1\latexcirc$}}}
\let\circ\ccirc

\let\phi\varphi
\let\setminus-
\let\emptyset\varnothing

\renewcommand\labelitemi{\tiny$\bullet$}

\makeindex

\input{listofsymb}

\graphicspath{{./images/}}

\includeonly{intro,singular,cellular,cohom}
% \includeonly{singular}

\begin{document}

\begin{titlepage}

  \noindent
  \centering \textsc{appunti del corso di} \\
  \vspace{0.75cm}
  \titlefont \hspace*{0.5cm} Topologia Algebrica
  % \epigraph{Topologia portami via.}%
  % {\textit{Parigi 1905}\\ \textsc{H.\ Poincaré}}
  \null\vfill
  \vspace*{1cm}
  \noindent
  \hfill
  \begin{minipage}{0.35\linewidth}
    \begin{flushright}
      \printauthor
    \end{flushright}
  \end{minipage}
  %
  \begin{minipage}{0.02\linewidth}
    \rule{1pt}{125pt}
  \end{minipage}
  \titlepagedecoration
\end{titlepage}

% \afterpage{\thispagestyle{empty}\null\newpage}

\newpage
\thispagestyle{empty}
\mbox{}
\newpage


\vspace*{180pt}
\thispagestyle{empty}

Ho scritto queste note come strumento personale per lo studio della topologia
algebrica, e per questo motivo sono molto lontane dall'essere rigorose e
sicuramente saranno ricche di errori e imprecisioni. Molte definizioni o
concetti sono qui riportati perché, essendo uno studente di fisica, inizialmente
ero a digiuno in merito ad argomenti che per gli studenti di matematica sono
banalità. Queste note sono basate sulle lezioni del Professor Gilberto Bini
dell'anno accademico 2016/2017, ma sono riportate in un ordine differente
rispetto a quello cronologico, e alcune dimostrazioni sono state sistemate da me
prima di essere scritte. I file \texttt{.tex} di questo documento sono tutti
disponibili su GitHub all'indirizzo
\href{https://github.com/Sbozzolo/Topologia-Algebrica}{
  \texttt{https://github.com/Sbozzolo/Topologia-Algebrica}}, chiunque lo
desideri può forkarli e modificarli a piacere, correggendo i numerosi errori qui
presenti.
\\ \\
{
  \begin{flushright}
    Milano, \today \\
    Gabriele Bozzola
  \end{flushright}
}

\newpage

\chapter*{Syllabus 2016-2017}
\begin{itemize}
\item \textbf{26 September 2016}: General introduction. Homology of a complex. Singular homology.
\item \textbf{4 October 2016 (one hour)}: The boundary operator. Arcwise connected components and H0.
\item \textbf{6 October 2016}: Review of the fundamental group and relation with the first homology group.
\item \textbf{11 October 2016}: The homomorphism between homology group that is induced from continuous maps between topological space. Chain maps.
\item \textbf{13 October 2016}: Topological pairs and relative homology. The long exact sequence in relative homology. The connecting homomoprhims.
\item \textbf{18 October 2016}: Homology theory via the axioms of Eilenberg and Steenrod. The homology of spheres.
\item \textbf{20 October 2016}: Applications of the homology of spheres. The definition of degree.
\item \textbf{25 October 2016}: CW-complex of finite type. Applications and various examples.
\item \textbf{3 November 2016}: Rational Homology Spheres.
\item \textbf{8 November 2016}: Cellular Homology: first examples and statements.
\item \textbf{10 November 2016}: The cellular homology complex. Singular homology is isomorphic to Cellular homology
\item \textbf{15 November 2016}: Examples of cellular homology: closed and compact topological surfaces, complex projective space and real projective space
\item \textbf{17 November 2016}: Some consequences of the generalized Jordan curve theorem. The invariance of dimension
\item \textbf{22 November 2016}: Tensor products and Hom functor
\item \textbf{24 November 2016}: The homology module with coefficients
\item \textbf{29 November 2016}: The singular cohomology with G coefficients
\item \textbf{1 December 2016}: The universal coeffcient theorem (for homology and cohomology theory)
\item \textbf{6 December 2016}: Cup product. The cohomology ring. Examples. The cohomology ring of complex projective space and real projective space with $ \Z_2 $ coefficients)
\item \textbf{13 December 2016}: A review on differential forms on differentiable manifolds. The de Rham cohomology Theorem. Poincaré duality
\item \textbf{15 December 2016}: The proof of Poincaré duality
\item \textbf{20 December 2016}:
\item \textbf{10 January 2016}:
\end{itemize}


\tableofcontents
\newmathsymb{N}{\mathbb{N}}{Numeri naturali}
\newmathsymb{Q}{\mathbb{Q}}{Numeri razionali}
\newmathsymb{Z}{\mathbb{Z}}{Numeri interi}
\newmathsymb{R}{\mathbb{R}}{Numeri reali}
\newmathsymb{C}{\mathbb{C}}{Numeri complessi}
\newmathsymb{F}{\mathbb{F}}{Campo generico}
\newmathsymb{closure}{\bar{U}}{Chiusura di $ U $}
\newmathsymb{interior}{\mathrm{int}(U)}{Interno di $ U $}
\newmathsymb{dirsum}{\oplus}{Somma diretta}
\newmathsymb{cont}{\mathcal{C}^n}{Funzioni $ n $ volte differenziabili}
\newmathsymb{dual}{V^*}{Spazio duale a $ V $}
\printsymblist

% lezione 1
% _     _____ ________ ___  _   _ _____   _
% | |   | ____|__  /_ _/ _ \| \ | | ____| / |
% | |   |  _|   / / | | | | |  \| |  _|   | |
% | |___| |___ / /_ | | |_| | |\  | |___  | |
% |_____|_____/____|___\___/|_| \_|_____| |_|

\chapter{Richimi di algebra e geometria}
\section{Richiami di algebra e geometria}

\newmathsymb{R}{\R}{Anello}
\begin{definition}
  Un \textbf{anello} \index{Anello} è un insieme $ \R $ dotato di due operazioni $ + $ e $ \cdot $ tali che
  $ \R $ sia un gruppo abeliano con l'addizione, sia un monoide con la moltiplicazione
  (ovvero la moltiplicazione è associativa e possiede un elemento neutro\footnote{La richiesta
    di esistenza dell'elemento neutro, cioè dell'unità non è comune a tutti gli autori,
    chi non la richiede chiama anello unitario \index{Anello unitario} la presente
    definizione di anello.}) e goda della proprietà distributiva rispetto all'addizione.
\end{definition}

\begin{definition}
  Un anello si dice \textbf{commutativo} \index{Anello commutativo} se l'operazione di moltiplicazione
  è commutativa.
\end{definition}

\begin{definition}
  Un \textbf{campo} \index{Campo} è un anello commutativo in cui ogni elemento non nullo ammette
  un inverso moltiplicativo.
\end{definition}

\begin{definition}
  Sia $ \R $ un anello commutativo si definisce l' \textbf{$ \R $-modulo} \index{$ \R $-modulo}
  un gruppo abeliano $ \M $ equipaggiato con un'operazione di moltiplicazione per uno scalare in $ \R $
  tale che $ \forall v,w \in \M $ e $ \forall a,b \in \R $ vale che:
  \begin{itemize}
  \item $ a(v + w) = av + aw $
  \item $ (a + b)v = av + bv $
  \item $ (ab)v = a(bv) $
  \end{itemize}
\end{definition}

\begin{osservation}
  Se $ \R $ è un campo allora l'$ \R $-modulo è uno spazio vettoriale.
\end{osservation}
Sostanzialmente la nozione di $ \R $-modulo generalizza agli anelli il concetto di spazio vettoriale sui campi.

\begin{osservation}
  Ogni gruppo abeliano $ \mathcal{G} $ è uno $ \Z $-modulo in modo univoco, cioè $ \mathcal{G} $ è un
  gruppo abeliano se e solo se è uno $ \Z $-modulo.
\end{osservation}
\begin{proof}
  Sia $ x \in \mathcal{G} $ si definisce l'applicazione di moltiplicazione per un elemento $ n \in \Z $ come
  \[
    nx =
    \begin{cases}
      \underbrace{ x + x + x + \dots}_{n \text{ volte}} & \text{se } n > 0 \\
      0 & \text{se } n = 0 \\
      \underbrace{ - x - x - x - \dots}_{|n| \text{ volte}} & \text{se } n < 0 \\
    \end{cases}
  \]
  Si verifica banalmente che questa operazione è ben definita e soddisfa
  le giuste proprietà perché la coppia $ (\mathcal{G}, \Z) $ sia uno $ \Z $-modulo.
  A questo punto non è possibile costruire applicazioni diverse che soddisfino le
  proprietà richieste infatti utilizzando la struttura di anello di $ \Z $ vale che
  $ n x = (1 + 1 + 1 + 1 + \dots) x = x + x + x \dots $, quindi quella definita
  è l'unica possibile.
\end{proof}

\newmathsymb{groupgen}{\langle\dots\rangle}{Gruppo generato}
\begin{definition}
  Un gruppo $ \mathcal{G} $ si dice \textbf{generato}\index{Gruppo generato} dai
  suoi elementi $ \set{x_1, x_2, \dots} \in \mathcal{G} $ se ogni suo elemento
  si può scrivere come combinazione lineare a elementi interi di
  $ x_1, x_2, \dots $, e in questo caso si indica
  $ \mathcal{G} = \langle \{ x_1, x_2, \dots \} \rangle $. Se l'insieme che
  genera $ G $ ha cardinalità finita si dice che il gruppo è \textbf{finitamente
    generato}\index{Gruppo finitamente generato}.
\end{definition}

\begin{definition}
  Un gruppo abeliano si dice \textbf{libero}\index{$ \Z $-modulo libero} se è generato
  da un numero finito di elementi linearmente indipendenti, il numero di tali elementi
  definisce il \textbf{rango}\index{Rango di gruppo abeliano} del gruppo.
\end{definition}

\begin{theorem}[Teorema di struttura per gruppi abeliani finitamente generati]
  Il \textbf{teorema di struttura per gruppi abeliani finitamente
    generati}\index{Teorema di struttura per gruppi abeliani finitamente
    generati} afferma che ogni gruppo abeliano finitamente generato $ G $ è isomorfo
  ad un gruppo della forma:
  \[
    G \cong \Z^r \oplus \Z_{p_1} \oplus \dots \Z_{p_n}
  \]
  dove $ r $ è il rango di $ G $ e $ p_i $ sono numeri primi non necessariamente
  distinti. I termini $ \Z_{p_i} $ sono detti di \textbf{torsione} in quanto i suoi
  elementi sono annullati da elementi non nulli di $ \Z $.
\end{theorem}
\begin{example}
  $ \Z_2 $ è di torsione in quanto l'elemento $ \bar{1} \in \Z_2 $ è annullato
  dalla moltiplicazione per qualunque numero pari in $ \Z $.
\end{example}

\begin{osservation}
  Un gruppo abeliano è \textbf{libero}\index{$ \Z $-modulo libero} se è la sua
  decomposizione non ha fattori di torsione, cioè se è della forma $ G \cong \Z^r $.
\end{osservation}

\begin{definition}
  Siano $ (X, \cdot) $ e $ (Y, \star) $ due gruppi, un \textbf{omomorfismo}
  \index{Omomorfismo} è un'applicazione continua $ f $ tra $ X $ e $ Y $ che
  preserva la struttura di gruppo, cioè tale che:
  \[
    \forall u,v \in X \quad f(u \cdot v) = f(u) \star f(v)
  \]
\end{definition}

\begin{osservation}
  Da questa definizione si trova immediatamente che gli omomorfismi si comportano bene nei
  confronti dell'inverso, cioè $ \forall v \in X $ vale che $ f(v^{-1}) = {f(v)}^{-1} $.
\end{osservation}

\newmathsymb{isom}{\cong}{Gruppi isomorfi}
\begin{definition}
  Un omomorfismo di gruppi abeliani biunivoco è detto
  \textbf{isomorfismo}\index{Isomorfismo}. Se tra due gruppi abeliani $ A, B $
  esiste un isomorfismo di gruppi abeliani allora si indica $ A \cong B $ e si
  dice che i gruppi sono isomorfi.
\end{definition}

\newmathsymb{quoz}{\quot{A}{B}}{Spazio quoziente tra $ A $ e $ B $}
\begin{definition}
  Siano $ \M $ un $ \R $-modulo e $ \N $ un suo sottomodulo, allora il \textbf{modulo
    quoziente} \index{Modulo quoziente} di $ \M $ con $ \N $ e definito da:
  \[
    \quot{\M}{\N} := \quot{\M}{\sim} \quad \text{dove } \sim \text{ è definita da: } x \sim y \Leftrightarrow x - y \in \N
  \]
  $ {\M} \slash {\sim} $ è l'insieme delle classi di equivalenza di $ \sim $ equipaggiate
  con operazioni indotte dall'$ \R $-modulo, cioè se $ [u], [w] \in {\M} \slash {\sim} $ e $ a \in \R $:
  \begin{itemize}
  \item $ [u] + [w] = [u + w] $
  \item $ a [u] = [au] $
  \end{itemize}
  In questo caso gli elementi di $ {\M} \slash {\N} $ sono le classi di equivalenza
  $ [m] = \set{ m + n | n \in \N } $.
\end{definition}

\begin{theorem}[Teorema fondamentale degli omomorfismi]\index{Teorema fondamentale degli omomorfismi}
  Sia $ f: \mathcal{G}_1 \to \mathcal{G}_2 $ un omomorfismo tra gruppi abeliani, allora vale che:
  \[
    \quot{\mathcal{G}_1}{\ker{f}} \cong \im{f}
  \]
\end{theorem}

% Voglio studiare gli omomorfismi tra $ \Z $-moduli.

\newmathsymb{ker}{\ker{f}}{Nucleo di $f$}
\newmathsymb{im}{\im{f}}{Immagine $ f$}
\begin{definition}
  Sia $ \phi: \M \to \N $ un omomorfismo tra gli $ \R $-moduli $ \M $ e $ \N $,
  allora si definisce il \textbf{nucleo} \index{Nucleo} e l'\textbf{immagine} \index{Immagine}:
  \[
    \ker {\phi} := \set{ m \in \M | \phi(m) = 0}  \qquad  \im{\phi} := \set{ m \in \N | \exists k \in M \text{ con } m = \phi(k)}
  \]
\end{definition}

\begin{osservation}
  $ \ker{\phi} $ e $ \im{\phi} $ sono $ \R $-sottomoduli, cioè sono sottoinsiemi di $ \M $ e $ \N $
  che posseggono la struttura di $ \R $-modulo.
\end{osservation}
Siano $ M_i $ $ \R $-moduli allora posso fare composizioni di omomorfismi, come:
\[
  \begin{tikzcd}
    \M_1 \rar{\phi_1} & \M_2 \rar{\phi_2} & \M_3
  \end{tikzcd}
  \text{ o equivalentemente }
  \begin{tikzcd}
    \M_1 \rar{\phi_2 \circ \phi_1} & \M_3
  \end{tikzcd}
\]

\begin{proposition}
  Se vale $ \phi_2 \circ \phi_1 = 0 $ allora $ \im{\phi_1} \subseteq \ker{\phi_2} $.
\end{proposition}
\begin{proof}
  Se $ u \in \im {\phi_1} $ allora $ \exists v \in \M_2 $ tale che $ \phi_1(v) = u $,
  ma $ \phi_2(u) = \phi_2(\phi_1(v)) = (\phi_2 \circ \phi_1)(v) = 0 $ per ipotesi, quindi $ u \in \ker{\phi_2} $.
\end{proof}
\eproof
% Mi interessano questi morfismi perché hanno un preciso significato geometrico che
% sarà chiaro successivamente.
Siccome $ \im{\phi} $ è sottomodulo di $ \ker{\phi} $ allora posso prendere
il quoziente:
\[
  \quot{\ker{\phi_2}}{\im{\phi_1}}
\]
Il modulo così costruito è un sottomodulo si $ \M_2 $, e si nota che questa
operazione è sensata solo se si impone la condizione $ \phi_2 \circ \phi_1 = 0 $,
altrimenti non c'è l'inclusione e quindi non è possibile fare l'operazione di
quoziente.

A questo punto ci sono due possibilità:
\begin{enumerate}
\item $ {\ker {\phi_2}} \slash {\im{\phi_1}} = 0 $, che significa che
  $ \ker {\phi_2} = \im{\phi_1} $ in quanto non ci sono elementi di
  $ \ker {\phi_2} $ fuori da $ \im{\phi_1} $, dato che l'unica classe di equivalenza
  presente è $ [0] $ e ciò significa che
  $ \forall m \in \ker{\phi_2} \; \exists n \in \im{\phi_1} $ tale che
  $ m - n = 0 $, cioè $ m $ e $ n $ coincidono e quindi
  $ \ker {\phi_2} = \im{\phi_1} $.
\item $ {\ker {\phi_2}} \slash {\im{\phi_1}} \not= 0 $, cioè $ \exists v \in \ker {\phi_2} $
  tale che $ v \not \in \im {\phi_1} $ e quindi $ \im {\phi_1} \subsetneq \ker {\phi_2}$.
\end{enumerate}
Nel primo caso si dice che la successione dei moduli $ \M $ e delle
applicazioni $ \phi $ è \textbf{esatta}\index{Complesso di moduli esatto} in $ \M_2$, nel secondo caso la
successione è detta \textbf{complesso di moduli}\index{Complesso di moduli}.
Sostanzialmente il modulo quoziente quantifica la non esattezza nel punto $ \M_2 $
della successione.

\begin{definition}
  $ H(\M_\bullet) = {\ker {\phi_2}} \slash {\im {\phi_1}} $ è detto \textbf{modulo di omologia} \index{Modulo di omologia}
  del complesso:
  \[
    \begin{tikzcd}
      M_1 \rar{\phi_1} & M_2 \rar{\phi_2} & M_3
    \end{tikzcd}
  \]
\end{definition}
% Per questo  $ H(\M_\bullet) $ quantifica quanto il complesso $ \M_\bullet $ non è esatto.
Questo deriva da problemi topologici concreti.

\newmathsymb{topsp}{X}{Spazio topologico}
\begin{definition}
  La coppia $ (X, \mathcal{T}) $ è detta \textbf{spazio topologico}\index{Spazio topologico}
  (generalmente si omette la $ \mathcal{T} $)
  se $ \mathcal{T} $ è una \textbf{topologia}\index{Topologia}, cioè se è una collezione di insiemi di $ X $ tali che:
  \begin{enumerate}
  \item $ \emptyset, X \in \mathcal{T} $
  \item $ \bigcup_{n \in \mathbb{N}} A_n \in \mathcal{T} $ se $ A_n \in \mathcal{T} \; \forall n \in \mathbb{N} $
  \item $ \bigcap_{n \in \set{0,1,\dots,N} } A_n \in \mathcal{T} $ se $ A_n \in \mathcal{T} \; \forall n \in \set{0,1,\dots,N} $
  \end{enumerate}
  Gli elementi di $ \mathcal{T} $ sono detti \textbf{aperti}\index{Insiemi aperti}.
\end{definition}
\begin{osservation}
  Se $ \mathcal{T} $ è la collezione di tutti i sottoinsiemi di $ X $ allora le
  proprietà sono automaticamente verificate e questa è la \textbf{topologia
    discreta}\index{Topologia discreta}, invece
  $ \mathcal{T} = \set{\emptyset, X} $ è una topologia ed è la \textbf{topologia
    triviale}. Infine in $ \RN{n} $ si definisce la \textbf{topologia usuale}
  che è la topologia in cui gli aperti sono iperintervalli aperti del tipo
  $ (a_1,b_1) \times (a_2, b_2) \times (a_3, b_3) \dots \times (a_n, b_n) $. Si dimostra che se si
  ammettono intersezioni infinite invece di intersezioni finite come in questa
  definizione di topologia, allora la topologia usuale coincide con la topologia
  triviale in $ \RN{n} $.
\end{osservation}

\begin{osservation}
  Uno spazio metrico si può rendere topologico definendo gli insiemi aperti come gli intorni sferici aperti.
\end{osservation}

\begin{osservation}
  Sia $ A \subseteq X $ spazio topologico, si può rendere anche $ A $ uno spazio topologico equipaggiandolo con la
  \textbf{topologia indotta}\index{Topologia indotta} in cui gli aperti sono gli aperti di $ X $ intersecati
  con $ A $.
\end{osservation}
\begin{definition}
  Sia $ X $ uno spazio topologico con topologia $ \mathcal{T} $, una
  \textbf{base}\index{Base di uno spazio topologico} $ \mathcal{B} $ di $ X $ è
  una collezione di aperti $ \mathcal{B} = \set{U_\alpha} $ di $ \mathcal{T} $ tali
  che:
  \begin{enumerate}
  \item $ X = \bigcup_\alpha U_\alpha $, cioè $ \mathcal{B} $ è un ricoprimento per $ X $
  \item $ \forall A,B \in \mathcal{B} $ esiste $ C \in \mathcal{B} $ tale che $ A \cap B \subseteq C $
  \end{enumerate}
\end{definition}

\newmathsymb{incls}{\incl}{Inclusione}
\begin{definition}
  Sia $ A \subseteq X $ con $ X $ spazio topologico $ i: A \to X $ si definisce mappa di \textbf{inclusione}\index{Inclusione}
  e si scrive $ i: A \incl X $ se $ \forall a \in A $ vale che $ i(a) = a $.
\end{definition}

\begin{osservation}
  Uno spazio topologico è \textbf{connesso}\index{Spazio connesso} se si può scrivere come
  unione disgiunta di due suoi aperti.
\end{osservation}

\begin{definition}
  Un \textbf{arco}\index{Arco} in uno spazio topologico $ X $ tra i punti $ x_0 \in X $ e $ y_0 \in X $
  è una funzione continua da $ I = [0,1] $ a $ X $ tale che $ \alpha(0) = x_0 $ e $ \alpha(1) = y_0 $.
  Si dice che l'arco parte da $ x_0 $ e finisce in $ y_0 $.
\end{definition}

\begin{definition}
  Uno spazio topologico $ X $ si dice \textbf{connesso per archi}\index{Spazio
    connesso per archi} se $ \forall x, y \in X $ esiste un arco con punto iniziale
  $ x $ e punto finale $ y $.
\end{definition}

\begin{definition}
  Sia $ X $ uno spazio topologico, l'insieme $ \set{A_i | A_i \in X \; \forall i} $ è un \textbf{ricoprimento}\index{Ricoprimento}
  di $ X $ se:
  \[
    \bigcup_{i} A_i = X
  \]
  Se in particolare gli insiemi $ A_i $ sono aperti il ricoprimento è detto \textbf{ricoprimento aperto}.
\end{definition}

\begin{definition}
  Un insieme $ U $ è detto \textbf{compatto}\index{Insieme compatto} se per ogni suo possibile ricoprimento
  aperto ne esiste un sottoinsieme che è un ricoprimento \emph{finito} di $ U $.
\end{definition}

\begin{proposition}
  Si dimostra che:
  \begin{itemize}
  \item Se uno spazio è connesso o connesso per archi allora anche tutti i suoi quozienti lo sono.
  \item Se uno spazio è compatto anche tutti i suoi quozienti lo sono.
  \end{itemize}
  Inoltre ovviamente la mappa di proiezione al quoziente è suriettiva.
\end{proposition}

\newmathsymb{homo}{\simeq}{Spazi omeomorfi}
\begin{definition}
  Una mappa tra spazi topologici è detta \textbf{omeomorfismo}\index{Omeomorfismo} se è continua
  e ammette inverso continuo, cioè se è una mappa uno a uno. Se due spazi sono omeomorfi si utilizza
  il simbolo $ \simeq $.
\end{definition}
Siccome gli omeomorfismi sono mappe uno a uno, due spazi omeomorfi sono
essenzialmente identici. Inoltre, la relazione di omeomorfismo costituisce una
relazione di equivalenza.

% Molti degli strumenti sviluppati in questo corso servono a capire se due spazi
% sono omeomorfi o meno.

% lezione 3
% _     _____ ________ ___  _   _ _____   _____
% | |   | ____|__  /_ _/ _ \| \ | | ____| |___ /
% | |   |  _|   / / | | | | |  \| |  _|     |_ \
% | |___| |___ / /_ | | |_| | |\  | |___   ___) |
% |_____|_____/____|___\___/|_| \_|_____| |____/


\section{Gruppo fondamentale}

\begin{definition}
  Sia $ X $ uno spazio topologico e $ x_0 $ un suo punto, allora un \textbf{laccio}\index{Laccio} è un arco in $ X $
  avente come punto di partenza e punto di arrivo il punto $ x_0 $. Un laccio $ C_{x_0} $ si dice \textbf{costante} se $ \forall t \in I $
  $ C_{x_0}(t) = x_0 $ con $ x_0 \in X $.
\end{definition}
\noindent
Vorrei strutturare l'insieme dei lacci in uno spazio $ X $ come un gruppo con l'operazione di giunzione
e avente come unità il laccio costante.

\begin{definition}
  Siano $ f, g $ due lacci, si definisce l'operazione $ \star $ detta \textbf{cammino
    composto}, o \textbf{giunzione}, \index{Cammino composto}
  \index{Giunzione!\vedi{Cammino composto}} come:
  \[
    (f \star g)(t) =
    \begin{cases}
      f(2t) & \text{se } 0 \leq t \leq \frac{1}{2} \\
      g(2t -1) & \text{se } \frac{1}{2} \leq t \leq 1
    \end{cases}
  \]
\end{definition}
Il cammino composto è un laccio di base $ x_0 $ percorso a velocità doppia,
metà del tempo percorso su $ f $ l'altra metà su $ g $.
Il problema di questa costruzione è non sempre la giunzione di un laccio con il
suo inverso è il laccio costante. Per questo si passa al quoziente rispetto la
relazione di omotopia.

\begin{definition}
  Sia $ X $ uno spazio topologico e $ x_0 \in X $ un suo punto, allora la coppia $ (X, x_0) $ è detta \textbf{spazio topologico puntato}.
  \index{Spazio topologico puntato}
\end{definition}

\newmathsymb{homotop}{\sim_H}{Relazione di omotopia}
\begin{definition}
  Sia $ (X, x_0) $ uno spazio topologico puntato e $ f: I \to X $ una mappa continua tale che $ f(0) = f(1) = x_0 \; \forall t \in I $,
  cioè un laccio di base $ x_0 $,
  si dice che una funzione continua $ g $ è \textbf{omotopicamente equivalente} a $ f $ ($ g \sim_H f $) se esiste una funzione
  continua $ F \colon I \times I \to X $ tale che:
  \begin{itemize}
  \item $ F(0,x) = f(x) \; \forall x \in I $
  \item $ F(1,x) = g(x) \; \forall x \in I $
  \item $ F(t,0) = x_0 \; \forall s \in I $
  \item $ F(t,1) = x_0 \; \forall s \in I $
  \end{itemize}
  La relazione $ \sim_H $ è detta \textbf{relazione di omotopia tra lacci}\index{Relazione di omotopia} \index{Omotopia! \vedi{Relazione di omotopia}}
  e si dimostra essere una relazione di equivalenza.
\end{definition}

\begin{figure}[htbp]
  \centering
  \begin{tikzpicture}
    \draw (0, 0) rectangle (3,3);
    \draw[-Latex] (-0.5, 1) -- (-0.5, 2);
    \draw[-Latex] (1, -0.5) -- (2, -0.5);
    \node[left] () at (-0.5, 1.5) {$ t $};
    \node[below] () at (1.5, -0.5) {$ x $};
    \node[right] () at (0, 1.5) {$ x_0 $};
    \node[left] () at (3, 1.5) {$ x_0 $};
    \node[above] () at (1.5, 0) {$ f(x) $};
    \node[below] () at (1.5, 3) {$ g(x) $};
    \draw (0,2) -- (3,2);
    \draw[-Latex] (0, 2) -- (1.5, 2);
    \node[below] () at (1.5, 2) {$ F(t,x) $};
    \draw (6,1.5) circle [x radius=2, y radius=1];
    \node[above] () at (6, 2.5) {$ f $};
    \draw (6, 1.5) circle (0.5);
    \node[above] () at (6, 1.5) {$ g $};
    \draw[-Latex] (4.5, 1.5) -- (5, 1.5);
    \draw[-Latex] (7.5, 1.5) -- (7, 1.5);
  \end{tikzpicture}
  \caption{Omotopia: deforma $ f $ in $ g $ in modo continuo.}
  \label{fig:lez3:homotopy}
\end{figure}
\newmathsymb{fondgroup}{\pi_1}{Gruppo fondamentale}
\begin{definition}
  Dato uno spazio topologico puntato $ (X,x_0) $ si definisce il \textbf{gruppo
    fondamentale}\index{Gruppo fondamentale} come l'insieme:
  \[
    \pi_1(X,x_0) = \quot{\set{f \colon I \to X | f \text{ continua}, f(0) = f(1) = x_0}}{\sim_H}
  \]
  equipaggiato con un'operazione l'operazione di giunzione, cioè se
  $ [f], [g] \in \pi_i(X,x_0) $, si definisce $ [f][g] := [f \star g] $, l'elemento
  neutro di questa operazione è il cammino costante
  $ 1_{\pi_1(X,x_0)} = [C_{x_0}] $ con $ C_{x_0}(t) = x_0 \; \forall t $. L'inverso di
  un elemento invece è $ [f]^{-1} = [\bar{f}] $ dove $ \bar{f} $ è il cammino
  percorso in verso opposto, cioè definito da $ \bar{f}(t) = f(1-t) $, in questo
  modo $ \bar{f}(0) = f(1) $ e $ \bar{f}(1) = f(0) $.
\end{definition}

Alcune proprietà del gruppo fondamentale:
\begin{enumerate}
\item $ \pi_1(X,x_0) $ è invariante omotopico, cioè se $ X \sim_H Y $, cioè se
  \[
    \exists f: X \to Y, g: Y \to X \; | \; f \circ g \sim_H 1_Y \text{ e }  g \circ f \sim_H 1_X
  \]
  allora $ \pi_1(X,x_o) \cong \pi_1(Y,f(x_0)) $. Questo in particolare porta alla seguente
  utile osservazione:
  \begin{osservation}
    Se due spazi topologici puntati hanno gruppi fondamentali non isomorfi allora
    non possono essere omotopicamente equivalenti.
  \end{osservation}
\item Se $ X $ è \textbf{contraibile}\index{Spazio contraibile} (cioè è
  omotopo ad un punto) allora vale che $ \pi_1(X,x_0) \cong 1 $, cioè il gruppo
  fondamentale è banale.
\item Si dimostra che:
  \begin{proposition}
    Se uno spazio tologico $ X $ è connesso per archi allora tutti i gruppi fondamentali
    degli spazi puntati $ (X,x_0) $ sono isomorfi, cioè si può omettere la dipendenza da $ x_0 $.
  \end{proposition}
  Questo intuitivamente è vero perché se gli spazi sono connessi per archi allora esistono cammini
  che collegano qualunque coppia di punti.
\end{enumerate}

\begin{definition}
  Uno spazio topologico connesso per archi si dice \textbf{semplicemente connesso}\index{Semplicemente connesso}
  se il suo gruppo fondamentale è banale.
\end{definition}

\begin{osservation}
  Tutti gli spazi contraibili sono semplicemente connessi, ma
  non tutti gli spazi semplicemente connessi sono contraibili, come ad esempio $ \Sph{2} $.
\end{osservation}

\begin{enumerate}
  \setcounter{enumi}{3}
\item $ \pi_1(\Sph{1}) \cong \Z $, infatti si può costruire la mappa:
  \begin{align*}
    \sigma \colon I & \to \Sph{1} \\
    t & \mapsto  \me^{2 \pi i t}
  \end{align*}
  Questa è tale che $ \sigma(0) = \sigma(1) = 1 $ quindi $ [\sigma] \in \pi_1(\Sph{1}) $ e:
  \begin{align*}
    \pi_1(\Sph{1}) & \to \Z \\
    [\sigma] & \mapsto  1
  \end{align*}
  Ogni elemento è multiplo di $ \sigma $ e il fattore di proporzionalità conta il numero di avvolgimenti
  con segno del cammino su sé stesso.
\item $ \pi_1(X \times Y) \cong \pi_1(X) \times \pi_1(Y) $
\item Il gruppo fondamentale si calcola o partendo da gruppi omotopi oppure utilizzando il \textbf{teorema di Seifert–van Kampen}, \index{Teorema di Seifert–van Kampen}
  il quale fornisce un metodo algoritmico per il calcolo.
\end{enumerate}

\begin{example}
  Si definisce:
  \[
    V_g =
    \begin{cases}
      \Sph{2} & \text{ se } g = 0 \\
      P_{\frac{4g}{N}} & \text{ se } g \geq 1 \in \mathbb{N}
    \end{cases}
  \]
  con $ P_{\frac{k}{N}} $ poligono con $ k $ lati e con identificazioni a coppie, come
  ad esempio nel caso $ g = 1 $ si ottiene un toro piatto identificando lati opposti
  di un quadrato.
  \begin{figure}[htbp]
    \centering
    \begin{subfigure}[htbp]{.45\linewidth}
      \centering{}
      \begin{tikzpicture}
        \fill[gray!20] (0,0) rectangle (3,3);
        \draw (0,0) rectangle (3,3);
        \draw[-Latex] (0,0) -- (0,1.5);
        \draw[-Latex] (3,3) -- (1.5,3);
        \draw[-Latex] (3,0) -- (3,1.5);
        \draw[-Latex] (0,0) -- (1.5,0);
        \node[left] () at (0,1.5) {$ a $};
        \node[above] () at (1.5,3) {$ b $};
        \node[right] () at (3,1.5) {$ a $};
        \node[below] () at (1.5,0) {$ b $};
      \end{tikzpicture}
      \caption{Toro piatto, o anche toro di Clifford}
      \label{fig:lez3:clifford_torus}
    \end{subfigure}
    \begin{subfigure}[htbp]{.45\linewidth}
      \centering
      \def\svgwidth{0.9\textwidth}
      \input{images/torus_generators.pdf_tex}
      \caption{Generatori di un toro}
      \label{fig:lez3:torus_generators}
    \end{subfigure}
    \label{fig:lez3:torus}
    \caption{Toro}
  \end{figure}
  Si usano simboli combinatori per descrivere l'identificazione: si definisce un verso di percorrenza, si assegnano delle lettere a ciascun
  lato e si scrivono in ordine tali lettere, aggiungendo un esponente $ -1 $ quando il verso di percorrenza è opposto. In questo caso quindi
  si ha $ aba^{-1}b^{-1} $. Questo si estende a poligoni con $ 4g $ lati e si usa l'identificazione
  $ a_1 b_1 a_1^{-1} b_1^{-1} \dots a_g b_g a_g^{-1} b_g^{-1} $.

  Si dimostra che queste sono varietà topologiche nel senso che verrà ora
  riportato, in particolare per $ g = 1 $ si ha un toro, per $ g = 2 $ un
  bitoro, \dots. $ g $ è detto \textbf{genere}\index{Genere}.

  \begin{definition}
    Una \textbf{varietà topologica}\index{Varietà topologica} $ \M $ è uno spazio topologico
    che localmente sembra uno spazio reale $ n $-dimensionale, cioè tale che esiste un
    interno $ n $ detto \textbf{dimensione}\index{Dimensione di una varietà topologica}
    tale ogni punto in $ \M $ possiede un intorno che è omeomorfo a $ \RN{n} $.
  \end{definition}
  Si trova con il teorema di Seifert-Van Kampen che:
  \[
    \pi_1(V_g) \cong
    \begin{cases}
      1 & \text{se } g = 0 \\
      \Z \oplus \Z & \text{se } g = 1 \\
      \langle a_1 b_1 \dots \Pi_{i=1}^g [a_i,b_i] = 1 \rangle & \text{se } g > 1
    \end{cases}
  \]
  Dove $ [,] $ è il commutatore, cioè esattamente $ a_1 b_1 a_1^{-1} b_1^{-1} \dots a_g b_g a_g^{-1} b_g^{-1} $.
  Solo per $ g = 0 $ o $ g = 1 $ si ottengono dei gruppi abeliani, ma io vorrei averlo sempre abeliano, quindi lo abelianizzo.
  \[
    \Ab{\pi_1(X)} = \quot{\pi_1(X)}{[\pi_1(X), \pi_1(X)]} = \quot{\pi_1(X)}{\pi_1'(X)}
  \]
  Chiaramente questo gruppo è abeliano e si calcola facilmente che $ \Ab{\pi_1(V_g)} \cong \Z^{2g} $ per $ g \geq 2 $,
  infatti il gruppo è generato su $ 2 g $ generatori $ a_1, b_1, a_2, b_2, \dots, a_g, b_g $ e poi
  si impone la relazione di identificazione e i commutatori diventano tutti banali.
  Si vedono facilmente anche i generatori, ad esempio per un toro sono riportati in figura.
  % L'abelianizzato è uno $ \Z $-modulo essendo un gruppo abeliano.
\end{example}


\subsection{Omomorfismo tra $ \RN{} $ e $ \RN{N} $}


\begin{proposition}
  Se $ f:X \to Y $ è una mappa continua suriettiva tra spazi topologici e se
  $ X $ è connesso per archi allora $ Y $ è connesso per archi. Questo vale in
  particolare se $ f $ è un omeomorfismo, cioè la connessione per archi è una
  proprietà invariante per omeomorfismi.
\end{proposition}

\begin{proof}
  Siano $ y_0, y_1 $ due punti di $ Y $. La funzione $ f $ è suriettiva, e
  dunque esistono $ x_0 $ e $ x_1 $ in $ X $ tali che $ f(x_0)=y_0 $ e
  $ f(x_1)=y_1 $. Dato che $ X $ è connesso, esiste un cammino
  $ \alpha:[0,1] \to X $ tale che $ \alpha(0)=x_0 $ e $ \alpha(1)=x_1 $. Ma la composizione di
  funzioni continue è continua, e quindi il cammino ottenuto componendo
  $ \alpha $ con $ f $: $ f \circ \alpha : [0,1] \to X \to Y $ è un cammino continuo che parte da
  $ y_0 $ e arriva a $ y_1 $.
\end{proof}
\hfill \newline \noindent
Si sa inoltre che:
\begin{proposition}
  $ \RN{n} $ è connesso per archi $ \forall n \in \mathbb{N} $.
\end{proposition}

È noto che $ \RN{} \not \simeq \RN{N} $ per $ n \geq 2 $, infatti basta togliere un punto a $ \RN{} $ che diventa sconnesso per archi
mentre $ \RN{N} $ rimane connesso per archi anche togliendogli un punto. In questa dimostrazione ho utilizzato
il seguente risultato fondamentale:
\begin{proposition}
  Se $ f: X \to Y $ è omeomorfismo tra spazi topologici allora $ f \rvert_U : U \to f(U) $ è omeomorfismo per ogni $ U \subseteq X $.
\end{proposition}
Nel caso considerato $ U = \Sph{n} \setminus {x_0} $, siccome ho trovato un
$ U $ per cui la funzione ristretta non è omeomorfismo $ f $ non può essere
omeomorfismo. Infatti l'immagine di un punto rimane un punto.

\noindent
Questo risultato vale anche che $ \RN{2} \not \simeq \RN{N} $ per $ n \geq 3 $, infatti:

\newmathsymb{homoto}{\homoto}{Omeomorfismo}
\begin{proof}
  Per assurdo $ f : \RN{2} \homoto \RN{N} $ è un omeomorfismo con
  $ n \geq 3 $, tolgo un punto da $ \RN{2} $, se $ f $ omeomorfismo anche la restrizione deve essere omeomorfismo, cioè
  $ \forall p \in \RN{2} \quad f:\RN{2} \setminus \set{p} \homoto \RN{N} \setminus \set{f(p)} $.
  Ma $ \RN{2} \setminus \set{p} \simeq \RN{} \times \mathcal{S}^1 $ con la mappa
  $ \vec{x} \mapsto \left( || \vec{x} ||, \frac{\vec{x}}{|| \vec{x} ||} \right) $ (dopo aver fatto
  una traslazione di $ p $ nell'origine, operazione che è certamente un omeomorfismo).
  % In pratica sto dicendo che il piano senza un punto è omeomorfo ad un cilindro infinito.
  Analogamente $ \RN{n} \setminus \set{f(p)} \simeq \RN{} \times \Sph{n-1} $. Quindi se esiste un omeomorfismo tra $ \RN{2} $ e
  $ \RN{n} $ significherebbe che $ \RN{} \times \Sph{1} \simeq \RN{} \times \Sph{n-1} $, ma quindi i gruppi fondamentali
  dovrebbero essere isomorfi:
  $ \pi_1 (\RN{} \times \Sph{1}) \simeq \pi_1(\RN{}\times \Sph{n-1}) $ ma
  $ \pi_1 (\RN{} \times \Sph{1}) = \Z $ infatti il gruppo fondamentale di un prodotto è il prodotto dei gruppi
  fondamentali e $ \pi_1(\RN{}) = 1 $, $ \pi_1(\Sph{1}) = \Z $ dato che i lacci omotopicamente distinti
  sono quelli che avvolgono il buco un numero differente di volte. Analogamente $ \pi_1(\RN{}\times \Sph{n-1}) = 1 $
  perché le sfere sono contraibili. Trovo quindi che dovrebbero essere isomorfi $ \pi_1 (\RN{} \times \Sph{1}) = \Z $
  e $ \pi_1(\RN{}\times \Sph{n-1}) = 1 $ che è assurdo.
\end{proof}
\hfill \newline \newline \noindent
Ho quindi dedotto proprietà topologiche a partire da considerazioni algebriche (con il gruppo fondamentale).
Il gruppo fondamentale è un invariante algebrico per problemi topologici, provo ad utilizzarlo per mostrare
l'analogo risultato per $ \RN{3} $.

% \begin{definition}
%   Si definisce il \textbf{gruppo fondamentale}\index{Gruppo fondamentale} di uno spazio topologico $ X $
%   connesso per archi attorno al punto $ x_0 \in X $
%   \[
%     \pi_1 (X, x_0) = \quot{\set{ g: \Sph{1} \to X | g \text{ continua}, g(1) = x_0}}{\sim}
%   \]
%   e $ \sim $ è la relazione di omotopia: $ g_1 \sim g_2 $ se $ \exists G: \mathcal{S}^1 \times I \to X  $ tale che
%   $ G(z,0) = g_1(z), G(z,1) = g_2(z), G(1,t) = x_o $ con $ G $ continua. In questo vedo $ \Sph{1} $ come sottospazio
%   di $ \RN{2} $ con la topologia indotta (il punto $ 1 $ è un punto della circonferenza vedendola come
%   insieme nello spazio complesso $ \Sph{1} = \set{ z \in \mathbb{C} | |z| = 1} $).
% \end{definition}
% Sostanzialmente il gruppo fondamentale è l'insieme dei lacci quozientato rispetto alla relazione di omotopia.
% Infatti $ g $ è un laccio dato che è un arco e il punto di partenza e il punto di arrivo necessariamente
% coincidono dato che $ g $ è definito su $ \Sph{1} $.
% Questo perché l'insieme dei lacci non è strutturabile come gruppo in quanto il laccio costante non è
% l'unità.

\begin{proposition}
  Non esiste omomorfismo tra $ \RN{3} $ e $ \RN{N} $.
\end{proposition}

\begin{proof}
  Come nel caso precedente suppongo esiste $ f $ omeomorfismo tra $ \RN{3} $ a
  $ \RN{n} $, tolgo $ q $ da $ \RN{3} $ e $ f(q) $ da $ \RN{n} $, quindi ottengo
  l'omomorfismo tra $ \RN{} \times \Sph{2} \simeq \RN{} \times \Sph{n-1} $, ma i gruppi
  fondamentali associati sono banali, quindi sono isomorfi, e non è possibile
  replicare il ragionamento utilizzato sopra.
\end{proof}
\hfill \newline \newline \noindent
Poincaré introdusse i gruppi di  omotopia superiore per risolvere questo e altri problemi.
\begin{definition}
  Si definiscono i \textbf{gruppi di omotopia superiore}\index{Gruppi di omotopia superiore} di uno spazio topologico $ X $
  attorno al punto $ x_0 $ per $ k \geq 2 $:
  \[
    \pi_k(X) (X, x_0) = \quot{\set{ g: \Sph{k} \to X | g \text{ continua}, \; g(p_0) = x_0}}{\sim}
  \]
  Con $ p_0 \in \Sph{k} $ e $ \sim $ relazione di omotopia.
\end{definition}
Studiare i gruppi di omotopia superiore è un problema aperto della topologia moderna.
Tuttavia si sa che:
\begin{enumerate}
\item $ \pi_k(\Sph{m}) = 1 \quad \text{per} \quad 1 \leq k < m \quad (m > 2)$
\item $ \pi_m(\Sph{m}) \simeq \Z \quad \text{per} \quad k = m $
\item $ \pi_1(\Sph{2}) = 1 $
\item $ \pi_2(\Sph{2}) \simeq \Z $
\item $ \pi_3(\Sph{2}) \simeq \Z $\footnote[$\dagger$]{Questo dà origine alla fibrazione di Hopf che ha molte applicazioni in fisica.}
\end{enumerate}

% Anche se non so calcolare i gruppi di omotopia superiore non vorrei buttarli via \dots
% Vorrei degli invarianti algebrici per problemi topologici, come i gruppi di omotopia.
Per evitare di utilizzare i gruppi di omotopia superiore introduco i gruppi di omologia.


%%% Local Variables:
%%% ispell-local-dictionary: "italiano"
%%% mode: latex
%%% TeX-master: "notes"
%%% End:
                             .
\chapter{Omologia singolare}

\section{Introduzione}

Si introduce la teoria dell'omologia per semplificare problemi, infatti la
teoria dell'omologia serve ad associare agli spazi topologici oggetti
algebrici meno complicati dei gruppi di omotopia. Sono stati sviluppati diversi
tipi di omologia:
\begin{itemize}
\item Omologia singolare
\item Omologia cellulare
\item Omologia persistente\footnote{Questa ha numerose applicazioni pratiche, come la ricostruzione di immagini.}
\item Omologia simpliciale
\end{itemize}
Quello che farò sarà associare ad ogni spazio topologico (anche patologico)
gruppi abeliani e omomorfismi a partire da applicazioni continue tra due spazi
topologici. Fino a quando non sarà espressamente indicato, lavoro sempre con
anello di base $ \Z $, che quindi rimane sottinteso a meno di scriverlo
esplicitamente.

\section{Simplessi singolari}

\newmathsymb{simplexstd}{\Delta_k}{Simplesso standard}
\begin{definition}
  In $ \RN{k+1} $ si definisce il \textbf{simplesso standard}\index{Simplesso standard} $ \Delta_k $ l'insieme:
  \[
    \Delta_k = \set{(x_1,x_2,\dots) \in \RN{k+1} | \forall i \; 0 \leq x_i \leq 1 \text{ e } \sum_{i=1}^{k+1}x_i = 1}
  \]
  Le coordinate $ x_i $ sono dette \textbf{coordinate baricentrali}\index{Coordinate baricentrali}.
\end{definition}

\begin{osservation} Alcuni esempi sono:
  \begin{itemize}
  \item $ \Delta_0 $ è un punto.
  \item $ \Delta_1 $ è un segmento, che è omeomorfo a $ [0,1] $.
    \begin{figure}[htbp]
      \centering
      \begin{tikzpicture}
        \draw[-Latex] (0,0) -- (3,0);
        \draw[-Latex] (0,0) -- (0,3);
        \draw[thick] (0,2) -- (2,0);
        \node[below] () at (2,0) {1};
        \node[left] () at (0,2) {1};
      \end{tikzpicture}
      \caption{1-Simplesso standard}
      \label{fig:lez1:1_standard_simplex}
    \end{figure}
  \item $ \Delta_2 $ è un triangolo
  \item $ \Delta_3 $ è un tetraedro
  \item \dots
  \end{itemize}
\end{osservation}
\begin{figure}[htbp]
  \centering
  \begin{subfigure}{.2\textwidth}
    \centering
    \begin{tikzpicture}
      \draw (0,0) circle (0.05);
    \end{tikzpicture}
    \caption{$ \Delta_0 $}
  \end{subfigure}
  \begin{subfigure}{.2\textwidth}
    \centering
    \begin{tikzpicture}
      \draw (0,0) -- (2,0);
    \end{tikzpicture}
    \caption{$ \Delta_1 $}
  \end{subfigure}
  \begin{subfigure}{.2\textwidth}
    \centering
    \begin{tikzpicture}
      \draw (0,0) -- (2,0) -- (1, 1.7) -- cycle;
    \end{tikzpicture}
    \caption{$ \Delta_2 $}
  \end{subfigure}
  \begin{subfigure}{.33\textwidth}
    \centering
    \def\svgwidth{0.56\textwidth}
    \input{images/Tetrahedron.pdf_tex}
    \caption{$ \Delta_3 $}
  \end{subfigure}%
  \caption{Simplessi standard}
  \label{fig:lez1:standard_simplexes}
\end{figure}

\begin{definition}
  Dato uno spazio topologico $ X $ si definisce il \textbf{$ k $-simplesso singolare}\index{$ k $-simplesso singolare}
  in $ X $ come un'applicazione continua $ \sigma: \Delta_k \to X $.
\end{definition}
Spesso conviene identificare il $ k $-simplesso con la sua immagine in $ X $.
In questo modo uno $ 0 $-simplesso è un punto in $ X $, mentre un $ 1 $-simplesso singolare potrebbe
essere sia un segmento che un punto (se la mappa è costante).
Siccome non c'è relazione tra la dimensione dello spazio di partenza e lo spazio di arrivo
(ad esempio la curva di Peano) il simplesso può deformare, ed è per questo che è detto singolare.

\begin{example}
  Un esempio di $ k $-simplesso singolare in cui è particolarmente evidente la possibilità di fare l'identificazione
  è la mappa identità: $ \Id{} \colon \Delta_k \to \Delta_k $.
\end{example}
\begin{osservation}
  Quando è possibile faccio un abuso di notazione e identifico la mappa con la sua immagine
  nello spazio topologico.
\end{osservation}

Voglio costruire un complesso di gruppi abeliani e definire l'omologia singolare
come l'omologia di tale complesso.

\begin{definition}
  Si definisce lo spazio delle $ k $-\textbf{catene
    singolari}\index{$ k $-catene singolari} come il gruppo generato da tutte le
  possibili applicazioni continue da $ \Delta_k $ a $ X $, cioè:
  \[
    S_k(X) = \langle \set{g | g \text{ $ k $-simplesso singolare in $ X $}} \rangle
  \]
  Cioè:
  \begin{align*}
  S_k(X) ={}& \{\text{combinazioni lineari finite a coefficienti interi: } \\
            & \sum_g n_g g \;|\; n_g \in \Z, g \; k-\text{simplessi singolari di } X \}
  \end{align*}
\end{definition}
$ S_k(X) $ è un gruppo abeliano con l'operazione somma definita naturalmente:
\[
  \sum_g n_g g + \sum_h n_h h =   \sum_g n_g g + \sum_g n_g^\star g = \sum_g (n_g + n_g^\star)g
\]
Inoltre $ \forall k < 0 $ si pone $ S_k(X) = 0 $. Un elemento generico di $ S_k(X) $
è una somma formale finita (cioè con un numero finito di coefficienti non nulli)
su tutti i possibili $ k $-simplessi singolari in $ X $.
% \begin{example}
%   \[
%     (n_1 g_1 + n_2 g_2 + 2 n_3 g_3) + (m_1 g_1 + m_4 g_4) = (n_1 + m_1)g_1 + n_2 g_2 + 2 n_3 g_3 + m_4 g_4
%   \]
% \end{example}
Questa è una somma con tutte le giuste proprietà. Lo zero è la catena con tutti
i coefficienti nulli, mentre l'inverso è la catena con i coefficienti opposti.
Si nota che le catene sono somme formali di mappe e non sono esse stesse mappe.

\begin{example}[$ k = 0$]
  Se $ k = 0 $ allora $ S_0(X) $ sono catene di punti ($ g_0 : \Delta_0 \to X $,
  identifico l'applicazione con il punto in $ X $ sapendo che l'immagine di un
  punto è un punto)
  \[
    S_0(X) = \set { \sum n_i p_i | n_i \in \Z, \; p_i \in X}
  \]
\end{example}

A questo punto considero la successione $ S_\bullet $ ($ S $ sta per singolare), cioè:
\[
  \dots \to S_{k+1}(X) \to S_k(X) \to S_{k-1}(X) \to \dots \to S_0(X)
\]
Per rendere $ S_\bullet $ un complesso bisogna le applicazioni tra i vari
$ S_k $, queste applicazioni saranno il bordo. A questo scopo noto
$ h: \Delta_1 \to X $ è arco, e posso ottenere una $ 0 $-catena prendendo i punti
estremi dell'arco, infatti il bordo di un $ 1 $-simplesso è uno $ 0 $-simplesso.
L'idea è quindi ottenere simplessi di ordine più piccolo prendendo il bordo dei
simplessi. Questa operazione si generalizza con l'operatore faccia.

\begin{figure}[htbp]
  \centering
  \begin{tikzpicture}
    \draw[-Latex] (0,0) -- (2,0);
    \draw[-Latex] (0,0) -- (0,2);
    \draw[thick] (0,1) -- (1,0);
    \draw plot [smooth cycle] coordinates {(5,2) (6,3) (7,3) (6,0) (4,0)};
    \node () at (7,1) {$ X $};
    \draw plot [smooth, tension = 1] coordinates {(6,2) (5,1) (5.5,0.5)};
    \node[circle] () at (6,2) {\textbullet};
    \node[circle] () at (5.5,0.5) {\textbullet};
    \draw[->] (0.75,0.5) -- (4.75,1);
  \end{tikzpicture}
  \caption{1-Simplesso singolare}
  \label{fig:lez1:1_standard_simplex_with_arc}
\end{figure}

\begin{definition}
  Sia $ \Delta_k $ un $ k $-simplesso standard con $ k \geq 0 $ si definisce l'operatore \textbf{faccia}\index{Operatore faccia}
  come la mappa $ F_i^{\;k}: \Delta_{k-1} \to \Delta_k $ tale che $ F_i^{\;k}(\Delta_{k-1}) $ è una faccia di $ \Delta_k $.
\end{definition}
L'operatore faccia prende un $ k $-simplesso standard e lo immerge in un qualche senso in un
simplesso più grande, ad esempio manda un punto in uno degli estremi di un segmento (nel caso $ k = 0 $),

\begin{example}[$ k = 2 $]
  Per $ k = 2 $ vale che:
  \[
    \Delta_2 = \set{ (x_1,x_2,x_3) \in \RN{3} | x_1 + x_2 + x_3 = 1, \; 0 \leq x_i \leq 1 \; \forall i}
  \]
  Si definisce la base $ e_0 = (1,0,0) \; e_1 = (0,1,0) \; e_2 = (0,0,1) $, voglio vedere il bordo del triangolo
  come facce.

  \begin{figure}[htbp]
    \centering
    \begin{tikzpicture}
      \draw[-Latex] (0,0) -- (4,0);
      \draw[-Latex] (0,0) -- (0,4);
      \draw[-Latex] (0,0) -- (-2,-2);
      \node[below] () at (2,0) {$ e_1 $};
      \node[right] () at (0,2) {$ e_2 $};
      \node[right, below] () at (-1,-1) {$ e_0 $};
      \draw (-1,-1) -- (2,0);
      \draw (0,2) -- (2,0);
      \draw (0,2) -- (-1,-1);
      \draw (5,0) -- (9,0) -- (7,3) -- cycle;
      \node[left] () at (5,0) {$ e_0 $};
      \node[right] () at (9,0) {$ e_1 $};
      \node[above] () at (7,3) {$ e_2 $};
      \node[below] () at (7,0) {$ F_2^{(2)}(\Delta_1) $};
      \node[left] () at (6,2) {$ F_1^{(2)}(\Delta_1) $};
      \node[right] () at (8,2) {$ F_0^{(2)}(\Delta_1) $};
    \end{tikzpicture}
    \caption{Azione dell'operatore faccia}
    \label{fig:lez1:standard_simplex_faces}
  \end{figure}
\end{example}

Il segmento faccia $ i $-esimo è quello che non contiene il vertice $ i $-esimo, cioè
\emph{dimentico} un punto e gli altri punti diventano vertici del simplesso.

In generale se $ \Delta_k $ è un simplesso standard si definisce la base canonica come (si noti
che la base canonica è ordinata):
\begin{gather*}
  e_0 = (1,0,0,\dots)                            \\
  e_1 = (0,1,0,\dots)                            \\
  e_2 = (0,0,1,\dots)                            \\
  \dots
\end{gather*}
Questi sono i vertici del simplesso, definisco l'azione dell'operatore faccia
come:
\[
  \begin{cases}
    F_i^{\; k}(e_j) = e_{j+1}     & \text{se } j \geq i \\
    F_i^{\; k}(e_j) = e_{j} & \text{se } j < i
  \end{cases}
\]

% Se fosse un tetraedro dimenticando punti ottengo triangoli e dimenticando
% triangoli ottengo punti, come è giusto.

\begin{exercise}
  Dimostrare che se $ [\cdot, \cdot] $ indica l'inviluppo convesso allora:
  \begin{enumerate}
  \item Per $ j > i $ vale che $ F_j^{\; k+1} \circ F_i^{\; k} = [e_0, \dots, \hat{e}_i, \dots, \hat{e}_j, \dots, e_k ] $.
  \item Per $ j \leq i $ vale che $ F_j^{\; k+1} \circ F_i^{\; k} = [e_0, \dots, \hat{e}_j, \dots, \hat{e}_{i+1}, \dots, e_k ] $.
  \end{enumerate}
  dove i cappucci indicano che quell'elemento è omesso.
\end{exercise}

\begin{definition}
  L'\textbf{inviluppo convesso}\index{Inviluppo convesso} di un insieme $ U $ in
  $ \RN{n} $ è il più piccolo insieme convesso che contiene $ U $, dove un
  insieme in $ \RN{n} $ si dice \textbf{convesso}\index{Insieme convesso} se
  contiene il segmento che unisce ogni coppia di punti dell'insieme.
\end{definition}

\begin{definition}
  Dato un $ k $-simplesso singolare $ \sigma: \Delta_k \to X $ una sua faccia è data dalla
  mappa $ \sigma^{(i)} \colon \Delta_{k-1} \to X $ cioè la restrizione di
  $ \sigma $ sulla faccia $ i $-esima del simplesso, cioè
  $ \sigma^{(i)} = \sigma \circ F_i^{\; k} $, si definisce quindi il
  \textbf{bordo}\index{Bordo} come la mappa:
  \begin{align*}
    \partial \colon \Sigma_k(X) & \to \Sigma_{k-1}(X) \\
    \sigma & \mapsto  \sum_{i=0}^{k}(-)^i \sigma^{(i)}
  \end{align*}
  dove $ \Sigma_k(X) $ indica lo spazio dei $ k $-simplessi singolari di $ X $.
  % di $ \sigma $ come $ \partial_k \sigma = \sum_{i=0}^{k}(-)^i \sigma^{(i)} $.
\end{definition}
Il bordo sostanzialmente corrisponde alla somma alterna delle facce.

\begin{figure}[htbp]
  \centering
  \begin{tikzpicture}
    \draw[-Latex] (0,0) -- (3,0);
    \draw[-Latex] (0,0) -- (0,3);
    \draw[-Latex] (0,0) -- (-1.5,-1.5);
    \draw (-1,-1) -- (2,0);
    \draw (0,2) -- (2,0);
    \draw (0,2) -- (-1,-1);
    \draw plot [smooth cycle] coordinates {(5,2) (6,3) (7,3) (6.5,0) (4,0)};
    \node () at (7.5,1) {$ X $};
    \draw plot [smooth, tension = 1] coordinates {(6,2) (5,1) (5.5,0.5)};
    \draw plot [smooth, tension = 1] coordinates {(6,2) (6.2,0.7) (5.5,0.5)};
    \draw[->] (0.75,0.5) -- (4.75,1);
    \node[above] () at (2.5, 0.75) {$ \sigma $};
    \draw[->] (0.35,-0.65) to [out=-30,in=-120]  (5.35,0.45);
    \node () at (2.5, -1.45) {$ \sigma^{(i)} $};
  \end{tikzpicture}
  \caption{Azione di $ \sigma $ e $ \sigma^{(i)} $}
  \label{fig:lez1:sigma}
\end{figure}

\begin{example}[$ k = 1 $]
  Per $ k = 1 $ vale che $ \partial_1 \sigma = p_1 - p_0 $, infatti:
  \begin{align*}
    \sigma^{0} = \sigma \circ F_0^{\; 1} = \sigma(1) = p_1 \\
    \sigma^{1} = \sigma \circ F_1^{\; 1} = \sigma(0) = p_0
  \end{align*}
  Il bordo è la somma con i segni alternati: $ \partial_1 \sigma = p_1 - p_0 $. Tecnicamente
  il bordo è una mappa quindi sarebbe più corretto scrivere
  $ \partial_1 \sigma = \sigma^{(1)} - \sigma^{(0)} $ dove l'azione di queste due mappe è quella di
  mandare un estremo dell'intervallo $ [0,1] $ in $ p_0 $ o $ p_1 $.
\end{example}

Si è quindi definito il bordo sui simplessi singolari, ma si può generalizzare
la definizione sull'intero gruppo di catene
$ \partial_k: S_k(X) \to S_{k-1}(X) $ estendendo la definizione per linearità
$ \partial_k \left( \sum_g n_g g\right) = \sum_g n_g \partial_k g $, dove $ g $ sono simplessi
singolari, che sono i generatori di $ S $. A questo punto si
$ (S_\bullet, \partial) $ è una successione di gruppi abeliani, per mostrare che è un
complesso bisogna verificare che $ \partial_k $ è un omomorfismo e che
soddisfa $ \partial_k \circ \partial_{k+1} = 0 $.

\begin{proposition}
  La mappa $ \partial \colon S_k(X) \to S_{k-1}(X) $ è un omomorfismo.
\end{proposition}
\begin{proof}
  \begin{gather*}
    \partial_k \left( \sum_g n_g g + \sum_g m_g g\right) = \partial_k \left( \sum_g(m_g + n_g)g \right) = \sum_g (m_g + n_g) \partial_k g = \\
    = \sum_g n_g \partial_k g + \sum_g m_g \partial_k g = \partial_k \left( \sum_g n_g g\right) + \partial_k \left( \sum_g m_g g \right)
  \end{gather*}
  Dove si è usato che la mappa di bordo è lineare.
\end{proof}
\hfill \newline \newline \noindent
Una volta verificato che $ \partial_k \circ \partial_{k+1} = 0 $ (spesso come notazione si pone $ \partial^2 = 0 $)
il complesso sarà:
\[
  \begin{tikzcd}
    \dots \rar & S_{k+1}(X) \arrow{r}{\partial_{k+1}} & S_k(X) \arrow{r}{\partial_k} & S_{k-1}(X) \arrow{r}{\partial_{k-1}} & \dots
  \end{tikzcd}
\]
\vspace*{-12pt}
\begin{proposition}
  Vale che $ \partial_k \circ \partial_{k+1} = 0 $.
\end{proposition}
\begin{proof}
  È sufficiente verificare la proprietà sui generatori, quindi se $ \sigma $ è un
  $ k $-complesso singolare, cioè $ \sigma : \Delta_k \to X $ continua:
  \begin{gather*}
    \partial_k \circ \partial_{k+1} \sigma = \partial_k \left( \sum_{j=0}^{k+1}(-)^j (\sigma \circ F_j^{\; k+1}) \right) =  \sum_{j=0}^{k+1}(-)^j \partial_k (\sigma \circ F_j^{\; k+1}) = \\
    = \sum_{j=0}^{k+1} (-)^j \sum_{i=0}^k (-)^i (\sigma \circ F_j^{\; k+1}) \circ F_i^{\; k} = \sum_{j = 0}^{k+1} \sum_{i = 0}^{k} (-)^{j+i} \sigma \circ F_j^{\; k+1} \circ F_{i}^{\; k} =
  \end{gather*}
  Separo le somme con $ i < j $ e quelle con $ i \geq j $:
  \[
    = \sum_{0 \leq i < j \leq k + 1} (-)^{i+j} \sigma \circ F_j^{\; k+1} \circ F_i^{\; k} + \sum_{0 \leq j \leq i \leq k} (-)^{i+j} \sigma \circ F_j^{\; k+1} \circ F_i^k =
  \]
  Usando la proprietà degli inviluppi convessi si trova che se $ j \leq i $ allora
  $ F_j^{\; k+1} \circ F_j^{\; k} = F_{i+1}^{\; k+1} \circ F_k^{k} $, infatti se
  $ j \leq i $ allora $ i + 1 \geq j $ quindi in entrambi i membri l'inviluppo convesso è
  $ [e_0, \dots \hat{e}_j, \dots, \hat{e}_{i+1}, \dots, e_k] $. Quindi:
  \[
    = \sum_{0 \leq i < j \leq k + 1} (-)^{i+j} \sigma \circ F_j^{\; k+1} \circ F_i^{\; k} + \sum_{0 \leq j < i \leq k} (-)^{i+j} \sigma \circ F_{i+1}^{\; k+1} \circ F_j^{\; k}  =  0
  \]
  Dove nell'ultimo si è rinominato nel secondo termine $ i + 1 $ con $ i $, e ciò produce un segno
  meno che annulla la somma.
  % Si nota che è di importanza cruciale il fatto che si è definito il bordo con i
  % segni alternati.
\end{proof}

% lezione 2
% _     _____ ________ ___  _   _ _____   ____
% | |   | ____|__  /_ _/ _ \| \ | | ____| |___ \
% | |   |  _|   / / | | | | |  \| |  _|     __) |
% | |___| |___ / /_ | | |_| | |\  | |___   / __/
% |_____|_____/____|___\___/|_| \_|_____| |_____|


% Sia $ X $ uno spazio topologico, voglio definire l'omologia singolare $ H_k(X) $, cioè il $ k $-esimo gruppo di omologia
% singolare. Costruisco il complesso $ (S_\bullet(X), \partial) $ con:
% \[
%   S_k(X) = \set{ \sum_g n_g g | g \text{ simplesso singolare, } n_g \in \Z }
% \]
% E $ \partial_k : S_k(X) \to S_{k-1}(X) $ applicazione di bordo con $ \partial_k(g) = \sum_{i=0}^k(-)^ig^{(i)} $ con $ g: \Delta_k \to X $, e poi lo estendo per
% linearità su tutti gli elementi di $ S $, dove $ g^{(i)} = g \circ F_i^{\; k} $.

% Siccome $ \partial_{k-1} \circ \partial_k = 0 $ si ha il complesso
% \[
%   \begin{tikzcd}
%     \dots \arrow{r}{\partial_{k+1}} & S_k(X) \arrow{r}{\partial_k} & S_{k-1}(X) \arrow{r}{\partial_{k-1}} & \dots
%   \end{tikzcd}
% \]
% Inoltre $ \partial_k \circ \partial_{k-1} $ è la mappa nulla dalle catene singolari di $ S_k(X) $
% a quelle di $ S_{k-2}(X) $, in questo modo $ (S_\bullet(X), \partial) $ è un complesso di gruppi abeliani.

\section{Omologia singolare}

\begin{definition}
  Si definisce l'\textbf{omologia singolare}\index{Omologia singolare} $ H_k(X) $ dello spazio topologico $ X $
  come l'omologia del complesso $ (S_\bullet(X),\partial) $, cioè:
  \[
    H_k(X) := H_k(S_\bullet(X)) = \quot{\ker{\partial_k}}{\im{\partial_{k+1}}}
  \]
\end{definition}
% Posso quindi calcolare l'omologia di $ (S_\bullet(X),\partial) $ come l'avevo definita
% in precedenza:
% \[
%   H_k(S_\bullet(X)) = \quot{\ker{\partial_k}}{\im{\partial_{k+1}}}
% \]
% Vale che $ \ker{\partial_k} = \set{c \in S_k(X) | \partial_k(c) = 0} $, cioè le $ k $-catene con
% bordo nullo, questi sono chiamati $ k $-cicli.

\begin{definition}
  Sia $ (S_\bullet(X),\partial) $ un complesso di moduli, gli elementi di $ \ker{\partial_k} $ sono detti
  \textbf{$ k $-cicli}\index{$ k $-ciclo}. Un $ k $-ciclo è quindi una $ k $-catena
  con bordo nullo:
  \[
    c \text{ ciclo } \Leftrightarrow \partial c = 0
  \]
  L'insieme dei $ k $-cicli è indicato con $ Z_k(X) $, cioè: $ Z_k(X) = \ker{\partial_k} $.
  Si indica invece con $ B_k(X) $ l'insieme dei \textbf{bordi}\index{$ k $-bordo}, cioè le $ k $-catene singolari
  che sono immagini di $ k+1 $-catene, cioè esplicitamente:
  \[
    B_k(X) = \set{\eta \in S_k(X) | \exists b \in S_{k+1}(X), \partial b = \eta}
  \]
\end{definition}
Per definizione si ha quindi che $ H_k(X) = {Z_k(X)} \slash {B_k(X)} $, cioè il
gruppo di omologia è formato dai cicli modulo i bordi. Esplicitamente gli
elementi di $ H_k(X) $ sono classi di equivalenza tali che se $ \llbracket c \rrbracket \in H_k(X) $
con $ \partial c = 0 $ e $ c_1 \in  \llbracket c \rrbracket  $ allora $ c_1 - c \in B_k(X) $ e
$ \partial c_1 = 0 $ quindi esiste $ b $ tale che $ c_1 - c = \partial b $. Cioè due elementi
stanno nella stessa classe di equivalenza se differiscono per un bordo:

\newmathsymb{homolog}{\sim_{hom}}{Relazione di omologia}
\begin{definition}
  Due elementi $ a,b $ si dicono \textbf{omologhi} \index{Elementi omologhi} se differiscono per un bordo.
  \[
    a \sim_{hom} b \Leftrightarrow \exists c \; | \; \partial_k c = a - b
  \]
\end{definition}

\begin{osservation}
  Vale che $ H_k(X) = 0 $ se e solo se $ B_k(X) = Z_k(X) $, cioè se ogni ciclo è
  un bordo, come si è già osservato. In generale si ha che
  $ B_k(X) \subseteq Z_k(X) $ e possono esserci cicli che non sono immagini di bordi.
\end{osservation}
% $ \partial_k c $ è il bordo di un $ k $-ciclo, se $ \partial_k c = 0 $ significa che il ciclo non ha bordo, inoltre
% se $ c = \partial_{k+1} b $ allora $ c $ è bordo di qualcosa: $ c $ è un bordo che non ha bordo. Questo tipo
% di oggetti è di interesse centrale
Scopo del corso è studiare $ H_k(X) $ e capire se si possono determinare a meno
di isomorfismi, quello che si trova è In alcuni casi è possibile calcolare
esplicitamente i gruppi di omologia, come nel caso dell'omologia cellulare.

\subsection{$ H_0(X) $}

\begin{proposition}
  Sia $ X $ uno spazio topologico connesso per archi, allora $ H_0 \cong \Z $, cioè è uno $ \Z $-modulo libero di rango 1.
  In effetti $ H_0(X) $ \emph{conta} le componenti connesse per archi in $ X $ e quindi dà informazioni di natura geometrica.
\end{proposition}
\begin{proof}
  Calcolo $ H_0 $ a partire dalla definizione di omologia:
  \[
    H_0(X) = \quot{Z_0(X)}{B_0(X)}
  \]
  Ho il complesso:
  \[
    \begin{tikzcd}
      \dots \arrow{r}{} & S_1(X) \arrow{r}{\partial_1} & S_0(X) \arrow{r}{\partial_0}  & 0
    \end{tikzcd}
  \]
  Quindi $ Z_0 = \ker{\partial_0} = S_0(X) $ in quanto ogni elemento di $ S_0(X) $ viene
  mandato in $ 0 $.

  % Ma $ Z_0(X) = \set{ c \in S_o(X) | \partial_0 c = 0} $ e $ S_0(X) = \set{ \sum n_i p_i | n_i \in \mathbb{N}, p_i \in X} $.
  % % Tecnicamente uno $ 0 $-simplesso singolare è una mappa $ \sigma_0 : \Delta_0 \to X $ tale che manda $ \Delta_0 = 1 $ in $ \sigma_0(1) = p_0 $ e per
  % % questo è naturale l'identificazione con i punti immagine dello spazio topologico.
  % Sia $ c \in S_0(X) $ allora $ c = \sum n_i p_i $, e vale che $ \partial_0(c) = \sum n_i \partial_0 (p_i) = 0 $, infatti
  % $ \partial_0 : S_0(X) \to S_{-1}(X) $, ma per $ k < 0 $ $ S_{k} = 0 $ per definizione.
  Quindi per ora ho che:
  \[
    Z_0(X) = \ker{\partial_0} = S_0(X) \; \Rightarrow \; H_0(X) = \quot{S_0(X)}{B_0(X)}
  \]
  Per definizione
  $ B_0(X) = \im{\partial_1} = \set{ x \in S_0(X) | \exists \alpha \in S_1(X), \; \partial_1(\alpha) = x}
  $. % $ \alpha $ è una catena.
  Ma $ S_0(X) $ è il gruppo libero generato dagli $ 0 $-simplessi singolari, che
  sono mappe $ \Delta_0 \to X $, e siccome $ \Delta_0 $ è un punto si possono identificare
  con i punti di $ X $, perciò si può immaginare formalmente $ S_0(X) $ come il
  gruppo libero generato dai punti di $ X $. $ B_0(X) $ è l'insieme delle coppie
  di punti di $ X $ che sono bordo di un $ 1 $-simplesso singolare, il quale è
  una mappa $ \Delta_1 \cong I \to X $, cioè è un arco. Siccome lo spazio è connesso per
  archi ogni coppia di punti è bordo di qualcosa, fissando un punto $ x \in X $
  sostanzialmente $ B_0(X) $ lo si può immaginare come $ X $ stesso e quindi
  $ H_0(X) \cong \Z $ in quanto quoziente tra un gruppo libero generato da un
  insieme di punti e l'insieme di punti stessi, quindi esiste un'unica classe di
  equivalenza che è quella di un punto, in quanto ogni coppia di punti è omologa
  essendo collegata da un arco.

  Se ci sono più componenti connesse per archi posso ripetere il ragionamento senza connettere componenti
  distinte, quindi trovo che:
  \[
    H_0(X) \cong \Z^{N_c}
  \]
  Dove $ N_c $ è il numero di componenti connesse per archi di $ X $ con
  $ N_c < + \infty $, in pratica $ H_0(X) $ è generato da un insieme formato da un
  punto per ogni componente connessa per archi.
  % Se $ X $ non è connesso per archi ma è composto da diverse componenti connesse
  % per archi allora si può applicare il precedente ragionamento per ciascuna
  % componente, e quindi si otterrebbe $ \Z^n $ dove $ n $ e il numero di
  % componenti.
  %
  % Per questo $ H_0 (X) \cong \Z $ generato dalla classe $ [p] \; \forall p \in X $ (con $ X $ connesso per archi).
  % Sia $ p_0 \in X $, allora $ q \sim_{hom} p_0 $ se e solo se $ \exists \alpha \in S_1(X) $ tale che $ q - p_0 = \partial_1 \alpha $.
  % Per questo motivo i punti sono tutti omologhi, infatti
  % essendo $ X $ connesso per archi esiste un arco $ \alpha $ che connette $ q $ e $ p_0 $, ma
  % per definizione gli archi sono applicazioni continue da $ \Delta_1 $ a $ X $ che hanno come bordo $ q - p_0 $.
  % Esiste quindi un'unica classe di equivalenza che è la classe di equivalenza di un punto.
  % Per questo il gruppo è omomorfo a $ \Z $.
\end{proof}
\hfill\newline\newline \noindent
La mappa che realizza questo isomorfismo è nota come grado.
\begin{definition}
  Si definisce la mappa \textbf{grado} \index{Grado} come l'applicazione che manda una catena in $ S_0(X) $ nella somma
  dei suoi coefficienti:
  \begin{align*}
    \deg \colon S_0(X)    & \to  \Z \\
    \sum n_i p_i & \mapsto  \sum n_i
  \end{align*}
\end{definition}

\begin{proposition}
  La mappa grado gode di alcune proprietà:
  \begin{enumerate}
  \item $ \deg $ è un omomorfismo di gruppi abeliani
  \item $ \deg $ è suriettivo
  \item $ \ker{\deg} \cong B_0(X) $
  \end{enumerate}
  Se dimostro questa proprietà utilizando il primo teorema fondamentale di isomorfismo:
  \[
    \quot{S_0(X)}{B_0(X)} \cong \im{\deg}
  \]
  Ma $ \deg $ è suriettiva, quindi $ \im{\deg} = \Z $, perciò:
  \[
    H_0(X) = \quot{S_0(X)}{B_0(X)} \cong \Z
  \]
\end{proposition}
Dimostro quindi questa proposizione.
\begin{proof}
  \begin{enumerate}
  \item
    Sia $ c_1 = \sum n_i p_i $ e $ c_2 = \sum m_i q_i $, bisogna mostrare che:
    \[
      \deg(c_1 + c_2) = \deg(c_1) + \deg(c_2)
    \]
    ma:
    \[
      c_1 + c_2 = \sum n_i p_i + \sum m_i q_i = \sum (n_i + m_i)r_i
    \]
    dove $ r_i $ è quello comune tra le catene, oppure è zero se
    l'elemento è presente in solo uno delle due catene.
    Quindi:
    \[
      \deg(c_1 + c_2) = \sum (n_i + m_i) = \sum n_i + \sum m_i = \deg(c_1) + \deg(c_2)
    \]
    Alternativamente in modo più semplice si può osservare l'azione di $ \deg $
    sui generatori di $ S_0(X) $, il quale possiede un solo generatore che viene
    mandato dalla mappa grado in $ 1 $, quindi si estende per linearità.
  \item
    La mappa è suriettiva, è sufficiente prendere un punto $ p \in X $
    e la controimmagine di $ m \in \Z $ è $ \deg^{-1}(m) = mp $
  \item
    Mostro che $ \ker{\deg} = B_0(X) $, e lo faccio msotrando che $ \ker{\deg} \subseteq B_0(X) $
    e che  $ \ker{\deg} \supseteq B_0(X) $.

    Inizio con $ \ker{\deg} \subseteq B_0(X) $: sia $ c \in \ker{\deg} $ cioè tale che $ \deg(c) = 0 $,
    se $ c = \sum n_i p_i $ allora $ \sum n_i = 0 $, voglio mostrare che $ c \in B_0(X) $,
    cioè che $ \exists b \in S_1(X) $ con $ \partial_1 b = c $.

    Fissato $ p_0 $ considero i $ p_i $, ci sono archi
    $ \lambda_i $ che li uniscono a $ p_0 $. $ b $ si può costruire in questo modo: siano
    $ \lambda_i : [0,1] \to X $ con $ \lambda_i(0) = p_0 $ e $ \lambda_i(1) = p_i $ allora:
    \begin{gather*}
      c - \partial\left(\sum n_i \lambda_i \right) =  c - \sum n_i \partial \lambda_i = c - \sum n_i (p_i - p_0) = \\
      = c - \sum n_i p_i
      + \sum n_i p_0 = p_0 \sum n_i = 0
    \end{gather*}
    In cui si è usato che per ipotesi $ c \in \ker{\deg} $ quindi $ \sum n_i = 0 $ e che $ c = \sum n_i p_i  $.
    Ma quindi $ c = \partial(\sum n_i \lambda_i) $ e definendo $ \sum n_i \lambda_i = b $ si è trovato l'elemento $ b $,
    per cui $ \ker{\deg} \subseteq B_0(X) $.

    Mi rimane da mostrare che $ B_0(X) \subseteq \ker{\deg} $: mostro che se $ c \in B_0(X) $ allora
    $ c \in \ker{\deg} $, cioè, $ \deg(c) = 0 $.
    Siccome $ c \in B_0(X) $ esiste $ b \in S_1(X) $ tale che $ c = \partial b $, ma $ S_1(X) $
    è lo spazio generato dagli $ 1 $-simplessi singolari, cioè dagli archi, quindi
    chiamando $ \lambda_i $ gli archi si può scrivere $ b = \sum m_i \lambda_i $.
    A questo punto:
    \[
      \deg(c) = \deg(\partial b) = \sum n_i \deg(\partial \lambda_i) = 0
    \]
    In quando $ \partial \lambda_i = \lambda_i(1) - \lambda_i(0) $ e l'azione dell'opertaore grado è quella di sommare i coefficienti,
    che sono opposti.

    Siccome $ \ker{\deg} = B_0(X) $ in particolare gli spazi sono isomorfi.
  \end{enumerate}
  Per questo si può utlizzare il primo teorema dell'isomorfismo, come indicato
  all'inizio di questa dimostrazione.
\end{proof}
% \hfill \newline\newline

\subsection{$ H_1(X) $}
% lezione 3 parte 2

% Cosa si può dire invece su $ H_1(X) $?

Sia $ X $ spazio topologico e $ x_0 \in X $, alla coppia $ (X, x_0) $ si associa
il gruppo fondamentale $ \pi_1(X,x_0) $, il quale in generale non è abeliano. Per
questo motivo conviene studiare la versione abelianizzata:
$ \Ab{\pi_1(X,x_0)} = {\pi_1(X,x_0)} \slash {\pi_1(X,x_0)'} $ dove $ ' $ indica il
  \textbf{gruppo derivato}\index{Gruppo derivato}, cioè il gruppo generato dai
  commutatori.
\[
  \pi_1(X,x_0)' = [\pi_1(X,x_0), \pi_1(X,x_0)] = \langle\set{[g,h] | g,h \in \pi_1(X,x_0)}\rangle
\]
Il gruppo derivato è il gruppo dei prodotti formali di elementi del tipo
$ aba^{-1}b^{-1} $, quando passo al quoziente questi oggetti si annullano e
quindi $ ab = ba $ (infatti
$ aba^{-1}b^{-1} = 1 \Rightarrow aba^{-1} = b \Rightarrow ab = ba $), per questo si ottiene il
gruppo abelianizzato.

Se $ X $ è connesso per archi allora mostrerò che $ \Ab{\pi_1(X,x_0)} \cong H_1(X) $,
quindi conoscendo il gruppo fondamentale si può calcolare anche
il primo gruppo di omologia, che quindi è sostanzialmente formato dai lacci
(modulo omotopia) che commutano tra loro.

% lezione 3 parte 2

\begin{osservation}
  Sia $ X $ uno spazio topologico connesso per archi e $ \mathcal{G} $ un gruppo
  abeliano se esiste un omomorfismo di gruppi $ \phi: \pi_1(X) \to \mathcal{G} $ allora
  esiste $ \phi' : \Ab{\pi_1(X)} \to \mathcal{G} $ omomorfismo di gruppi abeliani.
  \[
    \begin{tikzcd}
      \pi_1(X) \arrow{r}{\phi} \arrow{d}{P} & \mathcal{G} \\
      \Ab{\pi_1(X)} \arrow{ur}{\phi'}
    \end{tikzcd}
  \]
  dove $ P $ è la proiezione sul quoziente.
\end{osservation}
\begin{proof}
  La definizione di $ \phi' $ è naturale, questa è tale che
  $ \phi'(P(c)) = \phi(c) $, ma bisogna controllare se questa è ben definita, cioè se
  prendendo rappresentanti equivalenti si ottengono le stesse immagini, cioè se
  considerati $ c \sim_H d $ risulta che $ \phi(c) = \phi(d) $. Se
  $ c \sim_H d $ allora $ P(c) = P(d) $, e quindi $ c = d[x,y] $ per opportuni
  $ x $ e $ y $, in quanto gli elementi in $ \Ab{\pi_1(X)} $ differiscono per
  commutatori. Applicando $ \phi $ si ottiene $ \phi(c) = \phi(d[x,y]) $, siccome
  $ \phi $ è omomorfismo:
  \[
    \phi(d[x,y]) = \phi(d)\phi([x,y]) = \phi(d) \phi(xyx^{-1}y^{-1}) = \phi(d) \phi(x) \phi(y) \phi(x)^{-1} \phi(y)^{-1} = \phi(d)
  \]
  dove nell'ultimo passaggio ho utilizzato che il gruppo è abeliano.
  Si nota che questa osservazione dipende crucialmente dal fatto che il gruppo è abeliano.
\end{proof}
\hfill\newline\newline
Per dimostrare che $ \Ab{\pi_1(X)} \cong H_1(X) $ mi serve prima un lemma:
\begin{lemma}
  Se $ f \sim_H g $ allora $ f \sim_{hom} g $, cioè se $ f $ e $ g $ sono lacci che
  definiscono lo stesso elemento nel gruppo fondamentale allora differiscono per
  un bordo ($ f \sim_H g \Rightarrow f \sim_{hom} g $).
\end{lemma}
\begin{proof}
  Siccome $ f \sim_{H} g $ allora $ \exists F $ continua tale $ F: I \times I \to X $ tale che $ F(0,x) = f(x) $,
  $ F(1,x) = g(x) $ e $ F(t,0) = F(t, 1) = x_0 $.
  \begin{figure}[htbp]
    \centering
    \begin{tikzpicture}
      \draw (0, 0) rectangle (3,3);
      \draw[-Latex] (-0.5, 1) -- (-0.5, 2);
      \draw[-Latex] (1, -0.5) -- (2, -0.5);
      \node[left] () at (-0.5, 1.5) {$ t $};
      \node[below] () at (1.5, -0.5) {$ x $};
      \node[right] () at (0, 1.5) {$ C_{x_0} $};
      \node[left] () at (3, 1.5) {$ C_{x_0} $};
      \node[above] () at (1.5, 0) {$ f(x) $};
      \node[below] () at (1.5, 3) {$ g(x) $};
      \draw (0,2) -- (3,2);
      \draw[-Latex] (0, 2) -- (1.5, 2);
      \node[below] () at (1.5, 2) {$ F(t,x) $};
    \end{tikzpicture}
    \caption{Omotopia: deforma $ f $ in $ g $ in modo continuo.}
    \label{fig:lez3:homotopy_f_g}
  \end{figure}

  Voglio mostrare che $ f - g $ è bordo di un $ 2 $-simplesso.
  Identificando tutti i punti di un uno dei due intervalli con l'equivalenza $ {I \times I} \slash {\set{0} \times I} $
  si ottiene qualcosa che è omeomorfo a $ \Delta_2 $, $ F $ sullo spigolo $ \set{0} \times I $
  assume sempre lo stesso valore.
  \begin{figure}[htbp]
    \centering
    \begin{tikzpicture}
      \draw (0, 0) rectangle (3,3);
      \draw[ultra thick] (0,0) -- (0,3);
      % \node[right] () at (0, 1.5) {$ C_{x_0} $};
      % \node[left] () at (3, 1.5) {$ C_{x_0} $};
      \node[above] () at (1.5, 0) {$ f(x) $};
      \node[below] () at (1.5, 3) {$ g(x) $};
      \draw[-Latex] (4,1.5) -- (5,1.5);
      \draw (6,0) -- (7.7, 3) -- (9.4,0) -- cycle;
      \node[above, rotate = 60] () at (6.7, 1.3) {$ g(x) $};
      \node[above, rotate = -58] () at (8.7, 1.3) {$ K(x) = C_{x_0} $};
      \node[below] () at (7.7, 0) {$ f(x) $};
      \node[left] () at (6,0) {$ e_0 $};
      \node[] () at (6,0) {$ \bullet $};
      \node[right] () at (9.4,0) {$ e_1 $};
      \node[above] () at (7.7,3) {$ e_2 $};
    \end{tikzpicture}
    \caption{La relazione di equivalenza fa passare da un quadrato a un triangolo
      in quanto fa collassare un intervallo nel punto $ e_0 $}
    \label{fig:lez3:homotopy_f_g_to_triangle}
  \end{figure}

  Siccome $ F $ rimane costante sul sottospazio su cui su quozienta,
  dove vale sempre $ x_0 $,
  $ F $ induce $ F' \colon \Delta_2 \to X $ continua in cui $ e_0 $ viene mandato in $ x_0 $:
  \[
    \begin{tikzcd}
      I \times I \arrow{r}{F} \arrow{d}{P} & X \\
      \quot{I \times I}{0 \times I} \simeq \Delta_2 \arrow{ur}{F'}
    \end{tikzcd}
  \]
  Calcolo il bordo: $ \partial F' = F'^{(0)} - F'^{(1)} + F'^{(2)} = K - g + f $
  dove $ K $ è il cammino costante per definizione di omotopia, cioè è $ C_{x_0}$. Se $ K $
  fosse il bordo di qualcosa avrei finito ($ \partial w = f - g $). Ma $ K $ è il $ 2 $-simplesso
  singolare costante uguale a $ x_0 $, cioe $ K \colon \Delta_2 \to \set{x_0} $, quindi il suo bordo:
  \[
    \partial K = K^{(0)} - K^{(1)} + K^{(2)} =  K^{(2)}
  \]
  in quanto tutti i tre termini sono uguali a $ k \colon \Delta_1 \to \set{x_0} $, quindi $ \partial K = K^{(2)} = k $,
  cioè $ k $ è un bordo, perciò:
  \[
    \partial F' = \partial k - F'^{(1)} + F'^{(2)} \Rightarrow \partial F' - \partial k = f - g \Rightarrow \partial(F' - k) = f - g
  \]
  $ F' - k $ è $ 2 $-simplesso singolare, lo chiamo $ \sigma $ ed è tale che $ \partial \sigma = f - g $, quindi
  $ f $ e $ g $ sono omologhi e $ \sigma $ è il $ 2 $-simplesso singolare che realizza
  l'omologia.
\end{proof}

\begin{proposition}
  Se $ X $ è uno spazio topologico connesso per archi allora esiste un
  omomorfismo $ \phi \colon \Ab{\pi_1(X)} \to H_1(X) $, cioè si può passare dall'equivalenza
  omologica a quella omotopica.
\end{proposition}

\begin{proof}
  Per dimostrare che $ \Ab{\pi_1(X)} \cong H_1(X) $ trovo un omomorfismo di
  gruppi abeliani tra $ \pi_1(X) $ a $ H_1(X) $, infatti
  se costruisco $ \phi \colon \pi_1(X) \to H_1(X) $ omomorfismo di gruppi ottengo
  gratuitamente la mappa da $ \Ab{\pi_1(X)} $ a $ H_1(X) $ per l'osservazione precedente.
  \[
    \begin{tikzcd}
      \pi_1(X) \arrow{r}{\phi} \arrow{d}{P} & H_1(X) \\
      \Ab{\pi_1(X)} \arrow{ur}{\phi'}
    \end{tikzcd}
  \]
  Poi dovrò mostrare che questa mappa è invertibile, cioè $ \exists \psi:H_1(X) \to A_1(X) $ tale che $ \phi' \circ \psi = \Id{H_1(X)} $ e
  $ \psi \circ \phi' = \Id{\Ab{\pi_1(X)}} $.

  Per il lemma appena dimostrato una possibile costruzione di $ \phi $ è:
  \begin{align*}
    \phi:  \pi_1(X) & \to H_1(X) \\
    [f]_H & \mapsto [f]_{hom} = \llbracket f \rrbracket
  \end{align*}
  In tutto ciò non ho ancora utilizzato la connessione per archi.

  % Ora voglio costruire $ \phi' \colon \Ab{\pi_1(X)} \to H_1(X) $ e lo faccio ancora
  % senza l'ipotesi di connessione per archi.
  Mostro che $ \phi $ è omomorfismo, in questo modo anche $ \phi' $ lo è.

  Siano $ [f]_H, [g]_H \in \pi_1(X) $ voglio fare vedere che:
  \[
    \phi ( [f]_H [g]_H) = \phi([f]_H) + \phi([g]_H)
  \]
  Questo è verso se e solo se:
  \[
    \phi([f \star g]_H) = [f]_{hom} + [g]_{hom}
  \]
  Che è vera se e solo se:
  \[
    [f \star g]_{hom} = [f + g]_{hom}
  \]
  Questo è vero se e solo se i due rappresentati sono equivalenti, cioè se
  differiscono per un bordo, ovvero se:
  \[
    \exists T: \Delta_2 \to X \text{ $ 2 $-simplesso singolare tale che } \partial T = f + g - f \star g
  \]
  Cioè:
  \[
    \partial T = T^{(0)} - T^{(1)} + T^{(2)} = f + g - f \star g
  \]
  \begin{figure}[htbp]
    \centering
    \begin{subfigure}[htbp]{.45\linewidth}
      \begin{tikzpicture}
        \draw (5,0) -- (9,0) -- (7,3) -- cycle;
        \node[left] () at (5,0) {$ e_0 $};
        \node[right] () at (9,0) {$ e_1 $};
        \node[above] () at (7,3) {$ e_2 $};
        \node[below] () at (7,0) {$ g $};
        \node[below] () at (7,-0.5) {$ T^{(2)} $};
        \node[left] () at (5.95,1.7) {$ f \star g $};
        \node[left] () at (5.05,1.7) {$ T^{(1)} $};
        \node[left] () at (6.2,1.5) {};
        \node[right] () at (8,1.7) {$ f $};
        \node[right] () at (8.5, 1.7) {$ T^{(0)} $};
      \end{tikzpicture}
      \caption{Costruzione dell'omomorfismo}
      \label{fig:lez3:proof_homo_1}
    \end{subfigure}
    \begin{subfigure}[htbp]{.45\linewidth}
      \centering
      \begin{tikzpicture}
        \draw (5,0) -- (9,0) -- (7,3) -- cycle;
        \node[left] () at (5,0) {$ e_0 $};
        \node[right] () at (9,0) {$ e_1 $};
        \node[above] () at (7,3) {$ e_2 $};
        \node[below] () at (7,0) {$ g $};
        \node[left] () at (5.9,1.5) {$ \frac{1}{2} $};
        \node[left] () at (6.2,1.5) {\textbullet};
        \node[] () at (7,-0.025) {\textbullet};
        \node[right] () at (8,2) {$ f $};
        \draw (6.025, 1.525) -- (7,0);
      \end{tikzpicture}
      \caption{Costruzione dell'omomorfismo, deve avere valori costanti su rette parallele}
      \label{fig:lez3:proof_homo}
    \end{subfigure}
    \caption{Costruzione dell'omomorfismo}
  \end{figure}
  Una possibile costruzione parte tracciando la retta che congiunge
  due punti medi di due segmenti, quindi si richiede che $ T $ abbia
  valori costanti sulle rette parallele.
\end{proof}
Al momento la situazione è che ho $ \phi: \pi_1(X,x_0) \to H_1(X) $ omomorfismo di
gruppi ben definito anche con $ X $ non necessariamente connesso per archi, e
dato che $ H_1(X) $ è abeliano ho $ \phi': \Ab{\pi_1(X)} \to H_1(X) $ omomorfismo di
gruppi abeliani. L'omomorfismo costruito è in realtà un isomorfismo, come
afferma il \textbf{teorema di Hurewicz}.
\begin{theorem}[Teorema di Hurewicz\index{Teorema di Hurewicz}]
  Se $ X $ è uno spazio topologico
  connesso per archi allora  $ \phi \colon \Ab{\pi_1(X)} \to H_1(X) $
  è un isomorfismo, quindi $ \Ab{\pi_1(X)} \cong H_1(X) $.
\end{theorem}
\begin{proof}
  \emph{Sketch of proof, la dimostrazione completa è piuttosto noiosa}.
  Per dimostrare che $ \phi' $ è isomorfismo o dimostro che è iniettiva e suriettiva
  o che ammette un inverso. Procedo con la seconda possibilità: mostro che
  $ \exists \psi \colon H_1(X) \to \Ab{\pi_1(X)} $ tale che $ \psi $ è inverso di $ \phi' $.

  Considero un arco $ f \colon \Delta_1 \to X $ con $ f(0), f(1) \in X $.
  \begin{figure}[htbp]
    \centering
    \begin{tikzpicture}
      \draw (0,-0.75) rectangle (5.25,3);
      \node[right] () at (5.5,1.5) {$ X $};
      \node[above] () at (1,1) {$ x_0 $};
      \node[] () at (1,1) {\textbullet};
      \node[above] () at (2,2) {$ f(0) $};
      \node[] () at (2,2) {\textbullet};
      \node[above, right] () at (4,1) {$ f(1) $};
      \node[] () at (4,1) {\textbullet};
      \draw[-Latex] (1,1) to [out=-30,in=-50] (2,2);
      \node[right] () at (2,1.6) {$ \lambda_{f(0)} $};
      \draw[Latex-] (1,1) to [out=-60,in=-90] (4,1);
      \node[right] () at (3.9,0.5) {$ \bar{\lambda}_{f(1)} $};
      \draw[-Latex] (2,2) to [out=-30,in=90] (4,1);
    \end{tikzpicture}
    \caption{Dimostrazione della proposizione}
    \label{fig:lez3:sketch_of_proof}
  \end{figure}
  Siccome lo spazio è connesso per archi esiste un cammino da $ x_0 $ a $ f(0) $, cioè
  una funzione $ \lambda_{f(0)} \colon I \to X $ tale che $ \lambda_{f(0)} = x_0 $ e $ \lambda_{f(1)} = f(0) $.
  Lo stesso vale per $ x_0 $ e $ f(1) $. Questi archi sono orientati partendo da $ x_0 $, posso
  considerare il cammino con verso opposto $ \bar{\lambda}_{f(1)} $ e quindi costruire il laccio
  di base $ x_0 $: $ \lambda_{f(0)} \star f \star \bar{\lambda}_{f(1)} =: \tilde{f} $. Vale che $
  \psi(f) = \llbracket \tilde{f} \rrbracket $,
  dove $  \llbracket \tilde{f} \rrbracket = P \left([\tilde{f}]_H\right)$.
  Bisogna mostrare che:
  \begin{enumerate}
  \item $ \psi $ è ben definito, cioè se $ f \sim_{hom} g $ allora $ \psi(f) = \psi(g) $ e che $ \psi $
    non dipende dalla scelta del cammino.
  \item $ \psi $ è omomorfismo di gruppi
  \item $ \phi' \circ \psi = \Id{H_1(X)} $
  \item $ \psi \circ \phi' = \Id{\Ab{\pi_1(X)}} $
  \end{enumerate}
  \emph{Lo studente interessato può verificare queste asserzioni.}
  \begin{exercise}
    Verificarli.
  \end{exercise}
  Una volta verificati si trova in particolare che $ H_1(X) \cong \Ab{\pi_1(X)} $.
\end{proof}

\begin{example} \hfill
  \begin{itemize}
  \item $ H_1(V_g) \cong \Z^{2g} $ con $ g \geq 0 $, infatti si impone già la condizione di abelianizzazione nella costruzione di $ V_g $
  \item $ H_1(\bigvee_{i=1}^{k}\Sph{1}) \cong \Z^k $ con $ \bigvee_{i=1}^{k}\Sph{1} $ bouquet, cioè $ k $ circonferenze incollate in un punto,
    infatti c'è un termine $ \Z $ per ogni circonferenza.
  \item $ H_1(\RN{3} \setminus \Sph{1}) \cong \Z $ (è un toro tappato)
  \item $ H_1(U_1) \cong \Z_2 $ dove $ U_1 $ è il piano proiettivo reale $ \mathbb{P}^2(\RN{}) = {\RN{3} \setminus \set{0}} \slash {\sim} $
    con $ \vec{x} \sim \vec{y} $ se $ \vec{x} = a \vec{y} $ con $ a \in \RN{} $
  \item $ H_1(U_2) \cong \Z \oplus \Z_2 $ dove $ U_2 $ è la bottiglia di Klein.
    Infatti $ \pi_1(U_2) = \set{a, b | aba^{-1}b^{-1} = 1} $ per abeliannizzarlo bisogna
    porre $ aba^{-1}b = 1 $ e $ aba^{-1}b^{-1} = 1 $ cioè $ b^2 = 1 $ e $ a $ libero:
    $ \Ab{\pi_1(U_2)} = \set{\underset{\Z}{a}, \underset{\Z_2}{b} |  aba^{-1}b = 1 } $
    \begin{figure}[htbp]
      \centering
      \begin{subfigure}{.5\textwidth}
        \centering
        \def\svgwidth{0.26\textwidth}
        \input{images/Klein_bottle.pdf_tex}
        \caption{Bottiglia di Klein}
      \end{subfigure}%
      \begin{subfigure}{.5\textwidth}
        \centering
        \begin{tikzpicture}
          \draw (0,0) rectangle (3,3);
          \draw[-Latex] (0,0) -- (0,1.5);
          \draw[-Latex] (0,3) -- (1.5,3);
          \draw[-Latex] (3,0) -- (3,1.5);
          \draw[-Latex] (3,0) -- (1.5,0);
          \node[left] () at (0,1.5) {$ a $};
          \node[above] () at (1.5,3) {$ b $};
          \node[right] () at (3,1.5) {$ a $};
          \node[below] () at (1.5,0) {$ b $};
        \end{tikzpicture}
        \caption{Bottiglia di Klein, si nota che rispetto al toro di Clifford c'è
          una torsione nella $ a $ di destra}
      \end{subfigure}
      \caption{Bottiglia di Klein}
      \label{fig:lez3:klein_bottle}
    \end{figure}.
  \end{itemize}
\end{example}

\newmathsymb{bouquet}{\vee}{Bouquet}
\begin{definition}
  Siano $ (X,x_0) $ e $ (Y,y_0) $ due spazi topologici puntati, si definisce il \textbf{bouquet}\index{Bouquet}
  $ X \vee Y $ come lo spazio topologico definito da:
  \[
    X \vee Y = \quot{X \invamalg Y}{\sim}
  \]
  in cui $ \sim $ identifica $ x_0 $ con $ y_0 $. In pratica si incollano $ X $ e $ Y $ per lo stesso punto.
\end{definition}

% lezione 4
% _     _____ ________ ___  _   _ _____   _  _
% | |   | ____|__  /_ _/ _ \| \ | | ____| | || |
% | |   |  _|   / / | | | | |  \| |  _|   | || |_
% | |___| |___ / /_ | | |_| | |\  | |___  |__   _|
% |_____|_____/____|___\___/|_| \_|_____|    |_|

\section{Morfismi indotti}

% Se $ X $ è uno spazio topologico connesso per archi allora esiste
% l'isomorfismo:
% \[
%   \phi \colon \quot{\pi_1(X)}{[ \pi_1(X), \pi_1(X)]} \to H_1(X)
% \]
% Il problema è costruire
% \[
%   \psi \colon H_1(X) \to \quot{\pi_1(X)}{[ \pi_1(X), \pi_1(X)]}
% \]
% Tale che: $ \phi \circ \psi = \Id{H_1(X)} $ e $ \psi \circ \phi = \Id{\pi_1(X)} $.
% So calcolare $ H_0(X) $ e $ H_1(X) $ se voglio calcolare
% gli altri $ H_k(X) $? Prima guardo come si comportano i gruppi sotto
% l'azione di applicazioni continue:
Sia $ g \colon X \to Y $ mappa continua tra spazi topologici,
allora
$ g $ induce un'applicazione tra $ H_k(X) $ e $ H_k(Y) $.
Infatti, considero $ \sigma \colon \Delta_k \to X $ $ k $-simplesso singolare, posso
considerare la composizione con $ g $ definendno $ g' \colon \Delta_k \to Y $ con $ g' = g \circ \sigma $:
\[
  \begin{tikzcd}
    g' \colon \Delta_k \arrow{r}{\sigma} &  X  \arrow{r}{g} & Y
  \end{tikzcd}
\]
Siccome sia $ g $ che $ \sigma $ sono continue allora $ g' $ è continua, quindi è un
$ k $-simplesso singolare in $ Y $.
\newmathsymb{fsharp}{f_\sharp}{Applicazione indotta da $ f $ sulle catene}
Si definisce $ g_\sharp $ come l'estensione di $ g' $ su tutte le $ k $-catene
per linearità:
\begin{align*}
  g_\sharp \colon S_k(X) & \to S_k(Y) \\
  \sum_\sigma n_\sigma \sigma & \mapsto  \sum_\sigma n_\sigma g' =  \sum_\sigma n_\sigma ( g \circ \sigma )
\end{align*}
Questa mappa è ben definita ed è lineare quindi $ g_\sharp $ è un omomorfismo di
gruppi abeliani che manda $ k $-catene in $ S_k(X) $ in $ k $-catene
in $ S_k(Y) $.
Ora voglio ottenere un'applicazione a livello di omologia singolare,
quindi definisco $ g_\star $.
\newmathsymb{fstar}{f_\star}{Applicazione indotta da $ f $ sui gruppi di omologia}
\begin{align*}
  g_\star \colon H_k(X) & \to H_k(Y) \\
  [c]          & \mapsto [g_\sharp (c)]
\end{align*}
Si dice che $ g $ è \textbf{covariante} perché va da $ X $ a $ Y $,
cioè rispetta il verso della applicazione $ g $.
Devo verificare se questa applicazione è ben definita, cioè non se
dipende dalla scelta del rappresentate della classe.
Considero $ d \in S_k(X) $ tale che $ \partial d = 0 $, suppongo
che $ d \sim_{hom} c $, questo vale se e solo se $ [d] = [c] $ con $ \partial c = 0 $,
mi chiedo è vero che $ g_\star([d]) = g_\star([c]) $?
Devo cioè mostrare che $ g_\sharp (d) \sim_{hom} g_\sharp (c) $, ma questo è vero
se e solo se $ \exists \tau \in S_{k+1}(Y) $ tale che $ g_\sharp(d) - g_\sharp (c) = \partial \tau $.
Siccome $ g_\sharp $ è omomorfismo allora deve essere $ g_\sharp (d - c) = \partial \tau $,
ma $ d $ e $ c $ sono omologhi per ipotesi, quindi:
\[
  \exists u \in S_{k+1}(X) \; | \; \partial u = d - c
\]
Quindi $ g_\sharp(\partial u) = g_\sharp (d - c) $, e questo implica che $ [g_\sharp (d)] = [g_\sharp(c)] $, infatti
% vorrei che questo sia implicato $ g_\sharp (\partial u) = g_\sharp (d - c) $.
trovo $ \tau $ a partire da $ u $:
\begin{align*}
  g_\sharp (\partial u) & = g_\sharp \left(\sum_{i = 0}^{k+1} (-)^i u^{(i)}\right) = \sum_{i=0}^{k+1}(-)^i g_\sharp (u^{(i)}) =
                          \sum_{i=0}^{k+1}(-)^i g \circ u^{(i)} = \\
                        & = \sum_{i = 0}^{k+1} (-)^i g \circ \left( u \circ F_i^{\; k+1} \right) =
                          \sum_{i = 0}^{k+1} (-)^i  \left( g \circ u \right) \circ F_i^{\; k+1} = \\
                        & = \sum_{i = 0}^{k+1}(-)^i \left( g \circ u \right)^{(i)} = \partial \left( g \circ u \right)
\end{align*}
Ma quindi $ g_\sharp(\partial u) = \partial (g_\sharp (u)) $ cioè:
\[
  g_\sharp(d-c) = g_\sharp (\partial u) = \partial (g_\sharp (u)) = \partial \tau \quad \text{con } \tau = g_\sharp(u)
\]
% In conclusione:
% \begin{align*}
    %     g_\star \colon H_k(X) & \to H_k(Y) \\
    %     [c]_X & \to [g_\sharp (c)]_Y
                  %   \end{align*}
Quindi $ g_\star $ è ben definita ed è omomorfismo in quanto
è il passaggio a quoziente di omomorfismi.
Noto in particolare che ho mostrato che $ g_\sharp \circ \partial = \partial \circ g_\sharp $
in quanto l'ho mostrato sui generatori.

\begin{example}
  Sia $ j \colon \Sph{1} \to \Sph{2} $ l'immersione di un equatore in una sfera allora
  $ j_\star \colon H_1(\Sph{1}) \to H_1(\Sph{2}) $  è una mappa costante in quanto $ \Sph{2} $
  ha gruppo fondamentale banale quindi $ H_1(\Sph{2}) $ è banale.
  Si nota che $ j $ era iniettiva,  ma $ j_\star $ è costante quindi non è più iniettiva.
\end{example}
\begin{example}
  Se considero $ \Sph{1} = \set{z \in \mathbb{C} | |z| = 1 } $
  \begin{align*}
    f \colon \Sph{1} & \to \Sph{1} \\
    z & \to z^4
  \end{align*}
  Come è fatta $ f_\star \colon H_1(\Sph{1}) \to H_1(\Sph{1})$ ?
  Si sa che $ H_1(\Sph{1}) \cong \Z $ in quanto il gruppo fondamentale di $ \Sph{1} $
  è $ \Z $ che è già abeliano. C'è quindi un solo generatore, che posso prendere
  il simplesso singolare:
  \begin{align*}
    \sigma \colon \Delta_1 & \to \Sph{1} \\
    t & \to \me^{2 \pi i t}
  \end{align*}
  Cioè in pratica $ [\sigma] \to 1 $, il laccio si avvolge su sè stesso una volta.
  \begin{align*}
    f_\star \colon H_1(\Sph{1}) & \to H_1(\Sph{1}) \\
    [\sigma] & \mapsto [f_\sharp (\sigma)] = [f \circ \sigma]
  \end{align*}
  Si ha:
  \[
    \begin{tikzcd}
      \Delta_1 \arrow{r}{\sigma} & \Sph{1} \arrow{r}{f} & \Sph{1}
    \end{tikzcd}
  \]
  Con:
  \[
    \begin{tikzcd}
      t \arrow{r}{\sigma} & \me^{2 \pi i t} \arrow{r}{f} & \me^{8 \pi i t}
    \end{tikzcd}
  \]
  Quindi:
  \begin{align*}
    f \circ \sigma \colon \Delta_1 & \to \Sph{1} \\
    t & \mapsto \me^{8 \pi i t}
  \end{align*}
  Sostanzialmente $ f \circ \sigma $ è un cammino in $ \Sph{1} $ ed è
  quindi potenza di $ \sigma $, che è l'unico generatore:
  \[
    f \circ \sigma = \sigma^4 = \sigma \star \sigma \star \sigma \star \sigma
  \]
  Cioè avvolgo il laccio quattro volte, quindi:
  \begin{align*}
    f_\star \colon H_1(\Sph{1}) & \to H_1(\Sph{1}) \\
    [\sigma] & \mapsto [\sigma^4]
  \end{align*}
  Cioè:
  \begin{align*}
    f_\star \colon \Z & \to \Z \\
    1 & \mapsto 4
  \end{align*}
  $ f_\star $ è iniettivo ma non suriettivo (non tutti gli interi sono
  multipli di 4)
\end{example}

        %         Siano $ X, Y $ spazi topologici a partire da $ f \colon X \to Y $
        %         ho $ f_\star \colon H_k(X) \to H_k(Y) $ $ \forall k $

\begin{osservation}
  Siano $ X $ spazio topologico:
  $ \Id{X} \colon X \to X $ allora:
  \begin{align*}
    \left(\Id{X}\right)_\star \colon H_k(X) & \to H_k(X) \\
    [c] & \mapsto [\left(\Id{X}\right)_\sharp (c)] = [c]
  \end{align*}
  Quindi $ \left(\Id{X}\right)_\star $ è proprio l'identità
  a livello di gruppi di omologia, cioè:
  \[
    \left(\Id{X}\right)_\star = \Id{H_k(X)}
  \]
\end{osservation}

\begin{osservation}
  Siano $ X, Y, Z $ spazi topologici e $ f \colon X \to Y $,
  $ g \colon Y \to Z $ funzioni continue, allora $ g \circ f \colon X \to Z $
  è continua, si ha quindi:
  \[
    \begin{tikzcd}
      X \arrow{r}{f} &  Y  \arrow{r}{g} & Z
    \end{tikzcd}
  \]
  E:
  \[
    \begin{tikzcd}
      H_k(X) \arrow{r}{f_\star} &  H_k(Y) \arrow{r}{g_\star} & H_k(Z)
    \end{tikzcd}
  \]
  Sono ben definite $ g_\star \circ f_\star \colon H_k(X) \to H_k(Z) $ e
  $ \left(g \circ f\right)_\star \colon H_k(X) \to H_k(Z) $, vale che
  $ g_\star \circ f_\star =  \left(g \circ f\right)_\star $, infatti se $ \sigma $ è
  simplesso singolare (poi basta estendere per linearlità):
  \begin{gather*}
    \left(g \circ f\right)_\star ([\sigma]) = [ (g \circ f)_\sharp (\sigma) ] = [ (g \circ f) \circ \sigma ] =  [ g \circ (f \circ \sigma) ] = \\
    = [ g_\sharp (f \circ \sigma)] = [ g_\sharp \circ f_\sharp (\sigma) ] = (g_\star \circ f_\star) ([\sigma])
  \end{gather*}
\end{osservation}
Quindi sulla categoria degli spazi topologici questo
fornisce un funtore covariante, in quanto questa associazione
si comporta bene rispetto all'identità e alla composizione.

\section{Successioni esatte}

Considero due complessi $ (C_\bullet, \partial) $ e $ (C'_\bullet, \partial') $,
considero l'omomorfismo di $ \Z $-moduli $ F \colon (C_\bullet, \partial) \to (C'_\bullet, \partial') $
tale che $ \forall k $ si formi un diagramma commutativo,
cioè valga $ F \circ \partial = \partial' \circ F $
\[
  \begin{tikzcd}
    \dots \arrow{r}{\partial} &  C_{k+1}  \arrow{r}{\partial} \arrow{d}{F} &  C_{k}  \arrow{r}{\partial} \arrow{d}{F} & C_{k-1}  \arrow{r}{\partial} \arrow{d}{F} & \dots \\
    \dots \arrow{r}{\partial'} &  C'_{k+1}  \arrow{r}{\partial'} &  C'_{k}  \arrow{r}{\partial'}  &  C'_{k-1} \arrow{r}{\partial'} & \dots
  \end{tikzcd}
\]
Tutti i quadrati che si formano devono essere
commutativi. Si pone questa richiesta di commutatività
in quanto considerando $ f \colon X \to Y $ e quindi
$ F = f_\sharp \colon (S_\bullet(X), \partial) \to  (S_\bullet(Y), \partial') $ la condizione
di commutatività è $ f_\sharp \circ \partial = \partial' \circ f_\sharp $ che è
proprio quella che ho utilizzato prima per mostrare
che l'applicazione è ben definita a livello
di omologia (avevo usato $ g_\sharp \circ \partial = \partial \circ g_\sharp $).
Una funzione $ F $ fatta in questo modo è detta
\textbf{mappa tra complessi}\index{Mappa tra complessi}.

\begin{definition}
  Si definisce una \textbf{successione esatta corta}\index{Successione esatta corta} di
  gruppi la successione:
  \[
    \begin{tikzcd}
      A \arrow{r}{\alpha} & B \arrow{r}{\beta} & C
    \end{tikzcd}
  \]
  con $ \alpha $ omomorfismo iniettivo, $ \beta $ omomorfismo suriettivo e $ \ker{\beta} = \im{\alpha} $.
  Si nota che richiedere queste condizioni su $ \alpha $ e $ \beta $ è equivalente a scrivere la
  successione esatta come:
  \[
    \begin{tikzcd}
      0 \arrow{r}{} & A \arrow{r}{\alpha} & B\arrow{r}{\beta} & C \arrow{r}{} & 0
    \end{tikzcd}
  \]
  Infatti indicando le mappe sottointese con $ i \colon 0 \to A $ e $ j \colon C \to 0 $
  allora per l'esattezza vale che $ \ker{\alpha} = \im{i} = 0 $ in quanto $ i $ è
  omomorfismo, ma $ \ker{\alpha} = 0 $ signfiica che $ \alpha $ è iniettiva, inoltre
  $ \ker{j} = \im{\beta} = C $, quindi $ \beta $ è suriettiva. Quindi automaticamente
  $ C \cong \quot{B}{A} $ infatti per il teorema fondamentale degli omomorfismi
  $ \quot{B}{\ker{\beta}} \cong \im{\beta} \overset{\text{suriettività}}{=} C $, ma per
  l'esattezza $ \ker{\beta} = \im{\alpha} $ quindi $ \ker{\beta} = \alpha(A) $ ed essendo $ \alpha $
  iniettiva $ \alpha(A) \cong A $.
\end{definition}

\begin{definition}
  Si definisce una \textbf{successione esatta corta}\index{Successione esatta corta} di
  complessi la successione:
  \[
    \begin{tikzcd}
      0 \arrow{r}{} & A_\bullet \arrow{r}{\alpha} & B_\bullet \arrow{r}{\beta} & C_\bullet \arrow{r}{} & 0
    \end{tikzcd}
  \]
  con $ (A_\bullet, \partial^A) $, $ (B_\bullet, \partial^B) $ e $ (C_\bullet, \partial^C) $ complessi, e
  $ \alpha $ mappa tra complessi iniettiva, $ \beta $ mappa tra complessi suriettiva
  e deve valere che $ \forall k $ sia $ C_k \cong \quot{B_k}{A_k} $.
\end{definition}

                                                                                                         %                                                                                                          lezione 4 parte 2

In modo più esteso questo significa:
\[
  \begin{tikzcd}
    {} & 0 \arrow{d}{} & 0 \arrow{d}{} & 0 \arrow{d}{} & {} \\
    \dots \arrow{r}{} & A_{k+1} \arrow{r}{} \arrow{d}{} & A_{k} \arrow{r}{} \arrow{d}{} & A_{k-1} \arrow{r}{} \arrow{d}{} & \dots \\
    \dots \arrow{r}{} & B_{k+1} \arrow{r}{} \arrow{d}{} & B_{k} \arrow{r}{} \arrow{d}{} & B_{k-1} \arrow{r}{} \arrow{d}{} & \dots \\
    \dots \arrow{r}{} & C_{k+1} \arrow{r}{} \arrow{d}{} & C_{k} \arrow{r}{} \arrow{d}{} & C_{k-1} \arrow{r}{} \arrow{d}{} & \dots \\
    {} & 0 & 0 & 0 & {}
  \end{tikzcd}
\]
Le colonne sono successioni esatte corte di $ Z $-moduli, quindi
l'immagine di $ \alpha $ è uguale al nucleo e la mappa è iniettiva
perciò la prima riga è formata da zero (infatti se è
iniettiva il nucleo è zero), similmente siccome
la mappa $ \beta $ è suriettiva quindi l'ultima
riga è formata da zero.
Inoltre tutti i quadrati sono commutativi.
\subsection{Omomorfismo di connessione}
A partire da una successione esatta corta
posso passare all'omologia, se passo brutalmente
all'omologia non ottengo una successione esatta,
ma c'è il modo per indurre una successione esatta lunga:
\begin{theorem}
  Una successione esatta corta di complessi induce una successione
  esatta lunga tale che sia fatta così:
  \[
    \begin{tikzcd}
      \dots \arrow{r}{} & H_p(A_\bullet) \arrow{r}{\alpha_\star} & H_p(B_\bullet) \arrow{r}{\beta_\star} & H_p(C_\bullet) \arrow{r}{\delta}
      & H_{p-1}(A_\bullet) \arrow{r}{\alpha_\star} & \dots
    \end{tikzcd}
  \]
  Esatta signfiica che $ \forall p $:
  \begin{gather*}
    \im{\alpha_\star} = \ker{\beta_\star} \\
    \im{\beta_\star} = \ker{\delta} \\
    \im{\delta} = \ker{\alpha_\star}
  \end{gather*}
  $ \delta $ è detto \textbf{omomorfismo di connessione}\index{Omomorfismo di connessione}
  in quanto cambia il grado dell'omologia.

  La scrittura estesa della successione è:
  \[
    \begin{tikzcd}
      {} & \dots  \arrow{d}{} &  \dots  \arrow{d}{}  &  \dots  \arrow{d}{}  & {} \\
      \dots \arrow{r}{} & H_{p+1}(C_{k+1}) \arrow{r}{} \arrow{d}{} &  H_{p+1}(C_{k}) \arrow{r}{} \arrow{d}{} &  H_{p+1}(C_{k-1}) \arrow{r}{} \arrow{d}{} & \dots \\
      \dots \arrow{r}{} & H_p(A_{k+1}) \arrow{r}{} \arrow{d}{} & H_p(A_{k})  \arrow{r}{} \arrow{d}{} & H_p(A_{k-1})  \arrow{r}{} \arrow{d}{} & \dots \\
      \dots \arrow{r}{} & H_p(B_{k+1}) \arrow{r}{} \arrow{d}{} & H_p(B_{k})  \arrow{r}{} \arrow{d}{} & H_p(B_{k-1})  \arrow{r}{} \arrow{d}{} & \dots \\
      \dots \arrow{r}{} & H_p(C_{k+1}) \arrow{r}{} \arrow{d}{} & H_p(C_{k})  \arrow{r}{} \arrow{d}{} & H_p(C_{k-1})  \arrow{r}{} \arrow{d}{} & \dots \\
      \dots \arrow{r}{} & H_{p-1}(A_{k+1}) \arrow{r}{} \arrow{d}{}  &  H_{p-1}(A_{k}) \arrow{r}{} \arrow{d}{} &  H_{p-1}(A_{k-1}) \arrow{r}{} \arrow{d}{} & \dots \\
      {} & \dots &  \dots &  \dots & {}
    \end{tikzcd}
  \]
\end{theorem}
\begin{proof}
  Per dimostrare il teorema bisogna:
  \begin{enumerate}
  \item Dimostrare che $ \alpha_\star $ e $ \beta_\star $ sono ben definite
  \item Costruire l'omomorfismo di connessione e verificare che è effettivamente un omomorfismo
  \item Mostare che la successione è esatta, cioè che
    \begin{gather*}
      \im{\alpha_\star} = \ker{\beta_\star} \\
      \im{\beta_\star} = \ker{\delta} \\
      \im{\delta} = \ker{\alpha_\star}
    \end{gather*}
  \end{enumerate}
  \emph{Sketch of proof, la dimostrazione è lunga e noiosa.}

  Per costruire l'omomorfismo di connessione devo trovare un
  elemento in $ A_{k-1} $ a partire da uno in $ C_k $.
  Sia $ c \in C_k $ un ciclo, quindi tale che $ \partial c = 0 $,
  siccome $ \beta_k $ è suriettiva $ \exists b \in B_k $ tale che
  $ \beta_k(b) = c $, voglio recuperare un elemento $ a \in A_{k-1} $,
  in questo modo posso definire l'azione dell'omomorfismo
  di connessione con $ \delta \colon \llbracket c \rrbracket \mapsto \llbracket a \rrbracket $.
  \[
    \begin{tikzcd}
      {} & a \in A_{k-1} \arrow{d}{\alpha_{k-1}} \\
      b \in B_k \arrow{r}{\partial} \arrow{d}{\beta_k} & B_{k-1} \arrow{d}{\beta_{k-1}} \\
      c \in C_k \arrow{r}{\partial} & C_{k-1}
    \end{tikzcd}
  \]
  Prendo il bordo per passare a $ B_{k-1} $ ($ \partial b \in B_{k-1} $), poi
  applico $ \beta_{k-1} $ e usando la commutatività $ \beta_{k-1} \circ \partial = \partial \circ \beta_k $:
  \[
    \beta_{k-1}(\partial b) = \partial \beta_k (b) = \partial c = 0
  \]
  Quindi $ \beta_{k-1}(b) = 0 $, e quindi $ \partial b \in \ker{\beta_{k-1}} $, ma
  le colonne sono esatte quindi $ \partial b \in \im{\alpha_{k-1}} = \ker{\beta_{k-1}} $,
  perciò $ \exists a \in A_{k-1} $ tale che $ \alpha_{k-1}(a) = \partial b $, quindi
  a partire da $ c \in C_k $ ho associato un elemento $ a \in A_{k-1} $.
  Per scendere a livello di omologia $ a $ deve essere un ciclo,
  cioè $ \partial a = 0 $, per verificarlo apllico $ \alpha_{k-2} $ a $ \partial a $
  e uso la commutatività:
  \[
    \alpha_{k-2}(\partial a) = \partial \alpha_{k-1}(a) = \partial \partial b = 0
  \]
  Ma $ \alpha_{k-2} $ è iniettiva, quindi $ \partial a = 0 $.
  Sono partito da un $ k $-ciclo in $ C_k $ e
  ho trovato un $ k-1 $-ciclo in $ A_{k-1} $,
  che è quello che mi proponevo di fare.

  Ci sono un paio di dettagli da verificare:
  \begin{enumerate}
  \item È univoca la scelta dell'elemento $ b $? Se non lo è ci sono
    problemi?
  \item Se prendo in $ C_k $ un elemento $ c' $ che è omologo
    a $ c $ è sicuro che trovo un $ a' $ che è
    omologo ad $ a $?
  \end{enumerate}
  Se queste due problematiche non sono verificate l'applicazione a livello di
  non è ben definita.
  Verifico che comunque scelga una controimmagine di $ \beta_k $ si ottiene
  in $ A_{k-1} $ un elemento omologo ad $ a $:
  suppongo di aver scelto la controimmagine $ b' \in B_k $ e quindi
  valga $ \beta_k (b') = \beta_k (b) = c $, allora:
  \[
    \beta_k(b' - b) = 0 \iff b' - b \in \ker{\beta_k} = \im{\alpha_k}
  \]
  Quindi esiste $ a_0 \in A_k $ tale che $ \alpha_k(\alpha_0) = b' - b $, prendendo
  il bordo:
  \[
    \partial ( b' - b ) = \partial ( \alpha_k (a_0 )) \Rightarrow \partial b' - \partial b = (\partial \circ \alpha_k)(a_0) = \alpha_{k-1}(\partial a_0)
  \]
  Ma per come costruisco l'omomorfismo di connessione $ \partial b = \alpha_{k-1}(a) $,
  e analogamente $ \partial b' = \alpha_{k-1}(a') $:
  \[
    \alpha_{k-1}(a') - \alpha_{k-1}(a) = \alpha_{k-1}(\partial a_0) \Rightarrow \alpha_{k-1}(a' - a - \partial a_0) = 0
  \]
  Ma $ \alpha_{k-1} $ è iniettivo quindi $ a' - a - \partial a_0 = 0 $, e perciò  $ a' \sim_{hom} a $,
  in quanto $ a $ e $ a' $ differiscono per un bordo.

  Per quanto riguarda la seconda questione considero $ c'' \sim_{hom} c $ in $ C_k $
  allora mostro che $ a'' \sim_{hom} a $ in $ A_{k-1} $, e così facendo
  mostro che l'applicazione è ben definita.
  \[
    c'' \sim_{hom} c \iff \exists c_0 \in C_{k+1} \; | \; c'' - c = \partial c_0
  \]
  Ma per la suriettività $ \exists b, b'' $ tale che $ c = \beta_k(b) $,
  $ c'' = \beta_k(b'') $ e $ c_0 = \beta_{k+1}(c_0) $, quindi:
  \[
    \beta_k(b'') - \beta_k(b) = \partial c_0 \Rightarrow \beta_k(b'' - b) = \partial c_0  \Rightarrow
    \beta_k(b''-b) = \partial\left( \beta_{k+1}(b_0)\right) = \beta_{k}(\partial b_0)
  \]
  Quindi:
  \[
    \beta_k(b'' - b - \partial b_0) = 0 \Rightarrow b'' - b - \partial b_0 \in \ker{\beta_k} = \im{\alpha_k}
  \]
  Perciò $ \exists \tilde{a} \in A_k$ tale che $ b'' - b - \partial b_0 = \alpha_k (\tilde{a}) $, e
  applicando il bordo si ottiene $ \partial b'' - \partial b - \partial \alpha_k(\tilde{a}) = 0 $, quindi
  dalla definizione dell'omomorfismo di connessione e dalla commutatività:
  \[
    \partial b'' - \partial b = \partial \alpha_{k}(\tilde{a}) \Rightarrow \alpha_{k-1}(a'') - \alpha_{k-1}(a) = \alpha_{k-1}(\partial \tilde{a})
  \]
  Ma $ \alpha_{k-1} $ è omomorfismo iniettivo quindi
  $ a'' - a - \partial \tilde{a} = 0 $ cioè $ a'' - a = \partial \tilde{a} $,
  quindi siccome $ a'' $ e $ a $ differiscono per un bordo sono omologhi.

  Si può quindi definire $ \delta $ su $ \llbracket c \rrbracket \in H_p(C_k) $:
  \[
    \delta(\llbracket c \rrbracket) = \llbracket \alpha \circ \partial \circ \beta^{-1}(c) \rrbracket
  \]
  Questa è ben definita.
\end{proof}

\section{Omologia singolare relativa}
Sia $ X $ uno spazio topologico e $ A $ sottospazio generico di $ X $ (anche
improprio), cioè $ A \incl X$. Vorrei definire l'omologia singolare di $ X $ tenendo
presente la presenza di $ A $, cioè $ H_k(X,A) $, il $ k $-esimo gruppo di
omologia singolare dellla coppia $ (X, A) $. Sia $ S_k(A) $ lo spazio delle
$ k $-catene in $ A $, cioè lo spazio generato dai simplessi singolari in $ A $,
la mappa di inclusione $ i \colon A \to X $ induce una mappa
$ i_\sharp \colon S_k(A) \to S_k(X) $. Questa mappa è sicuramente iniettiva (basta vedere le
catene di $ A $ come catene di $ X $, per cui $ S_k(A) \subseteq S_k(X) $). A questo
punto la successione
\[
  \begin{tikzcd}
    0 \arrow{r}{h} & S_k(A) \arrow{r}{i_\sharp} & S_k(X) \arrow{r}{\beta} & \quot{S_k(X)}{S_k(A)} \arrow{r}{k} & 0
  \end{tikzcd}
\]
è esatta infatti $ h $ iniettiva e $ \beta $ suriettiva. Vale che:
\begin{gather*}
  \im{h} = \ker{i_\sharp} = 0  \\
  \ker{k} = \im{\beta} = \quot{S_k(X)}{S_k(A)} \\
  \ker{\beta} = \im{i_\sharp}
\end{gather*}
di cui l'ultima è valida in quanto il nucleo della proiezione su un sottospazio
è il sottospazio stesso e $ \im{i_\sharp} \cong S_k(A) $ in quanto $ i_\sharp $ è iniettiva.
Pongo come notazione $ \quot{S_k(X)}{S_k(A)} = S_k(X, A) $, in questo modo la
successione diventa:
\[
  \begin{tikzcd}
    0 \arrow{r}{} & S_k(A) \arrow{r}{i_\sharp} & S_k(X) \arrow{r}{\beta} & S_k(X,A) \arrow{r}{} & 0
  \end{tikzcd}
\]
A partire da questa successione posso costruire una successione esatta corta di
complessi (la mappa tra complessi è l'applicazione bordo):
\[
  \begin{tikzcd}
    {} & 0 \arrow{d}{} & 0 \arrow{d}{} & {} \\
    \dots \arrow{r}{} & S_{k+1}(A) \arrow{r}{} \arrow{d}{} &   S_k(A) \arrow{r}{} \arrow{d}{}   & S_{k-1}(A) \arrow{r}{} \arrow{d}{}  & \dots \\
    \dots \arrow{r}{} & S_{k+1}(X) \arrow{r}{} \arrow{d}{} &   S_k(X) \arrow{r}{} \arrow{d}{}   & S_{k-1}(X) \arrow{r}{} \arrow{d}{}  & \dots \\
    \dots \arrow{r}{} & S_{k+1}(X,A) \arrow{r}{} \arrow{d}{} & S_k(X,A) \arrow{r}{} \arrow{d}{} & S_{k-1}(X,A) \arrow{r}{} \arrow{d}{} & \dots \\
    {} & 0 & 0 & {}
  \end{tikzcd}
\]
I quadrati sono commutativi quindi questa successione esatta corta
di complessi ne induce una esatta lunga.
Si ottiene quindi:
\[
  \begin{tikzcd}
    \dots \arrow{r}{} & H_k(A) \arrow{r}{\alpha_\star} & H_k(B) \arrow{r}{\beta_\star} & H_k(X,A) \arrow{r}{\delta} & H_{k-1}(A) \arrow{r}{} & \dots
  \end{tikzcd}
\]
Si definisce quindi in questo modo l'\textbf{omologia singolare della
  coppia}\index{Omologia singolare relativa}\index{Omologia singolare della
  coppia ! \vedi{Omologia singolare relativa}} $ H_k(X,A) $.

\subsection{Successioni spezzanti}

\begin{definition}[Prima definizione]
  % Con $ \alpha $ iniettiva, $ \beta $ suriettiva e $ \quot{B}{A} \cong C $, cioè
  % $ \im{\alpha} = A = \ker{\beta} $ in quanto $ \beta $ è suriettiva.
  Si dice che una successione esatta corta di $ \Z $-moduli:
  \[
    \begin{tikzcd}
      0 \arrow{r}{} & A \arrow{r}{\alpha} & B \arrow{r}{\beta} & C \arrow{r}{} & 0
    \end{tikzcd}
  \]\textbf{spezza}\index{Successione spezza} se esiste un endomorfismo continuo
  $ \phi \colon B \to B $ idempotente (cioè tale che $ \phi^2 = \phi $) e tale che $ \ker{\phi} = \im{\alpha} = \ker{\beta} $
  oppure $ \im{\phi} = \im{\alpha} = \ker{\beta} $
\end{definition}

% lezione 5

Sia $ B = A \oplus C $ con $ A, C $ $ \Z $-moduli, in quello che segue il ruolo di
$ A $ e $ C $ può essere scambiato. A questi moduli sono associate la mappa
di inclusione e di passaggio al quoziente:
\begin{align*}
  i \colon A & \to A \oplus C \\
  a & \mapsto (a, 0)
\end{align*}
\begin{align*}
  j \colon A \oplus C & \to C \\
  (a,c) & \mapsto c
\end{align*}
La mappa $ i $ è iniettiva perché è un'inclusione, mentre $ j $ è suriettiva perché
è un passaggio al quoziente, si può quindi costruire la successione esatta corta:
\[
  \begin{tikzcd}
    0 \arrow{r}{} & A \arrow{r}{i} & B = A \oplus C \arrow{r}{j} & C \arrow{r}{} & 0
  \end{tikzcd}
\]
Ma esiste anche l'inclusione $ s \colon C \to B $ e quindi ho;
\[
  \begin{tikzcd}[nodes={row sep=5pt, column sep = 20pt}]
    C \rar{s} & A \oplus C \rar{j} & C \\
    c  \arrow[mapsto]{r}{} & (0,c) \arrow[mapsto]{r}{} & c
  \end{tikzcd}
\]
%  Con $ s \circ j \colon c \mapsto (0,c) \mapsto c $, mi piacerebbe che $ s \circ j = \Id{C} $.
Vale che e $ j \circ s = \Id{C} $. Se $ B $ è proprio somma diretta di $ A $ e $ C $
posso sempre fare questa costruzione, ma nelle successioni esatte generiche non è cosi.
Una successione spezza quando ha un comportamento come questo, e la mappa $ s $
tale che $ j \circ s = \Id{C} $ è detta \textbf{sezione dell'omomorfismo}\index{Sezione dell'omomorfismo}
$ j \colon B \to C $.
% Quindi se $ B $ è proprio somma diretta ho automaticamente $ s $ e $ s' $ con $ s' $
% quoziente.
% Questo è il prototio di successione che spezza.
\begin{definition}[Seconda definizione]
  % Siano $ A, B, C $ $ \Z $-moduli con $ \ker{\alpha} = 0 $, $ \im{\beta} = C$ e $ \ker{\beta} = \im{\alpha} $,
  % cioe una successione esatta,
  Si dice che la successione esatta di $ \Z $-moduli
  \[
    \begin{tikzcd}
      0 \arrow{r}{} & A \arrow{r}{\alpha} & B \arrow{r}{\beta} & C \arrow{r}{} & 0
    \end{tikzcd}
  \]
  \textbf{spezza}\index{Successione spezza} se esiste una sezione da $ C $ a $ B $
  o da $ B $ ad $ A $, cioè:
  \begin{gather*}
    \exists s \colon C \to B \text{ omomorfismo continuo tale che } \beta \circ s = \Id{C} \\
    \text{oppure} \\
    \exists s' \colon B \to A \text{ omomorfismo continuo tale che } s' \circ \alpha = \Id{A}
  \end{gather*}
\end{definition}
Questo è equivalente a dire che $ B = A \oplus s(C) $, infatti vale l'osservazione
\begin{osservation}
  Se la successione $ 0 \to A \to B \to C \to 0 $ spezza allora $ B \cong A \oplus s(C) $ con $ s $ sezione.
  Il viceversa l'ho già dimostrato, infatti se $ B $ si scrive come somma diretta
  la sezione è banale.
\end{osservation}
\begin{proof}
  Per dimostrare che $ B \cong A \oplus s(C) $ per prima cosa mostro che l'intersezione
  tra $ A $ e $ s(C) $ è vuota.

  Siccome $ \alpha $ è iniettiva allora $ \alpha(A) \cong A $, inoltre
  $ s(C) \subseteq B $ in quanto per ipotesi $ s \colon C \to B $. Sia
  $ x \in \alpha(A) \cap s(C) $, mostro che $ x = 0
  $.%cioè $ x \in \alpha(A) $ e $ x \in s(C) $ allora
  Siccome $ x \in \alpha(A) $ allora esiste $ a \in A $ tale che
  $ x = \alpha(a) $ e siccome $ x \in s(C) $ allora esiste $ k \in C $ tale che
  $ x = s(k) $, naturalmente $ \alpha(a) = s(k) $. Applicando $ \beta $ si ottiene
  $ (\beta \circ \alpha) (a) = (\beta \circ s)(k) $, ma
  $ \beta \circ \alpha = 0 $ in quanto la successione è esatta, quindi
  $ (\beta \circ s)(k) = 0 $. Ma $ s $ è sezione quindi
  $ \beta \circ s = \Id{C} $, quindi $ k = 0 $, ma siccome $ s $ è omomorfismo allora
  $ s(k) = 0 $, perciò $ x = s(k) = 0 $.

  A questo punto bisogna dimostrare che ogni elemento di $ B $ si scrive come somma
  di un elemento di $ \alpha(A) $ e di un elemento
  di $ s(C) $.

  Sia $ b \in B $ allora $ \beta(b) \in C $, ci sono due possibilità:
  \begin{enumerate}
  \item Se $ \beta(b) = 0 $ significa $ b \in \ker{\beta} = \im{\alpha} $, quindi $ b \in \im{a} $, cioè
    $ \exists \alpha \in A $ tale che $ b = \alpha(a) $ e quindi si scrive come elemento di $ A $ sommato
    a zero.
  \item Se $ \beta(b) = c \not = 0 $ allora $ b - s(t) \in B $,
    mostro che $ b - s(t) \in \ker{\beta} $ e quindi posso usare lo stesso ragionamento
    di prima.
    \[
      \beta(b - s(t)) = \beta(b) - \beta(s(t)) = t - t = 0 \Rightarrow \beta(b - s(t)) \in \ker{b} = \im{\alpha}
    \]
    Quindi esiste $ a' \in A $ tale che $ \alpha(a') = b - s(t) $ e quindi
    vale che $ b = s(t) + \alpha(a') $
  \end{enumerate}
  Siccome l'intersezione tra $ A $ e $ s(C) $ è vuota e ogni elemento di $ B $
  si può scrivere come somma di un elemento di $ A $ e di uno di $ s(C) $ allora
  $ B $ è somma diretta di $ A $ e $ s(C) $.
\end{proof}

Sostanzialmente una successione esatta corta spezza se vale il diagramma commutativo:
\[
  \begin{tikzcd}
    0  \rar & A \rar \dar{\cong} & B \rar  \dar{\cong} & C \rar \dar{\cong} & 0 \\
    0 \rar & A \rar &  A \oplus C \rar & C \rar & 0
  \end{tikzcd}
\]

\begin{example}[Successione non spezzante]
  Considero la successione:
  \[
    \begin{tikzcd}
      0 \arrow{r}{} & n \Z \arrow{r}{\alpha} & \Z \arrow{r}{\beta} & \quot{\Z}{n\Z} \arrow{r}{} & 0
    \end{tikzcd}
  \]
  Questa successione è esatta ma non spezza, infatti se spezzasse varrebbe che:
  \[
    \Z_n \oplus n \Z \cong \Z
  \]
  Ma questa non è possibile in quanto $ n \Z \cong \Z $ e $ \Z_n $ non è banale. Più
  precisamente si vede che non può esistere una sezione $ s \colon \Z_n \to Z $.
\end{example}

\begin{proposition}
  Le due definizioni di successione che spezza sono equivalenti, cioè
  se $ \exists s \colon C \to B $ tale che $ \beta \circ s = \Id{C} $ allora $ \exists \phi \colon B \to B $ tale che sia
  idempotente e che $ \ker{\phi} = \ker{\beta} $
\end{proposition}
\begin{proof}
  Una possibile costruzione è $ \phi = s \circ \beta  $, infatti:
  \[
    \phi^2 = s \circ \beta \circ s \circ \beta = s \circ \Id{C} \circ \beta = s \circ \beta = \phi
  \]
  Quindi $ \phi $ è idempotente. Siccome $ s $ omomorfismo $ \ker{\beta} \subseteq \ker{s \circ \beta } $,
  mostro che $ \ker{s \circ \beta} \subseteq \ker{\beta} $:
  \[
    \ker{\phi} = \ker{s \circ \beta} = \set{ b \in B | (s \circ \beta)(b) = 0}
  \]
  Quindi $ s(\beta(b)) = 0 $ cioè $  \beta \circ s \circ \beta (b) = 0 $ quindi $ \beta(b) = 0 $ che significa
  che $ b \in \ker{\beta} $. Ma quindi $ \ker{\beta} \subseteq \ker{s \circ \beta} \subseteq \ker{\beta} $ allora $ \ker{s \circ \beta} = \ker{\beta} $.
  Rimane da mostrare il viceversa.
  \begin{exercise}
    Mostrare che se esiste l'endomorfismo $ \phi $ allora si può costruire una sezione.
  \end{exercise}
  Le due definizioni sono quindi equivalenti.
\end{proof}

\section{Omologia singolare ridotta}

Fin ora ho parlato di omologia singolare $ H_k(X) $, omologia singolare relativa
$ H_k(X,A) $, ora introduco l'omologia singolare ridotta.

\begin{definition}
  Sia $ X $ uno spazio topologico e $ A = \set{x_0 \in X} $, è ben definita l'omologia
  relativa $ H_k(X, A) $, si definisce questa come \textbf{omologia singolare ridotta}\index{Omologia singolare ridotta}
  $ \tilde{H}_k(X) $.
  L'omologia singolare ridotta è l'omologia relativa ad un punto.
\end{definition}
Per costruire l'omologia singolare ridotta servono le $ k $-catene in $ X $ e le
$ k $-catene in $ \set{x_0} $
\[
  \begin{tikzcd}
    0 \arrow{r}{} & S_k(\set{x_0}) \arrow{r}{} & S_k(X) \arrow{r}{} & \quot{S_k(X)}{S_k({\set{x_0}})} = S_k(X, \set{x_0}) \arrow{r}{} & \dots
  \end{tikzcd}
\]
In $ S_k $ $ \sigma \colon \Delta_k \to \set{x_0} $ è simplesso sono le applicazioni costanti
dal $ k $-simplesso standard in $ \set{x_0} $. Quindi $ S_k(\set{x_0}) = \langle\sigma_k\rangle $, dato che $ \sigma_k $ è l'unica mappa che c'è.
\begin{lemma}[Omologia di un punto]
  Sia $ X = \set{x_0} $ con $ x_0 \in X $, allora:
  \[
    H_k(\set{x_0}) \cong
    \begin{cases}
      \Z & \text{se } k = 0 \\
      0 & \text{se } k \geq 1
    \end{cases}
  \]
\end{lemma}

\begin{proof}
  Il generico $ k $-simplesso singolare in $ X $ è una mappa
  continua $ \sigma_k \colon \Delta_k \to \set{x_0} $, quindi fissato $ k $ esiste
  un solo simplesso singolare, che è la mappa costante dal simplesso
  standard a $ x_0 $. Il generico $ S_k(X) $ quindi è il gruppo libero
  generato da questo simplesso singolare, cioè $ S_k(X) = \langle\sigma_k\rangle $.
  A questo punto fissato $ k $ si può computare semplicemente il bordo di $ \sigma_k $:
  % infatti dalla definizione di omologia singolare c'è il complesso:
  % \[
  %   \begin{tikzcd}
  %     \dots \arrow{r}{}  & S_{k+1}(\set{x_0}) \arrow{r}{\partial} &  S_{k}(\set{x_0}) \arrow{r}{\partial} &  S_{k-1}(\set{x_0}) \arrow{r}{} & \dots
  %   \end{tikzcd}
  % \]
  % Che corrisponde alla successione dei generatori:
  % \[
  %   \begin{tikzcd}
  %     \dots \arrow{r}{}  & \langle\sigma_{k+1}\rangle \arrow{r}{\partial} &  \langle\sigma_{k}\rangle \arrow{r}{\partial} & \langle\sigma_{k-1}\rangle \arrow{r}{} & \dots
  %   \end{tikzcd}
  % \]
  \[
    \partial \sigma_k = \sum_{i = 0}^k (-)^i \sigma_k^{(i)}  \text{ con }  \sigma_k^{(i)} \colon \Delta_{k-1} \overset{F_k^{\; i}}{\to} \Delta_k \overset{\sigma_k}{\to} \set{x_0}
    \text{ cioè } \sigma_k^{(i)}  = \sigma_{k-1}
  \]
  Fissato $ k $ nella sommatoria che calcola il bordo tutte le quantita sono uguali,
  quindi la somma a segni alterni è nulla oppure è uguale a $ \sigma_{k-1} $ a seconda
  della parità di $ k $.
  \[
    \partial \sigma_k =
    \begin{cases}
      0       & \text{se $ k $ dispari} \\
      \sigma_{k-1} & \text{se $ k $ pari}
    \end{cases}
  \]
  A questo punto si può calcolare facilmente il nucleo e l'immagine dell'operatore
  bordo:
  \[
    \ker{\partial_k} =
    \begin{cases}
      0       & \text{$ k $ dispari} \\
      S_k(X)  & \text{$ k \geq 2$ pari}
    \end{cases}
  \]
  E:
  \[
    \im{\partial_{k+1}} =
    \begin{cases}
      0       & \text{$ k $ dispari} \\
      S_k(X)  & \text{$ k \geq 2$ pari}
    \end{cases}
  \]
  Infatti, se $ k \geq 2 $ ed è pari:
  \begin{align*}
    \partial_k \colon S_k(\set{x_0}) & \to S_{k-1}(\set{x_0}) \\
    \sigma_k & \mapsto \sigma_{k-1}
  \end{align*}
  quindi solo lo $ 0 $ è mandato in $ 0 $, mentre se è dispari:
  \begin{align*}
    \partial_k \colon S_k(\set{x_0}) & \to S_{k-1}(\set{x_0}) \\
    \sigma_k & \mapsto 0
  \end{align*}
  quindi tutto viene mandato in $ 0 $
  Invece per $ k $ pari:
  \begin{align*}
    \partial_{k+1} \colon S_{k+1}(\set{x_0}) & \to S_{k}(\set{x_0}) \\
    \sigma_{k+1} & \mapsto \sigma_{k}
  \end{align*}
  quindi l'immagine è il generatore, cioè tutto $ S_{k}(X) $, mentre per $ k \geq 2$ pari:
  \begin{align*}
    \partial_k \colon S_k(\set{x_0}) & \to S_{k-1}(\set{x_0}) \\
    \sigma_k & \mapsto 0
  \end{align*}
  Quindi l'immagine è solo $ 0 $.

  A questo punto se $ k \geq 2 $ $ \im{\partial_{k+1}} = \ker{\partial_k} $, quindi:
  \[
    H_k(X) = \quot{\ker{\partial_{k}}}{\im{\partial_{k+1}}} \cong 0
  \]
  Invece se $ k = 0 $ vale che $ \ker{\partial_0} = S_0(X) $, mentre
  $ \im{\partial_1} = 0 $ quindi:
  \[
    \quot{\ker{\partial_{0}}}{\im{\partial_{1}}} \cong S_0(X)
  \]
  Questo è sostanzialmente l'unico caso in cui si può calcolare
  direttamente dalla definizione i gruppi di omologia.
\end{proof}

\begin{proposition}
  Vale che:
  \[
    \tilde{H}_k(X) \cong
    \begin{cases}
      \quot{H_0(X)}{\Z} & \text{se } k = 0 \\
      H_k(X) & \text{se } k \geq 1
    \end{cases}
  \]
\end{proposition}
\begin{proof}
  Per dimostrarlo uso la successione esatta lunga in omologia relativa:
  \[
    \begin{tikzcd}
      \dots \arrow{r}{} & H_{k+1}(\set{x_0}) \arrow{r}{} & H_{k+1}(X) \arrow{r}{} & \tilde{H}_{k+1}(X) \arrow{r}{} & H_k(\set{x_0}) \arrow{r}{} & \dots
    \end{tikzcd}
  \]
  Nel caso $ k \geq 1 $ tutti i gruppi di omologia del punto sono banali, quindi il complesso diventa:
  \[
    \begin{tikzcd}
      0 \arrow{r}{i} & H_{k+1}(X) \arrow{r}{\psi} &  \tilde{H}_{k+1}(X) \arrow{r}{j} &  0
    \end{tikzcd}
  \]
  La successione è esatta quindi $ \psi $ è iniettiva, ma è suriettiva essendo una
  proiezione al quoziente, quindi è un isomorfismo e perciò $ H_m(X) \cong \tilde{H}_m(X) $ per $ m \geq 2 $.
  % $ \ker{\psi} = \im{i} $, ma $ i \colon 0 \to H_{k+1}(X) $ omomorfismo,
  % perciò $ i(0) = 0 $ quindi $ \ker{\psi} = 0 $, cioè $ \psi $ è iniettiva.
  % $ \psi $ è iniettiva quindi $ \ker{\psi} = \im{0} = 0 $, ma è anche surietta, in quanto [MANCA].
  % Quindi $ \psi $ è isomorfismo e perciò $ H_m(X) \cong \tilde{H}_m(X) $ per $ m \geq 2 $.
  Mi rimane da calcolare il caso $ k = 1 $ e il caso $ k = 0 $.
  Considero la successione esatta:
  \[
    \begin{tikzcd}[nodes = {column sep = 10pt}]
      0 \rar & H_1(\set{x_0}) \rar & H_1(X) \rar & \tilde{H}_1(X) \rar & H_0(\set{x_0}) \rar & H_0(X) \rar & \tilde{H}_0(X) \rar & 0
    \end{tikzcd}
  \]
  So che $ H_1(\set{x_0}) = 0 $ quindi:
  \[
    \begin{tikzcd}[nodes = {column sep = 15pt}]
      0 \rar & H_1(X) \rar{\phi} & \tilde{H}_1(X) \rar{j} & H_0(\set{x_0}) \rar{i_\star} & H_0(X) \rar{\tau} & \tilde{H}_0(X) \rar & 0
    \end{tikzcd}
  \]
  Inoltre so sempre dall'omologia di un punto che $ H_0(\set{x_0}) $ è il gruppo libero di rango
  uno, sia la classe di $ \sigma_0 \colon \Delta_0 \to \set{x_0} $ il generatore. È definita una
  mappa di inclusione $ i \colon \set{x_0} \to X $ che induce
  \begin{align*}
    i_\star \colon H_0(\set{x_0}) & \to H_0(X) \\
    \llbracket\sigma_0\rrbracket & \mapsto \llbracket i \circ \sigma_0 \rrbracket = \llbracket \sigma_0 \rrbracket
  \end{align*}
  Poi si estende per linearità al generico elemento $ c = k \sigma_0 $ con $ k \in \Z $, inoltre
  si è usato che $ i \circ \sigma_0 = \sigma_0 $ perché $ \sigma_0 $ è lo $ 0 $-simplesso singolare
  costante che vale $ x_0 $. In particolare $ \im{i_\star} = H_0(\set{x_0}) $.
  Questa mappa indotta è iniettiva, infatti sia  $ c = k \sigma_0 \in H_0(\set{x_0}) $:
  \[
    i_\star(\llbracket c \rrbracket) = \llbracket 0 \rrbracket \iff \llbracket i \circ c \rrbracket = \llbracket 0 \rrbracket \iff \exists u \in S_1(X) \text{ tale che } i \circ c - 0 = \partial u \Rightarrow i \circ c = \partial u
  \]
  Ma $ c = k \sigma_0 $, quindi:
  \[
    k i \circ \sigma_0 = \partial u \Rightarrow k \circ \sigma_0 = \partial u \Rightarrow c = \partial u
  \]
  Ma quindi $ c $ e $ 0 $ differiscono per un bordo, quindi $ c $ è nella stessa classe
  di equivalenza di $ 0 $, cioè $ \llbracket c \rrbracket = \llbracket 0 \rrbracket $ e quindi $ \ker{i_\star} = 0 $, cioè $ i_\star $ è
  iniettiva.
  Quindi $ \ker{i_\star} = 0 $ da cui $ \im{j} = \ker{i_\star} = 0 $, perciò posso scrivere
  la successione esatta corta:
  \[
    \begin{tikzcd}
      0 \rar & H_1(X) \rar{\phi} & \tilde{H}_1(X) \rar{j} & 0
    \end{tikzcd}
  \]
  Siccome $ \phi $ è iniettiva ma e è anche suriettiva perché è proiezione
  sul quoziente allora è isomorfismo e quindi $ H_1(X) \cong \tilde{H}_1(X) $.
  [NON SONO SICURO DI QUESTO, RIASCOLTARE!!!]
  Ma siccome $ H_1(X) \cong \tilde{H}_1(X) $ allora la successione lunga iniziale diventa:
  \[
    \begin{tikzcd}
      0 \rar & H_1(X) \arrow{r}{j} & H_0(\set{x_0}) \arrow{r}{i_\star} &  H_0(X) \arrow{r}{\tau} & \tilde{H}_0(X) \arrow{r}{} & 0
    \end{tikzcd}
  \]
  Quindi ora $ j $ è iniettiva perciò sono $ 0 $ va in $ 0 $:
  \[
    \begin{tikzcd}
      0 \rar & H_0(\set{x_0}) \arrow{r}{i_\star} &  H_0(X) \arrow{r}{\tau} & \tilde{H}_0(X) \arrow{r}{} & 0
    \end{tikzcd}
  \]
  Ma $ \tau $ è suriettiva, quindi $ \im{\tau} = \tilde{H}_0(X) $, inoltre la successione è esatta
  quindi $ \ker{\tau} = \im{i_\star} = H_0(\set{x_0}) $, quindi $ \quot{H_0(X)}{H_0(\set{x_0})} \cong \tilde{H}_0(X) $
  infatti $ \quot{H_0(X)}{\ker{\tau}} \cong \im{\tau} $
  per il teorema fondamentale dell'isomorfismo.
  % E infine $ \im{\tau} = \tilde{H}_0(X) $ per la suriettività e $ \ker{\tau} = \Z $ per l'iniettività.

  Quindi ho trovato che $ \forall k \geq 1 $ i gruppi di omologia singolare e omologia singolare ridotta
  sono isomorfi, mentre per $ k = 0 $ ho trovato che:
  \[
    \tilde{H}_0(X) = \quot{H_0(X)}{H_0(\set{x_0})} \cong  \quot{H_0(X)}{\Z}
  \]
  Se voglio mostrare che $ H_0(X) \cong \tilde{H}_0(X) \oplus \Z $ basta che mostro che esiste una sezione,
  ovvero che la successione esatta corta:
  \[
    \begin{tikzcd}
      0 \rar & H_0(X) \rar & \tilde{H}_0(X) \rar & \Z \rar & 0
    \end{tikzcd}
  \]
  spezza. Questo è sempre vero, a meno di casi eccezionalmente patologici.
\end{proof}

\begin{example}
  Considero ad esempio $ H_k(\Sph{n}) $ con $ n \geq 1 $:
  \[
    H_k(\Sph{n}) \cong
    \begin{cases}
      \Z & \text{se } k \in \set{0,n} \\
      0 & \text{se } k \not \in \set{0,n}
    \end{cases}
  \]
  Fin ora so che:
  \[
    H_1(\Sph{n}) \cong
    \begin{cases}
      \Z & \text{se } n = 1 \\
      0 & \text{se } n \geq 2
    \end{cases}
  \]
  E che $ H_0(\Sph{n}) \cong \Z $ per $ n \geq 1 $, vorrei calcolare gli altri gruppi di omologia,
  ma per farlo mi servono altri strumenti.
\end{example}

\section{Assiomi di una teoria omologica}

\begin{definition}[Teoria omologica secondo Eilenberg e Steenrod]
  Una \textbf{teoria omologica}\index{Teoria omologica}\index{Steendord ! \vedi{Teoria omologica}}\index{Eilenberg ! \vedi{Teoria omologica}}
  sulla categoria di tutte le coppie di spazi topologici e mappe continue è
  un funtore che assegna ad ogni coppia di spazi $ (X, A) $ un gruppo
  abeliano $ H_p(X, A) $ per il quale si pone $ H_k(X) := H_{k}(X, \emptyset) $
  e ad ogni applicazione continua $ f \colon (X, A) \to (Y, B) $
  un omomorfismo $ f_\star \colon H_k(X, A) \to H_k(Y, B) $ con una trasformazione
  naturale $ \delta_k \colon H_k(X, A) \to H_{k-1}(A) $,
  detta \textbf{omomorfismo di connessione}\index{Omomorfismo di connessione}
  tale che siano soddisfatti i seguenti assiomi:
  \begin{enumerate}
  \item (Omotopia): se $ f \sim_H g $ con $ f, g \colon (X, A) \to (Y, B) $ mappe continue, allora $ f_\star = g_\star$.
    Dove $ f \sim_H g $ se esiste una funzione continua $ F \colon X \times I \to Y $ tale che $ F(x,0) = f(x) $,
    $ F(x, 1) = g(x) $ e $ F(a, t) \subseteq B $ $ \forall a \in A $ e $ \forall t \in I $.
  \item (Esattezza): Per ogni inclusione $ i \colon A \incl X $ e $ j \colon X \incl (X, A) $ la successione:
    \[
      \begin{tikzcd}
        \dots \arrow{r}{}  & H_p(A) \arrow{r}{i_\star} &  H_p(X) \arrow{r}{j_\star} &  H_p(X,A) \arrow{r}{\delta_p} & H_{p-1}(A) \arrow{r}{} & \dots
      \end{tikzcd}
    \]
    è esatta.
  \item (Dimensione): $ H_k (P) = 0 $ $ \forall k \not = 0 $ dove $ P $ è lo spazio formato da un solo punto.
  \item (Additività): Se $ X $ è la somma topologica di spazi $ X_\alpha $ allora $ H_p(X) = \bigoplus_\alpha H_p(X_\alpha) $
  \item (Escissione): Se $ U $ è un aperto in $ X $ tale che $ \bar{U} \subset \mathrm{int}(A) $ allora la mappa di
    inclusione di $ (X \setminus U, A \setminus U) $ in $ (X, A) $ induce un isomorfismo tra i gruppi di omologia:
    \[
      H_k(X \setminus U, A \setminus U) \cong H_k(X, A) \quad \forall k \in \mathbb{N}
    \]
    (cioè togliendo un opportuno insieme da $ (X,A) $ l'omologia non sente della escissione).
  \end{enumerate}
  Per trasformazione naturale si intende che $ \forall f \colon (X, A) \to (Y, B) $ il seguente diagramma è commutativo:
  \[
    \begin{tikzcd}
      H_p(X,A) \arrow{r}{\delta} \arrow{d}{f_\star} & H_{p-1}(A) \arrow{d}{f'_\star} \\
      H_p(Y,B) \arrow{r}{\delta} & H_{p-1}(B)
    \end{tikzcd}
  \]
  dove $ f' = f \big \lvert_A $.
  Mentre la richiesta che sia funtore significa che se $ f \colon (X, A) \to (Y, B) $ e $ g \colon (Y, B) \to (Z, C) $ sono
  mappe continue allora $ (g \circ f)_\star = g_\star \circ f_\star $ e che $ (\Id{X})_\star = \Id{H_k(X)} $.
\end{definition}
L'omologia singolare relativa soddisfa tutti questi assiomi,
ma non tutti sono stati ancora verificati, cioè l'omotopia e l'escissione.

\newmathsymb{topsum}{\invamalg}{Somma topologica}
\begin{definition}
  Sia $ \set{X_\alpha} $ una famiglia di spazi topologici, si definisce la \textbf{somma topologica}\index{Somma topologica}
  $ X = \invamalg_\alpha X_\alpha $
  come lo spazio topologico formato dall'unione disgiunta di tutti gli $ X_\alpha $ equipaggiato
  con la \textbf{topologia debole}\index{Topologia debole},
  ovvero un insieme è aperto se e solo se è aperto rispetto alla topologia di ogni $ X_\alpha $.
\end{definition}

% lezione 6
%  _     _____ ________ ___  _   _ _____    __
% | |   | ____|__  /_ _/ _ \| \ | | ____|  / /_
% | |   |  _|   / / | | | | |  \| |  _|   | '_ \
% | |___| |___ / /_ | | |_| | |\  | |___  | (_) |
% |_____|_____/____|___\___/|_| \_|_____|  \___/

\begin{proposition}
  Esiste almeno una teoria che soddisfa gli assiomi di Eilenberg e Steenrod.
\end{proposition}
\begin{proof}
  [MANCA DA SISTEMARE TUTTA QUESTA PROOF!!!]
  L'omologia singolare relativa soddisfa gli assiomi di Eilenberg e Steenrod,
  ne ripercorro la costruzione e metto in luce il collegamento con gli assiomi.

  Ho introdotto gli spazi di $ k $-catene e ho definito $ S_k(X,A) = \quot{S_k(X)}{S_k(A)} $,
  poi ho costruito con l'operatore di bordo $ \partial $ e la proiezione al quoziente $ P $
  la successione esatta corta:
  \[
    \begin{tikzcd}
     0 \rar & S_k(A) \rar & S_k(X) \rar & S_k(X,A) = \quot{S_k(X)}{S_k(A)} \rar & 0
    \end{tikzcd}
  \]
  Cioè in modo più esteso:
  \[
    \begin{tikzcd}
      {} & 0 \arrow{d}{} & 0 \arrow{d}{} & {} \\
      \dots \arrow{r}{} & S_{k+1}(A) \arrow{r}{\partial} \arrow{d}{} & S_k(A) \arrow{r}{} \arrow{d}{} & \dots \\
      \dots \arrow{r}{} & S_{k+1}(X) \arrow{r}{\partial} \arrow{d}{P} & S_k(X) \arrow{r}{} \arrow{d}{P} & \dots \\
      \dots \arrow{r}{} & S_{k+1}(X,A) \arrow{r}{\partial'} \arrow{d}{} & S_k(X,A) \arrow{r}{} \arrow{d}{}& \dots \\
      {} & 0 & 0 & {}
    \end{tikzcd}
  \]
  Posso costruire $ \partial '$  tale che i quadrati siano commutativi, cioè tale che $ P \circ \partial = \partial' \circ P $,
  infatti: sia $ c \in S_{k+1}(X) $ allora la sua immagine tramite $ P $ è la classe di equivalenza
  $ P(c) = [c' \in S_{k+1}(X) \text{ tale che } c'-c \in S_{k+1}(A)] $,
  allora definisco $ \partial'([c]_A) := [\partial c]_A $ dove con il pedice $ A $ indico la relazione
  di equivalenza indotta dalle $ k $-catene in $ A $, così la relazione è automaticamente soddisfatta.

  Devo verificare che $ \partial' $ è ben definita cioè che se prendo elementi
  equivalenti ottengo elementi equivalenti.
  Se $ c' \sim_A c $ allora $ \exists a \in S_{k+1}(A) $ tale che $ c' - c = a $,
  prendo il bordo $ \partial c' - \partial c = \partial a $, ma $ \partial a \in S_k(A) $ quindi
  $ \partial c' $ e $ \partial c $ differiscono per un elemento in $ S_k(A) $ e quindi
  $ \partial c' \sim_A \partial c $ perciò l'applicazione è ben definita.

  L'omologia relativa singolare è l'omologia singolare del complesso $ S_\bullet(X, A) $,
  cioè per definizione:
  \[
    H_k(X,A) = H_k(S_\bullet(X,A)) = \quot{\ker{S_k(X,A) \to S_{k-1}(X,A)}}{\im{S_{k+1}(X,A) \to S_{k}(X,A)}}
  \]
  Questo gruppo abeliano (in quanto è quoziente di gruppi abeliani) è detto
  gruppo di omologia relativa della coppia  $ (X, A) $. Se in particolare $ A = \emptyset $
  allora riottengo $ H_k(X, \emptyset) = H_k(X) $.
  Ho fatto un'associazione da una coppia a un gruppo, voglio verificare che questa
  sia funtoriale.

  Sia $ f \colon (X,A) \to (Y,B) $ continua tale che $ f(A) \subseteq B $, definisco:
  \begin{align*}
    f_\star \colon H_k(X,A) & \to H_k(Y,B) \\
    \llbracket c \rrbracket_ A & \mapsto  \llbracket f_\sharp(c) \rrbracket_ B
  \end{align*}
  I pedici $ A $ e $ B $ stanno a ricordare che l'omologia è relativa.
  Quindi ho;
  \[
    \begin{tikzcd}
      \dots \arrow{r}{} & S_{k+1}(X,A) \arrow{r}{} \arrow{d}{f_\sharp} & S_k(X,A) \arrow{r}{}  \arrow{d}{f_\sharp} & S_{k-1}(X,A) \arrow{r}{}  \arrow{d}{f_\sharp} & \dots \\
      \dots \arrow{r}{} & S_{k+1}(Y,B) \arrow{r}{} & S_k(Y,B) \arrow{r}{} & S_{k-1}(Y,B) \arrow{r}{} & \dots
    \end{tikzcd}
  \]
  $ f_\sharp $ esiste, infatti:
  \[
    \begin{tikzcd}
      S_{k}(X) \arrow{r}{f} \arrow{d}{} & S_k(Y)  \arrow{d}{}  \\
      \quot{S_k(X)}{S_k(A)} \arrow{r}{f_\sharp} & \quot{S_k(Y)}{S_k(B)}
    \end{tikzcd}
  \]
  $ f_\sharp $ esiste perché $ S_k(A) \to S_k(B) $ per la condizione $ f(A) \subseteq B $
  quindi simplessi singolari in $ A $ vanno a finire in simplessi singolari in $ B $.
  Una volta che ho $ f_\sharp $ posso scendere a livello di omologia prendendo la
  classe di equivalenza.

  Inoltre ho dimostrato che se $ (X,A) \overset{f}{\to} (Y,B) \overset{g}{\to} (Z,C) $,
  allora $ (g \circ f)_\star = g_\star \circ f_\star $ e $ (X,A) \overset{\Id{X}}{\to} (X,A) $
  allora $ (\Id{X})_\star = \Id{H_k(X,A)} $.

  Poi ho $ \delta $ omomorfismo di connessione $ \delta \colon H_{k+1}(X,A) \to H_k(A) $,
  ho che se ho una successione esatta di complessi:
  \[
    \begin{tikzcd}
      0 \arrow{r}{} & S_\bullet(A) \arrow{r}{} & S_\bullet(X) \arrow{r}{} & S_\bullet(X,A) \arrow{r}{} & \dots
    \end{tikzcd}
  \]
  Esiste una successione lunga in omologia:
  \[
    \begin{tikzcd}
      \dots \arrow{r}{} & H_k(A) \arrow{r}{} & H_k(X) \arrow{r}{} & H_k(X,A) \arrow{r}{\delta} & H_{k-1}(A) \rar &  \dots
    \end{tikzcd}
  \]

  Ho dimostrato che $ H_k(P) = H_k(P, \emptyset) = 0 $ per $ k \geq 1 $ e $ P $ insieme
  formato da un solo punto in $ X $,
  inoltre so che $ H_k (\invamalg_\alpha X_\alpha) \cong \bigoplus_\alpha H_k(X_\alpha) $ con $ X_\alpha $
  varie componenti connesse per archi.

  Mi rimangono da verificare l'invarianza omotopica e l'escissione.

  Quindi gli assiomi di Eilenberg e Steenrod non definiscono una teoria
  vuota ma c'è almeno una teoria a soddisfarli, che è l'omologia singolare
  relativa.

  In futuro verificherò gli assiomi mancanti.
\end{proof}
 % \hfill\newline\newline
\subsection{Omologia ridotta per una qualsiasi teoria omologica}

Sia $ X \not = \emptyset $ spazio topologico e $ p \in X $ punto
($ P = \set{p} $), allora sono ben definite le applicazioni di inclusione $ i $
e la mappa costante $ \epsilon $:
\begin{gather*}
  i \colon P \to X \\
  \epsilon \colon X \to P
\end{gather*}
Si ha che $ \epsilon \circ i = \Id{P} $ in quanto
$ P \overset{i}{\to} X \overset{\epsilon}{\to} P $. Dagli assiomi deriva l'esistenza di
un'applicazione indotta sui gruppi di omologia:
$ \epsilon_\star \colon H_0(X) \to H_0(P) $, questa è suriettiva perché per le proprietà
funtoriali $ (\epsilon \circ i)_\star = (\Id{p})_\star = \Id{H_0(p)} $ e
$ (\epsilon \circ i)_\star = \epsilon_\star \circ i_\star $ quindi $ \epsilon_\star \circ i_\star = \Id{H_0(p)} $, quindi:
\[
  \forall y \in H_0(P) \text{ vale che } (\epsilon_\star \circ i_\star)(y) = y \text{ quindi } \epsilon_\star (i_\star(y)) = y
\]
Sia $ i_\star(y) = x \in H_0(X) $ allora $ \epsilon_\star(x) = y $, quindi $ \epsilon_\star $ è suriettiva.
A partire da ciò posso costruire una successione esatta, infatti per ora ho:
\[
  \begin{tikzcd}
    H_0(X) \rar{\epsilon_\star} & H_0(P) \rar & 0
  \end{tikzcd}
\]
Per il teorema fondamentale degli omomorfismi:
\[
  \quot{H_0(X)}{\ker{\epsilon_\star}} \cong \im{\epsilon_\star} = H_0(P)
\]
Se ora considero la mappa iniettiva $ \alpha \colon \ker{\epsilon_\star} \incl H_0(X) $, quindi tale che
$ \im{\alpha} = \ker{\epsilon_\star} $, la successione corta è automaticamente esatta (infatti $ \epsilon_\star \circ \alpha = 0 $, dato
che in $ H_0(P) $ $ \ker{\epsilon_\star} $ è ridotto al solo $ 0 $):
\[
  \begin{tikzcd}
    0 \arrow{r}{} & \ker{\epsilon_\star} \arrow{r}{\alpha} & H_0(X) \arrow{r}{\epsilon_\star} & \arrow[bend left]{l}{i_\star} H_0(P) \arrow{r}{} & 0
  \end{tikzcd}
\]
% c'è $ i_\star \colon H_0(P) \to H_0(X) $:
% C'è la successione esatta:
% \[
%   \begin{tikzcd}
%     0 \arrow{r}{} & \ker{\epsilon_\star} \arrow{r}{} & H_0(X) \arrow{r}{\epsilon_\star} & \arrow[bend left]{l}{i_\star} H_0(P) \arrow{r}{} & 0
%   \end{tikzcd}
% \]
Inoltre, siccome $ \epsilon_\star \circ i_\star = \Id{H_0(p)} $, la successione spezza perché
esiste una sezione $ i_\star $, perciò
$ H_0(X) \cong \ker{\epsilon_\star} \oplus H_0(P) $. Si ha quindi che per qualsiasi teoria omologia
che soddisfa gli assiomi di Eilenberg e Steenrod (infatti ho utilizzato solo gli
assiomi), e quindi in particolare per l'omologia singolare relativa, si ha che
$ H_0(X) \cong \ker{\epsilon_\star} \oplus H_0(P) $.

\newmathsymb{coefgrup}{\mathcal{G}}{Gruppo dei coefficienti}
Generalmente si chiama $ H_0(P) $ il \textbf{gruppo dei coefficienti}\index{Gruppo dei coefficienti di una teoria omologica}
di una teoria omologica e viene denotato con $ \mathcal{G} $. Nell'omologia singolare relativa questo è $ \Z $.
Inoltre si definisce $ \ker{\epsilon_\star} = \tilde{H}_0(X) $ \textbf{gruppo di omologia ridotta di ordine zero}, quindi
ho trovato che $ H_0(X) \cong \tilde{H}_0(X) \oplus \mathcal{G} $.

Cosa sono invece gli $ \tilde{H}_k(X) $? Vorrei che fossero proprio $ H_k(X) $,
così come nel solo nel caso dell'omologia singolare.

\begin{proposition}
  In qualsiasi teoria omologica di Eilenberg e Steenrod, se $ \tilde{H}_k(X) $ sono
  i gruppi di omologia ridotta allora:
  \[
    H_k(X) \cong
    \begin{cases}
      \tilde{H}_0(X) \oplus \mathcal{G} & \text{se } k = 0 \\
      \tilde{H}_k(X) & \text{se } k \not= 0
    \end{cases}
  \]
  Con $ \epsilon \colon X \to P $ dove $ P = \set{p} $ con $ p \in X $.
\end{proposition}
\begin{proof}
  Considero $ F \colon (X, A) \to (P,P) $ con:
  \[
    F =
    \begin{cases}
      \epsilon_X \colon X \to P \\
      \epsilon_A \colon A \to P
    \end{cases}
  \]
  % Posso fare:
  % \[
  %   \begin{tikzcd}
  %     0 \arrow{r}{}        & \ker{\epsilon_X} \arrow{r}{} & H_k(X) \arrow{r}{\epsilon_\star} & H_k(P) \arrow{r}{}    & 0
  %   \end{tikzcd}
  % \]
  In generale:
  \[
    \begin{tikzcd}
      0 \arrow{r}{}        & \ker{\epsilon_X} \arrow{r}{} & H_k(X) \arrow{r}{\epsilon_\star} & H_k(P) \arrow{r}{}    & 0
    \end{tikzcd}
  \]
  Per $ k \geq 1 $ $ \ker{\epsilon_X} = H_k(X) $, in quanto per gli assiomi $ H_k(P) \cong 0 $ se $ k \geq 1 $, quindi
  la successione si riduce a:
  \[
    \begin{tikzcd}
      0 \rar & \ker{\epsilon_X} \rar & H_k(X) \rar & 0
    \end{tikzcd}
  \]
  Mentre per $ k = 0 $ ho che $ H_0(X) \cong \tilde{H}_0(X) \oplus \mathcal{G} $,
  quindi:
  \[
    \tilde{H}_k(X) =
    \begin{cases}
      H_k(X)               & \text{per } k \geq 1 \\
      \tilde{H}_0(X) \oplus \mathcal{G} & \text{per } k = 0 \\
    \end{cases}
  \]
  Questo lo posso fare anche nel caso di una coppia.
  \[
    \begin{tikzcd}
      0 \arrow{r}{}        & \ker{F_\star} \arrow{r}{} & H_k(X, A) \arrow{r}{} & H_k(p, p) \arrow{r}{} & 0
    \end{tikzcd}
  \]
  E si definisce $ \tilde{H}_k(X,A) = \ker{F_\star} $.
  Calcolo $ H_k(P,P) $ con $ P $ spazio formato da un solo punto in $ X $.
  So che c'è una successione esatta lunga per gli assiomi:
  \[
    \begin{tikzcd}
      \dots \arrow{r}{}        & H_k(P) \arrow{r}{}    & H_k(P) \arrow{r}{}    & H_k(P,P) \arrow{r}{}  & H_{k-1}(P) \arrow{r}{} & \dots
    \end{tikzcd}
  \]
  Cioè ho posto $ X = P $ e $ A = P $. Ma io conosco l'omologia di un punto, che è nulla
  per $ k \geq 1 $ mentre vale il gruppo dei coefficienti per $ k = 0 $.
  Supponendo $ k \geq 2 $ la successione diventa:
  \[
    \begin{tikzcd}
       0 \arrow{r}{} & H_k(P,P) \arrow{r}{}  & 0
    \end{tikzcd}
  \]
  E quindi $ H_k(P, P) = 0 $. Mentre se $ k = 1 $ allora:
  \[
    \begin{tikzcd}[nodes={column sep=10pt}]
      \dots \arrow{r}{}        & H_1(P) \arrow{r}{}    & H_1(P) \arrow{r}{}    & H_1(P,P) \arrow{r}{}  & H_{0}(P) \arrow{r}{}   &
      H_{0}(P) \arrow{r}{} & H_{0}(P,P) \rar       & 0
    \end{tikzcd}
  \]
  Cioè siccome $ H_1(P) \cong 0 $:
  \[
    \begin{tikzcd}
       0 \arrow{r}{}         & H_1(P,P) \arrow{r}{i}  & H_{0}(P) \arrow{r}{j}   & H_{0}(P) \arrow{r}{k} & H_{0}(P,P) \rar       & 0
    \end{tikzcd}
  \]
  Ma quindi ho $ H_0(P) \to H_0(P) $ che sarebbe $ H_0(A) \to H_0(X) $ e quindi la mappa che li collega è quella
  indotta dall'inclusione, che per $ X = A = P $ e l'indentità, ma per la funtorialità viene mandata nell'indentità,
  quindi $ j $ è isomorfismo. Per l'esattezza della successione $ \ker{j} = \im{i} $, quindi posso
  riscrivere la prima parte della successione come:
  \[
    \begin{tikzcd}
      0 \rar & H_1(P,P) \rar & 0
    \end{tikzcd}
  \]
  Da cui $ H_1(P,P) = 0 $. Similmente $ \ker{k} = \im{j} = H_0(P) $ quindi $ H_0(P,P) = 0 $ perché
  $ H_0(P,P) \cong \quot{H_0(P)}{\ker{k}} \cong \quot{H_0(P)}{H_0(P)} = 0 $.
  % \[
  %   \begin{tikzcd}[nodes={column sep=10pt, row sep=5pt}]
  %      {} & {} & 0  \arrow{dr}{} & {} & {}  & 0 \arrow{dr}{} & {} & {}  & {}    \\
  %      0 \rar & H_1(P,P) \arrow{rr}{}  \arrow{ur}{}  & {}  & H_{0}(P) \rar &  H_{0}(P) \arrow{rr}{}  \arrow{ur}{}
  %     & {} & H_{0}(P,P) \rar & 0
  %   \end{tikzcd}
  % \]
  % E quindi $ H_1(P,P) = 0 $ e $ H_0(P,P) = 0 $.
\end{proof}

\begin{corollary}
  Se $ X $ è uno spazio topologico contraibile allora $ \tilde{H}_k(X) = 0 $.
\end{corollary}
\begin{proof}
  Se $ X $ è contraibile allora $ X \sim_H P $ cioè $ \exists f \colon X \to P $ e $ \exists g \colon P \to X $
  continue tali che $ f \circ g \sim_H \Id{P} $ e $ g \circ f \sim_H \Id{X} $, quindi
  per la funtorialità e l'assioma dell'omotopia vale che passando a livello
  di omologia:
  $ f_\star \circ g_\star = \Id{H_k(P)} $ e  $ g_\star \circ f_\star = \Id{H_k(X)} $
  % con $ f_\star \colon H_k(X) \to H_k(P) $  e  $ g_\star \colon H_k(P) \to H_k(X) $,
  quindi $ f_\star $ e $ g_\star $ sono inversi l'una dell'altra, ma sempre per la funtorialità:
  $ (f \circ g)_\star = \left( \Id{P} \right)_\star $ e $ (g \circ f)_\star = \left( \Id{X} \right)_\star $.

  \[
   H_k(X) \cong H_k(P) \cong
    \begin{cases}
      \tilde{H}_k(X) = 0 & \text{se } k \geq 1 \\
      \tilde{H}_0(X) \oplus \mathcal{G} = \mathcal{G} & \text{se } k = 0
    \end{cases}
  \]
  Ma quindi $ \mathcal{G} = H_0(P) = \tilde{H}_0(X) \oplus \mathcal{G} $ e quindi $ \tilde{H}_0(X) = 0 $.
\end{proof}

Un secondo importantissimo corollario è l'omologia delle sfere.

\newmathsymb{spheres}{\Sph{n}}{$ n $-sfera}
\newmathsymb{disks}{\Disk{n}}{$ n $-disco}
\newmathsymb{updisk}{\Disk{n}_+}{Calotta superiore dell'$ n $-disco}
\section{Omologia delle sfere}
\begin{theorem}[Omologia di dischi e sfere]
  Siano per $ n \geq 1$:
  \begin{gather*}
    \Sph{n} = \set{\vec{x} \in \RN{n+1} | ||\vec{x}||^2 = 1} \\
    \Disk{n} = \set{\vec{x} \in \RN{n} | ||\vec{x}||^2 \leq 1} \\
    \Disk{n}_+ = \set{\vec{x} \in \RN{n+1} | ||\vec{x}||^2 \leq 1, x_n \geq 0}
  \end{gather*}
  Allora in una qualsiasi teoria omologica avente $ \mathcal{G} $ come
  gruppo dei coefficienti:
  \begin{align*}
    \tilde{H}_k(\Sph{n}) & \cong
    \begin{cases}
      \mathcal{G} & \text{se } k = n \\
      0 & \text{se } k \not = n
    \end{cases} \\
    H_k(\Disk{n}, \Sph{n-1}) & \cong
    \begin{cases}
      \mathcal{G} & \text{se } k = n \\
      0 & \text{se } k \not = n
    \end{cases} \\
    H_k(\Sph{n}, \Disk{n}_+) & \cong
    \begin{cases}
      \mathcal{G} & \text{se } k = n \\
      0 & \text{se } k \not = n
    \end{cases}
  \end{align*}
  Quindi $ \tilde{H}_k(\Sph{n}) \cong H_k(\Disk{n}, \Sph{n-1}) \cong H_k(\Sph{n}, \Disk{n}_+) $.
\end{theorem}

\begin{proof}
  Comincio calcolando $ H_k(\Sph{0}, \Disk{0}_+) $.
  Ho $ \Sph{0} = \set{-1, +1} $ e $ \Disk{0} = \set{+ 1} $.
  Siccome $ \Disk{0} \subseteq \Sph{0} $ per l'assioma dell'esattezza esiste
  una successione esatta in omologia:
  \[
    \begin{tikzcd}
      \dots \rar & H_k(\Disk{0}) \rar & H_k(\Sph{0}) \rar & H_k(\Sph{0}, \Disk{0}) \rar & H_{k-1}(\Disk{0}) \rar & \dots
    \end{tikzcd}
  \]
  Per $ k \geq 2 $ $ H_k(\Disk{0}) = H_{k-1}(\Disk{0}) \cong 0 $ perché $ \Disk{0} $ è un punto, quindi
  la successione diventa:
  \[
    \begin{tikzcd}
      0 \rar & H_k(\Sph{0}) \rar{i} & H_k(\Sph{0}, \Disk{0}) \rar{j} & 0
    \end{tikzcd}
  \]
  Ma per l'assione di additività, siccome $ \Sph{0} $ è la somma di due punti
  $ H_k(\Sph{0}) \cong 0 $, siccome $ i $ è iniettiva perché la successione è esatta
  ed è suriettiva perché essendo la successione esatta $ \im{i} = \ker{j} =  H_k(\Sph{0}, \Disk{0}) $
  allora è isomorfismo quindi anche $ H_k(\Sph{0}, \Disk{0}) \cong 0 $.
  Per calcolare i casi $ k = 1 $ e $ k = 0 $
  considero la successione esatta:
  \[
    \begin{tikzcd}[nodes={column sep=10pt, inner sep=4pt}]
      \dots \rar & H_1(\Disk{0}) \rar & H_1(\Sph{0}) \rar & H_1(\Sph{0}, \Disk{0}) \rar
      & H_0(\Disk{0}) \rar & H_{0}(\Sph{0}) \rar & H_0(\Sph{0}, \Disk{0}) \rar & 0
    \end{tikzcd}
  \]
  Cioè siccome l'omologia di un punto è nulla per $ k \not = 0 $:
  \[
    \begin{tikzcd}
       0 \rar & H_1(\Disk{0}, \Sph{0}) \rar{i}
      & H_0(\Disk{0}) \rar{j} & H_{0}(\Sph{0}) \rar & H_0(\Sph{0}, \Disk{0}) \rar & 0
    \end{tikzcd}
  \]
  Siccome $ \Disk{0} \incl \Sph{0} $ in quanto $ \set{+1} \incl \set{-1, +1} $ è inieittiva
  a livello di omologia per l'assioma di addittività $ j \colon H_0(\set{+1}) \to H_0(\set{-1}) \oplus H_0(\set{+1}) $
  è iniettiva, quindi $ \ker{j} = \im{i} = 0 $ quindi posso riscrivere la prima
  parte della successione come:
  \[
    \begin{tikzcd}
      0 \rar & H_1(\Sph{0},\Disk{0}) \rar & 0
    \end{tikzcd}
  \]
  Da cui $ H_1(\Sph{0},\Disk{0}) = 0 $ per lo stesso ragionamento di prima.
  Infine per definizione $ H_0(\Disk{0}) = \mathcal{G} $ e
  per l'additività $ H_0(\Sph{0}) = \mathcal{G} \oplus \mathcal{G} $ quindi
  $ H_0(\Sph{0}, \Disk{0}) \cong \quot{\mathcal{G} \oplus \mathcal{G}}{\mathcal{G}} \cong \mathcal{G} $.
  In conclusione:
  \[
    H_k(\Sph{0}, \Disk{0}) \cong
    \begin{cases}
      \mathcal{G} & \text{se } k = 0 \\
      0 & \text{se } k \not =  0
    \end{cases}
  \]
  Mostro che $ \tilde{H}_k(\Sph{n}) \cong H_k(\Disk{n}, \Sph{n-1}) $.
  % \[
  %   \tilde{H}_k(\Sph{n}) \cong
  %   \begin{cases}
  %     \mathcal{G} & \text{se } k = n \\
  %     0 & \text{se } k \not = n
  %   \end{cases}
  % \]
  Ho che $ \Sph{n-1} $ è il bordo di $ \Disk{n} $ quindi c'è una mappa naturale di inclusione
  e ponendo $ X = \Disk{n} $ e $ A = \Sph{n-1} $ è ben definita la successione esatta lunga:
  \[
    \begin{tikzcd}[nodes={column sep=4.5pt, inner sep=0.5pt}]
      \dots \rar & H_k(\Sph{n-1}) \rar & H_k(\Disk{n}) \rar & H_k(\Disk{n}, \Sph{n-1}) \rar
      & H_{k-1}(\Sph{n-1}) \rar & H_{k-1}(\Disk{n}) \rar & H_{k-1}(\Disk{n}, \Sph{n-1}) \rar & \dots
    \end{tikzcd}
  \]
  Per $ k \geq 1 $ $ H_k(\Disk{n}) = 0 $ perché $ \Disk{n} $ è contraibile, quindi ho la successione:
  \[
    \begin{tikzcd}[nodes={column sep=10pt}]
      0 \rar & H_k(\Disk{n}, \Sph{n-1}) \rar & H_{k-1}(\Sph{n-1}) \rar & H_{k-1}(\Disk{n}) \rar & H_{k-1}(\Disk{n}, \Sph{n-1}) \rar & \dots
    \end{tikzcd}
  \]
  Se $ k \geq 2 $ la successione si riduce a:
  \[
    \begin{tikzcd}[nodes={column sep=10pt}]
      0 \rar & H_k(\Disk{n}, \Sph{n-1}) \rar{i} & H_{k-1}(\Sph{n-1}) \rar & 0
    \end{tikzcd}
  \]
  Quindi $ i $ è inieittiva e suriettiva e perciò
  $ H_k(\Disk{n}, \Sph{n-1}) \cong H_{k-1}(\Sph{n-1}) \cong \tilde{H}_{k-1}(\Sph{n-1}) $.
  Per $ k = 1 $ ho la successione:
  \[
    \begin{tikzcd}[nodes={column sep=10pt}]
      0 \rar & H_1(\Disk{n}, \Sph{n-1}) \rar & H_0(\Sph{n-1}) \rar & H_0(\Disk{n}) \rar & H_0(\Disk{n}, \Sph{n-1}) \rar & 0
    \end{tikzcd}
  \]
  Quindi $ H_1(\Disk{n}, \Sph{n-1}) = 0 $ per i soliti motivi.
  Ma esiste $ i \colon \Sph{n-1} \incl \Disk{n} $ iniettiva, quindi esiste $ i_\star \colon H_0(\Sph{n-1}) \to H_0(\Disk{n}) $.
  Ma $ \Disk{n} $ è contraibile quindi posso prendere come generatore un punto di $ \Disk{n} $, e ne prendo
  uno sul bordo, cioè in $ \Sph{n-1} $, quindi:
  \begin{align*}
    i_\star \colon H_0(\Sph{n-1}) & \to H_0(\Disk{n}) \\
    \llbracket p \rrbracket & \mapsto \llbracket p \rrbracket
  \end{align*}
  Quindi $ H_0(\Disk{n}, \Sph{n-1}) = 0 $ in quanto $ H_0(\Sph{n-1}) \to H_0(\Disk{n}) $ è iniettiva
  e suriettiva e $ H_0(\Disk{n}, \Sph{n-1}) \cong \quot{H_0(\Disk{n})}{H_0(\Sph{n-1})} \cong 0 $.
  In conclusione ho trovato che $ H_k(\Disk{n}, \Sph{n-1}) = 0 $ per $ k \in \set{0,1} $.
  Mi rimane da verificare l'ultimo, osservo intanto che $ \Disk{n}_+ \simeq \Disk{n} $,
  quindi in quello che segue sostanzialmente ometto il $ + $.

  Considero la successione esatta ($ \Disk{n} \subseteq \Sph{n} $):
  \[
    \begin{tikzcd}
      \dots \rar & H_k(\Disk{n}) \rar & H_k(\Sph{n}) \rar & H_k(\Sph{n}, \Disk{n}) \rar & H_{k-1}(\Disk{n}) \rar & \dots
    \end{tikzcd}
  \]
  Per $ k \geq 2 $ ho che $ H_k(\Disk{n}) \cong 0 $ e che $ H_{k-1}(\Disk{n}) \cong 0 $ quindi
  la successione diventa:
  \[
    \begin{tikzcd}
      0 \rar & H_k(\Sph{n}) \rar & H_k(\Sph{n}, \Disk{n}) \rar & 0
    \end{tikzcd}
  \]
  Quindi $ H_k(\Sph{n}, \Disk{n}) \cong H_k(\Sph{n}) \cong \tilde{H}_k(\Sph{n}) $ per $ k \geq 2 $.
  Per $ k = 1 $ la successione è:
  \[
    \begin{tikzcd}[nodes={column sep=10pt}]
      0 \rar & H_1(\Sph{n}) \rar & H_1(\Sph{n}, \Disk{n}) \rar & H_0(\Disk{n}) \rar & H_0(\Sph{n}) \rar & H_0(\Sph{n}, \Disk{n}) \rar & 0
    \end{tikzcd}
  \]
  Ma $ H_0(X) $ conta le componenti connesse per archi di $ X $ quindi $ H_0(\Disk{n}) \cong H_0(\Sph{n}) $
  e per lo stesso motivo di prima $ H_0(\Sph{n}, \Disk{n}) \cong 0$
  \[
    H_k(\Sph{n}, \Disk{n}) =
    \begin{cases}
      H_k(\Sph{n}) & \text{se } k \geq 1 \\
      0 & \text{se } k = 0
    \end{cases}
  \]
  Rimane da vedere come si comportano i gruppi di omologia $ \tilde{H}_k(\Sph{n}) $ con $ k \geq 1 $.
  Per $ n = 0 $ è noto perché sono $ \Sph{0} $ sono due punti, per $ k = 0 $ anche perché sono connessi
  per archi, infine so che: $ H_k(\Sph{n}) \cong \tilde{H}_k(\Sph{n}) $ per $ k \geq 1 $, ma anche
  che $ H_k(\Sph{n}, \Disk{n}) \cong H_k(\Sph{n}) $, se mostro che $ H_p(\Sph{n}, \Disk{n}) \cong H_p(\Disk{n}, \Sph{n-1}) $
  allora $ H_k(\Sph{n}, \Disk{n}) \cong H_k(\Disk{n}, \Sph{n-1}) $.
  Ma ho mostrato che $ H_k(\Disk{n}, \Sph{n-1}) \cong H_{k-1}(\Sph{n-1}) $, quindi posso procedere
  per induzione:
  \[
    \tilde{H}_k(\Sph{n}) \cong H_k(\Sph{n}) \cong H_k(\Sph{n}, \Disk{n}) \cong H_k(\Disk{n}, \Sph{n-1}) \cong H_{k-1}(\Sph{n-1}) \cong \dots
  \]
  Per far vedere che $ H_p(\Sph{n}) \cong H_p(\Disk{n}, \Sph{n-1}) $ uso l'escissione:
  considero $ U $ intorno opportuno del polo nord di $ \Sph{n} $, per l'escissione:
  \[
    H_p(\Sph{n}, \Disk{n}) \cong H_p(\Sph{n} \setminus U, \Disk{n} \setminus U )
  \]
  Per l'equivalenza omotopica $  H_p(\Sph{n} \setminus U, \Disk{n} \setminus U ) \cong H_p(\Disk{n}, \Sph{n-1}) $, in pratica
  deformo il buco facendolo retrarre.
\end{proof}

% lezione 7

%  _      _____ ________ ___  _   _ _____   _____
% |  |   | ____|__  /_ _/ _ \| \ | | ____| |___  |
% |  |   |  _|   / / | | | | |  \| |  _|      / /
% |  |___| |___ / /_ | | |_| | |\  | |___    / /
% | _____|_____/____|___\___/|_| \_|_____|  /_/
\begin{corollary}
  Se il gruppo dei coefficienti è $ \Z $:
  \[
    H_k(\Sph{n}) \cong
    \begin{cases}
      \Z & \text{se } k \in \set{0, n} \\
      0 & \text{se } k \not \in \set{0, n}
    \end{cases}
  \]
\end{corollary}
Questo risultato ha numerose conseguenze, infatti ho trovato uno
strumento più fine del gruppo fondamentale che riesce a distinguere
spazi diversi.

\begin{corollary}
  $ \Sph{n} \simeq \Sph{m} $ se e solo se $ n = m $.
\end{corollary}
\begin{proof}
  Se $ n = m $ vale che $ \Sph{n} = \Sph{m} $ quindi in particolare
  $ \Sph{n} \simeq \Sph{m} $ con la mappa identità. Assumo $ n \not = m $
  e senza perdita di generalità pongo $ n > m $.

  Per assurdo $ \Sph{n} \simeq \Sph{m} $, quindi esiste un omomorfismo
  $ F: \Sph{n} \homoto \Sph{m} $, quindi esiste anche l'omomorfismo
  inverso $ G: \Sph{m} \homoto \Sph{n} $.
  Quindi esistono anche:
  \[
    F_\star: H_k(\Sph{n}) \to H_k(\Sph{m}) \quad \text{e} \quad G_\star: H_k(\Sph{m}) \to H_k(\Sph{n})
  \]
  Ma $ F \circ G = \Id{\Sph{n}} $ e $ G \circ F = \Id{\Sph{m}} $ perché sono omeomorfismi,
  ma utilizzando la funtorialità si trova quindi che:
  \[
    F_\star \circ G_\star = \Id{H_k(\Sph{m})} \quad \text{e} \quad G_\star \circ F_\star = \Id{H_k(\Sph{n})}
  \]
  Da cui si deduce che $ F_\star $ e $ G_\star $ sono continue e sono inverse l'una
  dell'altra.
  Vale quindi che:
  \[
    H_k(\Sph{n}) \cong H_k(\Sph{m}) \; \forall k \geq 0
  \]
  Se vale per ogni $ k $ in particolare vale per $ k = n $, cioè:
  \[
    H_n(\Sph{n}) = H_n(\Sph{m})
  \]
  Ma $ H_n(\Sph{n}) \cong \Z $ e $ H_n(\Sph{m}) \cong 0 $ da cui $ \Z \cong 0 $, che è assurdo.
\end{proof}

\begin{corollary}[Invarianza topologica della dimensione]
  $ \RN{n} \simeq \RN{m} $ se e solo se $ n = m $.
\end{corollary}
Come si è visto non si riesce a dimostrare questo corollario
utilizzano solo il gruppo fondamentale.
\begin{proof}
  Per assurdo esiste un omomorfismo $ f \colon \RN{n} \homoto \RN{m} $ con $ n > m > 2 $.
  Con i vincolo imposti su $ m $ e $ n $ gli spazi sono contraibili, quindi il gruppo
  fondamentale è in entrambi i casi banale. Togliendo un punto $ p \in \RN{n} $ e
  $ f(p) \in \RN{m} $, e restringendo $ f $ in modo da ottenere l'omomorfismo
  $ f' \colon \RN{n} \setminus \set{p} \homoto \RN{m} \setminus \set{f(p)} $.
  Si sa inoltre che per $ s \geq 2 $ vale che $ \RN{s} \setminus \set{q} \simeq \Sph{s-1} \times \RN{} $,
  infatti è sufficiente mandare a $ 0 $ il punto $ q $ con una traslazione
  (che è certamente un omomorfismo) e quindi si ha:
  \[
    \begin{aligned}[t]
      \RN{k} \setminus \set{q} & \to  \Sph{k-1} \times {\RN{}}^+ & \simeq \Sph{k-1} \times \RN{} \\
      \vec{x} & \mapsto \left( \vec{x}, \frac{\vec{x}}{|| \vec{x} ||} \right) &
    \end{aligned}
  \]
  Quindi:
  \[
    \RN{n} \setminus \set{p} \simeq \RN{m} \setminus \set{f(p)} \iff \Sph{n-1} \times \RN{} \simeq \Sph{m-1} \times \RN{}
  \]
  Si ha la tentazione di eliminare $ \RN{} $ dalla precedente relazione, ma
  questo non si può fare come mostrano alcuni casi molto patologici.
  Tuttavia è possibile passare alla omotopia sapendo che $ \Sph{k} \times \RN{} \sim \Sph{k} $,
  da cui $ \Sph{n-1} \sim \Sph{m-1} $. Ma l'omologia è invariante omotopico, cioè
  $ H_k(\Sph{n-1}) \cong H_k(\Sph{m-1}) $, utilizzando il trucco di prima scelgo $ k = n-1 $
  e quindi:
  \[
    H_{n-1}(\Sph{n-1}) \cong H_{n-1}(\Sph{m-1}) \iff \Z \cong 0
  \]
  Che è assurdo.
\end{proof}

\begin{corollary}
  $ \Sph{n-1} $ non è un retratto di deformazione di $ \Disk{n} $ per $ n \geq 2 $
\end{corollary}
\begin{proof}
  Si ricorda che:
  \[
    \Disk{n} = \set{ \vec{x} \in \RN{n} | || \vec{x} || \leq 1} \quad \Sph{n-1} = \partial \Disk{n} = \set{ \vec{x} \in \RN{n} | || \vec{x} || = 1}
  \]
  Chiaramente esiste $ i \colon \Sph{n-1} \incl \Disk{n} $.
  \begin{definition}
    Uno spazio topologico $ Y $ si dice \textbf{retratto di deformazione}\index{Retratto di deformazione} di un altro
    spazio topologico $ X $ tale che $ Y \incl X $ se esiste una funzione continua $ r\colon X \to Y $ che inverte a meno di omotopia
    la mappa di inclusione $ i\colon Y \to X $, cioè tale che soddisfa:
    \begin{enumerate}
    \item $ r\colon X \to Y $ continua
    \item $ i \circ r \sim \Id{X} $
    \item $ r \circ i = \Id{Y} $
    \end{enumerate}
    Una mappa che soddisfa queste condizioni è detta \textbf{retrazione}\index{Retrazione}.
  \end{definition}
  Suppongo per assurdo che $ \Sph{n-1} $ è un retratto di deformazione di $ \Disk{n} $, cioè che
  esiste una retrazione $ r $. Passando all'omologia:
  \begin{gather*}
    i_\star \colon H_k(\Sph{n-1}) \to H_k(\Disk{n}) \\
    r_\star \colon H_k(\Disk{n}) \to H_k(\Sph{n-1}) \\
    \left( i \circ r \right)_\star = (\Id{\Disk{n}})_\star \text{ e }  \left( r \circ i \right)_\star = (\Id{\Sph{n-1}})_\star
  \end{gather*}
  Quindi:
  \[
    i_\star \circ r_\star = \Id{H_k(\Disk{n})} \text{ e } r_\star \circ i_\star = \Id{H_k(\Sph{n-1})} \; \forall k \in \mathbb{N}
  \]
  In particolare considero $ k = n - 1 $:
  \begin{gather*}
    i_\star \colon H_n-1(\Sph{n-1}) \to H_n-1(\Disk{n}) \\
    r_\star \colon H_n-1(\Disk{n}) \to H_n-1(\Sph{n-1})
  \end{gather*}
  Cioè: $ i_\star \colon \Z \to 0 $. Considero un generatore $ \alpha $ di $ H_{n-1}(\Sph{n-1}) \cong \Z $, cioè tale
  che $ \langle\alpha\rangle = H_{n-1}(\Sph{n-1}) $ allora $ i_\star(\alpha) = 0 $ quindi $ r_\star \circ i_\star = 0 $, ma
  $ \left( r \circ i \right)_\star = \Id{\Sph{n-1}_\star} $ quindi significherebbe $ \Id{\Sph{n-1}_\star}(\alpha) = 0 $,
  cioè che $ \alpha = 0 $, che è assurdo perché $ \Z \not = \langle0\rangle $.
\end{proof}

\begin{theorem}[Teorema del punto fisso di Brouwer\index{Teorema del punto fisso}]
  Ogni funzione continua $ g \colon \Disk{n} \to \Disk{n} $ con $ n \geq 2 $ ammette almeno un punto fisso
  in $ \Disk{n} $, cioè:
  \[
    \exists \vec{x_o} \in \Disk{n} \; | \; g(\vec{x_0}) = \vec{x_0}
  \]
\end{theorem}

\begin{proof}
  Per assurdo $ g $ non ammette punto fisso cioè esisto $ \vec{x} \in \Disk{n} $
  tale che $ g(\vec{x}) \not = \vec{x} $. Sicuramente tuttavia $ g(\vec{x}) \in \Disk{n} $.
  Considero la retta $ l $ passante per $ \vec{x} $ e $ g(\vec{x}) $. Questa retta
  interseca il bordo di $ \Disk{n} $ in due punti $ \set{p_1, p_2} $:
  \[
    l \cap \partial \Disk{n} = l \cap \Sph{n-1} = \set{p_1, p_2}
  \]
  Definisco la mappa $ r \colon \Disk{n} \to \partial \Disk{n} = \Sph{n-1} $ tale che associ
  ad ogni punto del disco il punto di intersezione della retta $ l_{\vec{x}} $ che gli sta più
  vicino (infatti in $ \RN{n} $ è ben definita una nozione di distanza). La retta $ l_{\vec{x}} $
  è ben definita in quanto per due punti distinti (e per ipotesi  $ g(\vec{x}) \not = \vec{x} $)
  passa una e una sola retta.
  \begin{figure}[htbp]
    \centering
    \begin{tikzpicture}
      \draw (0,0) circle (2);
      \draw[-Latex] (-2.5, 0) -- (2.5,0);
      \draw[-Latex] (0, -2.5) -- (0, 2.5);
      \draw (-2, -2.33) -- (2, 3);
      \node[below, right] () at (0.5, 1) {$ g(\vec{x}) $};
      \node[] () at (0.5, 1) {\textbullet};
      \node[above] () at (-1, -1) {$ \vec{x} $};
      \node[] () at (-1, -1) {\textbullet};
      \node[left] () at (2, 3) {$ l $};
      \node[below, left] () at (-1.35, -1.47) {$ p_1 $};
      \node[] () at  (-1.35, -1.47) {\textbullet};
      \node[above, right] () at (1.04, 1.71) {$ p_2 $};
      \node[] () at  (1.04, 1.70) {\textbullet};
    \end{tikzpicture}
    \caption{Schema per $ n = 2 $}
    \label{fig:lez7:brouwer_proof_1}
  \end{figure}
  \begin{exercise}
    Dimostrare che $ r $ è continua.
  \end{exercise}
  Ho una mappa di inclusione naturale:
  \begin{align*}
      i \colon \Sph{n-1} & \to \Disk{n} \\
      \vec{x} & \mapsto \vec{x}
  \end{align*}
  Se dimostro che $ r $ è una retrazione trovo un assurdo per il corollario
  precedentemente dimostrato.
  Devo verificare $ r \circ i = \Id{\Sph{n-1}} $ e $ i \circ r \sim \Id{\Disk{n}} $.
  La prima uguaglianza è certamente vera perché se $ \vec{x} \in \partial \Disk{n} $
  allora l'intersezione del bordo del disco che gli sta più vicina corrisponde a
  $ \vec{x} $ stesso.
  Costruisco esplicitamente una relazione di omotopia per mostrare la seconda:
  Siccome $ \Disk{n} $ è convesso è ben definita $ G(t, \vec{x}) = (1-t)\vec{x}
  + t r(\vec{x}) $ con $ t \in [0,1] $. Questa è una buona omotopia in quanto $ \forall t, \vec{x} $:
  \begin{itemize}
  \item $ G $ è continua
  \item $ G(t, \vec{x}) \in \Disk{n} $
  \item $ G(0, \vec{X}) = \vec{x} $
  \item $ G(1, \vec{X}) = r(\vec{x}) $
  \end{itemize}
  Quindi $ r $ è retrazione ma questo è assurdo.
\end{proof}


\subsection{Teoria del grado}
\begin{definition}
  Ad ogni applicazione continua $ \phi \colon \Sph{n} \to \Sph{n} $ continua è possibile
  associare in modo univoco un numero intero, questo è il \textbf{grado}\index{Grado di una sfera}:
  \begin{align*}
    \phi_\star \colon H_n(\Sph{n}) & \to H_n(\Sph{n}) \\
    \alpha & \mapsto  \deg{(\phi)} \alpha
  \end{align*}
  con $ \alpha $ generatore.
\end{definition}

Si ha che $ H_n(\Sph{n}) \cong \Z $, quindi $ H_n(\Sph{n}) $ è il gruppo
libero di rango 1 generato da un singolo $ n $-ciclo che non è un bordo,
cioè esiste una mappa $ f \colon \Z \to H_n(\Sph{n}) $ tale che $ f(1) = \alpha $,
$ \alpha $ generatore, in questo modo $ H_n(\Sph{n}) = \langle\alpha\rangle $.
Considero $ \phi \colon \Sph{n} \to \Sph{n} $ continua con $ n \geq 1 $, questa induce
$ \phi_\star \colon H_n(\Sph{n}) \to H_n(\Sph{n}) $.
% Per $ n = 0 $ $ \phi_\star $ manda punti in punti, per $ n \geq 1 $:
L'azione di $ \phi_\star $ si calcola facilmente, infatti sia $ c \in H_n(\Sph{n}) $
allora $ c = p \alpha $ con $ p \in \Z $, quindi:
\[
  \phi_\star (c) = \phi_\star (p \alpha) = \phi_\star (\underbrace{\alpha + \alpha + \alpha + \dots}_{\text{|p| volte}}) =
  \underbrace{\phi_\star (\alpha) + \phi_\star (\alpha) + \dots}_{\text{|p| volte}} = p \phi_\star(\alpha)
\]
Ma $ \phi_\star(\alpha) \in H_n(\Sph{n}) $ quindi si deve poter scrivere come multiplo di $ \alpha $:
$ \phi_\star (\alpha) = d \alpha $ da cui: $ \phi_\star (c) = p d \alpha = d c $ con $ d \in \Z $.

\begin{osservation}
  Questo numero $ d $ viene fuori dall'immagine di un generatore, ma non dipende dalla
  scelta del generatore, infatti:
\end{osservation}
\begin{proof}
  Sia $ \beta $ un altro generatore, siccome $ \alpha $ è un generatore si può scrivere $ \beta = m \alpha $
  con $ m \in \Z $. Pongo come notazione:
  \[
    \phi_\star(\beta) = d(\beta) \beta \quad \phi_\star(\alpha) = d(\alpha) \alpha
  \]
  Allora:
  \[
    d(\beta) \beta = \phi_\star (\beta) = m \phi_\star (\alpha) = m d(\alpha) \alpha = d(\alpha) \beta
  \]
  Da cui $ d(\beta) \beta = \beta d(\alpha) $ cioè $ \left(d(\beta) - d(\alpha) \right) \beta = 0 $, siccome
  questo vale per ogni $ \alpha $ e $ \beta $ allora $ d(\alpha) = d(\beta) $.
\end{proof}

\begin{example}[$ n = 1 $]
  Ad esempio per $ n = 1 $ e $ p \in \mathbb{N} $ e la mappa
  \begin{align*}
    \phi \colon  \Sph{1} & \to  \Sph{1} \\
    z   & \mapsto  z^p
  \end{align*}
  Vale che $ \deg{(\phi)} = p $, infatti prendo un generatore di $ \Sph{1} $:
  \begin{align*}
    \sigma \colon \Delta_1 & \to \Sph{1} \\
    t & \mapsto \me^{2 \pi i t}
  \end{align*}
  Applicando la mappa:
  \begin{align*}
    \phi \circ \sigma \colon \Delta_1 & \to \Sph{1} \\
    t & \mapsto \me^{2 \pi i p t}
  \end{align*}
  Cioè $ \phi \circ \sigma = \sigma \star \sigma \star \dots = p \sigma $ volte, e quindi $ \deg{(\phi)} = p $.
\end{example}

\begin{proposition}
  Siano $ f,g \colon \Sph{n} \to \Sph{n} $ mappe continue, allora $ \deg{(g \circ f)} = \deg{(f)} \deg{(g)} $.
\end{proposition}
\begin{proof}
  Per la funtorialità $ (g \circ f)_\star = g_\star \circ f_\star $ quindi:
  \[
    (g \circ f)_\star (\alpha) = (g_\star \circ f_\star)(\alpha) \; \Rightarrow \; g_\star (f_\star (\alpha)) = g_\star(\deg{(f)}\alpha) = \deg{(f)} g_\star(\alpha) = \deg{(f)}\deg{(g)}\alpha
  \]
  Quindi:
  \[
    \deg{(f)}\deg{(g)}\alpha = (g \circ f)_\star (\alpha) = \deg{(g \circ f)} \alpha
  \]
  Siccome $ \alpha $ è generatore: $ \deg{(g \circ f)} = \deg{(f)} \deg{(g)} $.
\end{proof}

Voglio usare la teoria del grado per un'applcazione del teorema della palla pelosa.
\begin{proposition}
  Considero riflessione rispetto al sottospazio $ x_{n+1} = 0 $ in $ \RN{n+1} $
  \begin{align*}
    \rho \colon \Sph{n} & \to \Sph{n} \\
    (x_1, \dots, x_{n+1}) & \mapsto (x_1, \dots, - x_{n+1})
  \end{align*}
  Il grado di questa applicazione è $ - 1 $.
\end{proposition}
\begin{proof}
  La dimostrazione è per induzione.
  Per $ n = 1 $.
  \begin{align*}
    \rho \colon \Sph{1} & \to \Sph{1} \\
    (x_0,x_1) & \mapsto (x_0, -x_1)
  \end{align*}
  Considero il generatore $ \sigma $:
  \begin{align*}
    \sigma \colon \Delta_1 & \to \Sph{1} \\
    t & \mapsto \left(\cos(2 \pi t), \sin(2 \pi t)\right)
  \end{align*}
  Quindi:
  \begin{align*}
    \rho \circ \sigma \colon \Delta_1 & \to \Sph{1} \\
    t & \mapsto \left(\cos(2 \pi t), -\sin(2 \pi t))\right)
  \end{align*}
  Ma:
  \[
    \left(\cos(2 \pi t), -\sin(2 \pi t))\right) = \left(\cos(-2 \pi t), \sin(-2 \pi t))\right) = \left(\cos(2 \pi (1-t)), \sin(2 \pi (1-t)))\right)
  \]
  Quindi $ \rho \circ \sigma = \bar{\sigma} = - \sigma $ e quindi il grado è $ - 1 $.

  Suppongo che il risultato sia vero per $ \Sph{n-1} $ mostro che è vero anche per $ \Sph{n} $.

  % In $ \Sph{n} $ ho dei sottoinsiemi naturali:
  % \begin{gather*}
  %   \Disk{n}_+ = \set{ (x_1, \dots, x_{n+1}) \in \Sph{n} | x_1 \geq 0 } \\
  %   \Disk{n}_- = \set{ (x_1, \dots, x_{n+1}) \in \Sph{n} | x_1 \leq 0 }
  % \end{gather*}
  % Vale che $ \Disk{n}_+ \cap \Disk{n}_- = \set{ (x_1, \dots, x_{n+1}) \in \Sph{n} | x_1 \leq 0 } = \Sph{n-1} $.
  Ho dimostrato che
  \[
    \tilde{H}_p(\Sph{n}) \cong H_p(\Disk{n}, \Sph{n-1}) \cong H_p(\Sph{n}, \Disk{n})
  \]
  Quindi considerando anche che $ \rho $ induce una mappa $ \rho_\star $ a livello di omologia:
  \[
    \begin{tikzcd}
      H_n(\Sph{n}) \rar{\rho_\star} \arrow[leftrightarrow]{d}{\cong} & H_n(\Sph{n})  \arrow[leftrightarrow]{d}{\cong} \\
      H_n(\Disk{n}, \Sph{n-1}) &  H_n(\Disk{n}, \Sph{n-1})
    \end{tikzcd}
  \]
  Ho anche che $ H_n(\Disk{n}, \Sph{n-1}) \cong H_{n-1}(\Sph{n-1}) $, come ho dimostrato
  calcolando l'omologia delle sfere, quindi il diagramma diventa:
  \[
    \begin{tikzcd}
      H_n(\Sph{n}) \rar{\rho_\star} \arrow[leftrightarrow]{d}{\cong} & H_n(\Sph{n})  \arrow[leftrightarrow]{d}{\cong} \\
      H_{n-1}(\Sph{n-1}) \rar{\rho_\star^{(n-1)}} &  H_{n-1}(\Sph{n-1})
    \end{tikzcd}
  \]
  Ma per ipotesi induttiva per $ n - 1 $ il grado è $ - 1 $, quindi anche per $ n $ il grado è $ - 1 $.
\end{proof}
\hfill\newline\newline
% lezione 8

% Ho $ \Sph{n} = \set{ (x_1, \dots, x_{n+1}) \in \RN{n+1} | \sum_{i=1}^{n+1} x_i^2 = 1 } \subset \RN{n+1} $
% spazio topologico con la topologia indotta.

% Ho trovato che:
% \[
%   H_k(\Sph{n}) \cong
%   \begin{cases}
%     \Z & \text{se } k \in \set{0, n} \\
%     0 & \text{se } k \not \in \set{0, n}
%   \end{cases}
% \]
% Ho che $ f \colon \Sph{n} \to \Sph{n} $ induce $ f_\star \colon H_n(\Sph{n}) \to H_n(\Sph{n}) $ e ho definito il grado
% come:

% Prendo $ \alpha $ tale che $ \langle\alpha\rangle = H_n(\Sph{n}) $ con:
% \begin{align*}
%   H_n(\Sph{n}) & \to \Z \\
%   \alpha & \mapsto 1
% \end{align*}
% e $ f_\star (\alpha) = \deg{(f)} \alpha $. So che il grado è un invariante topologico per le sfere.


% Quindi il risultato finale è:
% \begin{lemma}
%   Se $ \rho \colon \Sph{n} \to \Sph{n} $ è la riflessione rispetto a $ \set{x_{n+1} = 0} $ allora
%   $ \deg{(\rho)} = - 1 $.
% \end{lemma}

% Perché voglio fare la riflessione?
Considero l'applicazione antipodale che è quella che scambia di segno tutte le componenti:
\begin{align*}
  A \colon \RN{n} & \to \RN{n} \\
  (x_1, \dots, x_{n}) & \mapsto (-x_, \dots, -x_n)
\end{align*}
Questa è continua e vale che $ A^2 = \Id{\RN{n}}$. Definisco per $ n \geq 2 $ la restrizione della trasformazione
antipodale su $ \Sph{n-1} $: $ a = A \lvert_{\Sph{n-1}} $, vale che $ a \colon \Sph{n-1} \to \Sph{n-1} $, infatti
$ \im{a} = \Sph{n-1} $. Quanto vale $ \deg{(a)} $?
Scrivo $ a $ come composizione di riflessioni:
\[
  a = \rho_n \circ \dots \circ \rho_1
\]
Per il risultato appena dimostrato:
\[
  \deg{(a)} = \deg{(\rho_n \circ \dots \circ \rho_1)} = \deg{(\rho_n)}\deg{(\rho_{n-1})}\dots\deg{(\rho_1)} = (-)^n
\]
Quindi $ \deg{(a)} = (-)^n $ e perciò cambia se $ n $ è pari o dispari.

\begin{corollary}
  La mappa antipodale non è omotopicamente equivalente all'identità su $ \Sph{n} $ su $ n $ è pari.
\end{corollary}
\begin{proof}
  Se le due applicazioni fossero omotope varrebbe che $ a_\star = (\Id{\Sph{n}})_\star $ quindi:
  \[
    \deg{(a)} = \deg{(\Id{\Sph{n}})} = (-)^{n+1} = 1
  \]
  Questo è vero solo se $ n + 1 $ è pari, ma se $ n $ è pari $ n + 1 $ non può esserlo.
\end{proof}

Ciò non dimostra che per $ n $ pari invece le due applicazioni sono omotope. Questa è una
dimostrazione avanzata che richiede i gruppi di omotopia superiori con i quali si dimostra
che se due applicazioni definite su $ \Sph{n} $ hanno lo stesso grado allora sono omotope.

\begin{corollary}
  Sia $ f \colon \Sph{n} \to \Sph{n} $ una mappa continua con $ n $ pari, allora esiste almeno
  un punto $ \vec{x_0} \in \Sph{n} $ tale che $ f(\vec{x_0}) = \pm x_0 $.
\end{corollary}
\begin{proof}
  Per assurdo $ f(\vec{x}) \not = \pm \vec{x} \; \forall \vec{x} \in \Sph{n} $. Sia $ F \colon \Sph{n} \times I \to \Sph{n} $
  con:
  \[
    F(\vec{x}, t) = \frac{t f(\vec{x}) + (1-t)\vec{x}}{|| t f(\vec{x}) + (1-t)\vec{x} ||}
  \]
  $ \forall \vec{x}, t $ vale che $ F(\vec{x}, t) \in \Sph{n} $.
  La norma al denominatore non è mai nulla per ipotesi, infatti $ || t f(\vec{x}) + (1-t) \vec{x} || = 0 $
  significa che $ t f(\vec{x}) = (1-t)\vec{x} $, quindi se $ t = 0 $ allora $ 0 = - \vec{x} $ ma $ \vec{x} = 0 \not \in \Sph{n} $,
  se $ t \not = 0 $ allora $ f(\vec{x}) = \left(\frac{t-1}{t}\right)\vec{x} $, ma $ \vec{x}, f(\vec{x}) \in \Sph{n} $
  quindi $ || f(\vec{x}) || = || \vec{x} || = 1 $ e quindi $ 1 = \big \rvert \frac{t-1}{t}\big \lvert $,
  ma $ t \in (0,1] $, quindi non è possibile trovare $ t $.

  Inoltre $ F(\vec{x}, 0) = \vec{x} $ e $ F(\vec{x}, 1) = f(\vec{x}) $ quindi $ F $ è una relazione di omotopia
  tra $ f $ e l'identità.

  Mostro che $ f $ è anche omotopa all'applicazione antipodale, così per la transitività della
  relazione di omotopia trovo l'assurdo.

  Si definisce  $ G\colon \Sph{n} \times I \to \Sph{n} $:
  \[
    G(\vec{x}, t) = \frac{-t \vec{x} + (1-t)f(\vec{x})}{|| -t \vec{x} + (1-t)f(\vec{x}) ||}
  \]
  Con i medesimi ragionamenti si trova che   $ \forall \vec{x}, t $ vale che $ G(\vec{x}, t) \in \Sph{n} $, e inoltre
  $ G(\vec{x}, 0) = f(\vec{x}) $ e $ G(\vec{x}, 1) = - \vec{x} $ quindi $ G $ realizza
  l'omotopia con l'applicazione antipodale.
\end{proof}


% lezione 14

%  _     _____ ________ ___  _   _ _____   _ _  _
% | |   | ____|__  /_ _/ _ \| \ | | ____| / | || |
% | |   |  _|   / / | | | | |  \| |  _|   | | || |_
% | |___| |___ / /_ | | |_| | |\  | |___  | |__   _|
% |_____|_____/____|___\___/|_| \_|_____| |_|  |_|

\subsection{Escissione e omotopia}

Dimostro che l'omologia singolare soddisfa gli assiomi verificando quelli
che mi mancano che sono l'escissione e l'omotopia.

\begin{theorem}
  Sia $ X $ uno spazio topologico, e $ A, B $ suoi sottospazi topologici
  con la topologia indotta tali che $ B \subseteq A \subseteq X $ e $ \bar{B} \subseteq \mathrm{int}(A) $,
  allora è possibile escindere il sottoinsieme $ B $ da $ A $ e da $ X $, cioè
  l'inclusione $ i \colon (X \setminus B, A \setminus B) \incl (X,A) $ induce un isomorfismo a livello
  di omologia $ i_\star \colon H_k(X \setminus B, A \setminus B) \homoto H_k(X,A) $.
\end{theorem}
\begin{proof}
  Sia $ \mathcal{U} = \set{A, X \setminus B} $ un ricoprimento per $ X $ (infatti
  $ X = A \cup X \setminus B $). Considero le $ k $-catene singolari in $ A $ $ S_k(A) $,
  le $ k $-catene singolari in $ X \setminus B $ $ S_k(X \setminus B) $ e definisco
  $ S_k^\mathcal{U}(X) $ il sottocomplesso generato dai simplessi singolari
  $ \sigma \colon \Delta_k \to X $ tali che $ \sigma $ sia un $ \mathcal{U} $-piccolo.

  \begin{definition}
    Un simplesso singolare in $ X $ si dice
    \textbf{$ \mathcal{U} $-piccolo}\index{$ \mathcal{U} $-piccolo} se esiste
    una suddivisione baricentrica di $ \Delta_k $ tale che abbia immagine in
    $ U_\alpha $, dove $ \mathcal{U} = \bigcup_\alpha U_\alpha $.
  \end{definition}

  \begin{definition}
    La definizione di \textbf{suddivisione baricentrica}\index{Suddivisione
      baricentrica} è piuttosto tecnica e noiosa, ma l'idea fondamentale è
    quella di spezzettare il simplesso a partire dal suo baricentro (che sempre
    univocamente determinabile) in tanti simplessi tanto piccoli da essere
    completamente contenuti negli insiemi del ricoprimento.
  \end{definition}

  Sostanzialmente $ S_k^\mathcal{U}(X) $ sono le catene somme di simplessi
  completamente contenuti in uno degli $ U_\alpha $ che formano il ricoprimento.

  \begin{exercise}
    Dimostrare che $ X = \mathrm{int}(A) \cup \mathrm{int}(X \setminus B) $.
  \end{exercise}
  Sostanzialmente è quindi possibile prendere insiemi aperti per formare il
  ricoprimento $ \mathcal{U} $, infatti se $ A $ e/o $ X \setminus B $ non fossero
  aperti potrei considerare il ricoprimento
  $ \mathcal{U} = \set{ \mathrm{int}(A), \mathrm{int}(X \setminus B)} $.

  La dimostrazione del teorema di escissione si basa ora su alcune assunzioni
  di natura algebrica.

  \begin{osservation}
    Vale che $ S_k^\mathcal{U}(X) = S_k(A) + S_k(X \setminus B) $. L'operazione di somma
    è ben definita perché i complessi sono $ \Z $-moduli. Per mostrare che
    questa asserzione è vera la verifico sui generatori
    $ \sigma \colon \Delta_k \to X $. Per definizione di $ S_k^\mathcal{U}(X) $ se
    $ \sigma \in S_k^\mathcal{U}(X) $ o $ \sigma \in S_k(A) $ oppure
    $ \sigma \in S_k(X\setminus B) $, quindi
    $ S_k^\mathcal{U}(X) \subseteq S_k(A) + S_k(X \setminus B) $. Inoltre se
    $ \sigma_1 \colon \Delta_1 \to A $ e
    $ \sigma_2 \colon \Delta_k \to X \setminus B $ allora
    $ \sigma_1 + \sigma_2 \colon \Delta_k \to X $ e naturalmente
    $ \sigma_1 + \sigma_2 \in S_k^\mathcal{U}(X) $, quindi vale l'inclusione e quindi vale
    l'uguaglianza.
  \end{osservation}

  \begin{osservation}[Terzo teorema degli omomorfismi\index{Terzo teorema degli omomorfismi}]
    Vale che $ S_k(A \setminus B) = S_k(A) \cap S_k(X \setminus B) $, questo è ovvio.
  \end{osservation}

  \begin{figure}[htbp]
    \centering
    \begin{tikzpicture}[scale=0.75]
      \draw plot [smooth cycle] coordinates {(0,0) (2,3) (3,4) (5,2) (2,-1)};
      \draw plot [smooth cycle] coordinates {(1.5,1) (2,2) (3,3) (4,2) (2,0)};
      \draw (2.5,1.5) circle (0.3);
      \node[above] () at (3,4) {$ X $};
      \node[right] () at (4,2) {$ A $};
      \node[] () at (2.5,1.5) {$ B $};
    \end{tikzpicture}
    \caption{Situazione}
    \label{fig:lez14:excision_proof}
  \end{figure}

  \begin{osservation}
    Vale che:
    \[
      \quot{S_k(X \setminus B)}{A \setminus B} \cong \quot{S_k^\mathcal{U}(X)}{S_k(A)}
    \]
    Infatti \dots
  \end{osservation}

  \begin{osservation}
    Vale che:
    \[
      \begin{tikzcd}
        S_k(X \setminus B) \rar \dar & S_k(X) \dar \\
        \quot{S_k(X \setminus B)}{S_k(A \setminus B)} \rar & \quot{S_k(X)}{S_k(A)}
      \end{tikzcd}
    \]
    Quindi usando le osservazioni precedenti è ben definito il diagramma:
    \[
      \begin{tikzcd}
        \quot{S_k(X \setminus B)}{S_k(A \setminus B)} \arrow{rd}{} \arrow{rr}{\cong} & {} & \quot{S_k^\mathcal{U}(X)}{S_K(A)} \arrow{ld}{}\\
        {} & \quot{S_k(X)}{S_k(A)} & {}
      \end{tikzcd}
    \]
    Passando all'omologia della coppia:
    \[
      \begin{tikzcd}
        H_p(X \setminus B, A \setminus B) \arrow{rd}{\phi} \arrow{rr}{\cong} & {} & H_P^\mathcal{U}(X,A) \arrow{ld}{\psi} \\
        {} & H_p(X,A) & {}
      \end{tikzcd}
    \]
    Per dimostrare l'escissione bisogna mostrare che $ \phi $ è un isomorfismo, e questo può essere
    ottenuto mostrando che $ \psi $ è isomorfismo (così sarebbe $ \phi $ composizione di isomorfismi),
    cioè devo dimostrare che $ H_p^\mathcal{U}(X, A) \cong H_p(X,A) $.

    Questo risultato dipende da altre due osservazioni:
    \begin{osservation}
      Se $ X $ è uno spazio topologico e $ \mathcal{U} = \set{U_\alpha} $ un suo ricoprimento di aperti
      allora $ H_p^\mathcal{U}(X) \cong H_p(X) $.
    \end{osservation}
    A questo punto voglio passare all'omologia relativa. Considero $ \mathcal{U} \cap A = \set{U_\alpha \cap A} $,
    questo è un ricoprimento aperto di $ A $. Definisco:
    \[
      S_k^\mathcal{U}(X,A) = \quot{S_k^\mathcal{U}(X)}{S_k^{\mathcal{U} \cap A}(A)}
    \]
    Ho quindi la successione esatta corta:
    \[
      \begin{tikzcd}
        0 \rar & S_k^{\mathcal{U} \cap A}(A) \rar & S_k^{\mathcal{U}}(X) \rar & S_k^{\mathcal{U}}(X,A) \rar & 0
      \end{tikzcd}
    \]
    Ma c'è anche la successione esatta corta:
    \[
      \begin{tikzcd}
        0 \rar & S_k(A) \rar & S_k(X) \rar & S_k(X,A) \rar & 0
      \end{tikzcd}
    \]
    Quindi è ben definita la mappa tra successioni corte:
    \[
      \begin{tikzcd}
        0 \rar & S_k^{\mathcal{U} \cap A}(A) \rar \dar & S_k^{\mathcal{U}}(X) \rar \dar & S_k^{\mathcal{U}}(X,A) \rar \dar & 0 \\
        0 \rar & S_k(A) \rar & S_k(X) \rar & S_k(X,A) \rar & 0
      \end{tikzcd}
    \]
    Queste successioni esatte corte ne inducono una esatta lunga in omologia:
    \[
      \begin{tikzcd}[nodes={column sep = 7 pt}]
        \dots \rar & H_i^{\mathcal{U} \cap A}(A) \rar \dar{\cong} & H_i^{\mathcal{U}}(X) \rar \dar{\cong} & H_i^{\mathcal{U}}(X,A) \rar \dar{?} &  H_{i_1}^{\mathcal{U} \cap A}(A) \rar
        \dar{\cong} & H_{i-1}^{\mathcal{U}}(X) \rar \dar{\cong} & \dots \\
        \dots \rar & H_i(A) \rar & H_i(X) \rar & H_i(X,A) \rar &  H_{i_1}(A) \rar & H_{i-1}(X) \rar & \dots
      \end{tikzcd}
    \]
    In queste condizioni si può applicare il lemma dei cinque con il quale si trova immediatamente
    che  $ H_p^\mathcal{U}(X, A) \cong H_p(X,A) $.

    \begin{lemma}[Lemma dei cinque\index{Lemma dei cinque}]
      Considerato il seguente diagramma formato da successioni esatte corte in cui ogni quadrato è commutativo:
      \[
        \begin{tikzcd}
          A \rar{i} \dar{\alpha} & B \rar{j} \dar{\beta} & C \rar{k} \dar{\gamma} & D \rar{l} \dar{\delta} & E \dar{\epsilon} \\
          A' \rar{i} & B' \rar{j} & C' \rar{k} & D' \rar{l} & E'
        \end{tikzcd}
      \]
      Se $ \alpha, \beta, \delta, \epsilon $ sono isomorfismi allora anche $ \gamma $ lo è.
    \end{lemma}
    \begin{proof}
      [MANCA LA PROVA (HATCHER 129)]
    \end{proof}

  \end{osservation}

\end{proof}

% lezione 14 parte 2

Ora verifico l'assioma di omotopia. Riepilogo alcuni concetti: Siano $ X, Y $
spazi topologici e $ A, B $ sottospazi topologici di $ X $ e $ Y $
rispettivamente, siano $ f_0 \colon (X, A) \to (Y,B) $ e
$ f_1 \colon (X,A) \to (Y,B) $ mappe continue con $ f_0(A) \subseteq B $ e
$ f_1(A) \subseteq B $, $ f_0 $ e $ f_1 $ si dicono omotopicamente equivalenti se esiste
una funzione continua $ F \colon I \times X \to Y $ tale che $ \forall x \in X $
$ F(0,x) = f_0(x) $, $ F(1,x) = f_1(x) $ e $ \forall t \in I $ $ F(t,a) \in B $.

\begin{theorem}
  Siano $ X, Y $ spazi topologici e $ A, B $ sottospazi topologici rispettivamente
  di $ X $ e $ Y $, e $ f_0, f_1 \colon (X, A) \to (Y, B) $ funzioni continue omotope,
  allora le mappe indotte da queste funzioni sui gruppi di omologia coincidono,
  cioè $ (f_0)_\star = (f_1)_\star $, con $ (f_0)_\star, (f_1)_\star \colon H_l(X,A) \to H_k(Y,B) $.
\end{theorem}
\begin{proof}
  L'intervallo $ I $ è omeomorfo al simplesso standard $ \Delta_1 $, introducendo
  $ \epsilon_1 \colon \Delta_0 \to I $ e $ \epsilon_0 \colon \Delta_0 \to I $ definiti da:
  $ \epsilon_i(e_0) = i $ con $ i \in \set{0,1} $, allora il bordo dell'intervallo è
  $ \partial I = \epsilon_1 - \epsilon_0 $. Chiamo $ I $ con abuso di notazione il simplesso che
  manda $ \Delta_1 $ in $ I $, Sia $ c \in S_q(X) $, cioè $ c \in S_q(X) $
  allora $ I \times c \in S_{q+1}(I \times X) $, infatti
  \begin{align*}
    I \times x \colon \Delta_1 \times \Delta_1 & \to I \times X \\
    (t,x) & \mapsto (I(t), c(x))
  \end{align*}
  Il bordo di questa catena è:
  \[
    \partial (I \times c) = \sum_i^{q} (-)^{i+1}(I \times c)^{(i+1)} = (I \times C)^{(0)} - (I \times C)^{(1)} + \dots
  \]
  Osservo che:
  [FIGURA]
  \[
    = \epsilon_1 \times c - \epsilon_0 \times c - I \times \partial c
  \]
  \begin{definition}
    Si definisce l'\textbf{operatore prisma}\index{Operatore prisma} definendo
    la sua azione sui simplessi singolari e poi estendendo per linearità:
    \begin{align*}
      D \colon S_q(X) & \to S_{q+1}(X) \\
      c & \mapsto I \times c
    \end{align*}
  \end{definition}
  \begin{exercise}
    Verificare che l'operatore prisma è un omomorfismo.
  \end{exercise}
  Questo operatore sostanzialmente prende un simplesso e restituisce il prisma
  in figura.
  Per quanto detto sopra vale che:
  \[
    \partial \circ D (c) + D \circ \partial (c) = \partial (I \times c) + I \times \partial c = \epsilon_1 \times c - \epsilon_0 \times c - \cancel{I \times \partial c}
    + \cancel{I \times \partial c}
  \]
  Cioè:
  \[
    \partial \circ D (c) + D \circ \partial (c) = \epsilon_1 \times x - \epsilon_0 \times c
  \]
  Nella figura questo sono la faccia superiore e inferiore del prisma.
  Si definiscono le sezioni del prisma, con $ t \in I $:
  \begin{align*}
    \eta_t  \colon X & \to I \times X \\
    x & \mapsto (t,x)
  \end{align*}
  Le sezioni a $ t = 0 $ e a $ t = 1 $ (e anche le altre, ma non interessano)
  inducono una mappa sulle catene:
  \begin{align*}
    (\eta_i)_\sharp \colon S_k(X) & S_k(I \times X) \\
    c & \mapsto \eta_i \circ c
  \end{align*}
  Ma $ (\eta_i \circ c)(x) = (i, c(x)) = \epsilon_i(c) $, quindi
  $ \partial \circ D (c) + D \circ \partial (c) = (\eta_1)_\sharp - (\eta_0)_\sharp $. Considero la relazione di
  omotopia $ F \colon I \times X \to Y $, per definizione vale che
  $ F(i,x) = f_i(x) $, e quindi $ (F \circ \eta_i)(x) = f_i(x) $, cioè $ F \circ \eta_i = f_i $.
  Essendo una funzione continua $ F $ induce una mappa sulle catene di simplessi:
  $ F_\sharp \colon S_k(I \times X, I \times A) \to S_k(Y,B) $.

  Considero $ D \colon S_q(X,A) \to S_{q+1}(I \times X, I \times A) $, posso definire
  $ G = F_\sharp \circ D $, questo è un omomorfismo tra $ S_k(X,A) $ e
  $ S_k(Y,B) $ in quanto composizione di omomorfismi.
  Sia $ c \in S_q(X,A) $ allora:
  \begin{gather*}
    \partial \circ G (c) = \partial (F_\sharp \circ D)(c) \\
    G \circ \partial (c) = (F_\sharp \circ D)(\partial c)
  \end{gather*}
  $ F_\sharp $ è un'applicazione tra complessi e si verifica che una chain map, cioè
  i quadrati che determina sono commutativi ($ F_\sharp \circ \partial = \partial \circ F_\sharp $). In questo modo
  \begin{gather*}
    \partial (F_\sharp \circ D)(c) + (F_\sharp \circ D)(\partial c) = F_\sharp \circ \partial \circ D (c) + F_\sharp \circ D \circ \partial (c) = \\
    = F_\sharp \circ ( \partial \circ D (c) + D \circ \partial (c)) = F_\sharp \circ ( (\eta_1)_\sharp - (\eta_0)_\sharp) (c)
  \end{gather*}
  Quindi $ \partial \circ G + G \circ \partial = (f_1)_\sharp - (f_0)_\sharp $.
  Passando a livello di omologia considero $ k $ un $ q $-ciclo in $ (X,A) $, quindi
  tale che $ \partial k = 0 $. Allora:
  \[
    (f_1)_\star(k) = [(f_1)_\sharp(k)]
  \]
  Ma:
  \[
    (f_1)_\sharp(k) = (f_0)_\sharp (k) + \partial \circ G (k) + \cancel{G \circ \partial (k)}
  \]
  Quindi in $ (Y,B) $ $ [(f_1)_\sharp(k)] = [(f_0)_\sharp(k)] $ in quanto differiscono per un bordo.
  Quindi $ (f_1)_\star(k) = (f_2)_\star(k) $, ma siccome questo è vero per ogni $ k $ allora
  deve essere $ (f_1)_\star = (f_2)_\star $.
\end{proof}

La mappa $ G $ è un esempio di omotopia di catena:
\begin{definition}
  Siano $ (A_\bullet, \partial^A) $ e $ (B_\bullet, \partial^B) $ complessi, e siano $ \phi, \psi \colon A_\bullet \to B_\bullet $
  mappe continue tra complessi, $ \phi $ e $ \psi $ si dicono \textbf{omotope}\index{Omotopia di catena}
  (\emph{chain homotopic}) se esiste una mappa tra complessi $ D \colon A_\bullet \to B_{\bullet + 1} $
  tale che $ \partial \circ D + D \circ \partial = \phi - \psi $. Si ha quindi il diagramma:
  \[
    \begin{tikzcd}
      \dots \rar & A_{i+1} \rar{\partial^A} & A_i \rar{\partial^A} & A_{i-1} \rar & \dots \\
      \dots \rar & B_{i+1} \rar{\partial^B} & B_i \rar{\partial^B} & B_{i-1} \rar & \dots
    \end{tikzcd}
  \]

\end{definition}

%%% Local Variables:
%%% ispell-local-dictionary: "italiano"
%%% mode: latex
%%% TeX-master: "notes"
%%% End:

%                         lezione 8 parte 2

\chapter{Omologia cellulare}

\section{CW-complessi}

Considero $ \Disk{n} $, vale che $ \partial\Disk{n} = \Sph{n-1} $, considerato lo spazio quoziente
$ X = \quot{\Disk{n}}{\partial \Disk{n}} $, questo è il quoziente del disco per la relazione
di equivalenza che fa collassare il bordo in un punto $ p $. Si trova che $ X \simeq \Sph{n} $.
In 2 dimensioni questo si visualizza facilmente: considerato il cerchio, si spinge il centro
in basso in modo da ottenere una superficie semisferica, quindi indentificare tutti i punti
del bordo con un unico punto vuol dire chiudere il cerchio ottenendo qualcosa di simile ad
una goccia, che è omeomorfa ad una sfera. In pratica quello che ho fatto è:
definisco $ X^{(0)} = P = \set{p} $ e $ \phi \colon \Sph{n-1} \to X^{(0)} $, posso definire:
\[
  X^{(1)} = X^{(0)} \cup_\phi \Disk{n}
\]
Dove con $ \cup_\phi $ si intende, con $ X, Y $ spazi topologici:
\newmathsymb{disgun}{\sqcup}{Unione disgiunta}
\[
  X^{(0)} \cup_\phi \Disk{n} = \quot{X^{(0)} \sqcup \Disk{n}}{p \sim \phi(q)} \quad \forall q \in \Sph{n-1}
\]
Quello che sto facendo in pratica è prendendo un punto e un disco, quindi
identifico il bordo del disco con il punto.

% Considero la proiezione al quoziente $ \pi \colon \Disk{n} \to \Sph{n} $
% con $ \Sph{n} \cong X^{(0)} \cup_\phi \Disk{n} $.

\begin{definition}
  Si dice che lo spazio topologico $ X $ è un \textbf{CW-complesso} di tipo finito\index{CW-complesso},
  dove C significa \emph{closure finite} e W \emph{weak topology} se è dato dai seguenti oggetti topologici:
  \begin{enumerate}
  \item Un insieme finito $ X^{(0)} = \set{p_1, \dots, p_n} $ detto \textbf{$ 0 $-scheletro}\index{$ 0 $-scheletro}
  \item Il \textbf{$ k $-scheletro}\index{$ k $-scheletro} $ X^{(k)} $ si costruisce induttivamente
    a partire da $ X^{(k-1)} $ attaccando opportunamente dei dischi nel modo seguente.
    Considero un numero finito di dischi $ k $-dimensionali $ \Disk{k}_\alpha $,
    detti \textbf{celle}\index{Cella} (o cella chiusa, mentre
    il loro interno è detto cella aperta) per ciascuno si definice una mappa
    continua di attaccamento $ \phi_\alpha \colon \partial\Disk{k}_\alpha \to X^{(k-1)} $, quindi si definisce:
    \[
      X^{(k)} = X^{(k-1)} \cup_\phi \bigcup_\alpha \Disk{k}_\alpha =\quot{ X^{(k-1)} \sqcup_\alpha \Disk{k}_\alpha}{x \sim \phi_\alpha(x)} \quad \forall \alpha \text{ e }
      \forall x \in \partial \Disk{k}_\alpha
    \]
  \item Esiste $ N \in \mathbb{N} $ tale che $ X^{(0)} \subseteq X^{(1)} \subseteq \dots \subseteq X^{(N)} =: X $
  \end{enumerate}
\end{definition}

\begin{osservation}
  % Si dimostra che in generale la cella chiusa non è omeomorfa all'immagine, mentre la cella
  % aperta lo è.
% In generale uno spazio ha numerose strutture di CW complesso.
La topologia è detta debole perché la topologia di unione disgiunta per tutti i $ k $-scheletri,
e questo è la topologia più debole di tutte. In questa topologia un insieme è aperto in $ X $
se e solo se è aperto la sua intersezione con tutti gli $ X^{(i)} $ è aperta.
\end{osservation}

% lezione 10

\subsection{Esempi di CW complessi}

\begin{example}[Circonferenza]
  Il caso più semplice in assoluto consiste nella costruzione di una circonferenza.
  Considero come $ 0 $-scheletro un punto e attacco un solo disco $ 1 $-dimensionale,
  questo è un intervallo. L'attaccamento consiste nel identificare i punti estremi del
  segmento con il punto dello $ 0 $-scheletro, e ciò dà origine a una circonferenza.
  \begin{figure}[htbp]
    \centering
    \begin{tikzpicture}
      \node () at (1,0) {\textbullet};
      \node[above] () at (1,0) {$ X^{(0)}$};
      \draw (2,0) -- (4,0);
      \node[above] () at (3,0) {$ \Disk{1} $};
      \node () at (2,0) {\textbullet};
      \node () at (4,0) {\textbullet};
      \node () at (5,0) {$ \Rightarrow $};
      \draw (7,0) circle (1);
      \node () at (8,0) {\textbullet};
      \node[above] () at (7,1) {$ X^{(1)}$};
    \end{tikzpicture}
    \caption{Circonferenza come CW complesso}
  \end{figure}
\end{example}
Questa costruzione può essere immediatamente generalizzata a una sfera generica, la quale
può essere vista come CW complesso formato da un punto e una sola cella.
\begin{example}[Sfere]
  Sia $ X^{(0)} = \set{p} $ con $ p \in \Sph{n} $ e sia
  $ \phi \colon \partial \Disk{n} \to \set{p} $ mappa costante, questa fa collassare il bordo in
  un punto, quindi
  $ \Sph{n} = X^{(0)} \cup_\phi \Disk{n} = {\Disk{n}} \slash {\partial \Disk{n}} $, cioè una sfera
  è formata da una $ 0 $-cella e una $ n $-cella.
\end{example}
\begin{example}[Sfere, seconda costruzione]
  Alternativamente una seconda possibile costruzione consiste nell'attaccare
  dischi all'equatore questi sono la calotta superiore e inferiore. Per
  costruire una circonferenza parto con $ X^{(0)} = \set{p_1, p_2} $ e attacco
  $ \Disk{1}_1 \cup \Disk{1}_2 $, con le mappe sono:
  \begin{gather*}
    \phi_1 \colon \partial \Disk{1}_1 \to X^{(0)} \quad \text{cioè} \quad \phi_1 \colon \set{-1, +1} \to \set{p_1, p_2} \\
    \phi_2 \colon \partial \Disk{1}_2 \to X^{(0)} \quad \text{cioè} \quad \phi_2 \colon \set{-1, +1} \to \set{p_1, p_2}
  \end{gather*}
  Definite da:
  \[
    \phi_1(1) = p_1 \quad \phi_1(-1) = p_2 \qquad  \phi_2(1) = p_2 \quad \phi_2(-1) = p_1
  \]
  A questo punto $ X^{(1)} \cup_\phi (\Disk{1}_1 \cup \Disk{1}_2) = \Sph{1} $.
  \begin{figure}[htbp]
    \centering
    \begin{tikzpicture}
      \node () at (1,-0.5) {\textbullet};
      \node[above] () at (1,0) {$ X^{(0)}$};
      \node[] () at (1,0) {$ \star $};
      \draw (2,0) -- (4,0);
      \draw (2,-0.5) -- (4,-0.5);
      \node[above] () at (3,0) {$ \Disk{1} $};
      \node[below] () at (3,-0.5) {$ \Disk{1} $};
      \node () at (2,0) {$ \star $};
      \node () at (4,0) {\textbullet};
      \node () at (2,-0.5) {\textbullet};
      \node () at (4,-0.5) {$ \star $};
      \node () at (5,-0.25) {$ \Rightarrow $};
      \draw (7,0) circle (1);
      \node () at (8,0) {\textbullet};
      \node () at (6,0) {$ \star $};
      \node[above] () at (7,1) {$ X^{(1)}$};
    \end{tikzpicture}
    \caption{Circonferenza come CW complesso nel secondo modo}
  \end{figure}

  \noindent
  Si può costruire la $ 2 $-sfera aggiungendo $ \Disk{2}_1 \cup \Disk{2}_2 $ con le mappe:
  \begin{gather*}
    \psi_1 \colon \partial \Disk{2}_1 \to X^{(1)} \\
    \psi_2 \colon \partial \Disk{2}_2 \to X^{(1)}
  \end{gather*}
  Cioè $ \psi_j \colon \Sph{1} \to X^{(1)} $, ovvero
  $ \psi_j \colon \Sph{1} \to \Sph{1} $ e quindi si può prendere l'identità. Si ottiene
  così una $ 2 $-sfera, cioè incollo sull'equatore due dischi. In questo modo si
  può procedere ad libidum.
\end{example}

\begin{example}[Toro]
  Considerato un toro $ T = \Sph{1} \times \Sph{1} $ una possibile costruzione
  è quella ottenuta partendo da un punto e attaccandoci due circonferenze
  che danno origine a quelle colorate in figura \ref{fig:lez3:clifford_torus},
  cioè $ X^{(0)} = \set{p} $, e poi:
  \[
    X^{(1)} = (\Disk{1}_1 \cup \Disk{1}_2) \cup_\phi X^{(0)}
  \]
  Con le mappe:
  \begin{gather*}
    \phi_1 \colon \set{-1, +1} \to \set{p} \\
    \phi_2 \colon \set{-1, +1} \to \set{p}
  \end{gather*}
  Il resto si ottiene attaccando un disco $ \Disk{2} $ che realizza
  l'identificazione che definisce il toro (si potrebbe dare un'espressione
  analitica, ma questa è molto brutta). La cella è
  $ X^{(2)} = (\Disk{2} \cup_\psi X^{(1)}) $ con
  \begin{gather*}
    \psi \colon \Sph{1}  \to X^{(1)} \\
    aba^{-1}b^{-1}
  \end{gather*}
\end{example}

\begin{example}[Prodotto di sfere]
  Siano $ X = \Sph{p} $ e $ Y = \Sph{q} $, questi spazi sono formati
  da una $ 0 $-cella, e da un'altra cella di dimensione $ p $ o $ q $
  con mappe di attaccamento costanti. Anche lo spazio $ X \times Y $ può essere
  strutturato come CW complesso.
  % Sia $ X = \Sph{p} $ allora una possibile struttura di CW
  % è data da una $ 0 $-cella $ e_0 $ e una $ p $-cella $ e_p $.
  % Allora $ X^{(0)} = \set{X} $, $ X^{(p)} = \set{X} \cup_f \Disk{p} $
  % e $ f \colon \Disk{p} \to \Sph{p} $ con $ f \big \vert_{\partial \Disk{p}} = \mathrm{cost.} $.
  % $ Y = \Sph{q} $ con $ f_0 $ $ 0 $-cella, $ f_q $ $ q $-cella,
  % la mappa di attaccamento è: $ g \colon \Disk{q} \to \Sph{q} $ con $ g \big \vert_{\partial \Disk{p}} = \mathrm{cost.} $.
  Questo possiede una $ 0 $-cella $ e_0 \times f_0 $, una $ p $-cella
  $ e_p \times f_0 $, una $ q $-cella $ e_0 \times f_q $ e una $ (p + q) $-cella
  $ e_p \times f_q $. Lo $ 0 $-scheletro è formato da $ \set{(x,y)} $ con $ x \in \Sph{p} $ e $ y \in \Sph{q} $,
  e le mappe di attaccamento sono:

  \noindent
  Per la $ p $-cella:
  \begin{align*}
    F_{p0} \colon \Disk{p} & \to \Sph{p} \times \Sph{q} \\
    z & \mapsto (f(z), y)
  \end{align*}
  Per la $ q $-cella:
  \begin{align*}
    F_{0q} \colon \Disk{q} & \to \Sph{p} \times \Sph{q} \\
    z & \mapsto (z, g(u))
  \end{align*}
  Per la $ (p+q) $-cella:
  \begin{align*}
    F_{pq} \colon \Disk{p+q} & \to \Sph{p} \times \Sph{q} \\
    (w,u) & \mapsto (f(w), g(u))
  \end{align*}
  Dove $ f $ e $ g $ sono le mappe che realizzano gli attaccamenti in $ X $ e in $ Y $,
  cioè tali che sul bordo del disco siano costanti.
\end{example}
% \begin{example}
%   La sfera $ \Sph{n} $ per $ n \geq 0 $ possiede numerose strutture di CW complesso,
%   ad esempio una $ 0 $-cella e una $ n $-cella, oppure un politopo gonfiato.
% \end{example}

\section{Spazi proiettivi}

\begin{definition}
  Si definice lo \textbf{spazio proiettivo reale}\index{Spazio proiettivo reale}
  $ \Pjr{n}= {\RN{n+1} \setminus \set{0}} \slash {\sim} $ con
  $ \vec{x} \sim \vec{y} $ se e solo se $ \vec{x} $ e $ \vec{y} $ sono multipli,
  cioè se esiste $ \lambda \in \RN{} \setminus \set{0} $ tale che $ \vec{x} = \lambda \vec{y} $.
\end{definition}
Fissato $ j \in \set{0, \dots, n} $ e definito
$ U_j = \set{[x_0 : \dots : x_n] \in \Pjr{n} | x_j \not = 0 } $ si definisce la mappa:
\begin{align*}
  \phi_j \colon U_j & \to \RN{n} \\
  [x_0 : \dots : xz_n] & \mapsto \left(\frac{x_0}{x_j}, \dots, \frac{x_n}{x_j}\right)
\end{align*}
Si dimostra che $ \phi_j $ è omeomorfismo. Queste mappe permettono di definire la
topologia dello spazio proiettivo: un sotto insieme di $ U_i $ è aperto se lo è
in $ \RN{n} $ attraverso $ \phi_i $, e un insieme generico $ A $ è aperto se lo
sono tutte le intersezioni con gli insiemi $ U_i $. In questo modo $ \Pjr{n} $ è
una varietà topologica di dimensione $ n $.

\begin{lemma}
  Si dimostra che $ \Pjr{n} \cong {\Sph{n}} \slash {H} $ con $ H = \set{\Id{\Sph{n}}, a_{\Sph{n}}} $, dove
  $ a $ è la mappa antipodale, cioè $ x \sim y \iff x = y \vee x = - y $.
\end{lemma}
\begin{proof}[Sketch]
  La mappa che realizza questo omeomorfismo è:
  \begin{align*}
    f \colon \Pjr{n} & \to \quot{\Sph{n}}{\sim}\\
    [\vec{x}]_{\Pjr{n}} & \mapsto \left[\frac{\vec{x}}{||\vec{x}||}\right]_\sim
  \end{align*}
  La dimostrazione che questa è un omeomorfismo è piuttosto laboriosa. Il motivo è
  comunque intuitivo: lo spazio proiettivo reale è formato dalle rette che passano
  per l'origine, le quali possono tutte essere identificate da un vettore di modulo
  1 su una semisfera.
\end{proof}
\eproof
Si trova quindi che:
\begin{itemize}
\item $ \Pjr{1} \simeq \Sph{1} \slash \sim \simeq \Sph{1} \simeq \RN{} \cup {\infty} $
\item $ \Pjr{2} = \mathbb{P}^1(\RN{}) \cup_\phi \Disk{2} $
  Ho $ {\Sph{2}} \slash {\sim} $, l'emisfero sud della sfera si identifica con quello
  nord per l'applicazione di antipodalità.
  $ \phi = a \big \lvert_{\Sph{1}} $ e $ \Sph{1} = \Pjr{1} $ e $ \Sph{1} = \partial \Disk{2} $,
  quindi:
  \begin{align*}
    \phi \colon \Sph{1} & \to \Sph{1} \\
    (x,y) & \mapsto (-x,-y)
  \end{align*}
\item Se considero $ \Pjr{2} \cup_\phi \Disk{3} $ con:
  \[
    \phi \colon \partial \Disk{3}  \to \Pjr{2}
  \]
  cioè il passaggio al quoziente:
  \[
    \phi \colon \Sph{2}  \to \quot{\Sph{1}}{H}
  \]
\end{itemize}

\begin{example}[Spazi proiettivi]
  Se $ X^{(k)} = \Pjr{k} \cup_\phi \Disk{k+1} $ con
  \[
    \phi \colon \partial \Disk{k+1}  \to \Pjr{k}
  \]
  Cioè:
  \[
    \phi \colon \Sph{k}  \to \Pjr{k}
  \]
  Cioè scelgo $ \phi $ come la proiezione sul quoziente da $ \Sph{k} $ a $ \mathbb{P}^k(\RN{}) = \quot{\Sph{k+1}}{H} $,
  questo è uno spazio compatto.
  $ \mathbb{P}^k(\RN{}) $ è uno spazio di Hausdorff, voglio mostrare che $ X^{(k+1)} \cong \mathbb{P}^{k+1}(\RN{}) $.
  Cerco un'applicazione continua biunivoca e chiusa $ \Phi \colon X^{(k+1)} \to  \mathbb{P}^{k+1}(\RN{}) $,
  cioè un omeomorfismo. Ho il digramma:
  \[
    \begin{tikzcd}
      \mathbb{P}^{k}(\RN{}) \sqcup \Disk{k+1} \arrow{r}{\eta} \arrow{d}{} &  \mathbb{P}^{k+1}(\RN{}) \\
      X^{(k+1)} \arrow{ru}{\Phi}
    \end{tikzcd}
  \]
  \begin{exercise}
    Dimostrare che $ \eta $ è continua e gode di tutte le buone proprietà.
  \end{exercise}
  So che $ i \colon  \mathbb{P}^{k}(\RN{}) \incl \mathbb{P}^{k+1}(\RN{}) $ (è un iperpiano all'infinito),
  quindi posso usare l'inclusione.

  Devo trovare una mappa $ j \colon \Disk{k+1} \to \mathbb{P}^{k+1}(\RN{}) $. $ i $ è ovvia:
  $ i([z_0, \dots, z_k]) = [z_0, \dots, z_k; 0] $, mentre $ j $:
  \[
    j \colon [z_0, \dots, z_k] \mapsto  \left[z_0, \dots, z_{k+1} = \sqrt{1 - \sum_{j=1}^k z_i^2}\right]
  \]
  Siccome $ \sum_{j=1}^k z_i^2 \leq 1 $ l'applicazione è ben definita, quindi $ \eta = (i,j) $.
\end{example}

\begin{example}
  Lo spazio proiettivo reale di dimensione $ n $ $ \Pjr{n} $ possiede una
  struttura di CW complesso con una $ 0 $-cella, una $ 1 $-cella, \dots, una
  $ n $-cella. Lo $ 0 $-scheletro è un punto, l'$ 1 $-scheletro è
  $ K^{(1)} = K^{(0)} \cup_{f_0} \Disk{1} \cong \Pjr{1} $ che è una retta proiettiva
  reale, il $ 2 $-scheletro è
  $ K^{(2)} = K^{(1)} \cup_{f_1} \Disk{1} \cong \Pjr{2} $ che è un piano proiettivo
  reale, e cosí via con $ f_j \colon \partial \Disk{j} \to K^{(j-1)} $ per
  $ j \geq 1 $. In generale ho
  $ \phi_j \colon \Disk{j} \to K^{(j-1)} \cong \Pjr{j-1} $. $ \Pjr{j-1} $ contiene
  $ \Pjr(j-2) $ come iperpiano all'infinito, ad esempio $ z_{j-1} = 0 $. Poi a
  $ \Pjr{j-2} $ incollo $ \Sph{j-2} $ tramite la mappa antipodale. Ad esempio
  per $ n = 2 $ $ \Pjr{2} $ contiene
  $ \Pjr{1} = \set{ [z_0 : z_1 : 0] | (z_0, z_1) \not = 0 } $. Attacco
  $ \Disk{2} $ su questo $ \Pjr{1} $ con:
  \begin{align*}
    f \colon \Sph{1} = \partial \Disk{2} & \to \Pjr{1} \\
    z & \mapsto [z]
  \end{align*}
  Che è la proiezione sul quoziente (mappa antipodale), infatti
  $ \Pjr{1} = \quot{\Sph{1}}{H} $. %Quindi $ \Pjr{2} = \Pjr{1} \cup_f \Disk{2} $.
  \begin{figure}[htbp]
    \centering
    \begin{tikzpicture}
      \fill[gray!20] (0,0) circle (1);
      \draw (0,0) circle (1);
      \draw[-Latex] (0.1,1) -- (0.1001,1);
      \draw[-Latex] (-0.0999,-1) -- (-0.101,-1);
      \node () at (1,0) {-};
      \node () at (-1,0) {-};
    \end{tikzpicture}
    \caption{$ \Pjr{1} $, attacco al disco una circonferenza con la mappa $ f $,
      che quindi identifica i punti antipodali.}
    \label{fig:lez10:projective}
  \end{figure}
\end{example}

\newmathsymb{pjc}{\Pjc{n}}{Spazio proiettivo complesso}
\newmathsymb{cstar}{\C^\star}{Piano complesso privato dell'origine}
Lo \textbf{spazio proiettivo complesso}\index{Spazio proiettivo complesso} invece è:
\[
  \Pjc{n} = \quot{\C^{n+1} \setminus \set{\vec{0}}}{\sim}
\]
con la relazione:
\[
  (z_0, \dots, z_n) \sim (w_0, \dots, w_n) \iff \exists \lambda \in \C^\star= \C \setminus \set{0} \text{ tali che } z_i = \lambda w_i \; \forall i \in \set{0, \dots, n}
\]
Questo è uno spazio compatto, connesso, di Hausdorff i cui punti si indicano con
$ p = [z_0 : \dots : z_n] $. Utilizzando la stessa costruzione dello spazio proiettivo
reale si rende $ \Pjc{n} $ una varietà topologica di dimensione reale $ 2n $.
% Fissato $ j \in \set{0, \dots, n} $ e definito
% $ U_j = \set{[z_0, \dots, z_n] \in \Pjc{n} | z_j \not = 0 } $ si definisce la mappa:
% \begin{align*}
%   \phi_j \colon U_j & \to \C^n \simeq \RN{2n} \\
%   [z_0, \dots, z_n] & \mapsto \left(\frac{z_0}{z_j}, \dots, \frac{z_n}{z_j}\right)
% \end{align*}
% Si dimostra che $ \phi_j $ è omeomorfismo. Queste mappe permettono di definire la
% topologia dello spazio proiettivo: un sotto insieme di $ U_i $ è aperto se lo è
% in $ \RN{2n} $ attraverso $ \phi_i $, e un insieme generico $ A $ è aperto se lo
% sono tutte le intersezioni con gli insiemi $ U_i $. In questo modo $ \Pjc{n} $ è
% una varietà topologica di dimensione $ 2n $.
\begin{example}[$ n = 1 $]
  $ \Pjc{1} $ è noto come retta complessa o sfera di Riemann, in quanto
  si trova che $ \Pjc{1} \simeq \Sph{2} $, infatti usando le mappe di isomorfismo $ \phi_i $:
  \[
    \Pjc{1} = \set{ [0:1] } \cup U =: \set{\infty} \cup U
  \]
  Con $ U $ intorno aperto omeomorfo a $ \RN{2} $. Ma la proiezione
  stereografica manda la sfera senza polo Nord in $ \RN{2} $, cioè
  $ \Sph{2} \setminus \set{N} \simeq \RN{2} $, quindi
  $ \Pjc{1} \setminus \set{\infty} \simeq \RN{2} $ e quindi
  $ \Pjc{1} \setminus \set{\infty} \simeq \Sph{2} \setminus \set{N} $. Questi sono spazi non compatti ma
  di Hausdorff, so che la compattificazione di Alexandroff\footnotemark sono spazi omeomorfi,
  ma la compattificazione di uno spazio meno un punto è lo spazio stesso: $ \Pjc{1} \simeq \Sph{2} $.
\end{example}
\footnotetext{\index{Compattificazione di Alexandroff}La compattificazione di
  Alexandroff consiste nell'aggiungere un punto ad uno spazio topologico per
  renderlo compatto.}

% In merito al generico spazio proiettivo complesso $ \Pjc{n} $ vorrei sapere
% quale è la struttura di CW complesso, quale è il suo gruppo fondamentale e quali
% sono i suoi gruppi di omologia.

Anche gli spazi proiettivi complessi ammettono struttura di CW complesso.
$ K^{(0)} $ è un punto $ \Pjc{0} $, poi si può costruire
$ K^{(2)} = K^{(0)} \cup_f \Disk{2} = \Sph{2} \simeq \Pjc{1} $ con $ f $ mappa costante,
che quindi fa collassare il bordo in un punto. Successivamente si ha
$ K^{(4)} = K^{(2)} \cup_g \Disk{4} $ infatti $ K^{(2)} = \Pjc{1} $, poi prendo
$ \Disk{4} $ so che $ \partial \Disk{4} = \Sph{3} $ e la mappa al quoziente è
$ \pi \colon \Sph{3} \to \Pjc{1} $ che è fatta così:
\begin{align*}
  \pi \colon \Sph{3} & \mapsto \Pjc{1} \\
  (z_0, z_1) & \mapsto [z_0 : z_1]
\end{align*}
Posso fare agire $ \Sph{1} $:
\begin{align*}
  \Sph{1} \times \Sph{3} & \to \Sph{3} \\
  (\lambda, (z_0, z_1)) & \mapsto (\lambda z_0, \lambda z_1)
\end{align*}
Siccome $ \lambda \in \Sph{1} $ allora $ | \lambda | = 1 $ e quindi $ | \lambda z_0 |^2 + | \lambda z_1 |^2 = 1 $.
Faccio il quoziente $ \Pjc{1} = \quot{\Sph{3}}{\Sph{1}} $ e $ \pi $ è la proiezione al
quoziente. Allora $ K^{(4)} = K^{(2)} \cup_\pi \Disk{4} \simeq \Pjc{2} $.

In generale $ K^{(2n - 2)} $ si costruisce prendendo $ \Disk{2n} $ e con la mappa
di proiezione $ \pi \colon \partial \Disk{2n} = \Sph{n-1} \to \Pjc{n-1} $, quindi $ K^{(2n -2)} \cup_\pi \Disk{2n} $.
$ \Pjc{n} $ è un CW complesso ottenuto attaccando celle di dimensione $ 2j $ per $ 0 \leq j \leq n $
Quindi ho una $ 0 $-cella, una $ 1 $-cella, \dots, una $ 2n $-cella.

\begin{osservation}
  In generale $ \Pjc{n} = \Pjc{n-} \cup \C^n $ quindi si può srotolare.
\end{osservation}

\begin{proposition}
  Vale che $ K^{(2n)} \simeq \Pjc{n} $.
\end{proposition}
\begin{proof}
  La dimostrazione è per induzione. Assumo che $ K^{2t} = \Pjc{t} $ per $ 0 \leq t \leq n-1 $.

  Sia
  \begin{align*}
    h \colon \Disk{2n} & \to \Pjc{n} \\
    (z_0, \dots, z_{n-1}) & \mapsto \left(z_0,\dots, \sqrt{1 - \sum_{i < n}|z_i|^2} \right)
  \end{align*}
  So che $ \partial \Disk{2n} = \Sph{2n-1} = \set{|z_0|^2 + \dots + |z_{n-1}|^2 = 1} $
  quindi $ h $ è ben definita in quanto la radice esiste, ed è continua.
  \begin{align*}
    h \big \lvert_{\partial \Disk{2n}} \colon \Sph{2n-1} & \to \Pjc{n-1} \\
    (z_0, \dots, z_{n}) & \mapsto [z_0, \dots, z_{n-1}, 0] = P
  \end{align*}
  Vale che:
  \[
    \left( h \big \lvert_{\partial \Disk{2n}}\right)^{-1}(P) = \set{ (\lambda_{z_0}, \dots, \lambda_{z_{n-1}}) | |\lambda| = 1} \simeq \Sph{1}
  \]
  Queste sono le preimmagini. $ h $ non è iniettiva.
  \[
    \begin{tikzcd}
      \Disk{2n} \rar{h} & \Pjc{n} \\
      \Pjc{n-1} \cup_\tau \Disk{n} \arrow{ur}{F} & {}
    \end{tikzcd}
  \]
  Dove $ \tau = h \big \lvert_{\partial \Disk{2n}} $, con $ P((z_0, \dots, z_{n-1})) = ([z_0, \dots, z_{n-1}]) $.
  $ F $ manda $ \Pjc{n-1} $ in $ \Pjc{n} $ banalmente e raccorda bene i dischi.
  $ F $ è iniettiva e suriettiva da uno spazio compatto a uno spazio di Hausdorff,
  quindi è un omeomorfismo.
\end{proof}

Ho che $ \pi_1(\Pjc{n}) = \set{1} $ $ \forall n \geq 1 $
infatti per $ n = 1 $ $ \pi_1(\Pjc{1}) \cong \pi_1(\Sph[1]) = \set{1} $.
Per induzione suppongo che $ \pi_1(\Pjc{n-1}) = \set{1} $, voglio
mostrare che $ \pi_1(\Pjc{n}) = \set{1} $. Per fa ciò
uso il teorema di Seifert-van Kampen.
\[
  \Pjc{n} = \Pjc{n-1} \cup_\pi \Disk{2n}
\]
Considero $ x \in \Disk{2n} $ e un aperto $ V $ disco centrato in $ x $
di raggio $ \epsilon $ piccolo, cioè $ V = \Disk{2n}_\epsilon(x) $. Poi prendo
$ U = \Pjc{n} \setminus \set{x} $ aperto.
Vale che $ V \sim \set{x} $, poi $ \Disk{2n} $ si ritrae al bordo, che
si attacca. $ U \simeq \Pjc{n-1} $.
Poi $ V \cap U $ è una specie di corona circolare in $ \Disk{2n} $,
quindi $ V \cap U \sim \Sph{2n -1} $. Quindi $ \pi_1(\Pjc{n}) = \Pjc{n-1} \cong \set{1} $.
È più interessante vedere l'omologia singolare.
Si trova che:
\[
  H_k(\Pjc{n}) \cong
  \begin{cases}
    \Z & \text{se } k \in \set{0,2,4,\dots, 2n} \\
    0 & \text{altrimenti}
  \end{cases}
\]
Con $ k = 1 $ è il gruppo fondamentale, quindi è nullo, e poi torna per $ n = 1 $.

Per comodità si introduce l'omologia cellulare di $ X $.

Se $ X $ è spazio topologico con struttura di CW complesso si introduce
l'omologia cellulare $ H_k^{CW}(X) $.

Si trova che $ H_k^{CW}(X) \cong H_k(X) $ e c'è un algoritmo per calcolare $ H_k^{CW}(X) $.

So che $ \Pjc{n} = \Pjc{n-1} \cup_\pi \Disk{2n} $. Fisso $ n $ voglio calcolare
$ H_s(\Pjc{n}^{(t)}, \Pjc{n}^{(t-1)}) $.
Calcolo per induzione.
So che $ H_k(\Pjc{1}) $ è a posto, voglio calcolare $ H_k(\Sph{m}) $ per induzione.

Mi piacerebbe che:
\[
  H_s(\Pjc{c}^{(t)}, \Pjc{n}^{(t-1)}) \cong H_s(\quot{\Pjc{n}^{(t)}}{\Pjc{n}^{(t-1)}})
\]

Questo è vero in generale, se $ A \subseteq X $ CW complessi allora:
\[
  H_k(X,A) \cong H_k(\quot{X}{A})
\]
Ma $ \Pjc{n}^{(t)} = \Pjc{n}^{(t-1)} \cup_\pi \Disk{2t} $, è come se collassa
quello che è in comune alle celle, cioè il bordo del dei dischi ad un punto,
cioè:
\[
  \quot{\Pjc{n}^{(t)}}{\Pjc{n}^{(t-1)}} \simeq \Sph{2t}
\]
Se $ s \not = 2 t $ allora $ H_s(\quot{\Pjc{n}^{(t)}}{\Pjc{n}^{(t-1)}}) = 0 $.
\[
  H_{2t}(\Pjc{n}^{(t)}, \Pjc{n}^{(t-1)}) \cong \Z
\]
E gli altri sono zero.

In generale $ X^{(k)}- X^{(k-1)} \cup_{f_1} \Disk{k}_{\alpha_1} \cup \dots \cup_{f_n} \Disk{k}_{\alpha_N} $
cosa è $ H_s(X^{(k)}, X^{(k-1)}) $.
\[
  H_s(X^{(k)}, X^{(k-1)}) \cong H_s(\quot{X^{(k)}}{X^{(k-1)}})
\]
Ma $ \quot{X^{(k)}}{X^{(k-1)}} $ è un bouquet, in quanto tutte le sfere hanno in comune
il punto a cui si è contratto $ X^{(k-1)} $.
Quindi:
\[
  H_s(X^{(k)}, X^{(k-1)}) \cong
  \begin{cases}
    \Z^N & \text{se } k = s \\
    0 & \text{se } k \not = s
  \end{cases}
\]

Considero $ \Pjc{n} $ ho che:
\[
  \Pjc{n-2} \incl \Pjc{n-1} \incl \Pjc{n}
\]
Quindi:
\[
  (\Pjc{n-1}, \Pjc{n-2}) \to (\Pjc{n}, \Pjc{n-1})
\]
e quindi
\[
  H_s(\Pjc{n-1}, \Pjc{n-2}) \to H_s(\Pjc{n}, \Pjc{n-1})
\]
Ma il primo è diverso da zero se $ s = 2n - 2 $ e
il secondo è diverso da zero se $ s = 2n $.

Da qui non ottengo informazioni di carattere generale,
cioè quello che sto dicendo è che non è semplicemente
la composizione della coppia allora uso un trucco.

Costruisco un'applicazione. FORSE

% Sia $ X $ un CW complesso e $ Y $ un CW complesso, è
% possibile dare una struttura di CW complesso anche a $ Z = X \times Y  $.
% Se $ \set{e_\alpha} $ sono le celle di $ X $ e $ \set{f_\beta} $
% quelle di $ Y $, allora $ \set{e_\alpha \times f_\beta} $ sono
% celle di $ Z $. Bisogna solo capire come sono fatte le
% mappe di attaccamento.

    %     lezione 8

\section{(ex-)Congettura di Poincaré}

Per $ n = 2 $ ho $ \Sph{2} $ è una $ 2 $-varietà topologica compatta e connessa
il cui gruppo fondamentale è banale e i gruppi di omologia noti, infatti ho
calcolato l'omologia di una sfera generica:
\[
  H_k(\Sph{n}) \cong
  \begin{cases}
    \Z & \text{ se } k \in \set{0,n} \\
    0 & \text{ se } k \not \in \set{0,n}
  \end{cases}
\]
In particolare ho $ H_0(\Sph{n}) \cong \Z $ ed è generato dalla classe di
omologia di un punto qualsiasi, mentre $ H_n(\Sph{n}) \cong \Z $ è generato dalla
classe di omologia di un $ n $-simplesso singolare $ \tau_n \colon \Delta_n \to \Sph{n} $.

\begin{proposition}
  Se $ \M $ è una $ 2 $-varietà topologica compatta e connessa tale che $ \forall k \geq 2 $
  $ H_k(\M) \cong H_k(\Sph{2}) $ allora $ \M \simeq \Sph{2} $.
\end{proposition}
\begin{proof}
  Esiste un teorema di classificazione delle varietà topologiche di dimensione
  $ 2 $ compatte e connesse, questo dice che una varietà topologia $ \M $
  compatta e connessa di dimensione $ 2 $ è omeomorfa a una delle seguenti tre
  categorie: $ \Sph{2} \simeq V_0 $, $ V_g $ oppure $ U_n $. Dove:
  \[
    V_g =
    \begin{cases}
      \Sph{2} & \text{se } g = 0 \\
      \quot{P_{4g}}{\sim} & \text{se } g \geq 1
    \end{cases}
  \]
  Dove $ \sim $ è l'identificazione $ a_1 b_1 a_1^{-1}b_1^{-1}\dots a_g b_g a_g^{-1}b_g^{-1} $,
  come ad esempio il toro, mentre:
  \[
    U_n =
    \begin{cases}
      \mathbb{P}^2(\RN{}) & \text{se } n = 1 \\
      \quot{P_{2n}}{\sim} & \text{se } n \geq 2
    \end{cases}
  \]
  Con $ \sim $ è l'identificazione $ a_1 a_1 \dots a_n a_n $, come ad esempio la
  bottiglia di Klein ($ abab^{-1} = aabb $). Tutti i $ V_g $ non sono omeomorfi
  tra loro, e similmente gli $ U_n $, e neppure gli $ U_n $ e i $ V_g $ sono
  vicendevolmente omeomorfi in quanto i primi sono non orientabili, mentre i
  secondi si. Escludo queste due possibilità confrontando i primi gruppi di
  omologia. So che:
  \[
    H_1(V_g) \cong
    \begin{cases}
      H_1(\Sph{2}) & \text{se } g = 0 \\
      \Z^{2g} & \text{se } g \geq 1
    \end{cases}
  \]
  Gli spazi $ V_g $ con $ g \geq 1 $ non hanno lo stesso tipo di omologia di
  $ \Sph{2} $ perché $ H_1(V_g) $ è non banale, mentre il gruppo fondamentale di
  $ \Sph{2} $ lo è. Similmente
  $ H_1(\mathbb{P}^2(\RN{})) \cong \pi_1(\mathbb{P}^2(\RN{})) \cong \Z_2 $, che non è
  banale, e $ H_1(U_n) \cong \Ab{\pi_1(U_n)} $, ma usando Seifert-van Kampen si trova che:
  \begin{gather*}
    \pi_1(U_n) = \langle a_1, \dots, a_n \; | \; a_1^2\dots a_n^2 = 1 \rangle \; \Rightarrow \\
    \Ab{\pi_1(U_n)} = \langle a_1, \dots, a_n \; | \;  c = a_1 \dots a_n = \pm 1 \rangle = \Z_2 \oplus \Z^{n-1}
  \end{gather*}
  Dove $ \Z_2 $ viene dal fatto che abelianizzando si ha
  $ c = a_1^2\dots a_n^2 = (a_1 \dots a_n)^2 = 1 $ quindi $ c = \pm 1 $, mentre
  $ \Z^{n-1} $ è il gruppo libero generato dai rimanenti. Questo non è banale,
  quindi l'unico spazio possibile è proprio $ \Sph{2} $.
\end{proof}
\eproof
Per $ \Sph{3} $ vale il seguente risultato, dimostrato da Perelman nel
2003, precedentemente noto come congettura di Poincaré:
\begin{proposition}[ex-Congettura di Poincaré]
  Se $ \M $ è una $ 3 $-varietà topologica compatta, connessa e semplicemente
  connessa tale che $ \forall k \geq 3 $ $ H_k(\M) \cong H_k(\Sph{3}) $ allora $ \M \simeq \Sph{3} $.
\end{proposition}
Se rinuncio alla richiesta di semplice connessione il risultato non vale più, ma
vale la seguente proposizione:
\begin{proposition}
  Se $ \M $ è una $ 3 $-varietà topologica compatta e connessa tale che $ \forall k \geq 3 $
  $ H_k(\M) \cong H_k(\Sph{3}) $ allora non si può concludere che $ \M \simeq \Sph{3} $.
\end{proposition}
\newmathsymb{mat}{M_n(\mathbb{F})}{Matrici quadrate di ordine $ n $ sul campo $ \mathbb{F} $}
\begin{proof}
  Costruisco un controesempio, noto come \textbf{spazio dodecaedrico di
    Poincaré}\index{Spazio dodecaedrico}, o anche spazio a omologia
  razionale\index{Spazio a omologia razionale ! \vedi{Spazio dodecaedrico}}.
  Costruirò una $ 3 $-varietà topologica compatta e connessa $ P $ con lo stesso
  tipo di omologia di una $ 3 $-sfera ma non omeomorfa a $ \Sph{3} $ in quanto
  il gruppo fondamentale è finito non abeliano di ordine 120. Parto da
  $ \Sph{3} $, posso scrivere:
  \[
    \Sph{3} \subseteq \mathbb{C}^2 \qquad \Sph{3} = \set{ (z_0, z_1) \in \mathbb{C}^2 | |z_0|^2 + |z_1|^2 = 1}
  \]
  Infatti $ z_0 = x + i y $ e $ z_1 = t + i w $ quindi $ |z_0|^2 = (x + iy)(x - iy) = x^2 + y^2 $
  e $ |z_0|^2 = (t + iw)(t - iw) = t^2 + w^2 $ e quindi ottengo:
  \[
    \Sph{3} = \set{ (x,y,t,w) \in \RN{4} | x^2 + y^2 + t^2 + w^2 = 1}
  \]
  Così come $ \Sph{1} $ ha una struttura di gruppo U(1) è possibile strutturare
  $ \Sph{3} $ come gruppo SU(2):
  \[
    \mathrm{SU(2)} = \set{A \in M_2(\mathbb{C}) | \det{A} = 1, \; AA^\dagger = \Id{2}}
  \]
  Quindi $ \mathrm{SU(2)} \subseteq \mathbb{C}^4 $, si dimostra che $ A \in \mathrm{SU(2)} $ se e solo se
  è della forma:
  \[
    \begin{pmatrix}
      \alpha & - \beta^\star \\
      \beta & \alpha^\star \\
    \end{pmatrix}
    \text{ con } \alpha,\beta \in \mathbb{C} \text{ e } |\alpha|^2 + |\beta|^2 = 1
  \]
  % Questo significa che i vettori in $ \mathbb{C}^2 $ $ (\alpha, \beta) $ e
  % $ (-\beta^\star, \alpha^\star) $ sono normalizzati e sono tra di loro ortogonali.
  In questo modo si costruisce immediatamente la corrispondenza buinivoca tra
  SU(2) e $ \Sph{3} $:
  \begin{align*}
    \mathrm{SU(2)} & \leftrightarrow \Sph{3} \\
    \begin{pmatrix}
      \alpha & - \beta^\star \\
      \beta & \alpha^\star \\
    \end{pmatrix} & \leftrightarrow (\alpha,\beta)
  \end{align*}
  In questo modo si può definire un prodotto su $ \Sph{3} $ rappresentando
  $ (x,y,t,w) \in \RN{4} $ come numeri complessi e passando alla controparte
  matriciale, dove il prodotto è definito naturalmente come prodotto riga per
  colonna, quindi una volta svolto il prodotto si torna alla notazione a quattro
  reali. A questo punto è triviale trovare l'identità e l'elemento inverso che
  permettono di dare a $ \Sph{3} $ la struttura di gruppo. Inoltre vedendo SU(2)
  come spazio topologico con topologia indotta da $ \mathbb{C}^4 $ SU(2) e
  $ \Sph{3} $ sono sia isomorfi come gruppi che omeomorfi come spazi topologici.

  La costruzione dello spazio dodecaedrico si basa sulle isometrie del dodecaedro $ D_{12} $,
  questo è un solido regolare con 12 facce, 30 spigoli e 20 vertici.
  \begin{figure*}[htbp]
    \centering
    \def\svgwidth{0.26\textwidth}
    \input{images/Dodecahedron.pdf_tex}
    \caption{Dodecaedro}
  \end{figure*}

  \noindent
  Il gruppo di isometrie del dodecaedro, cioè:
  \[
    \mathrm{Isom}(D_{12}) = \set{ g \colon \RN{3} \to \RN{3} | g \text{ regolare e } g(D_{12}) = D_{12}}
  \]
  Questo gruppo si può vedere come:
  \[
    \mathrm{Isom}(D_{12}) \cong A_5 \times \Z_2
  \]
  Dove $ A_5 $ è un sottogruppo di $ \mathrm{Isom}(D_{12}) $ ed è il gruppo
  alterno (cioè il gruppo delle permutazioni pari) su 5 elementi e quindi ha ordine 60.
  Le 60 trasformazioni che sono in $ A_5 $ sono l'identità, 24 rotazioni di $ \frac{2}{5} \pi $ attorno
  agli assi per i centri di facce opposti, 20 rotazioni di $ \frac{2}{3} \pi $ attorno
  agli assi per vertici opposti e 15 rotazioni di $ \pi $ attorno agli assi per
  i punti medi di spigoli opposti.
  $ \Z_2 $ invece è dovuto all'applicazione antipodale che è $ (x,y,z) \mapsto (z,y,z) $.
  $ A_5 $ è un sottogruppo finito di SO(3) che sono le rotazioni di $ \RN{3} $ attorno
  a una retta passante per l'origine, cioè:
  \[
    \mathrm{SO(3)} = \set{ R \in M_3(\RN{}) | \det{R} = 1, \;R^T R = \Id{3}}
  \]
  Per passare da SO(3) a $ \Sph{3} $ utilizzo la \textbf{rappresentazione spinoriale}\index{Rappresentazione spinoriale di SO(3)}
  (questo mi permette di passare dal dodecaedro che è tridimensionale alla $ 3 $-sfera).
  Sia $ \rho $ una rappresentazione di SU(2), cioè un omomorfismo:
  \[
    \rho \colon \Sph{3} = \mathrm{SU(2)} \to \mathrm{GL}(V)
  \]
  Con GL gruppo generale lineare e $ V $ è uno spazio vettoriale di dimensione
  3 (quindi $ V \cong \RN{3} $) scelgo lo spazio delle matrici antihermitiane a
  traccia nulla:
  \[
    V = \set{H \in M_2(\mathbb{C}) | H + H^\dagger = 0, \; \tr H  = 0}
  \]
  Si trova che $ V $ è generato da:
  \[
    E_1 =
    \begin{pmatrix}
      0 & i \\
      i & 0 \\
    \end{pmatrix}
    \quad
    E_2 =
    \begin{pmatrix}
      0 & 1 \\
      -1 & 0 \\
    \end{pmatrix}
    \quad
    E_3 =
    \begin{pmatrix}
      i & 0 \\
      0 & -i \\
    \end{pmatrix}
  \]
  Allora scelgo $ \rho(T) \colon H \mapsto THT^\dagger $.
  Perché $ \rho $ sia una rappresentazione bisogna verificare che se $ T \in \mathrm{SU(2)} $
  e $ H \in V $:
  \begin{enumerate}
  \item $ \rho $ omomorfismo
  \item $ \rho(T) $ lineare
  \item $ \rho(T) $ invertibile
  \item $ \rho(T)(H) \in V $
  \end{enumerate}
  Verifico ad esempio che $ \rho(T)(H) \in V $:
  \begin{gather*}
    THT^\dagger + TH^\dagger T^\dagger = 0 \iff T(H + H^\dagger)T^\dagger = 0 \iff H \in V \\
    \tr(THT^\dagger) = \tr(THT^{-1}) \overset{\text{ciclicità}}{=} \tr(H) = 0 \iff H \in V
  \end{gather*}
  Ho quindi $ \rho \colon \Sph{3} \to \mathrm{GL}(V) $, vorrei cercare di restringere a
  O($ V $) al posto di $ \mathrm{GL}(V) $. Per far ciò bisogna prima definire un
  prodotto scalare definito positivo, in modo da poter definire il concetto di
  isometria e considerare quindi O($ V $) come il gruppo di isometrie di $ \RN{3} $.
  Una possibile forma quadratrica naturale è in questo caso il determinante,
  infatti se $ H \in V $ allora:
  \[
    H =
    \begin{pmatrix}
      i a & c + i b \\
      -c + i b & - i a
    \end{pmatrix}
  \]
  Con $ a,b,c \in \RN{} $, infatti $ \det{H} = a^2 + b^2 + c^2 $ che è il consueto
  prodotto scalare in $ \RN{3} $. In questo modo $ V $ diventa uno spazio euclideo
  con prodotto scalare $ q = \det $.

  Con questa definizione $ \rho(T) \colon V \to V $ è un'isometria. Questo è vero se
  $ q(\rho(T)(H)) = q(H) $ cioè se $ \det(THT^\dagger) = \det{H} $, ma per Binet questo
  equivale a $ \det{T}\det{H}\det{T^\dagger} = \det{H} $, utilizzando il fatto che il
  determinante di una matrice è un numero complesso e quindi commuta questo
  equivale a $ \det{T}\det{T^\dagger}\det{H} = \det{H} $, sempre per Binet
  $ \det(TT^\dagger)\det{H} = \det{H} $, ma per ipotesi $ TT^\dagger = \Id{} $ quindi
  effettivamente $ \rho(T) $ è isometria, perciò:
  \[
    \rho \colon \Sph{3} \to \mathrm{O}_3(V) \; \text{cioè } \rho(T) \in \mathrm{O}_3(V)
  \]
  \vspace*{-20pt}
  \begin{exercise}
    Verificare che $ \rho $ è continua come applicazione tra spazi topologici
    equipaggiando $ \mathrm{O}_3(V) $ con la topologia indotta da $ \RN{9} $.
  \end{exercise}
  Essendo $ \rho $ continua manda compatti in compatti e connessi in connessi,
  quindi $ \rho(\Sph{3}= \mathrm{SU(2)}) $ è connesso in $ \mathrm{O}_3(V) $. Ma
  $ \mathrm{O}_3(V) $ non è connesso, e anzi è formato da due componenti
  connesse, una è SO(3), l'altra è SO(3) moltiplicata per una qualunque matrice
  di determinante $ - 1 $. Siccome $ \rho $ è omomorfismo
  $ \rho(\Id{}) = \Id{} $, quindi $ \rho(\Sph{3}) = \mathrm{SO(3)} $, in questo modo
  rappresento la $ 3 $-sfera come rotazioni in $ \RN{3} $. Si dimostra che
  $ \rho $ è suriettiva e $ \ker{\rho} = \set{(1,0,0,0), (-1,0,0,0)} $ elementi che
  corrispondono a $ \Id{} $ e $ -\Id{} $. A livello di gruppi per il primo
  teorema di omomorfismo:
  \[
    \quot{\Sph{3}}{\ker{\rho}} \cong \mathrm{SO(3)}
  \]
  Ad una rotazione in $ \RN{3} $ corrispondono due punti sulla $ 3 $-sfera che sono
  uno l'antipodale dell'altro.

  Ora ho $ A_5 \subseteq \mathrm{SO(3)} $ definisco
  $ G = \set{ T \in \Sph{3} | \rho(T) \in A_5} $, cioè sono tutti i punti della sfera a
  cui corrispondono le rotazioni in $ A_5 $. $ G $ è un gruppo, infatti se
  $ T, S \in G $ allora $ \rho(T), \rho(S) \in A_5 $ e
  $ \rho(TS) = \rho(T)\rho(S) \in A_5 $ in quanto $ A_5 $ gruppo e $ \rho $ omomorfismo.
  Inoltre $ \Id{} \in G $ in quanto $ \rho(\Id{}) \in A_5 $. Si definisce
  $ \phi = \rho \lvert_G \colon G \to A $, la quale per costruzione è suriettiva. Inoltre
  $ \ker{\phi} = \set{ T \in G | \phi(T) = \Id{}} $, ma $ \phi(T) = \rho(T) $, quindi
  $ T = \pm \Id{} $, cioè $ \ker{\phi} = \set{- \Id{}, + \Id{}} $. Si può costruire
  la successione esatta di gruppi:
  \[
    \begin{tikzcd}
      \Id{} \arrow{r}{} & \ker{\phi} \arrow{r}{} & G \arrow{r}{} & A_5 \arrow{r}{} & \Id{}
    \end{tikzcd}
  \]
  Essendo la successione esatta vale che $ A = {G} \slash {\ker{\phi}} $.
  $ G \subseteq \Sph{3} $ e ha ordine 120, inoltre $ \ker{\phi} $ è normale in
  $ G $. Si verifica formalmente che la successione spezza, quindi
  $ G \cong A_5 \oplus \ker{\phi} \cong \mathrm{Isom}(D_{12}) $, questo lo si intuisce per il fatto che sostanzialmente
  $ G $ e formato da $ (A_5, + \Id{}) $ e $ (A_5, - \Id{}) $.
  Definendo l'azione del gruppo su $ \Sph{3} $:
  \begin{align*}
    G \times \Sph{3} & \to \Sph{3} \\
    (g,x) & \to gx
  \end{align*}
  Il prodotto $ gx $ va inteso in questo modo: sia $ g $ che $ x $ sono punti di
  $ \Sph{3} $, quindi sono rappresentabili come matrici di SU(2) per cui è ben
  definito il prodotto. Considerando la proiezione sul quoziente
  $ \pi \colon \Sph{3} \to {\Sph{3}} \slash {G} $ si trova lo spazio dodecaedrico
  $ P := {\Sph{3}} \slash {G} $, il quale è connesso e compatto in quanto quoziente
  di uno spazio connesso e compatto. Bisogna verificare:
  \begin{enumerate}
  \item $ P $ è una $ 3 $-varietà
  \item $ \pi_1(P) $ non è banale
  \item $ H_k(P) \cong H_k(\Sph{3}) \; \forall k \in \mathbb{N}$
  \end{enumerate}
  Si dimostra che $ \pi $ è un rivestimento, cioè comunque si prenda un punto $ p \in P $
  esiste intorno di $ p $ a cui corrispondono 120 intorni disgiunti su $ \Sph{3} $.
  Siccome $ \Sph{3} $ è semplicemente connesso il rivestimento è universale.
  \begin{exercise}
    Dimostrare che $ \pi $ è rivestimento universale di $ P $ su $ \Sph{3} $.
  \end{exercise}
  Siccome $ P $ è rivestito da $ \Sph{3} $ è di dimensione 3 perché localmente è
  fatto come $ \Sph{3} $ che è una varietà $ 3 $-dimensionale. Dalla teoria
  generale dei rivestimenti si trova che $ \pi_1(P) \cong G $, quindi
  $ \pi_1(P) $ è non banale. Inoltre, $ P $ è connesso per archi perché passaggio
  al quoziente di insieme connesso per archi quindi $ H_0(P) \cong \Z $ cioè
  $ H_0(P) \cong H_0(\Sph{3}) $. Rimangono da calcolare i gruppi di omologia per
  $ k = 1 $ e $ k = 2 $, per far ciò considero $ \sigma \colon \Delta_k \to P $ un simplesso
  singolare, si ha il diagramma:
  \[
    \begin{tikzcd}
      {} & \Sph{3} \dar{\pi} \\
      \Delta_k \rar{\sigma} \arrow{ru}{} & P
    \end{tikzcd}
  \]
  Per il teorema di sollevamento siccome il rivestimento è universale
  $ \sigma $ si solleva e quindi vuol dire che c'è un elemento non banale in $ H_k(\Sph{3}) $,
  ma per $ k = 1 $ e per $ k = 2 $ l'omologia è nulla, quindi non può esserci
  qualcosa di non banale, per questo $ H_1(P) = 0 $ e $ H_2(P) = 0 $.

  Per calcolare $ H_3(P) $ si usa una tecnica geometrica basata
  sull'osservazione che $ \Sph{3} $ ha una struttura di CW-complesso, quella di
  una $ 0 $-cella e una $ 3 $-cella.

  \newmathsymb{eulerc}{e(X)}{Caratteristica di Eulero di $ X $}
  \begin{definition}
    Per un CW-complesso finito $ X $ si definisce la \textbf{caratteristica di Eulero}\index{Caratteristica di Eulero di un CW-complesso}
    come:
    \[
      e(X) = \sum_{i = 0}^n (-)^i a_i
    \]
    dove $ a_i $ è il numero di $ i $-celle, che per ipotesi è finito.
  \end{definition}
  Si dimostra che
  \begin{enumerate}
  \item La caratteristica di Eulero non dipende dalla scelta della struttura
    di CW-complesso.
  \item Vale la formula:
    \[
      e(X) = \sum_{i\geq0}(-)^i \rank{H_i(X)}
    \]
  \item Se $ \pi \colon X \to Y $ è un riversimento $ d $ a 1 allora vale che $ e(X) = d e(Y) $.
  \end{enumerate}
  Per $ P $:
  \[
    e(P) = \rank{H_0(P)} - \rank{H_1(P)} + \rank{H_2(P)} - \rank{H_3(P)} = 0
  \]
  Da cui $ \rank{H_3(P)} = 1 $ e quindi $ H_3(P) \cong \Z \oplus T $ dove $ T $ è una
  parte di torsione per il teorema di struttura, ma si dimostra che $ T $ è
  nulla. In questo modo si è costruito uno spazio che non soddisfa le ipotesi
  della congettura di Poincaré e per il quale non si può dedurre se sia o meno
  omeomorfo a una sfera.
\end{proof}
\eproof
Questo mostra che il gruppo fondamentale è uno strumento più fine
dei gruppi di omologia.

%     lezione 11

\section{Costruzione dell'omologia cellulare}

\begin{definition}
  Sia $ (Y,A) $ CW complessi con $ A \subseteq Y $
  la coppia $ (Y,A) $ si dice \textbf{buona}\index{Coppia Buona} se
  allora esiste un intorno aperto $ V $ di $ A $ in $ Y $ tale che $ A $
  sia un retratto di deformazione di $ A $.
\end{definition}
\begin{osservation}
  Siccome $ V $ è un intorno aperto di $ A $ vale
  che $ \bar{A} \subseteq \mathrm{int}(V) = V $. Questo è il requisito per poter
  applicare il teorema di escissione.
\end{osservation}

\begin{lemma}
  Per coppie buone $ (Y,A) $ la proiezione al quoziente
  \[
    q \colon (Y,A) \to \left(\quot{Y}{A}, \quot{A}{A} \right)
  \]
  induce un isomorfismo:
  \[
    q_\star \colon H_n(Y,A) \to  H_n(\quot{Y}{A}, \quot{A}{A})
  \]
\end{lemma}
\begin{proof}
  Essendo $ A $ retratto di deformazione di $ V $ esiste
  una mappa di inclusione $ i \colon A \to V $, per la funtorialità
  sono ben definite le mappe a livello di omologia:
  \[
    \begin{tikzcd}
      H_n(Y,A) \rar{i_\star} \dar{q_\star} & H_n(Y,V) \dar{q_\star} \\
      H_n(\quot{Y}{A}, \quot{A}{A}) \cong \tilde{H}_n(Y,A) \rar & H_n(\quot{Y}{A}, \quot{V}{A})
    \end{tikzcd}
  \]
  $  H_n({Y} \slash {A}, {A} \slash {A}) \cong \tilde{H}_n({Y} \slash {A}) $ in quanto il quoziente
  di $ A $ con sé stesso fa collassare $ A $ in un punto, quindi il gruppo di omologia
  è relativo ad un punto, e quindi è l'omologia ridotta.
  Ho la terna $ (A, V, Y) $ tale che $ A \subseteq V \subseteq Y $ allora c'è l'inclusione
  $ (Y,A) \incl (Y,V) $. A questa inclusione corrisponde la successione esatta lunga:
  \[
    \begin{tikzcd}
      \dots \rar & H_{n+1}(V, A) \rar & H_n(Y, A) \rar & H_n(Y,V) \rar & H_{n}(V,A) \rar & \dots
    \end{tikzcd}
  \]
  Ma $ V $ è omotopa ad $ A $ quindi $ H_n(V,A) \cong H_n(A,A) $ per l'assioma dell'omotopia.
  Ma $ H_n(A,A) \cong 0 $, infatti:
  $ H_n(A,A) \cong 0 \; \forall k $ in quanto il gruppo di omologia relativa
  di $ A $ con $ A $ stesso è definito dalla successione:
  \[
    \begin{tikzcd}
      0 \rar   & H_k(A) \rar  & H_k(A) \rar  & H_k(A,A) \rar & 0
    \end{tikzcd}
  \]
  Ma $ H_k(A) \cong H_k(A) $ quindi  $ H_k(A,A) \cong \quot{H_k(A)}{H_k(A)} \cong 0 $.
  La successione diventa:
  \[
    \begin{tikzcd}
      0 \rar & H_n(Y, A) \rar & H_n(Y,V) \rar & 0
    \end{tikzcd}
  \]
  Quindi $ H_k(Y,A) \cong H_n(Y,V) $.
  Se mostro che anche $ \quot{A}{A} $ è retratto di $ \quot{V}{A} $
  allora per  gli stessi motivi sono isomorfi
  $ H_n(\quot{Y}{A}, \quot{A}{A}) $ e $ H_n(\quot{Y}{A}, \quot{V}{A}) $.
  % siccome $ V $ ha lo stesso tipo di omotopia di $ A $.
  Quindi ho:
  \begin{align*}
    i \colon A & \to V \\
    r \colon V & \to A
  \end{align*}
  E $ \quot{A}{A} $ è retratto di $ \quot{V}{A} $,
  Infatti compongo $ i $ e $ r $ con le proiezioni al quoziente:
  \begin{align*}
    j \colon \quot{A}{A} & \to \quot{V}{A} \\
    \rho \colon \quot{V}{A} & \to \quot{A}{A}
  \end{align*}
  Sono tali che $ \rho \circ j = \Id{\quot{A}{A}} $ e $ j \circ \rho = \Id{\quot{V}{A}} $,
  in quanto per ipotesi $ r $ è retrazione per $ i $ e
  quindi $ \rho $ è retrazione per $ j $.
  Io ho $ A \subseteq V \subseteq Y $, faccio l'escissione di $ A $:
  \[
    \begin{tikzcd}
      H_n(Y,A) \rar{\cong} \dar{q_\star} & H_n(Y,V)  \dar{q_\star}& H_n({Y} \setminus {A}, {V} \setminus {A}) \lar{\cong} \dar{q_\star} \\
      H_n(\quot{Y}{A}, \quot{A}{A}) \rar{\cong} & H_n(\quot{Y}{A}, \quot{V}{A}) & H_n(\quot{Y}{A} \setminus \quot{A}{A}, \quot{V}{A} \setminus \quot{A}{A}) \lar{\cong}
    \end{tikzcd}
  \]
  Ho $ \quot{A}{A} \subseteq \quot{V}{A} \subseteq \quot{Y}{A} $.
  Inoltre
  \[
    H_k(\quot{Y}{A}, \quot{V}{A}) \cong H_n(\quot{Y}{A} \setminus \quot{A}{A}, \quot{V}{A} \setminus \quot{A}{A})
  \]
  per l'assioma di escissione che posso applicare in quanto vale che $ \bar{A} \subset \mathrm{int}(V) $.
  In questo è necessario che la coppia sia buona, ma nei CW complessi è sempre così, come si può
  verificare.
  \begin{exercise}
    Dimostrare che la coppia formata da un $ k $-scheletro e un $ k-1 $-scheletro è buona.
  \end{exercise}
  La $ q_\star $ di destra è un isomorfismo perché la sua restrizione sul complementare di $ A $
  in $ Y $ è un omeomorfismo. Per la commutatività del diagramma $ q_\star $ è isomorfismo.
\end{proof}
\begin{corollary}
  Se $ (Y,A) $ è una coppia buona allora vale che $ \tilde{H}_k(\quot{Y}{A}) \cong H_k(Y,A) $.
\end{corollary}

\begin{lemma}
  Vale che:
  \[
    \tilde{H}_k(\Sph{n}_{\alpha_1} \vee \dots \vee \Sph{n}_{\alpha_t}) \cong \bigoplus \tilde{H}_k(\Sph{n}_{\alpha_j}) \cong
    \begin{cases}
      \Z^t & \text{se $ k = n $} \\
      0 & \text{se $ k \not =  n $}
    \end{cases}
  \]
  dove $ t $ è il numero di sfere.
\end{lemma}
\begin{proof}
  Lavoro con $ n $ fissato, conosco l'omologia delle sfere,
  in particolare quella ridotta è:
  \[
    \tilde{H}_k(\Sph{n}) \cong
    \begin{cases}
      \Z     & \text{se } k = n \\
      0      & \text{se } k \not = n
    \end{cases}
  \]
  Per $ k = 0 $ e $ k = 1 $ so calcolare i gruppi di omologia perché sono
  il gruppo fondamentale è il suo abelianizzato,
  mi metto quindi nel caso $ k \geq 2 $.
  Nel caso $ k \geq 2 $ omologia ridotta coincide con quella usuale.
  So anche calcolare i gruppi di omologia nel caso di una sfera,
  cioè $ t = 1 $.

  La dimostrazione è per induzione:
  suppongo di conoscere $ \tilde{H}_k(\Sph{n}_1 \vee \dots \Sph{n}_{t-1}) $
  voglio calcolare $ \tilde{H}_k(\Sph{n}_1 \vee \dots \Sph{n}_t) $.
  Come notazione pongo $ Z_t = \Sph{n}_1 \vee \dots \Sph{n}_t $ e $ B = \Sph{n}_t $,
  cioè vale che $ Z_t = Z_{t-1} \vee B $.

  L'ipotesi induttiva è:
  \[
    \tilde{H}_k(Z_{t-1}) \cong
    \begin{cases}
      \Z^{t-1} & \text{se $ k = n $} \\
      0 & \text{se $ k \not = n $}
    \end{cases}
  \]
  Siccome ci sono delle naturali mappe di inclusione vale
  la successione esatta lunga in omologia relativa:
  \[
    \begin{tikzcd}[nodes={column sep=10pt}]
      \dots \rar & H_k(Z_{t-1}) \rar & H_k(Z_t) \rar & H_k(Z_t, Z_{t-1}) \rar & H_{k-1}(Z_{t-1}) \rar & \dots
    \end{tikzcd}
  \]
  Se $ k \not = n $ e siccome $ k \geq 2 $ allora $ H_k(Z_{t-1}) \cong 0 $ per ipotesi induttiva.
  % e $  H_{k-1}(Z_{t-1}) \cong 0 $
  Ma come dimostrato nel lemma precedente vale che
  \[
    H_k(Y,A) \cong \tilde{H}_k(\quot{Y}{A})
  \]
  Quindi:
  \[
    \begin{tikzcd}
      H_k(Z_t, Z_{t-1}) \cong \tilde{H}_k(\quot{Z_t}{Z_{t-1}}) \cong \tilde{H}_k(\Sph{n}_t) \cong 0 %\tilde{H}_k(B)
    \end{tikzcd}
  \]
  quindi la successione è:
  \[
    \begin{tikzcd}
      0 \rar   & H_k(Z_t) \rar  & 0
    \end{tikzcd}
  \]
  e quindi $ H_{k}(Z_t) = 0 $ siccome la successione è esatta.

  Se invece $ k = n $ allora vale la successione esatta:
  \[
    \begin{tikzcd}[nodes={column sep=7pt}]
      \dots \rar & H_{n+1}(Z_t, Z_{t-1}) \rar & H_n(Z_{t-1}) \rar & H _n(Z_t) \rar  & H_n(Z_t, Z_{t-1}) \rar & H_{n-1}(Z_{t-1}) \rar & \dots
    \end{tikzcd}
  \]
  Ma
  \[
    H_{n+1}(Z_t, Z_{t-1}) \cong \tilde{H}_{n+1}(\quot{Z_t}{Z_{t-1}}) \cong \tilde{H}_{n+1}(\Sph{n}) \cong 0
  \]
  E:
  \[
    H_{n}(Z_t, Z_{t-1}) \cong \tilde{H}_{n}(\quot{Z_t}{Z_{t-1}}) \cong \tilde{H}_{n}(\Sph{n}) \cong \Z
  \]
  Mentre $ H_{n-1}(Z_{t-1}) \cong 0 $ e $ H_{n}(Z_{t-1}) \cong \Z^{t-1} $ per ipotesi induttiva quindi:
  \[
    \begin{tikzcd}
      0 \rar   & \Z^{t-1} \rar & H_n(Z_t) \rar & \Z \rar & 0
    \end{tikzcd}
  \]
  Quindi siccome la successione è spezzante $ H_n(Z_t) \cong \Z^t $.
  [PERCHÈ LA SUCCESSIONE SPEZZA?]
\end{proof}

\begin{lemma}
  Sia $ X $ un CW complesso finito i cui $ k $-scheletri sono $ X^{(0)} \subseteq  \dots \subseteq X^{(n)} \subseteq \dots \subseteq X^{(N)} = X $,
  allora vale che:
  \begin{enumerate}
  \item
    \[
      H_k(X^{(n)}, X^{(n-1)}) \cong
      \begin{cases}
        \Z^{a_n} & \text{se } k = n \text{ con $ a_n $ numero di $ n $-celle} \\
        0 & \text{se } k \not = n
      \end{cases}
    \]
  \item
    \[
      H_k(X^{(n)}) \cong
      \begin{cases}
        0 & \text{per } k > n \\
        H_k(X) & \text{per } k < n
      \end{cases}
    \]
  \end{enumerate}
\end{lemma}
\begin{proof}
  \begin{enumerate}
  \item
    % Considero $ q \colon (Y,A) \to \left( \quot{Y}{A}, \quot{A}{A} \right) $ (lo spazio di arrivo
    % è uno spazio puntato in cui il quoziente rispetto $ A $ fa collassare $ A $ in un punto).
    % Si ha il risultato che dimostro dopo:
    % Se $ (Y,A) $ è una coppia buona allora:
    % \[
    %   H_k(Y,A) \cong H_k(\quot{Y}{A}, \quot{A}{A}) \cong \tilde{H}_k(\quot{Y}{A})
    % \]
    La coppia $ (X^{(n)}, X^{(n-1)}) $ è una coppia buona, quindi vale che:
    \[
      H_k(X^{(n)}, X^{(n-1)}) \cong \tilde{H}_k(\quot{X^{(n)}}{X^{(n-1)}})
    \]
    Ma
    \[
      X^{(n)} = X^{(n-1)} \cup_f \Disk{n}_{\alpha_i} \cup_{f_1} \dots \cup_{f_t} \Disk{n}_{\alpha_t} = \Sph{n}_{\alpha_1} \vee \dots \vee \Sph{n}_{\alpha_t}
    \]
    L'identificazione $ \quot{X^{(n)}}{X^{(n-1)}} $ fa collassare i bordi in un punto, quindi ottengo
    un bouquet.
    Per il lemma precedente:
    \[
      \tilde{H}_k(\Sph{n}_{\alpha_1} \vee \dots \vee \Sph{n}_{\alpha_t}) \cong \bigoplus \tilde{H}_k(\Sph{n}_{\alpha_j})
    \]
    % Quindi ho automaticamente il risultato che volevo mostrare.
    Se $ k > n $ l'omologia di ogni sfera è nulla, mentre
    se $ k < n $ è $ \Z $ per ogni cella.
  \item
    Considero la successione esatta della coppia $ (X^{(n)}, X^{(n-1)}) $:
    \[
      \begin{tikzcd}[nodes={column sep=7pt}]
        \dots \rar & H_{k+1}(X^{(n)}, X^{(n-1)}) \rar & H_k(X^{(n-1)}) \rar & H_k(X^{(n)}) \rar & H_{k}(X^{(n)}, X^{(n-1)}) \rar & \dots
      \end{tikzcd}
    \]
    Nel punto precedente ho calcolato i gruppi di omologia relativa:
    se $ k \not \in \set{n, n-1} $ allora sia
    $ H_{k+1}(X^{(n)}, X^{(n-1)}) $ che $ H_k(X^{(n)}, X^{(n-1)}) $ sono nulli
    quindi la successione diventa:
    \[
      \begin{tikzcd}
        0 \rar   & H_k(X^{(n-1)}) \rar  & H_k(X^{(n)}) \rar & 0
      \end{tikzcd}
    \]
    Quindi $ H_k(X^{(n-1)}) \cong H_k(X^{(n)}) $. Noto che per $ k \not = 0 $ vale che
    $ H_k(X^{(0)}) \cong 0 $ in quanto $ X^{(0)} $ sono punti,
    ma quindi:
    \[
      H_k(X^{(n)}) \cong  H_k(X^{(n-1)}) \cong H_k(X^{(n-2)}) \cong \dots \cong H_k(X^{(0)}) \cong 0
    \]
    Quindi per $ k > n $ sono tutti banali in quanto sicuramente $ k \not \in \set{n, n-1} $,

    Se $ k < n $ considero la successione esatta lunga della coppia $ (X^{(n+1)}, X^{(n)}) $:
    \[
      \begin{tikzcd}[nodes={column sep=7pt}]
        \dots \rar & H_{k+1}(X^{(n+1)}, X^{(n)}) \rar & H_k(X^{(n)}) \rar & H_k(X^{(n+1)}) \rar & H_{k}(X^{(n+1)}, X^{(n)}) \rar & \dots
      \end{tikzcd}
    \]
    Se $ k < n $ sicuramente $ k + 1 \not = n + 1 $ quindi $ H_{k+1}(X^{(n+1)}, X^{(n)}) \cong 0 $,
    ma anche $ H_k(X^{(n+1)}, X^{(n)}) \cong 0 $
    quindi ho la successione:
    \[
      \begin{tikzcd}
       0\rar & H_k(X^{(n)}) \rar & H_k(X^{(n+1)}) \rar & 0
      \end{tikzcd}
    \]
    Da cui $ H_k(X^{(n)}) \cong H_k(X^{(n+1)}) $.
    Quindi:
    \[
      H_k(X^{(n)}) \cong H_k(X^{(n+1)}) \dots \cong H_k(X^{(N)}) = H_k(X)
    \]
  \end{enumerate}
\end{proof}

Sia $ X $ un CW complesso di tipo finito, voglio costruire un complesso $ (S_\bullet^{CW}, d^{CW}) $ e
voglio mostrare che l'omologia di questo complesso, detta omologia cellulare, è isomorfa con
l'omologia singolare:
\[
  H_k^{CW}(X) = H_k(S^{CW}_\bullet(X)) \qquad H_k^{CW} = H_k(X^{(k)}, X^{(k-1)})
\]
% $ S_k^{CW}(X) := H_k(X^{(k)}, X^{(k-1)}) $
So che $ (X^{(k+1)}, X^{(k)}) $ è una coppia
e ho la successione esatta in omologia:
\[
  \begin{tikzcd}[nodes={column sep=10pt}]
    \dots \rar  & H_{k+1}(X^{(k+1)}) \rar & H_{k+1}(X^{(k+1)}, X^{(k)}) \rar & H_k(X^{(k)})   \rar & H_k(X^{(k+1)}) \rar & \dots
  \end{tikzcd}
\]
Poi ho la coppia $ (X^{(k)}, X^{(k-1)}) $ e quindi la successione
\[
  \begin{tikzcd}[nodes={column sep=10pt}]
    \dots \rar & H_{k+1}(X^{(k)}, X^{(k-1)}) \rar & H_k(X^{(k-1)}) \rar & H_k(X^{(k)}) \rar & H_{k}(X^{(k)}, X^{(k-1)}) \rar & \dots
  \end{tikzcd}
\]
Incrociando le successioni e considerando che $ H_k(X^{(k+1)}, X^{(k)}) \cong 0$:
\[
  \begin{tikzcd}[nodes={column sep=1pt, inner sep=2pt, outer sep=1pt}]
  0            \arrow{dr}{}      & {}                   & {}                                         & {}                           & 0  & {}\\
  {}                              & H_k(X^{(k-1)})   \arrow{dr}{\sigma}                 & {}                                    & H_k(X^{(k+1)})  \arrow{ur}{} & {} & {}\\
  {}                             & {}             & H_k(X^{(k)})   \arrow{dr}{j_k}   \arrow{ur}{\tau} & {}                           & {} & {}\\
  {}  & H_{k+1}(X^{(k+1)}, X^{(k)})  \arrow{ur}{\delta_{k+1}} \arrow{rr}{d_{k+1}^{CW}} & {}      & H_k(X^{(k)}, X^{(k-1)})  \arrow{rr}{d_{k}^{CW}}   \arrow{dr}{\delta_k}  & {} &  H_{k-1}(X^{(k-1)}, X^{(k-2)}) \\
  \dots \arrow{ur}{} & {} & {} & {} & H_{k-1}(X^{(k-1)}) \arrow{ur}{j_{k-1}} \arrow{dr}{} & {} \\
  {} & {} & {} & \dots \arrow{ur}{} & {} & \dots
\end{tikzcd}
\]
Cioè definisco $ d_k^{CW} = j_{k-1} \circ \delta_k $. Devo mostrare che questo è un complesso, cioè $ d^2 = 0 $,
quindi posso definire l'omologia:
\[
   d_k^{CW} \circ  d_{k+1}^{CW} = j_{k-1} \circ \delta_k \circ j_k \circ \delta_{k+1} = 0
\]
Infatti $ \delta_k \circ j_k $ è composizione in una successione esatta quindi è nulla.
$ d^{CW} $ è un operatore di bordo.

\begin{definition}
  Sia $ X $ un CW complesso, siano $ S_k^{CW}(X) := H_k(X^{(k)}, X^{(k-1)}) $ e $ d_k^{CW} = j_{k-1} \circ \delta_k $
  con $ j_k \colon H_k(X^{(k)}) \to H_k(X^{(k)}, X^{(k-1)}) $ e $ \delta_k \colon H_k(X^{(k)}, X^{(k-1)}) \to  H_{k-1}(X^{(k-1)})  $,
  allora si definisce \textbf{omologia cellulare}\index{Omologia cellulare}
  come l'omologia del complesso $ (S_\bullet^{CW}, d^{CW}) $.
\end{definition}

\begin{proposition}
  L'omologia cellulare è isomorfa all'omologia singolare.
\end{proposition}
\begin{proof}
  Ora voglio mostrare che l'omologia è isomorfa a quella singolare, devo mostrare
  che:
  \[
    H_k(X) \cong H^{CW}_k(X) := \quot{\ker{d_k^{CW}}}{\im{d_{k+1}^{CW}}}
  \]
  Avevo mostrato che
  $ H_k(X^{(k-1))}) = 0 $ quindi se $ n = k - 1 $ allora $ H_k(X^{(k-1)}) = 0 $ quindi
  $ j_k $ è iniettiva in quanto $ \ker{j_k} = \im{\sigma} = 0 $.
  Osservo che siccome $ \tau $ è suriettiva $ \im{\tau} = H_k(X^{(k+1)}) $, ma per
  il teorema fondamentale degli omeomorfismi $ \quot{H_k(X^{(k)})}{\ker{\tau}} \cong \im{\tau} $
  quindi $ H_k(X^{(k+1)}) \cong \quot{H_k(X^{(k)})}{\ker{\tau}} $. Ma $ H_k(X^{(k+1)}) \cong H_k(X) $,
  quindi ho che: $ \quot{H_k(X^{(k)})}{\ker{\tau}} \cong H_k(X) $. Inoltre siccome
  la successione è esatta $ \ker{\tau} = \im{\delta_{k+1}} $ quindi
  nel complesso ho che
  \[
    H_k(X) \cong \quot{H_k(X^{(k)})}{\im{\delta_{k+1}}}
  \]
  Inoltre ho $ j_k \colon H_k(X^{(k)}) \to H_k(X^{(k)}, X^{(k-1)}) $ che è tale che:
  \[
    j_k\left(\im{\delta_{k+1}}\right) = \im{j_k \circ \delta_{k+1}}
  \]
  infatti se $ z \in j_k\left(\im{\delta_{k+1}}\right) $ allora esiste $ u $ tale che
  $ z = j_k(\delta_{k+1}(u)) = j_k \circ \delta_{k+1}(u) $, e se $ w \in \im{j_k \circ \delta_{k+1}} $ allora esiste $ r $ tale che
  $ w = j_k \circ \delta_{k+1}(r) = j_k(\delta_{k+1}(r)) $.
  Quindi $ j_k\left(\im{\delta_{k+1}}\right) = \im{d_{k+1}^{CW}} $, e perciò\footnote{Usando il fatto che se
  $ j $ è iniettiva $ \quot{A}{B} \cong \quot{j(A)}{j(B)} $}:
  \[
    H_k(X) \cong \quot{j_k\left(H_k(X^{(k)})\right)}{\im{d^{CW}_{k+1}}}
  \]
  % Se faccio vedere che $ \im{d_{k+1}^{CW}} = \ker{d_k^{CW}} $ ho finito.
  Mi rimane da mostrare che $ j_k\left(H_k(X^{(k)})\right) \cong \ker{d_k^{CW}} $.
  Per l'esattezza vale che:
  \[
    j_k\left(H_k(X^{(k)})\right) = \im{j_k} = \ker{\delta_k}
  \]
  Ma $ \ker{\delta_k} = \ker{d_k^{CW}} $ in quanto se $ z \in \ker{\delta_k} $:
  \[
    \delta_k(z) = 0 \overset{\text{iniettività}}{\iff} j_{k-1} \circ \delta_k(z) = 0 \Rightarrow z \in \ker{j_k \circ \delta_k} = \ker{d_k^{CW}}
  \]
  % Ho $ j_{k-1} \colon \ker{\delta_k} \to \ker{d_k^{CW}} $ e $ j_{k-1} \left(\ker{\delta_k} \right) \subseteq \ker{d_{k}^{CW}} $.
  Infine, se $ w \in \ker{d_k^{CW}} $ allora $ 0 = d_k^{CW}(w) = j_{k-1} \circ \delta_k (w) $,
  ma $ j_{k-1} $ è iniettivo e perciò $ \delta_k(w) = 0 $ quindi $ w \in \ker{\delta_k} $.
\end{proof}


\begin{osservation}
  So che $ S_k^{CW}(X) = H_k(X^{(k)}, X^{(k-1)}) \cong \Z^{a_k} $ con $ a_k $ numero di celle,
  cosa posso dire su $ d_k^{CW} \colon S_k^{CW}(X) \to S_{k-1}^{CW}(X) $ ?

  Siccome $ S_k^{CW}(X) \cong \Z^{a_k} $ e c'è un fattore $ \Z $ per ogni cella
  posso considerare $ S_k^{CW}(X) $ generato da una base formata da
  $ k $-celle $ e_1, \dots, e_{a_k} $, e similmente $ S_{k-1}^{CW}(X) $ generato
  da $ k - 1 $-celle $ f_1, \dots, f_{a_{k-1}} $. Siccome $ d_k^{CW}(e_j) $ è un elemento in
  $ S_{k-1}^{CW}(X) $ si può scrivere come combinazione lineare a coefficienti interi di $ f_m $:
  \[
    d^{CW}_k (e_j) = \sum_m A_{jm}f_m
  \]
  Coma si calcolano gli $ A_{jm} $?
  Prendo $ e_j $ un generatore di $ H_k(X^{(k)}, X^{(k-1)}) \cong \bigoplus_j H_k(\Sph{k}_j) $.
  $ e_j $ genera il bordo di una cella.

  Posso rileggere gli $ H_k $:
  \[
    H_k(\Sph{k}_k) \cong H_k(\quot{\Disk{k}}{\partial \Disk{k}}) \cong H_k(\Disk{k}, \partial \Disk{k}) =
    H_k(\Disk{k}, \Sph{k-1}) \cong H_{k-1}(\Sph{k-1})
  \]
  Sono partito da $ \Sph{k} $ e sono arrivato in $ \Sph{k-1} $.
  $ d_k^{CW} $ è la mappa in omologia indotta dall'applicazione:
  \[
    \begin{tikzcd}
      \partial \Disk{k} \rar{\phi_k} \arrow{dr}{\eta_j} & X^{(k-1)} \dar & {}\\
      {} & \quot{X^{(k-1)}}{X^{(k-2)}} = \bigvee_\alpha \Sph{k-1}_\alpha \rar{u_\beta} & \Sph{k-1}_\beta
    \end{tikzcd}
  \]
  Allora $ A_{jb} = \deg{(u_\beta \circ \eta_j)} $.
\end{osservation}

 % lezione 12
%  _     _____ ________ ___  _   _ _____   _ ____
% | |   | ____|__  /_ _/ _ \| \ | | ____| / |___ \
% | |   |  _|   / / | | | | |  \| |  _|   | | __) |
% | |___| |___ / /_ | | |_| | |\  | |___  | |/ __/
% |_____|_____/____|___\___/|_| \_|_____| |_|_____|

\subsection{Calcolo dell'omologia cellulare di alcuni spazi}

\subsubsection{Spazi $ V_g $}

Gli spazi $ V_g $ sono definiti a partire dai poligoni
regolari con $ 4g $ lati, quozientati con l'identificazione
$ a_1 b_1 a_1^{-1} b_1^{-1} \dots a_g b_g a_g^{-1} b_g^{-1} $.
Di questi spazi conosco già $ H_0(V_g) \cong \Z $ in quanto
sono connessi per archi e $ H_1(V_g) \cong \Z^{2g} $ in quanto
conosco il gruppo fondamentale.

A questi spazi è possibile dare la struttura di CW complessi
con una $ 0 $-cella che è il punto in cui tutti i vertici del
poligono collassano sotto la proiezione $ \pi \colon P_{4g} \to V_g $
quindi la $ 0 $-cella è $ x = \pi(v) $, dove $ v $ sono i vertici.
Poi $ 2 g $ $ 1 $-celle $ \alpha_1, \dots, \alpha_g, \beta_1, \dots, \beta_g $ con
$ \alpha_i = \pi(a_i) $ e $ \beta_i = \pi(b_1) $. La funzione di attaccamento
è:
\begin{align*}
  f_1 \colon \partial \Disk{2} & \to X^{(0)} \\
  \pm 1 & \mapsto x
\end{align*}
Infine una $ 2 $-cella
che è l'immagine dell'interno, cioè ottenuta con la mappa
di attaccamento
\[
  f \colon \Sph{1} \to X^{(1)} = \Sph{1}_1 \vee \dots \vee \Sph{1}_{2g}
\]
Calcolo l'omologia di $ V_g $ con $ g \geq 1 $, il complesso
è:
\[
  \begin{tikzcd}
    0 \rar & S^{CW}_2(V_g) \rar & S^{CW}_(V_g) \rar & S^{CW}_0(V_g) \rar & 0
  \end{tikzcd}
\]
Ma:
\begin{gather*}
  S^{CW}_0(V_g) = H_0(X^{(0)}, X^{(-1)}) = H_0(X^{(0)}, \emptyset) = H_0(X^{(0)}) = H_0(X) \cong Z \\
  S^{CW}_1(V_g) = H_1(X^{(1)}, X^{(0)}) \cong \Z^{2g} \\
  S^{CW}_2(V_g) = H_2(X^{(2)}, X^{(0)}) \cong \Z
\end{gather*}
Quindi il complesso è:
\[
  \begin{tikzcd}
    0 \rar & \Z \rar{d_2} & \Z^{2g} \rar{d_1} & \Z \rar{d_0} & 0
  \end{tikzcd}
\]
Voglio calcolare i gruppi di omologia di questo complesso a partire dalla
definizione di omologia e so che questi gruppi sono isomorfi ai gruppi di
omologia singolare. $ d_0 $ è la mappa nulla per costruzione quindi
$ \ker{d_0} \cong \Z $, mentre cosa è $ d_1 \colon \Z^{2g} \to \Z $? $ \Z^{2g} $ è in
corrispondenza con le celle, quindi posso prende come generatori $ e_n $ celle
in $ X^{(1)} $. $ e_n $ è una $ 1 $-cella, quindi
$ e_n \in H_1(X^{(1)}, X^{(0)}) $, ma
$ H_1(X^{(1)}, X^{(0)}) \cong H_1(\quot{X^{(1)}}{X^{(0)}}) $, inoltre
$ X^{(1)} = X^{(0)} \cup_{f_1} \Disk{1}_1 \dots \cup_{f_{2g}} \Disk{1}_{2g} $. Quando
collasso i bordi ad un punto mi rimane un bouquet:
\[
  = h_1(\Sph{1}_1 \vee \dots \vee \Sph{1}_{4g}) \cong \bigoplus_n H_1(\Sph{1}_n)
\]
Quindi $ e_n \in H_1(\Sph{1}_n) $ per qualche $ n $. Poi ho:
\[
  \begin{tikzcd}
    \Sph{1}_n \rar{r} \arrow{dr}{} & X^{(1)} \dar{s} \\
    \quot{X^{(1)}}{X^{(0)}} & {}
  \end{tikzcd}
\]
E $ d_1(e_n) = \sum A_{kn}f_k $, con  $ A_{kn} = \deg {(r \circ s)} $.
[...]
Quello che trovo è $ \im{d} = 0 $, quindi:
\[
  H_0^{CW}(V_g) = \quot{\ker{d_}}{\im{d_1}} = \Z
\]
Faccio la stessa cosa con $ d_2 $, ho $ \im{d_2} = 0 $ e $ \ker{d_2} = \Z^{2g} $.
Considero:
\[
  \begin{tikzcd}
    \partial \Disk{2} \rar{f} & X^{(1)} \dar  \\
    {} & \quot{X^{(1)}}{X^{(0)}}
  \end{tikzcd}
\]
La mappa verticale non fa nulla in quanto fa collassare un punto in un punto.
Quindi ho:
\[
  \begin{tikzcd}
    \Sph{1} \rar{f} & \Sph{1} \vee \dots \Sph{1} \\
    {} & \Sph{1}
  \end{tikzcd}
\]
xHo $ d_2 \colon S^{CW}_2(V_g) \to S_1^{CW}(V_g) $ cioè:
$ d_2 \colon H_2(X^{(2)}, X^{(1)}) \to H_1(X^{(1)}, X^{(0)}) $,
ma:
\[
  H_2(X^{(2)}, X^{(1)}) \cong H_2(\quot{X^{(2)}}{X^{(1)}}) \cong H_2(\Sph{2}) \cong H_2(\Disk{2}, \partial \Disk{1})
  \cong H_2(\Disk{2}, \Sph{1}) \cong H_1(\Sph{1})
\]
Quindi:
\begin{align*}
  d_2 \colon H_1(\Sph{1}) & \to H_1(\Sph{1}_1 \vee \dots \vee \Sph{1}_{2g}) \\
  1 & \mapsto 0
\end{align*}
Infatti $ a_1 + b_1 - a_1 - b_1 \dots = 0$
Quindi $ \ker{d_2} = \Z $ e $ \im{d_2} = 0 $,
$ H_2^{CW}(V_g) = \quot{\ker{d_2}}{\im{d_3}} \cong \Z $ e
$ H_1^{CW}(V_g) = \quot{\ker{d_1}}{\im{d_2}} \cong \Z^{2g} $.
Si nota che
\[
  \rank{H_0(V_g)} - \rank{H_1(V_g)} + \rank{H_2(V_g)} = 1 - 2g + 1 = 2 - 2g
\]
Questa è la caratteristica di Eulero di $ V_g $: $ \chi(V_g) = 2 - 2g $.

\begin{osservation}
  Se $ X $ è un CW complesso finito allora $ H_k(X) = 0 $ se non ci sono $ k $-celle,
  infatti $ H_k(X) \cong H_k^{CW} \cong 0 $ se non ci sono $ k $-celle.
\end{osservation}

\begin{osservation}
  Se $ X $ è un CW complesso finito con $ a_n $ $ n $-celle allora $ H_n(X) $ è
  generato da al più $ a_n $ elementi, infatti $ H_n(X) \cong H_n^{CW}(X) $ che è
  quoziente di $ S_n^{CW}(X) \cong \Z^{a_n} $.
\end{osservation}

\begin{corollary}
  Se $ X $ è CW complesso finito $ H_k(X) $ è un gruppo abeliano finitamente
  generato, infatti $ H_k(X) \cong H_k^{CW}(X) $ quoziente di un gruppo abeliano
  libero finitamente generato.
\end{corollary}

\begin{theorem}[Teorema di struttura per gruppi abeliani liberi finitamente generati]
  \index{Teorema di struttura per gruppi abeliani liberi finitamente generati}
  Se $ \mathcal{G} $ è un gruppo abeliano libero finitamente generato di rango
  $ \rank{\mathcal{G}} $ allora:
  \[
    \mathcal{G} \cong \Z^{\rank{\mathcal{G}}} \oplus T_k
  \]
  Dove $ T_k $ è il sotto gruppo di torsione di $ \mathcal{G} $.
\end{theorem}

\begin{definition}[Numero di Betti e caratteristica di Eulero]
  Se $ X $ è un CW complesso allora si definisce il $ k $-esimo
  numero di Betti come $ b_k(X) = \rank{H_k(X)} $, e si definisce
  la caratteristica di Eulero di $ X $ come:
  \[
    \chi(X) = \sum_{k=0}^{N} (-)^k b_k(X)
  \]
\end{definition}

\subsubsection{Spazi proiettivi}

\begin{osservation}
  Se $ X $ è un CW complesso che non ha celle in indici consecutivi
  allora $ H_k(X) $è abeliano libero con una base in corrispondenza
  con le $ k $-celle.
\end{osservation}
\begin{proof}
  Infatti ho il complesso:
  \[
    \begin{tikzcd}
      \dots \rar & S^{CW}_{k+1}(X) \rar & S_k^{CW}(X) \rar & S_{k-1}^{CW}(X) \rar & \dots
    \end{tikzcd}
  \]
  Cioè:
  \[
    \begin{tikzcd}
      \dots \rar & \Z^{a_{n=1}} \rar & \Z^{a_n} \rar & \Z^{a_{n-1}} \rar & \dots
    \end{tikzcd}
  \]
  Almeno alcuni di questi sono zeri, quindi alcuni differenziali sono zero.
\end{proof}

Ad esempio se $ Y = \Pjc{n} $ ho una $ 0 $-cella, una $ 1 $-cella, \dots, e una
$ n $-cella, quindi la struttura del complesso è:
\[
  \begin{tikzcd}
    \dots \rar & S^{CW}_{2n}(Y) \rar & 0 \rar & S_{2n - 2}^{CW}(X) \rar & \dots \rar & 0
  \end{tikzcd}
\]
Quindi $ H_{2n}^{CW}(Y) = \quot{S^{CW}_{2n}(Y)}{\set{0}} \cong S_{2n}^{CW}(Y) \cong \Z $,
ecc, cioè:
\[
  H_{2k}^{CW}(Y) \cong \Z \text{ per } k \in \set{0, \dots, n}
\]
La caratteristica di Eulero è:
\[
  \chi(Y) = \sum_{k=0}^N (-)^{2k} = n + 1
\]

\begin{example}
  Sia $ X_1 = \Sph{n} $, $ X_2 = \Sph{n} $ e $ Z = X_1 \times X_2 $, $ Z $ è un CW
  complesso finito, e siccome $ \Sph{n} $ ha una $ 0 $-cella e una $ n $-cella,
  allora $ Z $ ha una $ 0 $-cella, due $ n $-celle e una $ 2n $-cella.
  Quindi ho:
  \[
    \begin{tikzcd}
      0 \rar & S_{2n}^{CW}(Z) \rar & 0 \rar & \dots \rar & S_n^{CW}(Z) \rar & \dots \rar & 0
    \end{tikzcd}
  \]
  Quindi:
  \[
    H_k(Z) \cong
    \begin{cases}
      \Z & \text{se } k \in \set{0, 2n} \\
      \Z^2 & \text{se } k = n \\
      0 & \text{se } k \not \in \set{0, n, 2n}
    \end{cases}
  \]
\end{example}

\subsubsection{Spazi proiettivi reali}
Lo spazio proiettivo reale è più complicato. So che
$ \Sph{1} \simeq \Pjr{1} $ e $ U_1 \simeq \Pjr{2} $.

Le celle del proiettivo reale sono una $ 0 $-cella, una $ 1 $-cella, \dots, una
$ n $-cella, e siccome non ci sono buchi non si può usare lo stesso ragionamento
di prima. So che;
\[
  X^{(k+1)} = X^{(k)} \cup_f \Disk{k+1} \text{ con } f \colon \partial \Disk{k+1} \to X^{(k)}
\]
Cioè $ f \colon \Sph{k} \to X^{(k)} = \Pjr{k} \cong \quot{\Sph{k}}{\sim} $ quindi
$ f $ è la proiezione. Ho il complesso per $ \Pjr{2} $:
\[
  \begin{tikzcd}
    0 \rar & S^{CW}_{2}(\Pjr{2}) \rar & S_1^{CW}(\Pjr{2}) \rar & S_{0}^{CW}(\Pjr{2}) \rar & 0
  \end{tikzcd}
\]
Cioè:
\[
  \begin{tikzcd}
    0 \rar{t_3} & \Z \rar{t_2} & \Z \rar{t_1} & \Z \rar{t_0} & 0
  \end{tikzcd}
\]
Devo calcolare nuclei e immagini. So che:
\[
  \begin{cases}
    \im{t_3} = 0 \\
    \ker{t_3} = 0
  \end{cases}
\]
E:
\[
  \begin{cases}
    \im{t_0} = 0\\
    \ker{t_0} = \Z
  \end{cases}
\]
Trovo che:
\begin{align*}
  t_2 \colon & \Z \to \Z \\
  x & \mapsto 2 x
\end{align*}
Ho il diagramma:
\[
  \begin{tikzcd}
    \Sph{1} = \partial \Disk{2} \rar{\mathcal{A}} \dar{\mathcal{A}} & \Pjr{1} \dar{\Id{}} \\
    {} & \Pjr{1}
  \end{tikzcd}
\]
La mappa $ \mathcal{A} $ è la proiezione sul gruppo generato dalla mappa antipodale.
Siccome è una mappa 2 a 1 viene quello che ho scritto.
So che $ H_1(\Pjr{2}) \cong \quot{\Z}{2 \Z} $ quindi:
\[
  \quot{\ker{t_1}}{\im{t_2}} \cong \quot{\Z}{2 \Z}
\]
$ t_1 \colon \Z \to \Z $ viene da:
\[
  \begin{tikzcd}
    \Sph{0} \rar{\Id{}} \arrow{rd}{} & \Pjr{0} \dar{\Id{}} \\
    {} & \Pjr{0}
  \end{tikzcd}
\]
Quindi il grado è zero.

Nel complesso ho che:
\[
  \begin{tikzcd}
    0 \rar{0} & \Z \rar{2} & \Z \rar{0} & \Z \rar{0} & 0
  \end{tikzcd}
\]
Trovo che: $ H_0 \cong \Z $, $ H_1 \cong \quot{\Z}{2 \Z} $, $ H_2 \cong 0 $.
In generale ho:
\[
  \begin{tikzcd}
    \Sph{k} \rar \arrow{rd}{} & \Pjr{k} \\
    {} & \quot{\Pjr{k}}{\Pjr{k-1}} \cong \Sph{k}
  \end{tikzcd}
\]
Cioè ho che:
\[
  \begin{tikzcd}
    0 \rar{0} & \Z \rar{t_n} & \Z \rar{t_{n-1}} & \dots \rar{t_1 }& \Z \rar{t_0} & 0
  \end{tikzcd}
\]
In generale il comportamento è differente per $ n $ pari o per $ n $
dispari, nel caso $ n = 3 $:
\[
  \begin{tikzcd}
    0 \rar{0} & \Z \rar{t_3} & \Z \rar{t_{2}} & \Z \rar{t_1}& \Z \rar{t_0} & 0
  \end{tikzcd}
\]
$ d_4 \colon 0 \to \Z $, quindi $ \im{t_4} = 0 $ e $ \ker{t_4} = 0 $,
mentre $ d_3 \colon \Z \to \Z $ e ho:
\[
  \begin{tikzcd}
    \Sph{2} = \partial \Disk{3} \rar \arrow{rd}{\mathcal{A}} & \Pjr{3} \\
    {} & \quot{\Pjr{3}}{\Pjr{2}} \cong \Sph{2}
  \end{tikzcd}
\]
Si trova che il grado di $ \mathcal{A} $ è zero e:
\begin{align*}
  t_3 \colon \Z & \to \Z \\
  x & \mapsto \deg{t_3} x
\end{align*}
Alla fine si ha che:
\[
  H_k(\Pjr{3}) =
  \begin{cases}
    \Z & \text{se } k = 0 \\
    \Z_2 & \text{se } k = 1 \\
    0 & \text{se } k = 2 \\
    \Z & \text{se } k = 3
  \end{cases}
\]

         %          lezione 13
Sia $ \Pjr{n} $ ho il complesso:
\[
  \begin{tikzcd}
    0 \rar & \Z \rar & \Z \rar & \dots & \Z \rar & 0
  \end{tikzcd}
\]
Ad esempio per $ n = 3 $ ho:
\[
  \begin{tikzcd}
    0 \rar & \Z \rar & \Z \rar & \Z \rar  & \Z \rar & 0
  \end{tikzcd}
\]
I bordi sono:
\[
  \begin{tikzcd}
    \Sph{2} = \partial \Disk{3} \rar \arrow{dr}{\Delta} & \Pjr{2} \dar \\
    {} & \quot{\Pjr{2}}{\Pjr{1}} \simeq \Sph{1}
  \end{tikzcd}
\]
Devo trovare il grado di $ \Delta $. Se $ \alpha $ è un generatore di
$ H_2(\Sph{2}) $ allora si definisce il grado di $ \Delta $ con
$ \Delta_\star(\alpha) = \deg{\Delta} \alpha $. Un generatore di $ \Sph{2} $ è
$ \tau_2 \colon \Disk{2} \to \Sph{2} $ mappa proiezione sul bordo.

[FIGURA]

Poi ho $ \tau' \colon \Disk{2} \to \Sph{2} $ identificazione.
\[
  [\tau'] \in H_2(\Sph{2}) \Rightarrow [\tau'] = m[\tau_2] = \deg{\Delta_\star}[\tau-2]
\]
Cioè
$ \tau_2 \big \lvert_{\partial \Disk{2}} = \tau' \big \lvert_{\partial \Disk{2}} $ sul bordo si
comportano come l'identità, cioè la cella viene mandata in $ \Sph{2} $ meno un
punto. Poi $ A \circ \tau_2 \big \lvert_{\mathrm{int} \Disk{2}} = \tau' \big \lvert_{\mathrm{int} \Disk{2}} $
con $ A $ mappa antipodale, quindi:
\[
  \deg{\Delta_\star} = 1 + (-1)^{2 + 1} = 1 - 1 = 0
\]
Quindi il complesso è:
\[
  \begin{tikzcd}
    0 \rar & \Z \rar{0} & \Z \rar{2} & \Z \rar{0}  & \Z \rar & 0
  \end{tikzcd}
\]
Questo si generalizza immediatamente a $ n $ generico.
Per $ n $ pari:
\[
  \begin{tikzcd}
    0 \rar & \Z \rar{2} & \dots & \rar & \Z \rar & \Z \rar{0} & \Z \rar & 0
  \end{tikzcd}
\]
Per $ n $ dispari:
\[
  \begin{tikzcd}
    0 \rar & \Z \rar{0} & \dots & \rar & \Z \rar & \Z \rar{2} & \Z \rar & 0
  \end{tikzcd}
\]
Si ha l'alternanza di applicazione costante e moltiplicazione per 2.

\section{Successione di Mayer-Vietoris}
\index{Successione di Mayer-Vietoris}

\begin{theorem}
  Sia $ X $ uno spazio topologico e $ A $ e $ B $ sottospazi aperti di $ X $ con la
  topologia indotta, se $ X = A \cup B $ allora esiste la successione esatta di complessi:
  \[
    \begin{tikzcd}
      0 \rar & S_p(A \cap B) \rar & S_p(A) \oplus S_p(B) \rar & S_p(A \cup B) \rar & 0
    \end{tikzcd}
  \]
  Quindi esiste una successione esatta lunga in omologia:
  \[
    \begin{tikzcd}[nodes = {column sep = 10 pt}]
      \dots \rar & H_p(A \cap B) \rar & H_p(A) \oplus H_p(B) \rar & H_p(A \cup B) \rar{\delta} & H_{p-1}(A \cap B) \rar & \dots
    \end{tikzcd}
  \]
  % $ \delta $ è l'omomorfismo di connessione ed è ben definito nel solito modo.
\end{theorem}
\begin{proof}
  Esistono le mappe di inclusione sono $ i_A \colon A \cap B \incl A $ e $ i_B \colon A \cap B \incl B $ quindi
  è ben definita:
  \begin{align*}
    i_\sharp \colon S_p(A \cap B) & \to S_p(A) \oplus S_p(B) \\
    c & \mapsto (i_A^\sharp (c), i_B^\sharp (c))
  \end{align*}
  Ma ci sono anche le inclusioni $ j_A \colon A \incl X $ e $ j_B \colon B \incl X $,
  quindi è ben definita
  \begin{align*}
    j_\sharp \colon  S_p(A) \oplus S_p(B) & \to S_p(A \cup B) = S_p(X) \\
    (a,b) &  \mapsto j_A^\sharp(a) - j_B^\sharp(b)
  \end{align*}
  La successione è esatta, infatti sia $ c \in S_p(A \cap B) $:
  \[
    j_\sharp \circ i_\sharp (c) = (i_A^\sharp (c), i_B^\sharp (c)) = i_A^\sharp (c) -  i_B^\sharp (c) = 0
  \]
  In quanto sugli elementi di $ S_p(A \cap B) $ $ i_A^\sharp $ e $ i_B^\sharp $ agiscono
  allo stesso modo.
\end{proof}

\begin{osservation}
  Questa non è la forma più generale del teorema di Mayer-Vietoris, il quale
  ammette anche la possibilità che $ A $ e $ B $ non siano aperti ma che
  $ X = \mathrm{int} A \cup \mathrm{int} B $, tuttavia questa possibilità si rivela
  necessaria solo in casi patologici.
\end{osservation}
Nel dimostrare il seguente teorema si dà per noto il seguente risultato, la cui
dimostrazione è noiosa e poco istruttiva:
\begin{lemma}
  Se $ f \colon \Disk{n} \to \Sph{n} $ è un embedding allora $ \tilde{H}_k(\Sph{r} \setminus f(\Disk{r})) = 0 $.
\end{lemma}

\begin{theorem}[Teorema di Jordan generalizzato\index{Teorema di Jordan generalizzato}]
  Sia $ f \colon \Sph{r} \to \Sph{n} $ un \textbf{embedding}\index{Embedding}, cioè una funzione
  continua tale che $ f(\Sph{r}) \simeq \Sph{r} $, allora:
  \[
    \tilde{H}_i(\Sph{n} \setminus f(\Sph{r})) \cong
    \begin{cases}
      \Z & \text{se } i = n - r - 1 \\
      0 & \text{se }  i \not = n - r - 1
    \end{cases}
  \]
  Ovvero $ \tilde{H}_i(\Sph{n} \setminus f(\Sph{r})) \cong \tilde{H}_i(\Sph{n - r - 1}) $.
\end{theorem}
\begin{proof}
  La dimostrazione è per induzione su $ r $. Per $ r = 0 $ ho che
  $ \Sph{0} = \set{+1,-1} $, e quindi $ f(\Sph{0}) = \set{p,q} $ essendo un
  embedding. Ho che:
  \[
     \Sph{n} \setminus f(\Sph{0}) = (\Sph{n} \setminus \set{p}) \setminus \set{q} \simeq \RN{n} \setminus \set{\vec{0}} \sim \Sph{n-1}
  \]
  Quindi siccome l'omologia è invariante omotopico:
  \[
    \tilde{H}_i(\Sph{n} \setminus f(\Sph{0})) \cong \tilde{H}_i(\Sph{n - 1})
  \]

  Suppongo di conoscere il risultato per $ r - 1 $: sia
  $ f \colon \Sph{r} \to \Sph{n} $ embedding, considero i due emisferi
  $ \Disk{r}_+ $ e $ \Disk{r}_- $, vale che:
  $ \Disk{r}_+ \, \cup \, \Disk{r}_- = \Sph{r} $ e
  $ \Disk{r}_+ \, \cap \, \Disk{r}_- = \Sph{r-1} $. Sia
  $ U_+ = \Sph{n} \setminus f(\Disk{r}_+) $ e
  $ U_- = \Sph{n} \setminus f(\Disk{r}_-) $, intendo usare Mayer-Vietoris, infatti $ U_- $ e
  $ U_+ $ sono aperti in quanto sono complementari di chiusi in $ \Sph{n} $.
  Ho che:
  \begin{gather*}
    U_+ \cup U_- = \left(\Sph{n} \setminus f(\Disk{r}_+) \right) \cup \left( \Sph{n} \setminus f(\Disk{r}_-) \right) =
    \Sph{n} \setminus (f(\Disk{r}_+) \cap  f(\Disk{r}_-)) = \\
   \overset{\text{$ f $ è embedding}} {=}  \Sph{n} \setminus (f(\Disk{r}_+  \cap  \Disk{r}_-)) = \Sph{n} \setminus f(\Sph{r-1})
  \end{gather*}
  Mentre:
  \begin{gather*}
    U_+ \cap U_- = \left(\Sph{n} \setminus f(\Disk{r}_+) \right) \cap \left(\Sph{n} \setminus f(\Disk{f}_-) \right) =
    \Sph{n} \setminus (f(\Disk{r}_+) \cup  f(\Disk{r}_-)) = \\
    \overset{\text{$ f $ è embedding}} {=}  \Sph{n} \setminus (f(\Disk{r}_+  \cup  \Disk{r}_-)) = \Sph{n} \setminus f(\Sph{r})
  \end{gather*}
  Per Mayers-Vietoris c'è:
  \[
    \begin{tikzcd}[nodes={column sep = 5pt, inner sep = 2pt}]
      \dots \rar & H_{i+1}(U_+) \oplus H_{i+1}(U_-) \rar & H_{i+1}(U_+ \cup U_-) \rar & H_i(U_+ \cap U_-) \rar & H_i (U_+) \oplus H_i(U_-)
      \rar & \dots
    \end{tikzcd}
  \]
  Da cui, utilizzando il precedente lemma ($ H_i(U_\pm) \cong 0 $):
  \[
    \begin{tikzcd}
      0 \rar & H_{i+1}(\Sph{n} - f(\Sph{r-1})) \rar & H_i(\Sph{n} \setminus f(\Sph{r})) \rar & 0
    \end{tikzcd}
  \]
  Da cui passando all'omologia ridotta
  $ \tilde{H}_i(\Sph{n} \setminus f(\Sph{r})) \cong \tilde{H}_{i+1}(\Sph{n} - f(\Sph{r-1})) \cong \tilde{H}_{i+1}(\Sph{n-r}) \cong \tilde{H}_i(\Sph{n-r-1}) $
  per ipotesi induttiva.
\end{proof}

Questo risultato generalizza il teorema di Jordan\index{Teorema di Jordan} che
dice che se $ C $ è una curva semplice (cioè che non si autointerseca) chiusa in
$ \RN{2} $ allora $ C $ divide $ \RN{2} $ in due componenti connesse.

\begin{example}
  Sia $ f \colon \Sph{1} \to \Sph{2} $ embedding allora:
  \[
    \tilde{H}_i(\Sph{2} \setminus f(\Sph{1})) \cong
    \begin{cases}
      \Z & \text{se } i = 0 \\
      0 & \text{se } i \not = 0
    \end{cases}
  \]
\end{example}

\begin{proposition}
  Il teorema di Jordan generalizzato implica il teorema di Jordan.
\end{proposition}
\begin{proof}
  Ho $ \Sph{2} \setminus \set{p} \cong \RN{2} $, per proiezione stereografica, sia
  $ f \colon \Sph{1} \to \Sph{2} $ embedding, $ f(\Sph{1}) $ è una curva chiusa
  semplice, la denoto con $ C = f(\Sph{1})$. Mostro che
  $ H_0(\RN{2} \setminus \set{q}) = \Z^2 $ e quindi
  $ \RN{2} \setminus \set{q} $ ha due componenti connesse. Siccome $ \Sph{2} $ è una
  varietà topologica ammette un intorno di $ q $ che denoto con $ D $ che è
  omeomorfo a $ \Disk{2} $. Poi prendo un intorno $ D' $ contenuto in $ D $.
  Se $ K = \Sph{2} \setminus C $ allora:
  \begin{gather*}
    (K \setminus D') \cup D = K \\
    (K \setminus D') \cap D' \sim_H \Sph{1} \; (\text{infatti è una corona circolare})
  \end{gather*}
  Inoltre $ D \sim_H \set{q} $ quindi $ K \setminus D' \sim K \setminus \set{q} $.
  [FIGURA]
  % So che
  % \[
  %   \tilde{H}_i(\Sph{2} \setminus C ) \cong
  %   \begin{cases}
  %     \Z & \text{se } i = 0 \\
  %     0 & \text{se } i \not = 0
  %   \end{cases}
  % \]
  % Ci sono due componenti connesse, infatti per $ i = 0 $ l'omologia è $ \Z^2 $ (infatti
  % l'omologia ridotta toglie uno $ \Z $).
  % % Bisogna dimostrare che $ \tilde{H}_1(\Sph{2} \setminus C) \cong \tilde{H}_1(\RN{2} \setminus C) $.
  % Mostro che $ \tilde{H}_0(\Sph{2} \setminus C) \cong \tilde{H}_0(\RN{2} \setminus C) $.
  % Un modo per farlo è  con Mayers-Vietoris. Sia $ q \in C $,
  % $ \Sph{2} \setminus \set{p} \simeq H_{\RN{2}} \setminus C' $ con $ C' \simeq C $.
  % Voglio mostrare che  $ \tilde{H}_0(\Sph{2} \setminus C) \cong \tilde{H}_0(\RN{2} \setminus C) $
  % implica $ H_0((\Sph{2} \setminus C) \setminus \set{q}) \cong H_0(\RN{2} \setminus C) $.
  % So che per il teorema di Jordan generalizzato $ H_0(\Sph{2} \setminus C) \cong \Z^2 $,
  % devo mostrare che $ H_0(\Sph{2} \setminus C \setminus \set{q}) \cong H_1(\Sph{2} \setminus C) $.
  % Ma questo è ovvio perché togliere un punto non sconnette. In realtà questo non
  % è vero.
  % Voglio mostrare che $ H_q(\Sph{2} \setminus C) \cong H_q((\Sph{2} \setminus C) \setminus \set{q}) $.
  % Prendo $ V(q) $ intorno di $ q $ omeomorfo a $ D $, poi prendo $ D' $
  % e considero $ \Sph{2} \setminus C = K $, quindi $ K \setminus D $.
  % [FIGURA]
  % Uso Mayer-Vietoris $ K = (K \setminus D') \cup D $. So che $ D \sim P $ punto e
  % $ (K \setminus D') \cap D \sim \Sph{1} $ (è una corona circolare).
  Quindi per Mayers-Vietoris ho la successione esatta lunga ($ P $ è l'insieme
  formato da un solo punto):
  \[
    \begin{tikzcd}[nodes={column sep=3pt, outer sep = 1pt, inner sep = 1pt}]
      \dots \rar & H_1(\Sph{1}) \rar & H_1(K \setminus \set{q}) \oplus H_1(P) \rar & H_1(K) \rar & H_0(\Sph{1}) \rar & H_0(K \setminus \set{q})
      \oplus H_0(P) \rar & H_0(K) \rar & 0
    \end{tikzcd}
  \]
  So che $ H_1(K) \cong 0 $ per il teorema di Jordan generalizzato e $ H_0(K) \cong \Z^2 $,
  quindi la successione si riduce a:
  \[
    \begin{tikzcd}%[nodes={column sep=3pt, outer sep = 1pt, inner sep = 1pt}]
      0 \rar & H_0(\Sph{1}) \rar & H_0(K \setminus \set{q}) \oplus H_0(P) \rar & H_0(K) \rar & 0
    \end{tikzcd}
  \]
  Cioè:
  \[
    \begin{tikzcd}%[nodes={column sep=3pt, outer sep = 1pt, inner sep = 1pt}]
      0 \rar & \Z \rar & H_0(K \setminus \set{q}) \oplus \Z \rar & \Z^2 \rar & 0
    \end{tikzcd}
  \]
  E quindi $ H_0(K \setminus \set{q}) \cong \Z^2 $.
\end{proof}

\begin{theorem}[Invarianza topologica della dimensione]\index{Invarianza topologica della dimensione}
  Se $ M $ è una varietà topologica di dimensione $ m $ e $ N $ una varietà
  topologica di dimensione $ n $ con $ M \simeq N $ allora $ m = n $, cioè la dimensione
  di una varietà topologica è un invariante topologico: se due spazi topologici sono
  omeomorfi allora hanno la stessa dimensione.
\end{theorem}
\begin{proof}
  Mostro inizialmente che la dimensione di una varietà topologica è legata al
  gruppo di omologia della sfera. Sia $ x \in M $ allora siccome $ M $ è una
  varietà topologica esiste un intorno aperto di $ x $ $ \Disk{m}(x) $, questo
  intorno è omeomorfo al disco aperto $ m $-dimensionale. Sia
  $ U = M \setminus \Disk{m}(x) $, $ U $ è chiuso perché complementare in $ M $ di un
  aperto. Vale che $ \bar{U} = U \subseteq M \setminus \set{x} $ perciò
  $ U \subseteq M \setminus \set{x} \subseteq M $ e quindi posso fare l'escissione:
  \[
    H_i(M, M \setminus \set{x}) \cong H_i(M \setminus U, M \setminus U \setminus \set{x})
  \]
  Ma:
  \[
    M \setminus U  = M \setminus (M \setminus \Disk{m}(x)) = \Disk{m}(x)
  \]
  Quindi:
  \[
    H_i(M, M \setminus \set{x}) \cong H_i(\Disk{m}(x), \Disk{m}(x) \setminus \set{x}) \cong H_i(\Disk{m}, \Disk{m} \setminus \set{\vec{0}})
  \]
  In particolare passando all'omologia ridotta, indicando con $ \Disk{m}_0 =  \Disk{m} \setminus \set{\vec{0}}$:
  \[
    \tilde{H}_i(M, M \setminus \set{x}) \cong  \tilde{H}_i(\Disk{m}, \Disk{m}_0)
  \]
  \begin{figure}[htbp]
    \centering
    \begin{tikzpicture}
      \fill[gray!30] plot [smooth cycle] coordinates {(0,0) (1,1) (3,1) (4,0) (2,-2)};
      \draw plot [smooth cycle] coordinates {(0,0) (1,1) (3,1) (4,0) (2,-2)};
      \fill[gray!10] (2,0) circle (0.6);
      \node () at (2,0) {\textbullet};
      % \node[right] () at (2.6,0) {$ V $};
      \node[above] () at (2,0) {$ x $};
      \draw (2,0) circle (0.6);
      \node[above] () at (2,-1.8) {$ U $};
      \node[right] () at (4,0) {$ M $};
    \end{tikzpicture}
    \caption{Situazione}
    \label{fig:lez13:dimension_topological_invariance}
  \end{figure}

  L'immersione di $ \Disk{m}_0 $ in $ \Disk{m} $ induce una
  successione esatta lunga in omologia relativa ridotta:
  \[
    \begin{tikzcd}[nodes = {column sep = 7pt}]
      \dots \rar & \tilde{H}_k(\Disk{m}_0) \rar & \tilde{H}_k(\Disk{m}) \rar & \tilde{H}_k(\Disk{m}, \Disk{m}_0) \rar &
      \tilde{H}_{k-1}(\Disk{m} \setminus \set{0}) \rar & \tilde{H}_{k-1}(\Disk{m}) \rar & \dots
    \end{tikzcd}
  \]
  Cioè essendo l'omologia ridotta dei dischi sempre nulla:
  \[
    \begin{tikzcd}[nodes = {column sep = 10pt}]
      0 \rar & \tilde{H}_k(\Disk{m}) \rar & \tilde{H}_k(\Disk{m}, \Disk{m} \setminus \set{0}) \rar &
      0
    \end{tikzcd}
  \]
  Per cui:
  \[
     \tilde{H}_i(\Disk{m}, \Disk{m}_0) \cong \tilde{H}_{i-1}(\Disk{m}_0)
  \]
  Ma $ \Disk{m}_0 \sim_H \Sph{m-1} $ quindi
  $ \tilde{H}_i(\Disk{m}_0) \cong \tilde{H}_i(\Sph{m-1}) $
  e perciò:
  \[
    \tilde{H}_i(M, M \setminus \set{x}) \cong \tilde{H}_i(\Sph{m-1})
  \]
  A questo punto diventa semplice collegare due varietà differenti: se $ M \simeq N $
  allora:
  \[
    \tilde{H}_i(M, M \setminus \set{x}) \cong \tilde{H}_i(N, N \setminus \set{y})
  \]
  Cioè:
  \[
    \tilde{H}_i(\Sph{m-1}) \cong \tilde{H}_i(\Sph{n-1})
  \]
  Quindi necessariamente $ m = n $.
\end{proof}
\begin{osservation}
  Non vale il viceversa, come ad esempio un toro e una sfera, che hanno la stessa
  dimensione topologica ma non sono omeomorfi.
\end{osservation}

%%% Local Variables:
%%% ispell-local-dictionary: "italiano"
%%% mode: latex
%%% TeX-master: "notes"
%%% End:

\chapter{Coomologia singolare}

% Lezione 15

Si è trovato che per $ n $ pari:
\[
  H_i(\Pjr{n}) \cong
  \begin{cases}
    \Z & \text{se } i = 0 \\
    \Z_2 & \text{se $ i $ pari e $ i < n $} \\
    0 & \text{altrimenti}
  \end{cases}
\]
Mentre per $ n $ dispari:
\[
  H_i(\Pjr{n}) \cong
  \begin{cases}
    \Z & \text{se } i = 0,n \\
    \Z_2 & \text{se $ i $ pari e $ i < n $} \\
    0 & \text{altrimenti}
  \end{cases}
\]
Queste espressioni sono poco simmetriche, ma è possibile
migliorarne l'estetica cambiando i coefficienti con il
prodotto tensore.

\section{Prodotto tensore}

Sia $ A, B $ gruppi abeliani, è ben definito il prodotto
cartesiano:
\[
  A \times B = \set{(a,b) | a \in A, b \in B}
\]
Sia $ F(A,B) $ il gruppo libero generato dalle coppie $ (a, b) \in A \times B $.
% in notazione additiva.
% Il gruppo $ F(A,B) $ è abeliano in quanto
% $ A $ e $ B $ lo sono, e l'operazione di somma è:
% \[
%   (a_1, b_1 ) + (a_2, b_2) = (a_1 + a_2, b_1 + b_2)
% \]
\newmathsymb{tensprod}{\otimes}{Prodotto tensore}
\begin{definition}
  Se $ A, B $ sono $ \Z $-moduli si definisce il \textbf{prodotto
    tensore}\index{Prodotto tensore} tra $ A $ e $ B $, come:
  \[
    A \otimes B = \quot{F(A,B)}{R(A,B)}
  \]
  Dove $ F(A,B) $ è il gruppo libero generato da $ A \times B $
  % con operazione $ (a_1, b_1) + (a_2, b_2) = (a_1 + a_2, b_1 + b_2) $,
  e $ R(A,B) $ il gruppo generato in $ F(A,B) $ dalle espressioni:
  \begin{gather*}
    (a_1 + a_2, b) - (a_1, b) - (a_2, b) \\
    (a, b_1 + b_2) - (a, b_1) - (a, b_2) \\
    n (a, b) - (na, b) \\
    n (a, b) - (a, nb)
  \end{gather*}.
  Gli elementi di $ A \otimes B $ sono generati dai simboli
  $ a \otimes b $ con $ a \in A $ e $ b \in B $ imponendo le relazioni:
  \begin{gather*}
    (a_1 + a_2) \otimes b = a_1 \otimes b + a_2 \otimes b \\
    a \otimes (b_1 + b_2) = a \otimes b_1 + a \otimes b_2 \\
    n (a \otimes b) = (na) \otimes b \\
    n (a \otimes b) = a \otimes (nb)
  \end{gather*}
  Infatti il quoziente manda a zero le espressioni in $ R(A,B) $.
\end{definition}
\begin{osservation}
  Il generico elemento di $ A \otimes B $ non è della forma $ a \otimes b $, m
  a è una combinazione lineare di oggetti di questo tipo, detti
  tensori puri, i quali generano $ A \otimes B $ come modulo.
\end{osservation}
\begin{example}
  Si parla di cambiamento di coefficienti in questo senso: considero
  $ \RN{3} \otimes \mathbb{C} $, una base di questo spazio è data da:
  \[
    \set{e \otimes 1, e \otimes i, f \otimes 1, f \otimes i, g \otimes 1, g \otimes i}
  \]
  Il generico elemento di $ \RN{3} \otimes \mathbb{C} $ è:
  \begin{gather*}
    a_1(e \otimes 1) + a_2(e \otimes i) + a_3(f \otimes 1) + a_4(f \otimes i) + a_5(g \otimes 1) + a_6(g \otimes i) = \\
    = (a_1e + a_3f + a_5g) \otimes 1 + (a_2e + a_4f + a_6g) \otimes i = \\
    = e \otimes (a_1 + a_2i) + f \otimes (a_3 + a_4i) + g \otimes (a_5 + a_6i)
  \end{gather*}
  Nell'ultima riga si vede che ora si hanno dei vettori nella base di partenza
  ma con coefficienti complessi. Questa è la complessificazione di $ \RN{3} $.
\end{example}
\begin{proposition}[Proprietà universale\index{Proprietà universale! \vedi{Prodotto tensore}}]
  Sia $ G $ un gruppo abeliano e $ \psi \colon A \times B \to G $ un'applicazione bilineare continua,
  allora esiste un unico omomorfismo $ \phi \colon A \otimes B \to G $ tale che il diagramma:
  \[
    \begin{tikzcd}
      A \times B \rar{\psi} \dar{\pi} & G \\
      A \otimes B \arrow{ru}{\phi} & {}
    \end{tikzcd}
  \]
  è commutativo, con:
  \begin{align*}
    \pi \colon A \times B & \to A \oplus B \\
    (a,b) & \mapsto a \otimes b
  \end{align*}
  In pratica $ \psi $ fattorizza per il prodotto tensoriale
  ($ \psi = \phi \circ \pi $). La proprietà è detta universale perché esiste mostra che
  esiste un solo prodotto tensoriale, ed è equivalente a dire
  che:
  \[
    \hom{A,B} \cong A^* \otimes B
  \]
\end{proposition}
\begin{proof}
  Si costruisce $ \phi $ in modo tale che
  $ \phi(a \otimes b) = \phi(\pi(a,b)) = \psi(a,b) $, e quindi bisogna solo verificare che sia ben
  definita. Considero un elemento $ c \otimes d $ equivalente a $ a \otimes b $, cioè tali
  che $ (a,b) - (c,d) \in R(A,B) $, devo mostrare che
  $ \phi(a \otimes b) = \phi(c \otimes d) $, cioè che
  $ \psi((a,b)) = \psi((c,d)) $, ovvero che $ \psi((a,b)) - \psi((c,d)) = 0 $, ma
  $ (a,b) - (c,d) \in R(A,B) $ e:
  \[
    \psi((c,d) - (a,b)) = \sum_\alpha \psi((r_\alpha, s_\alpha)) = \sum_\alpha \phi \left(\pi((r_\alpha, s_\alpha))\right) = 0
  \]
  con $ (r_\alpha,s_\alpha) $ base di $ R(A,B) $. Tali elementi al quoziente vanno a zero,
  ma $ \phi $ è un omomorfismo per costruzione (dato che per ipotesi $ \psi $ lo è, e
  il prodotto tensoriale è bilineare) quindi $ \phi(\pi((r_\alpha, s_\alpha))) = 0 $.
\end{proof}
\begin{example}
  Siano $ V, W $ spazi vettoriali reali, gli spazi
  $ \tilde{V} = V \otimes \mathbb{C} $, $ \tilde{W} = W \otimes \mathbb{C} $ sono spazi
  vettoriali complessi. La proprietà universale permette di estendere in modo
  univoco le funzioni lineari $ f \colon V \to W $ a funzioni lineari
  $ \tilde{f} : \tilde{V} \to \tilde{W} $ e quindi lavorare con $ \tilde{V} $ e
  $ \tilde{W} $ esattamente come se fossero usuali spazi vettoriali complessi.
\end{example}
Un'altra importante proprietà del prodotto tensore è il suo comportamento
rispetto agli omomorfismi.

\begin{proposition}
  Siano $ f \colon A \to B $ e $ g \colon A' \to B' $ omomorfismi, posso definire l'azione di
  sui generatori di $ A \otimes A' $:
  \begin{align*}
    f \otimes g \colon A \otimes A' & \to B \otimes B' \\
    a \otimes a' & \to f(a) \otimes g(a')
  \end{align*}
  Estendendo per linearità si definisce $ f \otimes g $ su tutto $ A \otimes A' $, il quale è
  per definizione omomorfismo di gruppi abeliani.
\end{proposition}

\begin{proposition}
  Vale che $ A \otimes B \cong B \otimes A $, cioè il prodotto tensore è simmetrico.
\end{proposition}
\begin{proof}
  Se per la proprietà universale (con $ G = B \otimes A $) trovo una mappa bilineare continua
  $ \psi \colon A \times B \to A \otimes B $ allora esiste un omomorfismo
  $ \phi_1 \colon A \otimes B \to B \otimes A $, quindi posso scambiare $ A $ e
  $ B $ e trovare un secondo omomorfismo
  $ \phi_2 \colon B \otimes A \to A \otimes B $, e quindi mostrare che $ \phi_1 $ e $ \phi_2 $ sono inverse.
  Sia:
  \begin{align*}
    \psi \colon A \times B & \to B \otimes A \\
    (x,y) & \mapsto y \otimes x
  \end{align*}
  Questa applicazione è continua e bilineare, allora per l'universalità sono ben
  definite $ \phi_1 $ e $ \phi_2 $, e:
  \[
    \begin{tikzcd}[nodes = {row sep=5pt}]
      A \otimes B \rar{\phi_1} & B \otimes A \rar{\phi_2} & A \times B \\
      a \otimes b \arrow[mapsto]{r} & b \otimes a \arrow[mapsto]{r} & a \otimes b
    \end{tikzcd}
  \]
  Quindi $ \phi_1 \circ \phi_2 = \Id{A \otimes B} $, e analogamente  $ \phi_2 \circ \phi_1 = \Id{B \otimes A} $.
\end{proof}
\eproof
Un'ulteriore proprietà da analizzare è il comportamento rispetto alle successioni
esatte. Considero una successone esatta corta di $ \Z $-moduli:
\[
  \begin{tikzcd}
    0 \rar{} & R \rar{\alpha} & F \rar{\beta} & A \rar & 0
  \end{tikzcd}
\]
Considero $ G $ gruppo abeliano, allora ho:
\[
  \begin{tikzcd}
    R \otimes G \rar{\alpha'} & F \otimes G \rar{\beta'} & A \otimes G
  \end{tikzcd}
\]
Questa successione è esatta? Per verificarlo utilizzo un lemma:
\begin{lemma}
  Se $ A $ è uno $ \Z $-modulo allora $ A \otimes \Z \cong A $.
\end{lemma}
\begin{proof}
  Costruisco esplicitamente l'isomorfismo. Siano $ \tau $ e $ \sigma $ definiti da:
  \begin{align*}
    \tau \colon A & \to A \otimes \Z \\
    a & \mapsto a \otimes 1
  \end{align*}
  E:
  \begin{align*}
    \sigma \colon A \otimes \Z & \to A \\
    \tilde{a} \otimes n & \mapsto n \tilde{a}
  \end{align*}
  Mostro che sono omomorfismi:
  \[
    \tau(a + b) \otimes 1 = a \otimes 1 + b \otimes 1 = \tau(a) + \tau(b)
  \]
  E:
  \begin{gather*}
    \sigma(\tilde{a} \otimes n + \tilde{b} \otimes m ) = \sigma(n \tilde{a} \otimes 1 + m \tilde{b} \otimes
    1) = \sigma ((n \tilde{a} + m\tilde{b}) \otimes 1) = \\ = n \tilde{a} + m \tilde{b} =
    \sigma(\tilde{a} \otimes n) + \sigma(\tilde{b} \otimes m)
  \end{gather*}
  Poi $ \sigma $ e $ \tau $ sono inversi, infatti:
  \[
    \begin{tikzcd}[nodes={row sep = 5 pt}]
      A \rar{} & A \otimes \Z \rar & A \\
      a \arrow[mapsto]{r}{\tau} & a \otimes 1 \arrow[mapsto]{r}{\sigma} & a
    \end{tikzcd}
  \]
  E:
  \[
    \begin{tikzcd}[nodes={row sep = 5 pt}]
      A \otimes \Z \rar & A \rar & A \otimes Z \\
      a \otimes n \arrow[mapsto]{r}{\sigma} & n \tilde{a} \arrow[mapsto]{r}{\tau} & n \tilde{a} \otimes 1 = \tilde{a} \otimes n
    \end{tikzcd}
  \]
  Quindi $ \tau $ e $ \sigma $ costituiscono isomorfismi tra $ A \otimes \Z $ e $ A $.
\end{proof}
\begin{example}
  Considero la successione esatta corta:
  \[
    \begin{tikzcd}
      0 \rar{} & n \Z \rar{\alpha} & \Z \rar{\beta} & \quot{\Z}{n \Z} \rar & 0
    \end{tikzcd}
  \]
  In particolare per $ n = 6 $:
  \[
    \begin{tikzcd}
      0 \rar{} & 6 \Z \rar{\alpha} & \Z \rar{\beta} & \quot{\Z}{6 \Z} \rar & 0
    \end{tikzcd}
  \]
  L'applicazione $ \alpha $ è la moltiplicazione per $ 6 $, mentre $ \beta $ è il passaggio
  alla classe modulo $ 6 $, in questo modo $ \beta \circ \alpha = 0 $. Tensorizzando per $ \Z $:
  \[
    \begin{tikzcd}[nodes = {row sep = 3pt}]
      6 \Z \otimes \Z \rar{\alpha \otimes \Id{}} & \Z \otimes \Z \rar{\beta \otimes \Id{}} & \quot{\Z}{6 \Z} \otimes \Z  \\
      x \otimes y \rar[mapsto] & 6x \otimes y  & {} \\
      {} & x \otimes y \rar[mapsto] & \bar{x} \otimes y \\
    \end{tikzcd}
  \]
  Con $ \bar{x} $ classe modulo 6 di $ x $. La successione è esatta perché
  passando all'isomorfismo descritto nel precedente lemma la successione è:
  \[
    \begin{tikzcd}[nodes = {row sep = 3pt}]
      0 \rar{} & 6 \Z \rar{\alpha} & \Z \rar{\beta} & \Z_6 \rar & 0 \\
    \end{tikzcd}
  \]
  La quale è esatta.
  % Usando il lemma precedente cerco il
  % nucleo di $ \alpha \otimes \Id{} $, quindi risolvo $ 6 xy = 0 $ da cui
  % $ x = 0 $ o $ y = 0 $, e in entrambi i casi $ x \otimes y = 0 $ è zero, quindi
  % siccome il nucleo è banale $ \alpha \otimes \Id{} $ è inieittiva. L'applicazione
  % $ \beta \otimes \Id{} $ è suriettiva, infatti se prendo $ \bar{k} \otimes u $ con
  % $ \bar{k} \in \Z_6 $ e $ u \in \Z $, la preimmagine è chiaramente
  % $ k \otimes u $. Il nucleo di $ \beta \otimes \Id{} $ è dato da
  % $ \bar{x} \otimes y = 0 $, cioè $ \bar{x} y = 0 $ in $ \Z_6 $ via
  % isomorfismo. Se $ m = 6t $ allora $ \bar{m} = 0 $ quindi
  % $ \ker{\beta \otimes \Id{}} $ contiene $ \Z \otimes \Z $, cioè
  % $ \Z \otimes \Z \subseteq \ker{\beta \otimes \Id{}} $. Ma
  % $ \bar{m} \otimes n = n \bar{m} \otimes \Id{} = \bar{nm} \otimes \Id{} $, quindi
  % $ mn = 6t $, ma $ n $ è intero, quindi $ m $ deve essere multiplo di 6, e
  % quindi è vera anche l'inclusione inversa. La successione è quindi esatta.
\end{example}
\begin{example}
  Considero la stessa successione di prima, ma ora tensorizzo per $ {\Z} \slash {4 \Z}  \cong \Z_4 $:
  \[
    \begin{tikzcd}[nodes = {row sep = 3pt}]
       6 \Z \otimes \Z_4 \rar{\alpha \otimes \Id{}} & \Z \otimes \Z_4 \rar{\beta \otimes \Id{}} & \Z_6 \otimes \Z_4  \\
       x \otimes \bar{y} \rar[mapsto] & 6x \otimes \bar{y} & {}\\
       {} & x \otimes \bar{y} \rar[mapsto] & \bar{x} \otimes \bar{y}
    \end{tikzcd}
  \]
  Considero in particolare l'applicazione:
  \begin{align*}
    6 \Z \otimes \Z_4 & \to \Z \otimes \Z_4 \\
     x \otimes \bar{y} & \mapsto 6x \otimes \bar{y}
  \end{align*}
  Questa ha un nucleo non banale, usando il lemma
  precedente:
  \begin{align*}
    \Z \otimes \Z_4 & \to \Z_4 \\
     6x \otimes \bar{y} & \mapsto \overline{6xy}
  \end{align*}
  E l'elemento $ x = 1 $ e $ y = 2 $, che non è nullo, viene mandato in
  $ \overline{12} $ che è $ 0 $ in $ \Z_4 $.

  % ad esempio l'elemento $ 6 \otimes \bar{2} $ finisce in
  % Cerco il nucleo in isomorfismo: $ 6 x \bar{k} = \bar{0} $, ma questo ha soluzioni non banali,
  % ad esempio $ 6 \otimes \bar{2} $, infatti va nella classe $ \bar{12} $ che è $ \bar{0} $, ma quindi
  % non è iniettiva.
\end{example}
Da questi esempi si nota che in generale successioni esatte non vanno in successioni esatte,
cioè $ R \otimes G \to F \otimes G \to A \otimes G $ non è sempre esatta. Per poter dire qualcosa di generale conviene
fare la seguente osservazione:
\begin{osservation}
Considero $ \alpha \otimes \Id{} \colon R \otimes G \to F \otimes G $ allora:
\[
  \quot{F \otimes G}{(\alpha \otimes \Id{})(R \otimes G)} \cong \quot{F}{\alpha(R)} \otimes G
\]
\end{osservation}
\begin{proof}
  Costruisco esplicitamente l'isomorfismo. Sia $ \eta $ l'omomorfismo definito da:
  \begin{align*}
    \eta \colon \quot{F}{\alpha(R)} \otimes G & \to \quot{F \otimes G}{(\alpha \otimes \Id{})(R \otimes G)} \\
    [\alpha] \otimes g & \mapsto [\alpha \otimes g]'
  \end{align*}
  Questa mappa è ben definita, infatti considero un elemento $ b $ equivalente
  ad $ a $, cioè tale che $ b = a + \alpha(r) $ con $ r \in R $. Ma quindi
  $ b \otimes g = (a + \alpha(r)) \otimes g = a \otimes g + \alpha(r) \otimes g $ e quindi
  $ a \otimes g $ e $ b \otimes g $ differiscono per un elemento
  $ \alpha(r) \otimes g \in (\alpha \otimes \Id{})(R \otimes G) $ e perciò sono equivalenti, verificando che
  $ \eta $ è ben definita in quanto manda elementi equivalenti in elementi
  equivalenti. L'applicazione è quindi ben definita e lineare, l'inversa è
  chiaramente la mappa $ [a \otimes g]' \mapsto [a] \otimes g $, che è ben definita per il
  medesimo ragionamento.
\end{proof}
\eproof
Ma a questo punto $ {F} \slash {\alpha(R)} \otimes G \cong A \otimes G $, infatti per il teorema fondamentale
degli omomorfismi:
\[
  \quot{F}{\im{\alpha}} = \quot{F}{\ker{\beta}} \cong \im{\beta} = A
\]
Quindi
$ A \otimes G \cong {F \otimes G} \slash {{(\alpha \otimes \Id{})(R \otimes G)}} $. In questo modo posso sempre
costruire una successione esatta tensorizzando, rinunciando all'iniettività di
$ \alpha \otimes \Id{} $, ma mantenendo
$ \ker{\beta \otimes \Id{}} = \im{\alpha \otimes \Id{}} $ e $ \beta \otimes \Id{} $ suriettiva:
\[
  \begin{tikzcd}
    0 \rar & \ker{\alpha \otimes \Id{}} \rar{i} & R \otimes G \rar{\alpha \otimes \Id{}} & F \otimes G \rar{\beta \otimes \Id{}} & A \otimes G \rar & 0
  \end{tikzcd}
\]
$ i $ è iniettiva perché è un inclusione, mentre $ \beta \otimes \Id{} $ è suriettiva
in quanto è una proiezione al quoziente. Si mantiene $ \ker{\beta \otimes \Id{}} = \im{\alpha \otimes \Id{}} $
in quanto tensorizzando si perde l'esattezza solo a sinistra.

\begin{definition}
  Se $ A $ è uno $ \Z $-modulo una successione esatta corta del tipo:
  \[
    \begin{tikzcd}
      0 \rar & R \rar{\alpha} & F \rar{\beta} & A \rar & 0
    \end{tikzcd}
  \]
  con $ R $ e $ F $ $ \Z $-moduli liberi è detta \textbf{risoluzione di
    $ A $}\index{Risoluzione di $ A $} oppure \textbf{presentazione di
    $ A $}\index{Presentazione di $ A $! \vedi{Risoluzione di $ A $}}.
\end{definition}

\begin{osservation}
  Esiste sempre almeno una risoluzione di $ A $ ottenuta prendendo $ F $ è il
  gruppo libero generato da $ A $ e $ R $ il gruppo delle relazioni da imporre
  per riottenere $ A $. Tensorizzando:
  \[
    \begin{tikzcd}
      0 \rar & \ker{\alpha \otimes \Id{}} \rar & R \otimes G \rar{\alpha \otimes \Id{}} & F \otimes G \rar{\beta \otimes \Id{}} & A \otimes G \rar & 0
    \end{tikzcd}
  \]
  Potrebbero comunque esserci altre successioni esatte:
  \[
    \begin{tikzcd}
      0 \rar & R' \rar{\alpha} & F' \rar{\beta} & A \rar & 0
    \end{tikzcd}
  \]
  Tensorizzando:
  \[
    \begin{tikzcd}
      0 \rar & \ker{\alpha' \otimes \Id{}} \rar & R' \otimes G \rar{\alpha \otimes \Id{}} & F' \otimes G \rar{\beta \otimes \Id{}} & A \otimes G \rar & 0
    \end{tikzcd}
  \]
\end{osservation}

\newmathsymb{torsion}{\tor{}}{Modulo di torsione}
\begin{definition}
  Si chiama \textbf{modulo di torsione}\index{Modulo di torsione} di $ A $ e di $ G $ il
  gruppo $ \ker{\alpha \otimes \Id{}} $, e lo si indica con $ \tor{A,G} $. Quindi vale che:
    \[
    \begin{tikzcd}
      0 \rar & \tor{A,G} \rar & R \otimes G \rar{\alpha \otimes \Id{}} & F \otimes G \rar{\beta \otimes \Id{}} & A \otimes G \rar & 0
    \end{tikzcd}
  \]
\end{definition}
\begin{lemma}
  Il modulo di torsione non dipende dalla scelta della risoluzione di $ A $, cioè con risoluzioni
  differenti si ottengono moduli di torsione isomorfi.
\end{lemma}

\begin{lemma}
  Se $ F_1 $ è un gruppo libero allora $ \tor{A,F_1} \cong 0 $, e quindi il modulo
  di torsione è dovuto alla parte di torsione di $ G $.
\end{lemma}
\begin{proof}
  Considero una presentazione di $ A $:
  \[
    \begin{tikzcd}
      0 \rar & R \rar & F \rar & A \rar & 0
    \end{tikzcd}
  \]
  Tensorizzo per $ F_1 $:
  \[
    \begin{tikzcd}
      0 \rar & \tor{A,F_1} \rar & R \otimes F_1 \rar{\phi} & F \otimes F_1 \rar & A \otimes F_1 \rar & 0
    \end{tikzcd}
  \]
  La mappa $ \phi = \alpha \otimes \Id{} $ è iniettiva, infatti $ R \cong \Z^r $, $ F \cong \Z^n $ e $ F_1 \cong \Z^{n_1} $,
  quindi $ \phi \colon \Z^r \otimes \Z^{n_1} \to \Z^n \otimes \Z^{n_1} $, cioè:
  \begin{align*}
    \Z^n \otimes \Z^{n_1} & \to  \Z^n \otimes \Z^{n_1} \\
    \vec{v} \otimes \vec{w} & \mapsto \alpha(\vec{v}) \otimes \vec{w}
  \end{align*}
  \vspace*{-20pt}
  \begin{exercise}
    Mostrare che $ \Z^s \otimes \Z^r \cong \Z^{sr} $.
    Hint:  $ \set{e_1 \otimes f_j} $ è una base di $ \Z^s \otimes \Z^r $ se $ \set{e_1} $
    e $ \set{f_j} $ lo sono per $ \Z^s $ e $ \Z^r $, mostrarlo.
  \end{exercise}
  Quindi se $ \set{g_k} $ è una base per $ \Z^{rn_1} $ e $ \set{h_l} $ per $ \Z^{nn_1} $
  si può scrivere:
  \begin{gather*}
     \vec{v} \otimes \vec{w} \; \mapsto \; \sum_k \mu_k g_k \\
     \alpha{\vec{v}} \otimes \vec{w} \; \mapsto \; \sum_l \nu_l h_l
   \end{gather*}
   Considerando $ \vec{u} \otimes \vec{w} $ con $ \vec{u} \not = \vec{w} $ questo
   elemento ha una espansione su $ \set{g_k} $ ma $ \alpha $ è iniettiva quindi
   necessariamente anche $ \alpha({\vec{u}}) \otimes \vec{w} $ ha un'espansione diversa su
   $ \set{h_l} $ e quindi corrisponde ad un elemento diverso in quanto
   $ \set{h_l} $ è una base, questo significa che $ \phi $ è iniettiva e quindi per
   l'esattezza della successione deve essere $ \tor{A, F_1} \cong 0 $.
\end{proof}

\begin{proposition}
  Se $ A $ e $ B $ sono $ \Z $-moduli allora $ \tor{A,B} \cong \tor{B,A} $.
\end{proposition}
\begin{proof}\emph{Sketch of proof. La dimostrazione completa è lunga e articolata}
  La dimostrazione è un diagram chase.
  Considero una risoluzione di $ B $ e di $ A $:
  \[
    \begin{tikzcd}
      0 \rar{} & R_B \rar{\alpha} & F_B \rar{\beta} & B \rar & 0
    \end{tikzcd}
  \]
  \[
    \begin{tikzcd}
      0 \rar{} & R_A \rar{\alpha} & F_A \rar{\beta} & A \rar & 0
    \end{tikzcd}
  \]
  Tensorizzo questa per $ B $:
  \[
    \begin{tikzcd}
      0 \rar{} & \tor{A,B} \rar & R_A \otimes B \rar{\alpha} & F_A \otimes B \rar{\beta} & A \otimes B \rar & 0
    \end{tikzcd}
  \]
  Tensorizzo $ B $ per $ F_A $ e $ R_A $ e $ A $ per $ R_B $ e $ F_B $, usando
  la simmetria di può costruire il diagramma:
  \[
    \begin{tikzcd}[nodes={column sep= 10 pt}]
      {} & {} & {} & {} & 0 \dar & {} \\
      {} & {} & \dots \dar & \dots \dar & \tor{B,A} \dar{\rho} & {} \\
      {} & 0 \rar & R_A \otimes R_B \rar \dar & F_A \otimes R_B \dar{\eta} \rar{\sigma} & A \otimes R_B \dar{\tau}  \rar & 0 \\
      {} & 0 \rar & R_A \otimes F_B \rar{\psi} \dar{\phi} & F_A \otimes F_B \dar{\beta} \rar & A \otimes F_B \dar  \rar & 0 \\
      0 \rar & \tor{A,B} \rar{\alpha} & R_A \otimes B \rar \dar & F_A \otimes B \rar \dar & A \otimes B  \rar \dar & 0 \\
      {} & {} & 0 & 0 & 0 & {} \\
    \end{tikzcd}
  \]
  Bisogna risalire da $ \tor{A,B} $ a $ \tor{B,A} $ e viceversa, dopodichè bisogna mostrare
  che le mappe così costruite sono omomorfismi e sono tra di loro inverse.
  Costruisco questa scaletta:
  \[
    \begin{tikzcd}[nodes={column sep= 10 pt}]
      {} & {} & {} & t \in \tor{B,A}   \\
      {} & {} & y' \in F_A \otimes R_B \rar & y'' \in A \otimes R_B  \uar  \\
      {} & y \in R_A \otimes F_B \rar  & \tilde{y} \in F_A \otimes F_B \uar  &  {}  \\
      x \in \tor{A,B} \rar & x' \in R_A \otimes B \uar & {} & {} \\
    \end{tikzcd}
  \]
  \begin{enumerate}
  \item Sia $ x \in \tor{A,B} $
  \item Si definisce $ x' = \alpha(x) $, che è unico e non nullo
    siccome $ \alpha $ è iniettiva
  \item Siccome $ \phi $ è suriettiva è ben definito non nullo $ y $,
    ma non è detto che sia unico
  \item Si definisce $ \tilde{y} = \psi(y) $ che è unico e non nullo
    siccome $ \psi $ è iniettiva
  \item Usando la commutatività del quadrato si trova che
    $ \tilde{y} \in \ker{\beta} $, ma la successione è esatta, quindi
    $ \ker{\beta} = \im{\eta} $ e perciò $ \exists y' $ tale che
    $ \eta(y') = \tilde{y} $. Potrebbe non essere unico
  \item Si definisce $ y'' = \sigma(y') $. Potrebbe essere nullo
  \item Utilizzando ancora la commutatività si ha che $ y'' \in \ker{\tau} = \im{\rho} $
    e quindi esiste $ t $ in $ \tor{B, A} $
  \end{enumerate}
  Rimane da verificare che tutto sia ben definito. Se ad ogni passaggio si trova
  sempre un solo elemento allora naturalmente la scaletta può essere percorsa in
  salita e in discesa e quindi è un isomorfismo.
\end{proof}

\begin{lemma}
  Siano $ A, B, C $ gruppi abeliani, per la bilinearità del prodotto tensore
  vale che:
  \[
    \tor{A, B} \oplus \tor{A, C} \cong \tor{A, B \oplus C}
  \]
\end{lemma}
\begin{proof}
  considero una presentazione di $ A $:
  \[
    \begin{tikzcd}
      0 \rar & R \rar & F \rar & A \rar & 0
    \end{tikzcd}
  \]
  Tensorizzo per $ B \otimes C $:
  \[
    \begin{tikzcd}[nodes = {column sep = 15pt}]
      0 \rar & \tor{A, B \oplus C} \rar & R \otimes (B \oplus C) \rar & F \otimes (B \oplus C)\rar & A \otimes (B \oplus C) \rar & 0
    \end{tikzcd}
  \]
  Ma posso anche tensorizzare separatamente per $ B $ e $ C $:
  \[
    \begin{tikzcd}[nodes = {column sep = 15pt, row sep = 10 pt}]
      0 \rar & \tor{A, B } \rar & R \otimes B \rar & F \otimes B \rar & A \otimes B \rar & 0 \\
      0 \rar & \tor{A, C} \rar & R \otimes C \rar & F \otimes C \rar & A \otimes C \rar & 0
    \end{tikzcd}
  \]
  Sommandole:
  \[
    \begin{tikzcd}[nodes = {column sep = 8pt, inner sep = 2pt, outer sep = 0.5pt}]
      0 \rar & \tor{A, B} \oplus \tor{A, C} \rar & R \otimes B \oplus R \oplus C \rar & F \otimes B \oplus F \oplus C \rar & A \otimes B \oplus A \otimes C  \rar & 0
    \end{tikzcd}
  \]
  Ma quindi:
  \[
    \tor{A, B} \oplus \tor{A, C} \cong \tor{A, B \oplus C}
  \]
  Essendo il modulo di torsione unico a meno di isomorfi.
 \end{proof}
 \begin{corollary}
   Se $ G $ è un gruppo libero $ \tor{A, G} \cong 0 $.
\end{corollary}
\begin{proof}
  Infatti $ \tor{A, G} \cong \tor{G, A} \cong 0 $ perché $ G $ libero.
\end{proof}

\section{Cambiamento di coefficienti}

\begin{example}
  Considero lo spazio $ X_n = \Disk{2} \cup_{f_n} \Sph{1} $ dove la mappa di attaccamento è:
  \begin{align*}
    f_n \colon \partial \Disk{2} = \Sph{1} & \to \Sph{1} \\
    z & \mapsto z^n
  \end{align*}
  $ X_n $ è un CW complesso con una $ 0 $-cella, una $ 1 $-cella e una $ 2 $-cella,
  quindi il complesso $ S_\bullet^{CW}(X) $ è:
  \[
    \begin{tikzcd}
      0 \rar{d_3} & \Z \rar{d_2} & \Z \rar{d_1} & \Z \rar{d_0} & 0
    \end{tikzcd}
  \]
  Per calcolare l'omologia bisogna conoscere le mappe di bordo.
  \begin{itemize}
  \item $ d_0 $ è la moltiplicazione per $ 0 $ in quanto manda
    tutto $ \Z $ in $ 0 $ per questo $ \ker{d_0} = \Z $
  \item Per ottenere $ d_1 $ osservo che $ X_n $ è connesso per archi in quanto
    quoziente di uno spazio connesso per archi e quindi deve risultare che
    $ H_0(X_n) = \Z $. Siccome $ H_0(X) = \ker{d_0} \slash \im{d_1} $ e
    $ \ker{d_0} = 0 $ allora $ \im{d_1} = 0 $ e quindi $ d_1 $ è anch'esso la
    moltiplicazione per zero.
  \item Per determinare $ d_2 $ uso la consueta formula: $ d_2 $ è la
    moltiplicazione per la somma dei gradi della mappe ottenuta attaccando
    $ \Disk{2} $, facendo collassare il complesso in un bouquet di sfere e
    selezionando una a una queste sfere. In questo caso l'operazione di collasso
    non fa nulla, e siccome c'è una sola cella la seleziona della sfera è unica:
    \[
      \begin{tikzcd}
        \partial \Disk{2} = \Sph{1} \rar{f_n} \arrow{rd}{\phi} & X^{(1)} \dar{\pi}  & {} \\
        {} & \quot{X^{(1)}}{X^{(0)}} = X^{(1)} \rar{\eta} & {}
      \end{tikzcd}
    \]
    $ d_2 $ è la moltiplicazione per $ \deg{(\phi)} $, ma $ \pi $ e $ \eta $ sono
    identità, quindi $ \deg{(f_n \circ \pi \circ \eta)} = \deg{(f_n)} = n $.
    Per questo $ \ker{d_2} = 0 $ e $ \im{d_2} = n\Z $.
  \end{itemize}
  In questo modo
  \[
    H_k(X_n) =
    \begin{cases}
      \quot{\Z}{0} \cong \Z & \text{se } k = 0 \\
      \quot{\Z}{n \Z} \cong \Z_n & \text{se } k = 1 \\
      \quot{0}{0} & \text{se } k = 2 \\
      0 & \text{se } k \geq 3
    \end{cases}
  \]
  Siccome non sono soddisfatto dalla estetica di questa soluzione vorrei cambiare
  coefficienti.
\end{example}
\eproof
Sia $ G $ un gruppo abeliano e $ X $ uno spazio topologico, considero il complesso
$ (S_\bullet(X) \otimes G, \partial \otimes \Id{G}) $:
\[
  \begin{tikzcd}
    \dots \rar & S_{p+1}(X) \otimes G \rar{\partial \otimes \Id{G}} & S_p(X) \otimes G \rar{\partial \otimes \Id{G}} & S_{p-1}(X) \otimes G \rar & \dots
  \end{tikzcd}
\]
Un modo compatto per scrivere il complesso è $ (S_\bullet(X;G), \partial) $. Ora i
coefficienti non sono più in $ \Z $, ma in $ G $. Definisco l'omologia singolare
a coefficienti in $ G $ come l'omologia singolare di questo complesso. Se
$ G = \Z $ si torna alla consueta omologia singolare. La domanda che sorge spontanea è
% se $ X $ è uno spazio topologico e $ G $ un gruppo abeliano,
che relazione c'è tra $ H_k(X) \oplus G $ e $ H_k(X;G) $? Vale che $ H_k(X) \oplus G \cong H_k(X;G) $?

\begin{theorem}[Teorema dei coefficienti universali\index{Teorema dei coefficienti universali}]
  La successione:
  \[
    \begin{tikzcd}[nodes = {column sep = 10pt}]
      0 \rar & H_k(S_\bullet(X)) \otimes G \rar & H_k(S_\bullet(X) \otimes G) \rar & \tor{H_{k-1}(S_\bullet(X)), G} \rar & 0
    \end{tikzcd}
  \]
  spezza in modo non naturale, cioè non esiste un'unica sezione. Si ha quindi che
  $ H_k(S_\bullet(X) \otimes G) \not \cong H_k(S_\bullet(X)) \otimes G $ ma c'è un pezzo
  di torsione. % cioè vale che:
  % \begin{gather*}
  %   H_k(X;G) = H_k(S_\bullet \otimes G) \cong \\
  %   \cong H_k(S_\bullet(X)) \otimes G \oplus \tor{H_k(S_\bullet), G} = H_k(X) \otimes G \oplus \tor{H_k(X), G}
  % \end{gather*}
  % Ho le successioni:
  % \[
  %   \begin{tikzcd}[nodes = {column sep = 10pt}]
  %     {} & 0 \dar & \dots \dar & {} & {} \\
  %     {} & B_p \dar & S_{p+1} \dar & \dots \dar &  \\
  %     0 \rar & Z_p \dar \rar & S_p \dar \rar & B_{p-1} \rar \dar  & 0 \\
  %     {} & H_p \dar & S_p \dar & Z_{p-1} \dar & {} \\
  %     {} & 0 & \dots & \dots & {}
  %   \end{tikzcd}
  % \]
  % Quando tensorizzo escono fuori delle torsioni.
  % \[
  %   \begin{tikzcd}[nodes = {column sep = 10pt}]
  %     {}     & \dots  \dar               & \dots \dar            & 0   \dar              & {} \\
  %     {}     & B_p \otimes G \dar      & S_{p+1} \otimes G\dar   & \tor{H_{p-1},G} \dar            &    \\
  %     0 \rar & Z_p \otimes G \dar \rar & S_p \otimes G \dar \rar & B_{p-1} \otimes G \rar \dar & 0  \\
  %     {}     & H_p \otimes G \dar      & S_p \otimes G \dar      & Z_{p-1} \otimes G \dar      & {} \\
  %     {}     & 0                 & \dots                 & \dots                 & {}
  %   \end{tikzcd}
  % \]
  % La successione orizzontale è esatta in quanto $ B_{p-1} $ è libero e quindi
  % $ \tor{B_{p-1}, G} \cong \tor{G, B_{p-1}} \cong 0 $.
\end{theorem}
\begin{example}
  Considero $ X_9 $, so che $ H_1(X_9) \cong \Z_9 $, quindi $ H_1(X_9) \otimes \Z_6 = \Z_9 \otimes \Z_6 $.
  Gli elementi di $ \Z_9 \otimes \Z_6 $ sono del tipo $ [n]_9 \otimes [m]_6 $, questi sono 54 elementi,
  ma molti possono essere zero. In effetti vale che:
  \begin{lemma}
    $ \Z_n \otimes \Z_m \cong \Z_d $ dove $ d $ è il massimo comune divisore tra $ n $ e $ m $.
  \end{lemma}
  \begin{exercise}
    Verificare il precedente lemma. Un modo per farlo è costruire esplicitamente
    l'isomorfismo:
    \begin{align*}
      \Z_n \otimes \Z_m & \to \Z_d \\
      [a]_n \otimes [b]_m & \mapsto [ab]_d
    \end{align*}
  \end{exercise}
  Cambiando i coefficienti ottengo quindi:
  \[
    H_k(X_9; \Z_6) \cong
    \begin{cases}
      H_0(X_9) \otimes \Z_6 \cong Z_6 & \text{se } k = 0 \\
      H_1(X_9) \otimes \Z_6 \cong Z_3 & \text{se } k = 1 \\
      H_2(X_9) \otimes \Z_6 \cong 0 & \text{se } k = 2 \\
      H_3(X_9) \otimes \Z_6 \cong 0 & \text{se } k \geq 3
    \end{cases}
  \]
  % Considero le successioni:
  % \[
  %   \begin{tikzcd}
  %     0 \rar & Z_1(X_9) \rar & S_1(X_9) \rar & B_0(X_9) \rar & 0
  %   \end{tikzcd}
  % \]
  % E
  % \[
  %   \begin{tikzcd}
  %     0 \rar & B_1(X_9) \rar & Z_1(X_9) \rar & H_1(X_9) \rar & 0
  %   \end{tikzcd}
  % \]
  % Questa non è esatta, ma anzi definisce l'omologia, e in questo caso non spezza
  % perché $ H_1(X_9) $ ha torsione.
  Mentre con il teorema dei coefficienti universali:
  \[
    H_k(X_9; \Z_6) \cong
    \begin{cases}
      H_0(X) \otimes G \oplus \tor{H_{-1}(X),G} & \text{se } k = 0 \\
      H_1(X) \otimes G \oplus \tor{H_0(X),G} & \text{se } k = 1 \\
      H_2(X) \otimes G \oplus \tor{H_1(X),G} & \text{se } k = 2
    \end{cases}
  \]
  Ma $ \tor{H_{-1}(X), G} \cong 0 $ in quanto $ H_{-1} \cong 0 $, quindi
  $ H_0(X_9, \Z_6) \cong \Z_6 $. Poi $ \tor{H_0(X), G} = \tor{\Z, \Z_6} = 0 $ in quanto
  $ \Z $ è libero, quindi $ H_1(X_9, \Z_6) \cong \Z_9 \otimes \Z_6 \cong \Z_3 $. Infine
  $ \tor{H_1(X), G} \cong \tor{\Z_9, \Z_6} \cong \Z_3 $. Quindi:
  \[
    H_k(X_9, \Z_6) \cong
    \begin{cases}
      \Z_6 & \text{se } k = 0 \\
      \Z_3 & \text{se } k = 1 \\
      \Z_3 & \text{se } k = 2
    \end{cases}
  \]
  Come si nota i gruppi sono differenti.
\end{example}

\begin{osservation}
  Esempi di gruppi di coefficienti che si possono utilizzare senza problemi sono
  $ \Z $, $ \Z_n $, $ \mathbb{Q} $, $ \RN{} $, $ \mathbb{C} $, $ \mathbb{F} $.
  In questi casi si ha che:
  \[
    H_k(X, \mathbb{F}) \cong H_k(X) \otimes \mathbb{F}
  \]
  Infatti questi sono moduli liberi e quindi non hanno torsione.
\end{osservation}

\begin{osservation}
  In generale se $ G $ è un gruppo abeliano finitamente generato c'è il teorema di struttura
  per cui $ G \cong \Z^n \oplus T $, per cui dal teorema dei coefficienti universali:
  \begin{gather*}
    H_k(X;G) \cong H_k(X) \otimes G \oplus \tor{H_{n-1}(X), G} \cong \\
    \cong H_n(X) \otimes (\Z^n \oplus T) \oplus \tor{H_{n-1}(X), \Z^n \oplus T}
  \end{gather*}
  Usando la bilinearità del prodotto tensore:
  \[
    H_n(X) \otimes (\Z^n \oplus T) \cong H_n(X) \otimes \Z^r \oplus H_n(X) \otimes T
  \]
  Questo in generale dipende da $ X $, ma se in particolare $ X $ è un CW
  complesso finito, allora anche $ H_n(X) $ è finitamente generato, quindi
  $ H_n(X) \cong \Z^{s_n} \oplus T' $ per cui vale che:
  \begin{gather*}
    H_n(X) \otimes \Z^{r_n} \cong (\Z^{s_n} \oplus T') \otimes \Z^r \cong \Z^{s_n r_n} \oplus \Z^{r_n} \otimes T' \cong \Z^{sr} \\
    H_n(X) \otimes T = (\Z^{s_n} \oplus T') \otimes T \cong T' \otimes T
  \end{gather*}
  infatti
  $ \Z^k \otimes T' = (\Z \otimes \Z \dots ) \otimes T' = (\Z \otimes T')^k = 0 $ in quanto
  $ T' $ è di torsione. [PERCHÉ'????].

  Poi ho
  $ \tor{H_{n-1}(X), \Z^{r_n} \oplus T} = \tor{H_{n-1}(X), \Z^n} \oplus \tor{H_{n-1}(X), T} $
  per un lemma precedente, quindi in questo caso, siccome $ \Z $ è libero quindi
  $ \tor{H_{n-1}(X), \Z^{r_n}} = (\tor{H_{n-1}(X), \Z})^{r_n} = 0 $, allora:
   \begin{gather*}
     \tor{H_{n-1}(X), \Z^{r_n} \oplus T} \cong \tor{H_{n-1}(X), \Z^{r_n}} \oplus \tor{H_{n-1}(X), T} = \\
     = \tor{H_{n-1}(X), T} = \tor{\Z^{s_{n-1}} \oplus T'_{n-1}, T}
   \end{gather*}
   Quindi:
   \[
     H_n(X; G) \cong \Z^{s_n r_n} \oplus T_n' \oplus T \oplus \tor{T_{n-1}', T}
   \]
   Dove $ H_k(X) \cong \Z^{s_k} \oplus T_k' $ e $ G \cong \Z^r \oplus T $.
   $ H_n(X;G) $ ha quindi una parte libera e delle parti di torsione che si calcolano
   sapendo fare $ \Z_h \otimes \Z_k $ (infatti $ T $ e $ T' $ sono fatte così).
 \end{osservation}

 \begin{exercise}
   Considerare la successione:
   \[
     \begin{tikzcd}
       0 \rar & h \Z \rar & \Z \rar & \quot{\Z}{h \Z} \rar & 0
     \end{tikzcd}
   \]
   Calcolare il modulo di torsione.
 \end{exercise}

\section{Coomologia singolare}

\newmathsymb{hom}{\hom{A,B}}{Spazio degli omomorfismi da $ A $ a $ B $}
Dato uno spazio topologico $ X $ e un gruppo abeliano $ G $ ho costruito le
catene in $ X $ a coefficienti in $ G $ e ho definito l'omologia singolare a
coefficienti in $ G $ come l'omologia di questo complesso. Posso fare anche un'altra
costruzione, considero lo spazio degli omomorfismi da $ S_k(X) $ a $ G $
$ \hom{S_k(X), G} $. A questo punto posso considerare il duale del complesso delle catene:
\[
  \begin{tikzcd}
    \dots \rar & S_{p+1}(X) \rar{\partial} & S_p(X) \rar{\partial} & S_{p-1}(X) \rar & \dots
  \end{tikzcd}
\]
Un elemento di $ \hom{S_p(X), G} $ è un omomorfismo $ \phi \colon S_p(X) \to G $, componendo $ \phi $
con $ \partial \colon S_{p+1}(X) \to S_p(X) $ ottengo  $ \phi' = \phi \circ \partial \colon S_{p+1}(X) \to G $, quindi la composizione
per il bordo è un'operazione controvariante perché inverte il verso. Ho il complesso degli spazi
di omomorfismi:
\[
  \begin{tikzcd}[nodes = {column sep = 14 pt}]
    \dots \rar & \hom{S_{p-1}(X),G} \rar{\delta} & \hom{S_p(X),G} \rar{\delta} & \hom{S_{p+1}(X),G} \rar & \dots
  \end{tikzcd}
\]
Come notazione si pone $ \hom{S_p(X),G} = S^p(X;G) $. $ \delta $ è il \textbf{cobordo}\index{Cobordo},
che non è nient'altro che la composizione per il bordo:
\begin{align*}
  \delta \colon S^p(X;G) & \to S^{p+1}(X;G) \\
  \phi & \mapsto \phi \circ \partial = \delta(\phi)
\end{align*}
Questo è un operatore di bordo, cioè $ \delta^2 = 0 $, infatti:
\[
  \delta^2(\phi) = \delta(\delta(\phi)) = \delta (\phi \circ \partial) = \phi \circ \partial^2 = 0
\]
Questo è un complesso.
\begin{definition}
  Si chiama \textbf{coomologia singolare}\index{Coomologia singolare} di uno
  spazio topologico $ X $ con coefficienti in $ G $, e si indica con
  $ H^p(X; G) $ l'omologia del complesso degli omomorfismi $ S^\bullet(X;G) $.
\end{definition}
Quindi per definizione la coomologia singolare è $ H^p(X;G) = H_p(\hom{S_\bullet(X), G}, \delta) $.
\begin{definition}
  Siano $ A, B $ $ \Z $-moduli, considero una risoluzione di $ A $:
  \[
    \begin{tikzcd}
      0 \rar & R \rar & F \rar & A \rar & 0
    \end{tikzcd}
  \]
  Passando agli omomorfismi la successione si gira e si aggiunge il \textbf{conucleo}\index{Conucleo}
  \[
    \begin{tikzcd}
      0 \rar & \hom{A,B} \rar & \hom{F,B} \rar{\beta} & \hom{R,B} \rar{\gamma} & \coker{\beta} \rar & 0
    \end{tikzcd}
  \]
  \newmathsymb{coker}{\coker{f}}{Conucleo di $ f $}
  Dove il conucleo di una mappa continua tra spazi topologici $ f \colon X \to Y $ è definito
  da:
  \[
    \coker{f} := \quot{Y}{\im{f}}
  \]
  Il conucleo è esattamente quel gruppo che rende esatta la successione, infatti
  usando il primo teorema degli isomorfismi e l'esattezza della successione:
  \[
    \coker{\beta} = \quot{\hom{R,B}}{\im{\beta}} \cong \quot{\hom{R,B}}{\ker{\gamma}} \cong \im{\gamma}
  \]
  Quindi automaticamente $ \gamma $ è suriettiva e perciò la successione diventa esatta.
  Esistono anche altre presentazioni, ma si dimostra che tutti i conuclei sono isomorfi,
  questo gruppo è proprio il \textbf{modulo di estensione di $ A $ e $ B $}\index{Modulo di estensione}.
\end{definition}
\begin{lemma}
  Se $ F $ è libero allora $ \ext{F, G} \cong 0 $ con $ G $ gruppo abeliano generico.
\end{lemma}
\begin{proof}
  Considero la presentazione:
  \[
    \begin{tikzcd}
      0 \rar & 0 \rar & F \rar & F \rar & 0
    \end{tikzcd}
  \]
  Passando agli omomorfismi ho che il conucleo è zero infatti:
  \[
    \begin{tikzcd}
      0 \rar & \hom{F,G} \rar & \hom{F,G} \rar & 0 \rar{\gamma} & \ext{F,G} \rar & 0
    \end{tikzcd}
  \]
  Quindi $ \ext{F,G} = \im{\gamma} = 0 $.
\end{proof}
\eproof
A questo punto ho due possibilità: costruire i gruppi di omologia singolare
$ H_p(X) $ e considerare gli omomorfismi tra tali gruppi e $ G $, oppure
costruire il gruppo di coomologia, cioè prima considerare gli omomorfismi, e
quindi costruire l'omologia. Quello che si trova è che in generale queste
due costruzioni sono differenti, cioè:
\[
  \hom{H_p(X), G} \not \cong H^p(X ; G)
\]

\begin{example}
  Considero la successione esatta corta:
  \[
    \begin{tikzcd}
      0 \rar & 4 \Z \rar & \Z \rar & \quot{\Z}{4\Z} \rar & 0
    \end{tikzcd}
  \]
  E scelgo come gruppo $ G = \Z_6 $.
  Quando prendo il duale la successione si inverte essendo controvariante,
  e rimane esatta solo a sinistra. Per renderla esatta anche a destra bisogna
  aggiungere un termine analogo al modulo di torsione, in modo che la successione
  sia:
  \[
    \begin{tikzcd}[nodes = {column sep = 18 pt}]
      0 \rar & \hom{\Z_4, \Z_6} \rar & \hom{\Z, \Z_6} \rar & \hom{4\Z, \Z_6} \rar & \ext{\Z_4, \Z_6} \rar & 0
    \end{tikzcd}
  \]
  La presenza di questi moduli è responsabile della non uguaglianza tra i gruppi
  $ \hom{H_p(X), G} $ e $ H^p(X ; G) $, come formalizza il teorema dei
  coefficienti universali.
\end{example}
\begin{theorem}[Teorema dei coefficienti universali\index{Teorema dei coefficienti universali}]
  Le successioni esatte corte:
  \[
    \begin{tikzcd}[nodes = {column sep = 12 pt}]
      0 \rar & \ext{H_{n-1}(X), G} \rar & H^n(X;G) \arrow[bend left]{l}{} \rar & \hom{H_n(X), G} \rar & 0
    \end{tikzcd}
  \]
  E:
  \[
    \begin{tikzcd}[nodes = {column sep = 18 pt}]
      0 \rar & H_n(X) \oplus G \rar & H_n(X; G) \rar & \arrow[bend left]{l}{} \tor{H_{n-1}(X), G} \rar & 0
    \end{tikzcd}
  \]
  Spezzano in modo non naturale (cioè non esiste una sola sezione), e quindi:
  \begin{gather*}
    H_n(X; G) \cong H_n(X) \oplus G \oplus \tor{H_{n-1}(X), G} \\
    H^n(X; G) \cong \hom{H_n(X), G} \oplus \ext{H_{n-1}(X), G}
  \end{gather*}
\end{theorem}
%
% lezione 17
%
% Sia $ X $ spazio topologico e $ G $ gruppo abeliano, ho
% \[
%   \begin{tikzcd}[nodes={row sep = 5 pt}]
%     H_\bullet \oplus G \rar & \lar H^\bullet(X;G) \\
%     \hom{H_\bullet(X); G} \rar & \lar H_\bullet(X;G)
%   \end{tikzcd}
% \]
% Quale è esattamente la relazione tra questi gruppi?
% La risposta è data dal teorema dei coefficienti universali.
% \begin{theorem}
%   Esiste una successione esatta spezzante in modo non naturale, cioè possono
%   esistere più sezioni in omologia singolare e in coomologia, queste sono:
%   \[
%     \begin{tikzcd}[nodes = {column sep = 12 pt}]
%       0 \rar & \ext{H_{p-1}(X), G} \rar & \arrow[bend left]{l}{} H^p(X;G) \rar & \hom{H_p(X), G} \rar & 0
%     \end{tikzcd}
%   \]
%   E:
%   \[
%     \begin{tikzcd}[nodes = {column sep = 18 pt}]
%       0 \rar & H_n(X) \oplus G \rar & H_n(X; G) \rar & \arrow[bend left]{l}{} \tor{H_{n-1}(X), G} \rar & 0
%     \end{tikzcd}
%   \]
% \end{theorem}
\begin{proof}
  La dimostrazione per le due successioni è praticamente identica, dimostro
  quella in coomologia. In quello che segue spesso ometto per brevità lo spazio
  topologico $ X $ come argomento. Voglio costruire la successione:
  \[
    \begin{tikzcd}[nodes = {column sep = 12 pt}]
      0 \rar & \ext{H_{p-1}(X), G} \rar & \arrow[bend left]{l}{} H^p(X;G) \rar & \hom{H_p(X), G} \rar & 0
    \end{tikzcd}
  \]
  Per definizione $ H^p(X;G) $ è l'omologia del complesso delle cocatene $ S^p $ con il cobordo,
  dove $ S^p(X;G) = \hom{S_p(X), G} $ e il cobordo è:
  \begin{align*}
    \delta \colon S^p(X;G) & \to S^{p+1}(X;G) \\
    \phi & \mapsto \phi \circ \partial
  \end{align*}
  La dimostrazione richiede che si costruisca un diagramma, quindi elenco alcune
  successioni esatte, omettendo per concisione l'esplicita dipendenza dallo spazio
  topologico, che si intende essere $ X $.

  \noindent
  Per definizione $ H_p(X) = Z_p \slash B_p $ (cicli modulo i bordi), quindi ho:
  \[
    \begin{tikzcd}
      0 \rar & B_p \rar{i} & Z_p \rar{\pi} & H_p \rar & 0
    \end{tikzcd}
  \]
  Non necessariamente questa spezza perché $ H_p $ può essere di torsione.
  Poi ho:
  \[
    \begin{tikzcd}
      0 \rar & Z_p \rar & S_p \rar{\partial} & \arrow[bend left]{l}{i} B_{p-1} \rar & 0
    \end{tikzcd}
  \]
  Questa spezza perché tra le catene singolari ci sono quelle che si esprimono
  come bordo e quindi c'è una sezione, che sui generatori (sono entrambi gruppi
  liberi) agisce come:
  \begin{align*}
    i \colon B_{p-1} \to S_p \\
    \partial c & \mapsto c
  \end{align*}
  In questo modo$ \partial \circ i = \Id{B_{p-1}} $.
  Poi ho a partire da:
  \[
    \begin{tikzcd}
      0 \rar & B_p \rar & Z_p \rar & H_p \rar & 0
    \end{tikzcd}
  \]
  Passando agli omomorfismi:
  \[
    \begin{tikzcd}
      0 \rar & \hom{H_p, G} \rar & \hom{Z_p, G} \rar{t_p} & \hom{B_p, G} \rar & \ext{H_p, G} \rar & 0
    \end{tikzcd}
  \]
  Per definizione ho che:
  \[
    \ext{H_p,G} \cong \quot{\hom{B_p,G}}{\im{t_p}}
  \]
  Oltre a ciò ho la successione:
  \[
    \begin{tikzcd}
      0 \rar & Z_{p+1} \rar & S_{p+1} \rar & B_p \rar & 0
    \end{tikzcd}
  \]
  Passando agli omomorfismi:
  \[
    \begin{tikzcd}
      0 \rar & \hom{B_p, G } \rar & \hom{S_{p+1}, G} \rar & \dots
    \end{tikzcd}
  \]
  Poi ho la successione:
  \[
    \begin{tikzcd}
      0 \rar & Z_{p-1} \rar & S_{p-1} \rar & B_{p-2} \rar & 0
    \end{tikzcd}
  \]
  Prendendo gli omomorfismi (non c'è il modulo di estensione in quanto i gruppi sono liberi):
  \[
    \begin{tikzcd}
      0 \rar & \hom{B_{p-2}, G} \rar & \hom{S_{p-1}, G} \rar & \hom{Z_{p-1}, G} \rar & 0
    \end{tikzcd}
  \]
  Infine, siccome ho la successione spezzante:
  \[
    \begin{tikzcd}
      0 \rar & Z_p \rar & S_p \arrow[bend left]{l}{\phi} \rar & B_{p-1} \rar & 0
    \end{tikzcd}
  \]
  È ben definita la sezione, e quindi posso definire la mappa:
  \begin{align*}
    \Phi \colon \hom{Z_p, G} & \to \hom{S_p, G} \\
    \alpha \colon Z_p \to G & \mapsto \alpha \circ \phi \colon S_p \to G
  \end{align*}
  cioè $ \Phi = \alpha \circ \phi $.
  Il mio obiettivo è trovare la successione esatta:
  \[
    \begin{tikzcd}
      0 \rar & \ext{H_{p-1}, G} \rar{\beta_1} & H^p(X; G) \rar{\beta_2} & \hom{H_p, G} \rar & 0
    \end{tikzcd}
  \]
  Mettendo insieme le successioni costruite ottengo un diagramma su cui posso
  fare diagram chase:
  \[
    \begin{tikzcd}
      {} & 0 & \dots & \dots & {} \\
      {} & \ext{H_{p-1},G} \uar & \hom{S_{p+1},G} \uar & \lar{\sigma} \hom{B_p,G} \uar & 0 \lar \\
      0 \rar & \hom{B_{p-1}, G} \uar{\lambda_2} \rar{\alpha_1} & \rar{\alpha_2} \hom{S_p,G} \uar{\delta} & \hom{Z_p,G} \rar \uar{\tau_2} \arrow[bend left]{l}{\Phi}  & 0 \\
      0 & \lar \hom{Z_{p-1}, G} \uar{\lambda_1} & \lar{\Delta} \hom{S_{p-1},G} \uar{\delta} & \hom{H_p,G} \uar{\tau_1}  & {} \\
      {} & \dots \uar & \dots \uar & 0 \uar{} & {}
    \end{tikzcd}
  \]
  Costruisco $ \beta_2 $.
  Per definizione:
  \[
    H^p(X;G) = \quot{\ker{\delta\colon S^p(X;G) \to S^{p+1}(X;G)}}{\im{\delta \colon S^{p-1}(X;G) \to S^p(X;G)}}
  \]
  Se $ \llbracket f \rrbracket \in H^p(X; G) $ allora $ f \in S^p $ e $ \delta(f) = 0 $. Applicando
  $ \sigma \circ \tau_2 \circ \alpha_2 $ a $ f $ e usando la commutatività:
  \[
    \sigma \circ \tau_2 \circ \alpha_2 (f) = \delta(f) = 0
  \]
  Quindi $ \sigma \circ \tau_2 \circ \alpha_2 (f) = 0 $, ma $ \sigma $ è iniettiva e quindi
  $ \tau_2(\alpha_2(f)) = 0 $, cioè
  $ \alpha_2(f) \in \ker{\tau_2} = \im{\tau_1} $ per l'esattezza della successione e quindi
  $ \exists g \in \hom{H_p, G} $ tale che $ \tau_1(g) = \alpha_2(f) $, e quindi ho trovato un
  elemento $ g $ a partire da $ f $. Pongo $ \beta_2(f) = g_f $. Per verificare che
  l'applicazione sia ben definita devo controllare che cambiando rappresentante
  della classe di equivalenza $ \llbracket f \rrbracket $ si ottenga il medesimo
  $ g_f $, equivalentemente posso verificare che l'associazione che ho definito
  mandi tutto il modulo $ \im{\delta \colon S^{p-1}(X;G) \to S^p(X;G)} $ in zero. Sia
  $ \delta (h) \in \im{\delta \colon S^{p-1}(X;G) \to S^p(X;G)} \subseteq S^p $ devo verificare che
  $ \beta_2(\delta(h)) = 0 $. Per trovare l'elemento $ g_{\delta(h)} $ applico $ \sigma \circ \tau_2 \circ \alpha_2 $
  e uso la commutatività:
  \[
    \sigma \circ \tau_2 \circ \alpha_2 (\delta(h)) = \delta \circ \delta (h) = 0
  \]
  Quindi per l'iniettivita di $ \sigma $ $ \tau_2(\alpha_2(\delta(h))) = 0 $ perciò
  $ \alpha_2(\delta(h)) \in \ker{\tau_2} = \im{\tau_1} $ e quindi esiste
  $ v \in \hom{H_p,G} $ tale che $ \alpha_2(\delta(h)) = \tau_1(v) $ e quindi si definisce
  $ \beta_2(\delta(h)) = v $. Devo mostrare che $ v = 0 $ per mostrare che
  $ \beta_2 $ è ben definita, ma $ \tau_1 $ è iniettiva, quindi posso mostrare che
  $ \alpha_2 \circ \delta (h) = 0 $ per mostrare che $ v = 0 $. Ma ho:
  \[
    \begin{tikzcd}
      S_p \rar{\partial} & S_{p-1} \rar{h} & G
    \end{tikzcd}
  \]
  \[
    \begin{tikzcd}
      Z_p \rar{i} & S_{p} \rar{h \circ \partial} & G
    \end{tikzcd}
  \]
  E $ \alpha_2(h \circ \partial) $ agisce sugli elementi di $ Z_p $ per restituire un elemento
  in $ G $, quindi sarà la mappa $ h \circ \partial \circ i \colon Z_p \to G $. Ma in
  $ Z_p $ ci sono solo quelli di bordo nullo, cioè se $ c \in Z_p $:
  \[
    (h \circ \partial \circ i)(c) = h \circ \partial(c) = h(0) = 0
  \]
  Quindi:
  \[
    \alpha_2 \circ \delta(h) = 0 \quad \Rightarrow \quad \tau_1(v) = 0
  \]
  Ma quindi $ v = 0 $ in quando $ \tau_1 $ è iniettiva. Ma questo significa
  che $ \beta_2 $ è ben definita:
  \begin{align*}
    \beta_2 \colon H^p(X;G) & \to \hom{H_p,G} \\
    \llbracket f \rrbracket & \mapsto g_f \; | \; \tau_1(g_f) = \alpha_2(f)
  \end{align*}

  Ora costruisco $ \beta_1 \colon \ext{H_{p-1}, G} \to H^p(X;G) $. Parto da $ u \in \ext{H_{p-1}, G} $,
  $ \lambda_2 $ è suriettiva, quindi esiste $ \tilde{u} \in \hom{B_{p-1}.G} $ tale che
  $ \lambda_2(\tilde{u}) = u $, poi ho che $ \alpha_1(\tilde{u}) \in \hom{S_p, G} = S^p $,
  quindi posso definire:
  \begin{align*}
    \beta_1 \colon \ext{H_{p-1}, G} & \to H^p(X,G) \\
    u & \mapsto \llbracket\alpha_1(\tilde{u})\rrbracket \; | \; \lambda_2(\tilde{u}) = u
  \end{align*}
  Per poter scendere a livello di omologia devo mostrare che se
  $ u \in \ext{H_{p-1},G} $ allora $ \alpha_1(\tilde{u}) \in S^p $ è un cociclo, cioè
  $ \delta(\alpha_1(\tilde{u})) = 0 $, ma per la commutatività:
  \[
    \delta \circ \alpha_1 = (\sigma \circ \tau_2 \circ \alpha_2)(\alpha_1) = \sigma \circ \tau_2 \circ \alpha_2 \circ \alpha_1 = 0
  \]
  In quanto $ \alpha_2 \circ \alpha_1 = 0 $ perché la successione è esatta, e quindi
  $ \delta(\alpha_1(\tilde{u})) = 0 $. Bisogna mostrare che $ \beta_1 $ è ben definita, cioè
  comunque scelga la preimmagine $ \tilde{u} $ ottengo sempre la medesima classe
  di equivalenza in $ H^p(X,G) $. Se $ \tilde{u} $ non fosse unico, ma se
  esistessero $ \tilde{u}_1, \tilde{u}_2 $ tali che
  $ \lambda_2(\tilde{u}_1) = \lambda_2(\tilde{u}_1) = u $, siccome $ \lambda_2 $ è un omomorfismo
  $ \lambda_2(\tilde{u}_1 - \tilde{u}_2) = 0 $, quindi
  $ \tilde{u}_1 - \tilde{u}_2 \in \ker{\lambda_2} = \im{\lambda_1} $ per l'esattezza della
  successione, quindi esiste $ V \in \hom{Z_{p-1}, G} $ tale che
  $ \lambda_1(V) = \tilde{u}_1 - \tilde{u}_2 $. Ma $ \Delta $ è suriettiva, quindi esiste
  $ w \in \hom{S_{p-1}, G} $ tale che $ \Delta(w) = V $. Quindi per la commutatività:
  \[
    \delta(w) = \alpha_1 \circ \lambda_1 \circ \Delta (w) = \alpha_1 \circ \lambda_1 (V) = \alpha_1 ( \tilde{u}_1 - \tilde{u}_2 )
  \]
  Quindi:
  \[
    \alpha_1(\tilde{u}_1) - \alpha_2(\tilde{u}_2) = \delta(w)
  \]
  Le immagini differiscono per un cobordo quindi danno origine alla stessa
  classe di equivalenza e perciò $ \beta_1 $ è ben definita.
  \\ \\
  Così ho costruito le due applicazione che mi servivano, ma non ho ancora
  finito, devo mostrare che $ \beta_2 $ è suriettiva, $ \beta_1 $ iniettiva, che
  $ \im{\beta_2} = \ker{\beta_1} $ e che la successione spezza.

  Dimostro che $ \beta_1 $ è iniettiva mostrando che il suo nucleo è banale.
  Considero $ u \in \ker{\beta_1} $, quindi tale che $ \beta_1(u) = 0 $, allora
  $ \llbracket\alpha_1(\tilde{u})\rrbracket = 0 $. Questo è vero se
  $ \alpha_1(\tilde{u}) $ è un cobordo, cioè esiste $ z $ tale che
  $ \alpha_1(\tilde{u}) = \delta z $. Applicando $ \alpha_1 $ e usando la commutatività:
  \[
    \alpha_1 \circ \alpha_1 (\tilde{u}) = \delta \circ \tau_2 \circ \alpha_2 \circ \alpha_1 (\tilde{u}) = 0
  \]
  Ma $ \alpha_1 $ è iniettiva quindi $ \tilde{u} = 0 $, ma
  $ u = \lambda_2(\tilde{u})$ quindi $ u = 0 $ siccome $ \lambda_2 $ è omomorfismo, e perciò
  $ \ker{\beta_1} = 0 $, e quindi $ \beta_1 $ è iniettiva.

  Per dimostrare che $ \beta_2 $ è suriettiva considero $ v \in \hom{H_p, G} $, allora
  $ \Phi(\tau_1(v)) \in \hom{S_p,G} $ applicando $ \delta $ e usando la commutatività:
  \[
    \delta \circ \Phi(\tau_1(v)) = (\sigma \circ \tau_2 \circ \alpha_2) \circ \Phi \circ \tau_1 (v) = \sigma \circ \tau_2 \circ (\alpha_2 \circ \Phi) \circ \tau_1 (v)
  \]
  Per trovare l'azione di $ \alpha_2 $ considero la successione spezzante:
  \[
    \begin{tikzcd}
      0 \rar & Z_p \rar{\psi} & S_p \arrow[bend left]{l}{\phi} \rar & B_{p-1} \rar & 0
    \end{tikzcd}
  \]
  Quindi ho:
  \begin{align*}
    \alpha_2 \colon \hom{S_p, G} & \to \hom{Z_p, G} \\
    \omega \colon S_p \to G & \mapsto \omega \circ \psi \colon Z_p \to G
  \end{align*}
  E quindi:
  \[
    (\alpha_2 \circ \Phi)(\omega) = \alpha_2(\omega \circ \phi) = \omega \circ \phi \circ \psi = \omega
  \]
  Quindi $ \alpha_2 \circ \Phi = \Id{\hom{Z_p,G}} $ e quindi
  $ \sigma \circ \tau_2 \circ \tau_2 (v) = 0 $ in quanto
  $ \tau_2 \circ \tau_1 = 0 $, dato che la colonna è esatta, quindi
  $ \delta(\Phi(\tau_1(v))) = 0 $, cioè $ \Phi(\tau_1(v) $ è un cociclo in
  $ S_p $ ed è tale che $ \beta_2(\llbracket\Phi(\tau_1(v))\rrbracket) = v $. Faccio
  agire $ \beta_2 $:
  \[
    \beta_2 \colon \llbracket \Phi(\tau_1(v)) \rrbracket \mapsto x \; | \; \tau_1(x) = \alpha_2 \circ \Phi(\tau_1(v)))= \tau_1(v)
  \]
  Essendo $ \tau_1 $ iniettivo ho che $ x = v $ quindi $ \beta_2(\llbracket\Phi(\tau_1(v))\rrbracket) = v $.

  Ora devo mostrare che $ \im{\beta_1} = \ker{\beta_2} $. Mostro che
  $ \im{\beta_1} \subseteq \ker{\beta_2} $. Sia $ u \in \ext{H_{p-1},G} $ allora
  $ \beta_1(u) = \llbracket\alpha_1(\tilde{u})\rrbracket $, applicando $ \beta_2 $:
  \[
    \beta_2 \colon \llbracket \alpha_1(\tilde{u}) \rrbracket \mapsto k \; | \; \tau_1(k) = \alpha_2 \circ \alpha_1(k) = 0
  \]
  Ma $ \tau_1 $ è iniettivo quindi $ \beta_2(\llbracket \alpha_1(\tilde{u}) \rrbracket) = 0 $
  e quindi $ \beta_1(u) = \llbracket \alpha_1(\tilde{u}) \rrbracket \in \ker{\beta_2} $.

  Ora mostro che $ \ker{\beta_2} \subseteq \im{\beta_1} $, sia
  $ \llbracket f \rrbracket \in H^p(X;G) $, se
  $ \beta_2(\llbracket f \rrbracket) = 0 $ allora $ \alpha_2(f) = \tau_1(0) $ quindi
  $ \alpha_2(f) = 0 $, quindi $ f \in \ker{\alpha_2} = \im{\alpha_1} $ per l'esattezza della
  successione, e quindi esiste $ f' \in \hom{B_{p-1}, G} $ tale che
  $ \alpha_1(f') = f $. Definendo $ u = \lambda_2(f') $ si ha:
  \[
    \beta_1 \colon u \mapsto \llbracket \alpha_1(\tilde{u}) \rrbracket \; | \; \lambda_2(\tilde{u}) = u = \lambda_2(f')
  \]
  Quindi $ \tilde{u} - f' \in \ker{\lambda_2} = \im{\lambda_1} $, quindi esiste $ \eta \in \hom{Z_{p-1}, G} $
  tale che $ \lambda_1(\eta) = \tilde{u} - f' $, ma $ \Delta $ è suriettivo, quindi esiste
  $ \chi \in \hom{S_{p-1},G} $ tale che $ \Delta(\chi) = \eta $, quindi per la commutatività:
  \[
    \alpha_1(\tilde{u}) = \alpha_1(f' + \lambda_1(\eta)) = \alpha_1 \circ f' + \alpha_1 \circ \lambda_1 \circ \Delta (\chi) = \alpha_1 \circ f' + \delta \chi
  \]
  Passando alla classe di equivalenza in coomologia:
  \[
    \llbracket \alpha_1(\tilde{u}) \rrbracket = \llbracket \alpha_1 \circ f' \rrbracket
  \]
  E quindi $ \beta_1(u) = \llbracket \alpha_1 \circ f' \rrbracket = \llbracket f \rrbracket $, cioè $ \llbracket f \rrbracket \in \im{\beta_1} $.
  \\ \\ \noindent
  La successione è quindi esatta, ma bisogna ancora verificare che spezza:
  \[
    \begin{tikzcd}
      0 \rar & \ext{H_{p-1}(X), G} \rar{\beta_1} & H^p(X;G) \rar{\beta_2} & \hom{H_p(X), G} \rar \arrow[bend left]{l}{\rho} & 0
    \end{tikzcd}
  \]
  Sia $ y \in \hom{H_p, G} $, definisco $ \rho(y) $ come
  $ \rho(y) = \llbracket \Phi(\tau_1(y)) \rrbracket$, in questo modo per costruzione
  $ \beta_2 \circ \rho = \Id{\hom{H_p,G}} $, infatti $ \rho $ percorre
  il diagramma in modo inverso a $ \beta_2 $.
\end{proof}

\begin{example}[Coomologia dello spazio proiettivo reale]
  Soche l'omologia dello spazio proiettivo reale con $ n = 3 $ è:
  \[
    H_p(\Pjr{3}) \cong
    \begin{cases}
      \Z & \text{se } p = 0 \\
      \Z_2 & \text{se } p = 1 \\
      0 & \text{se } p = 2 \\
      \Z & \text{se } p = 3
    \end{cases}
  \]
  Applico il teorema dei coefficienti universali, per ogni $ p \in \mathbb{N} $:
  \[
    H^p(\Pjr{3}, \Z_2) \cong \hom{H_p(\Pjr{3}), \Z_2} \oplus \ext{H_{p-1}(\Pjr{3}), \Z_2}
  \]
  Quindi:
  \begin{gather*}
    H^0(\Pjr{3}, \Z_2) \cong \hom{\Z, \Z_2} \\
    H^1(\Pjr{3}, \Z_2) \cong \hom{\Z_2, \Z_2} \oplus \ext{\Z, \Z_2} \\
    H^2(\Pjr{3}, \Z_2) \cong \hom{0, \Z_2} \oplus \ext{\Z_2, \Z_2} \cong \ext{\Z_2, \Z_2} \\
    H^3(\Pjr{3}, \Z_2) \cong \hom{\Z, \Z_2} \oplus \ext{0, \Z_2} \cong \hom{\Z, \Z_2}
  \end{gather*}
  Calcolo i gruppi che mi mancano:
  \[
    \hom{\Z, \Z_2} = \set{\phi \colon \Z \to \Z_2 } \cong \Z_2
  \]
  Infatti considero l'azione sui generatori, devo decidere dove mandare il generatore di $ \Z $,
  che è $ 1 $, lo posso mandare in $ 0 $ o in $ 1 $, quindi ho due possibili applicazioni, e quindi
  lo spazio degli omomorfismi è isomorfo a $ \Z_2 $. Considerazioni analoghe valgono per
  \[
    \hom{\Z_2, \Z_2} = \set{\phi \colon \Z_2 \to \Z_2 } \cong \Z_2
  \]
  Infatti $ 0 $ deve andare in $ 0 $ essendo un omomorfismo.
  $ \ext{\Z, \Z_2} \cong 0 $ in quanto $ \Z $ è libero.
  % Per calcolare $ \ext{\Z, \Z_2} $ considero la risoluzione:
  % \[
  %   \begin{tikzcd}
  %     0 \rar & 0 \rar & \Z \rar & \Z \rar & 0
  %   \end{tikzcd}
  % \]
  % Passando agli omomorfismi:
  % \[
  %   \begin{tikzcd}
  %     0 \rar & \hom{\Z, \Z_2} \rar & \hom{\Z, \Z_2} \rar & 0 \rar & 0
  %   \end{tikzcd}
  % \]
  % Quindi $ \ext{\Z, \Z_2} \cong 0 $.
  Invece per $ \ext{\Z_2, \Z_2} $ considero la risoluzione:
  \[
    \begin{tikzcd}
      0 \rar & 2 \Z \rar{i} & \Z \rar{\pi} & \Z_2 \rar & 0
    \end{tikzcd}
  \]
  Passando agli omomorfismi:
  \[
    \begin{tikzcd}
      0 \rar & \Z_2 \rar & \Z_2 \rar & \hom{2\Z, \Z_2} \rar & \ext{\Z_2, \Z_2} \rar & 0
    \end{tikzcd}
  \]
  Tra $ \Z_2 $ e $ \Z_2 $ l'unica possibile mappa iniettiva è l'isomorfismo, quindi
  la successione spezza e $ \ext{\Z_2, \Z_2} \cong \hom{\Z, \Z_2} \cong \Z_2 $.
  Nel complesso quindi:
  \[
    H^k(\Pjr{3}, \Z_2) \cong \Z_2
  \]
  In realtà si dimostra che $ \forall n \in \mathbb{N} $:
  \[
    H^k(\Pjr{n}, \Z_2) \cong \Z_2
  \]
  Mentre:
  \[
    H^k(\Pjc{n}, \Z) \cong \Z
  \]
\end{example}

% lezione 18

\section{Prodotto cup}

\begin{example}
  Sia $ X = \Pjc{2} $ e $ Y = \Sph{2} \vee \Sph{4} $, è vero che $ X $ e $ Y $ sono
  omotopicamente equivalenti? Mi aspetto che non lo siano in quanto $ X $ è una
  varietà topologica, mentre $ Y $ no, dato che possiede un punto (quello a cui
  le due sfere sono incollate) che non possiede un intorno omeomorfo a
  $ \RN{n} $. Per verificarlo posso usare gli invarianti topologici che
  conosco.
  \subsubsection{Gruppo fondamentale}
  Con Seifert-Van Kampen si trova che $ \pi_1(X) \cong \set{1} $ e
  $ \pi_1(Y) \cong \set{1} $, e quindi i gruppi fondamentali sono isomorfi.
  \subsubsection{Gruppi di omologia}
  Per calcolare i gruppi di omologia utilizzo la struttura di CW complesso,
  sia $ X $ che $ Y $ sono formati da una $ 0 $-cella, una $ 2 $-cella e una
  $ 4 $-cella, quindi il complesso delle catene è:
  \[
    \begin{tikzcd}
      0 \rar & S_4^{CW} \rar & S_3^{CW} \rar & S_2^{CW} \rar & S_1^{CW} \rar & S_0^{CW} \rar & 0
    \end{tikzcd}
  \]
  E in entrambi i casi questa si riduce a:
  \[
    \begin{tikzcd}
      0 \rar & \Z \rar & 0 \rar & \Z \rar & \Z \rar & \Z \rar & 0
    \end{tikzcd}
  \]
  Quindi entrambi gli spazi hanno come gruppi di omologia
  $ H_k(X) \cong H_k(Y) \cong \Z $ per $ k \in \set{0,2,4} $.
  \subsubsection{Gruppi di coomologia}
  Con il teorema di coefficienti universali $ H^k(\bullet;G) \cong \hom{H_k(\bullet), G} \oplus \ext{H_{k-1}(\bullet), G} $,
  quindi essendo uguali i gruppi di omologia:
  \[
    H^k(X;G) \cong H^k(Y;G) \cong
    \begin{cases}
      G & \text{se } k = 0 \\
      0 & \text{se } k = 1 \\
      G & \text{se } k = 2 \\
      0 & \text{se } k = 3 \\
      G & \text{se } k = 4 \\
    \end{cases}
  \]
  (Infatti $ H_k(X) \cong \Z $ e quindi $ \hom{\Z, G} \cong G $).

  Ho quindi bisogno di strumenti più fini, per questo e per altri motivi rendo i
  gruppi di coomologia un anello.

\end{example}

\subsection{Richiami di algebra degli anelli}

\begin{definition}
  Un anello commutativo $ \R $ si dice \textbf{dominio di integrità}\index{Dominio di integrità}
  se il prodotto tra qualsiasi coppia di elementi non nulli è un elemento non nullo, cioè
  vale che se $ ab = 0 $ allora o $ a = 0 $ o $ b = 0 $ $ \forall a, b \in \R $.
\end{definition}

\begin{proposition}
  In un dominio di integrità $ \R $ valgono le leggi di cancellazione del prodotto, cioè:
  \[
    \forall a,x,y \in \R \quad ax = ay \Rightarrow x = y
  \]
\end{proposition}

\begin{definition}
  Un \textbf{ideale}\index{Ideale} $ I $ di un anello commutativo $ \R $ è un
  sottoinsieme di $ \R $ tale che $ \forall a,b \in \R $ e $ \forall x,y \in I $ sia
  $ a x + b y \in I $.
\end{definition}

\newmathsymb{idgen}{(i)}{Ideale generato da $ i $}
\begin{definition}
  Un \textbf{dominio a ideali principali}\index{Dominio a ideali principali}
  (PID, \emph{principal ideal domain}) è un dominio di integrità in cui ogni
  ideale è principale\index{Ideale principale}, cioè generato da un solo
  elemento, cioè $ \forall I $ ideale esista $ i \in A $ tale che
  $ I = (i) = \set{a i | a \in A} $. Con la scrittura $ (i) $ si indica l'ideale
  generato.
\end{definition}

\newmathsymb{pid}{PID}{Dominio a ideali principali}
\begin{example}
  Esempi di PID sono $ \Z, \RN{}, \mathbb{F}, \mathbb{K}[x] $.
\end{example}

\subsection{Prodotto cup}

\newmathsymb{cuppr}{\cup}{Prodotto cup}
Sia $ H^\star (X,\R) := \bigoplus_k H^k(X,\R) $, definisco il prodotto
cup $ \cup $ tale che renda $ (H^\star (X,\R), +, \cup) $ un anello.
Gli elementi di $ H^\star (X,\R) $ sono somme formali finite
del tipo $ \sum_i \alpha^i $ con $ \alpha^i \in H^i(X, \R) $.

Sia $ X $ uno spazio spazio topologico, e $ \R $ un PID,
$ S^k(X; \R) $ e $ S^l(X; \R) $ sono gli insiemi delle cocatene,
voglio costruire una mappa:
\begin{align*}
  \cup \colon S^k(X; \R) \times S^l(X; \R) & \to S^{k+l}(X, \R) \\
  (\phi, \psi) & \mapsto \phi \cup \psi
\end{align*}
E quindi passare a livello di coomologia in modo da fornire la struttura
ad anello.

Se $ \phi \cup \psi \in S^{k+l}(X; \R) $ significa che $ \phi \cup \psi \in \hom{S_{k+1}(X), \R} $
e quindi $ \phi \cup \psi \colon S_{k+l}(X) \to \R $, e l'azione di questa mappa può essere
definita solo sui simplessi singolari e quindi estesa per linearità su tutto
lo spazio delle catene. Sia $ \sigma \colon \Delta_{k+l} \to X $ un simplesso singolare, si
può anche vedere il simplesso standard come inviluppo convesso di punti:
\[
  \Delta_{k+l} = [v_0, \dots, v_k, v_{k+1}, \dots, v_{k+l}]
\]
E quindi si può restringere il simplesso singolare sulla parte generata
dai primi $ k $ punti e su quella generata dagli ultimi $ l $:
\[
  \sigma \lvert_{[v_0, \dots, v_k]} \colon \Delta_k \to X \quad \sigma \lvert_{[v_{k+1}, \dots, v_{k+l}]} \colon \Delta_l \to X
\]
A questo punto la definizione dell'azione di $ \phi \cup \psi $ su $ \sigma $ risulta naturale:
\[
  (\phi \cup \psi)(\sigma) = \phi \left(\sigma \lvert_{[v_0, \dots, v_k]} \right) \cdot \psi \left(\sigma \lvert_{[v_{k+1}, \dots, v_{k+l}]}\right)
\]
Questa definizione è ben posta, il prodotto tra i due termini è infatti il prodotto in $ \R $.
Per passare a livello di coomologia (indicando con abuso di notazione $ \cup^\star = \cup $):
\begin{align*}
  \cup \colon H^k(X; \R) \times H^l(X; \R) & \to H^{k+l}(X; \R) \\
  (\llbracket \phi \rrbracket, \llbracket \psi \rrbracket) & \mapsto \llbracket \phi \circ \psi \rrbracket
\end{align*}
Verifico che questa applicazione è ben definita. Si ha che $ \phi $ e $ \psi $ sono
cocicli, cioè $ \delta \phi = \delta \psi = 0 $, e tutti gii altri elementi della classe differiscono
per un cobordo da $ \phi $ e $ \psi $, cioè sono della forma $ \phi + \delta\phi_1 $ e $ \psi + \delta\psi_1 $.
L'applicazione è ben definita se:
\begin{enumerate}
\item $ \phi \cup \psi $ è un cociclo
\item Elementi omologhi in $ H^k(X; \R) \times H^l(X; \R) $ vengono
  mandati in elementi omologhi in $ H^{k+l}(X; \R) $.
\end{enumerate}
Per verificare la prima di queste si utilizza il seguente lemma:
\begin{lemma}
  Vale che $ \delta(\phi \cup \psi) = \delta \phi \cup \psi + (-)^k \phi \cup \delta \psi $, quindi se $ \phi $ e $ \psi $ sono cocicli,
  anche $ \phi \cup \psi $ lo è.
\end{lemma}
\begin{exercise}
  Verificare il lemma.
\end{exercise}
Per verificare la seconda richiesta mostro che esiste $ \eta \in S^{k+l-1}(X) $ tale che:
\[
  (\phi + \delta\phi_1) \cup (\psi + \delta\psi_1) = \phi \cup \psi + \delta \eta
\]
Utilizzando il precedente lemma si ha che:
\begin{gather*}
  \delta(\phi \cup \psi_1) = \cancel{\delta \phi \cup \psi_1} + (-)^k (\phi \cup \delta\psi_1) \; \Rightarrow \; \phi \cup \delta \psi_1 = (-)^k \delta(\phi \cup \psi_1) = \delta ((-)^k \phi \cup \psi_1) \\
  \delta(\phi_1 \cup \psi) = \delta \phi_1 \cup \psi + \cancel{(-)^{k-1}(\phi_1 \cup \delta \psi)} \; \Rightarrow \; \delta \phi_1 \cup \psi = \delta(\phi_1 \cup \psi) \\
  \delta(\phi_1 \cup \delta \psi_1) = \delta \phi_1 \cup \delta \psi_1 + \cancel{(-)^{k-1}\phi_1 \cup \delta^2 \psi_1} \; \Rightarrow \; \delta \phi_1 \cup \delta \psi_1  =  \delta(\phi_1 \cup \delta \psi_1)
\end{gather*}
Ma quindi definendo $ \eta = (-)^k \phi \cup \delta \psi_1 + \phi_1 \cup \psi + \phi_1 \cup \delta \psi_1 $:
\[
  (\phi + \delta \phi_1) \cup (\psi + \delta \psi_1) = \phi \cup \psi + \phi \cup \delta \Psi_1 + \delta \phi_1 \cup \psi + \delta \phi_1 \cup \delta \psi_1 = \phi \cup \psi + \delta \eta
\]
La mappa è quindi ben definita a livello di coomologia e quindi si può dare la struttura
ad anello a $ H^\star(X,\R) $.

Se in particolare, come da qui in avanti assumo, $ X $ è connesso per archi:
\[
  H^0(X; \R) \cong \hom{H_0(X), \R} \cong \hom{\Z, \R} \cong \R
\]
Dove $ \hom{\Z, \R} \cong \R $ in quanto per specificare un omomorfismo da
$ \Z $ a $ \R $ mi basta dire quale è l'immagine di $ 1 $, la quale può essere
un qualunque elemento di $ \R $. Ma $ \R $ è unitario, quindi possiede un
elemento unità, e quindi si definisce l'unità in $ H^0(X; \R) $ e quindi in
tutto $ H^\star(X, \R) $ come l'elemento che corrisponde a $ \Id{\R} $ e che quindi
corrisponde anche a $ \Id{\hom{H_0(X), \R}} $, cioè $ \Id{} \colon \llbracket \phi \rrbracket \mapsto \llbracket \phi \rrbracket $.
Osservo che in $ H^\star(X; \R) $:
\[
  \llbracket \phi \rrbracket \cup \llbracket \Id{} \rrbracket = \llbracket \phi \cup \Id{} \rrbracket = \llbracket \phi \rrbracket = \llbracket \Id{} \cup \phi \rrbracket = \llbracket \Id{} \rrbracket \cup \llbracket \phi \rrbracket
\]
Quindi $ H^\star(X, \R) $ è un anello unitario, ma in generale non commutativo.

\begin{lemma}
  Siano $ X $ e $ Y $ spazi topologici omotopicamente equivalenti allora gli
  anelli di coomologia sono isomorfi (come anelli).
\end{lemma}
\begin{proof}
  Se $ X $ è equivalente a $ Y $ allora i gruppi di omologia sono isomorfi, cioè
  $ \forall k \; H_k(X) \cong H_k(Y) $, per il teorema dei coefficienti universali anche i
  gruppi di coomologia sono isomorfi come $ \Z $-moduli, infatti tale teorema
  afferma che:
  \[
    H^k(X; \R) \cong \tor{H_k(X), \R} \oplus \hom{H_{k-1}(X), \R}
  \]
  E siccome i gruppi di omologia sono isomorfi anche $ H^k(X; \R) $ lo sono,
  cioè $ H^\star(X) \cong H^\star(Y) $ come gruppi, devo mostrare che l'isomorfismo è anche
  di anelli. Se $ X \sim_H Y $ significa che esiste una mappa continua
  $ f \colon X \to Y $ e una $ g \colon Y \to X $ tali che
  $ f \circ g \sim_H \Id{Y} $ e $ g \circ f \sim_H \Id{X} $. Essendo $ f $ continua è ben
  definita
  \begin{align*}
    f_\sharp \colon S_k(X) & \to S_k(Y) \\
    \sigma & \mapsto f \circ \sigma
  \end{align*}
  Ma anche:
  \begin{align*}
    f^\sharp \colon S^k(Y) & \to S^X(X) \\
    \phi & \mapsto f^\sharp(\phi) = \phi(f_\sharp)
  \end{align*}
  Quindi si può passare alla coomologia:
  \begin{align*}
    f^\star \colon H^k(Y) & \to H^k(X) \\
    \llbracket \phi \rrbracket & \mapsto \llbracket f^\sharp \circ \phi \rrbracket
  \end{align*}
  Questa mappa è un omomorfismo di anelli, cioè:
  \begin{gather*}
    f^\star(\llbracket\phi\rrbracket + \llbracket\psi\rrbracket) = f^\star(\llbracket\phi\rrbracket) + f^\star(\llbracket\psi\rrbracket) \\
    f^\star(\llbracket\phi\rrbracket \cup \llbracket\psi\rrbracket) = f^\star(\llbracket\phi\rrbracket) \cup f^\star(\llbracket\psi\rrbracket)
  \end{gather*}
  Infatti, il comportamento rispetto alla somma è vero perché è vero anche come $ \Z $-moduli,
  mentre per il prodotto:
  Considero $ \sigma \colon \Delta_{k+l} \to X $ simplesso singolare:
  \begin{gather*}
    (f^\sharp(\phi) \cup f^\sharp(\psi))(\sigma) = (f^\sharp(\phi)(\sigma\lvert_{[v_0, \dots, v_k]})) (f^\sharp(\psi)(\sigma\lvert_{[v_{k+1}, \dots, v_{k+l}]})) = \\
    = \phi(f_\sharp(\sigma\lvert_{[v_o, \dots, v_k]}))\psi(f_\sharp(\sigma\lvert_{[v_{k+1}, \dots, v_{k+l}]})) = \phi \cup \psi (f_\sharp (\sigma)) = (f^\sharp(\phi \cup \psi))(\sigma)
  \end{gather*}
  Si può applicare il medesimo ragionamento anche per $ g $ e per l'assioma omotopico
  $ (f \circ g)^\star = (\Id{Y})^\star $ e $ (g \circ f)^\star = (\Id{X})^\star $, ma quindi $ f^\star $ e $ g^\star $ sono
  una l'inversa dell'altra ed essendo anche omomorfismi sono isomorfismi.
\end{proof}

\begin{example}
  Se $ X = \Sph{n} $, so che:
  \[
    H_k(\Sph{n}) \cong
    \begin{cases}
      \Z & \text{se } k \in \set{0,n} \\
      0 & \text{altrimenti}
    \end{cases}
  \]
  Ma $ \ext{H_k(\Sph{n}), \Z} \cong 0 $ dato che $ H_k(\Sph{n}) $ è libero, e quindi
  per il teorema dei coefficienti universali:
  \[
    H^k(\Sph{n}) \cong \hom{H_k(\Sph{n}), \Z} \cong
    \begin{cases}
      \hom{\Z, \Z} \cong \Z & \text{se } k \in \set{0,n} \\
      \hom{0, \Z} \cong 0 & \text{altrimenti}
    \end{cases}
  \]
  Ho che $ H^0(\Sph{n}) = \langle\Id{}\rangle $ e $ H^n(\Sph{n}) = \langle\alpha\rangle $ con $ \alpha $ opportuno
  generatore. La tabella di moltiplicazione tra questi generatori è:
  \[
    \begin{array}{c|cc}
      \cup & \Id{} & \alpha \\ \hline
      \Id{} & \Id{} & \alpha \\
      \alpha  & \alpha & 0 \\
    \end{array}
  \]
  Dove $ \alpha^2 = 0 $ in quanto $ \alpha^2 $ è in $ H^{2n}(X, G) `= 0 $. Quindi e il
  generico elemento è della forma $ a + b \alpha $ con $ a,b \in \Z $ e
  $ \alpha^2 = 0 $, cioè:
  \[
    H^\star(\Sph{n}) \cong \quot{\Z[\alpha]}{(\alpha^2)}
  \]
  Questo significa che $ H^\star(\Sph{n}) $ è formato da polinomi in $ \alpha $ in cui
  tutti i monomi che contengono un fattore $ \alpha^2 $ sono nulli.
\end{example}
\begin{example}
  A questo punto si posseggono gli strumenti necessari per risolvere il problema della
  distinzione tra $ \Pjc{2} $ e $ \Sph{2} \vee \Sph{4} $.
  Per Mayer-Vietoris $ H^\star(\Sph{2} \vee \Sph{4}) \cong H^\star(\Sph{2}) \oplus H^\star(\Sph{4}) $, quindi
  \[
    H^\star(\Sph{2} \vee \Sph{4}) \cong \quot{\Z[\alpha]}{(\alpha^2)} \oplus \quot{\Z[\beta]}{(\beta^2)}
  \]
  Successivamente dimostrerò che:
  \[
    H^\star(\Pjc{n}) \cong \quot{Z[x]}{(x^{n+1})}
  \]
  Dove $ x $ è un generatore di $ H^{2}(\Pjc{2}) $.
  Ora mostro quindi che $ \Sph{2} \vee \Sph{4} \not \sim_H \Pjc{2} $.
  \begin{gather*}
    H^\star(\Pjc{2}) = \set{a_0 + a_1 x + a_2 x^2 | x^3 = 0} \\
    H^\star(\Pjc{2}) = \set{(b_0 + b_1 \alpha, a_0 + a_1 \beta) | \alpha^2 = 0, \; \beta^2 = 0}
  \end{gather*}
  Se questi gruppi fossero isomorfi ci sarebbe una corrispondenza:
  \[
    x \leftrightarrow (b_0 + b_1 \alpha, a_0 + a_1 \beta)
  \]
  Ma quindi anche:
  \[
    x^3 = 0 \leftrightarrow (b_0^3 + 3 b_0^2 b_1 \alpha, a_0^3 + 3 a_0^2 a_1 \alpha)
  \]
  Ma se fosse un isomorfismo $ 0 $ dovrebbe andare in $ 0 $, cioè:
  \[
    \begin{cases}
      b_0^3 + 3 b_0^2 b_1 \alpha = 0 \Rightarrow b_0 = 0  \\
      a_0^3 + 3 a_0^2 a_1 \beta = 0 \Rightarrow a_0 = 0
    \end{cases}
  \]
  Cioè:
  \[
    x \leftrightarrow  (b_1 \alpha, a_1 \beta)
  \]
  Ma prendendo il quadrato avrei che:
  \[
    x^2 \leftrightarrow  (0, 0)
  \]
  Che è assurdo.
\end{example}

\begin{theorem}
  Siano $ x, y $ i generatori rispettivamente di $ H^1(\Pjr{n}; \Z_2) $ e $ H^2(\Pjc{n}; \Z) $,
  cioè:
  \[
    \langle x \rangle = H^1(\Pjr{n}, \Z_2) \cong \Z_2 \quad \langle y \rangle = H^2(\Pjc{n}, \Z) \cong \Z
  \]
  allora vale che:
  \begin{gather*}
    H^\star(\Pjr{n}; \Z_2) \cong \quot{\Z_2[x]}{(x^{n+1})} \\
    H^\star(\Pjc{n}; \Z) \cong \quot{\Z[y]}{(y^{n+1})}
  \end{gather*}
\end{theorem}
\begin{proof}
  La dimostrazione per i due risultati è la stessa, lo dimostro per il caso
  reale. La dimostrazione è per induzione, e in ciò che segue è sottinteso che
  il gruppo di coefficienti è $ \Z_2 $.
  Per $ n = 1 $ è noto che $ \Pjr{1} \simeq \Sph{1} $, e quindi ho già calcolato
  l'anello di coomologia:
  \[
    H^\star(\Pjr{1}) \cong \quot{\Z[x]}{(x^2)}
  \]
  Per $ n > 1 $ considero due indici $ i, j $. Mostro che posso restringermi al
  caso in cui $ i + j = n $. Se $ i + j < n $ considero
  $ u \colon \Pjr{k} \to \Pjr{n} $, ho che $ u^\star \colon H^l(\Pjr{n}) \homoto H^l(\Pjr{l}) $
  con $ l \leq j $, ma quindi:
  \begin{gather*}
    0 + \alpha_i \in H^i(\Pjr{n}) \mapsto u^\star(\alpha_i) \not = 0 \\
    0 + \alpha_j \in H^j(\Pjr{n}) \mapsto u^\star(\alpha_j) \not = 0
  \end{gather*}
  Ma $ u^\star(\alpha_i \cup \alpha+j) \not = 0 $ e $ \alpha_i \cup \alpha_j \in H^{i+j}(\Pjr{k}) $. Se $
  u^\star(\alpha_i \cup \alpha_j) = 0 $ quindi $ u^\star(\alpha_i) \cup u^\star(\alpha_j) = 0 $, ma $ u^\star(\alpha_i) = 0 $
  e $ u^\star(\alpha_j) \not = 0 $ e ili prodotto cup non manda in zero. In altri termini
  se $ i + j < n $ allora $ \alpha_i \cup \alpha_j $ è generatore di $ H^{i+j}(\Pjr{n}) $ e
  mi riconduco al caso precedente. Posso quindi fissare $ i, j $ tali che $ i +
  j = n $, e prendo $ \alpha_i, \alpha_j $ generatori tali che $ \langle\alpha_i\rangle = H^i(\Pjr{n}) $ e
  $ \langle\alpha_j\rangle = H^j(\Pjr{n}) $.

  Per definizione $ \Sph{n} = \set{(x_0, \dots, x_n) \in \RN{n+1} | \sum x_i^2 = 1} $,
  considero:
  \begin{gather*}
    \Sph{i} = \set{(x_0, \dots, x_{i}, 0, 0, \dots, 0) \in \Sph{n}} \\
    \Sph{j} = \set{(0, 0, \dots, x_{n-j}, \dots, x_n) \in \Sph{n}}
  \end{gather*}
  Se $ i + j = n $ queste due sottosfere si intersecano in due punti:
  \[
    \Sph{i} \cap \Sph{j} = \set{(0,\dots,\pm1,\dots,0)}
  \]
  \begin{figure}[htbp]
    \centering
    \begin{tikzpicture}
      \draw (0,0) circle (1);
      \draw[name path=line1]
      (0,1) to[out=-120, in=150] (0,-1);
      \draw[name path=line2]
      (-1,0) to[out=-50, in=-150] (1,0);
      \draw[dashed, name path=line3]
      (0,1) to[out=-40, in=60] (0,-1);
      \draw[dashed, name path=line4]
      (-1,0) to[out=30, in=130] (1,0);
      \path [name intersections={of=line1 and line2,by=E}];
      \node [fill=black,inner sep=1pt] at (E) {};
      \path [name intersections={of=line3 and line4,by=F}];
      \node [fill=black,inner sep=1pt] at (F) {};
    \end{tikzpicture}
    \caption{Intersezione tra $ \Sph{i} $ e $ \Sph{j} $}
    \label{fig:lez18:intersection}
  \end{figure}
  So che $ \Pjr{n} = \Sph{n} \slash \sim \; = \Pjr{n-1} \cup_\pi \Disk{n} $, dove $ \sim $ è la
  relazione antipodale. Quindi $ \Pjr{n} - \set{p} $ è retratto di deformazione
  di $ \Pjr{n-1} $.
  Costruisco il seguente diagramma commutativo:
  \[
    \begin{tikzcd}[nodes = {column sep = 10pt, outer sep = 0pt, inner sep = 1.5pt}]
      H^i(\Pjr{n}) \times H^j(\Pjr{n}) \rar{\cup} & H^n(\Pjr{n}) \\
      H^i(\Pjr{n}, \Pjr{n} \setminus \Pjr{j}) \times H^j(\Pjr{n}, \Pjr{n} \setminus \Pjr{i}) \uar{\cong} \rar{\cup} \dar{\cong} & H^n(\Pjr{n}, \Pjr{n} \setminus \set{p}) \dar{\cong} \uar{\cong} \\
      H^i(\RN{n}, \RN{n} \setminus \RN{j}) \times H^j(\RN{n}, \RN{n} \setminus \RN{i})  \rar{\cup} & H^n(\RN{n}, \RN{n} \setminus \set{\vec{0}})
    \end{tikzcd}
  \]
  Infatti ho $ \cup \colon H^i(\Pjr{n}) \times H^j(\Pjr{n}) \to H^n(\Pjr{n}) $. Poi ho la successione
  esatta lunga in omologia:
  \[
    \begin{tikzcd}[nodes = {column sep = 5pt, inner sep = 1.5pt}]
      H^{n-1}(\Pjr{n}) \rar & H^{n-1}(\Pjr{n} \setminus \set{p}) \rar & H^{n-1}(\Pjr{n}, \Pjr{n} \setminus \set{p}) \rar & H^n(\Pjr{n}) \rar & H^n(\Pjr{n} \setminus \set{p})
    \end{tikzcd}
  \]
  Ma $ H^{n-1}(\Pjr{n}) \cong H^{n-1}(\Pjr{n-1}) $ quindi $ H^{n-1}(\Pjr{n} \setminus \set{p}) \cong H^{n-1}(\Pjr{n}) $
  quindi $ H^n(\Pjr{n}, \Pjr{n} \setminus \set{p}) \cong H^n(\Pjr{n}) $.
  Poi ho $ \Pjr{i} \cong \Sph{i} \setminus \sim $ e $ \Pjr{j} \cong \Sph{j} \setminus \sim $, quindi
  $ H^i(\Pjr{n}, \Pjr{n} \setminus \Pjr{j}) \times H^j(\Pjr{n}, \Pjr{n} \setminus \Pjr{i}) $ vanno
  in $ H^i(\Pjr{n}) \times H^i(\Pjr{n}) $.

  Poi $ \Pjr{n} \setminus (\Pjr{n} \setminus \Pjr{j} \cup \Pjr{n} \setminus \Pjr{i}) = \Pjr{n} \setminus (\Pjr{n} \setminus (\Pjr{j}) \cap \Pjr{i}) = \Pjr{n} \setminus \set{p} $
  con $ p = [0,\dots,1,0,\dots,0] $.

  Faccio escissione con $ U_i = \set{[x_0, \dots, x_n] | x_i \not = 0} \cong \RN{n} $,
  ma $ U_i = \left(\frac{x_0}{x_i}, \dots, \frac{x_n}{x_i} \right) $ è contraibile.

  Prendo $ \Pjr{n} \setminus U_i $ e faccio l'escissione in omologia e poi prendo il duale, così
  ottengo:
  \begin{gather*}
    H^n(\Pjr{n}, \Pjr{n} \setminus \set{p}) \cong H^n(\Pjr{n} \setminus (\Pjr{n} \setminus U_i)), (\Pjr{n} \setminus \set{p}) \setminus (\Pjr{n} \setminus U_i) \cong \\
    \cong H^n(U_i, U_i \setminus \set{p}) = H^n(\RN{n}, \RN{n} \setminus \set{\vec{0}})
  \end{gather*}

  Poi c'è il prodotto cup in basso in quanto ho la successione esatta:
  \[
    \begin{tikzcd}
      H^{n-1}(\RN{n}) \rar & H^{n-1}(\RN{i} \setminus \RN{j}) \rar & H^n(\RN{n}, \RN{n} \setminus \RN{j}) \rar & H^n(\RN{n})
    \end{tikzcd}
  \]
  Ma $ H^n(\RN{n}) \cong 0 $, quindi la successione è:
  \[
    \begin{tikzcd}
      0 \rar & H^{n-1}(\RN{i} \setminus \RN{j}) \rar & H^n(\RN{n}, \RN{n} \setminus \RN{j}) \rar & 0
    \end{tikzcd}
  \]
  E quindi:
  \[
     H^{n-1}(\RN{i} \setminus \RN{j}) \cong  H^n(\RN{n}, \RN{n} \setminus \RN{j}) \cong H^{n-1}(\Sph{n-j-1}) \cong H^{i-1}(\Sph{i-1})
  \]

  Con l'ipotesi induttiva costruisco gli isomorfismi in alto a sinistra.
  Mancano da dimostrare delle cose.

\end{proof}

\section{Coomologia di de Rham}

\begin{definition}
  Uno spazio topologico $ \M $ è una \textbf{varietà differenziabile}\index{Varietà differenziabile}
  se è una varietà topologica di dimensione $ n $ e possiede un \textbf{atlante differenziabile}\index{Atlante},
  ovvero una collezione di coppie $ (U_\alpha, \phi_\alpha) $ tale che:
  \begin{enumerate}
  \item $ U_\alpha $ sono intorni aperti omeomorfi a intorni in $ \RN{n} $, e l'omeomorfismo
    è realizzato dalla mappa $ \phi_\alpha $, detta \textbf{carta}\index{Carta di varietà differenziabile}.
  \item Ogni punto $ p \in \M $ possiede almeno un intorno $ U_\alpha $ che lo
    contiene. Questo significa in particolare che $ \set{U_\alpha} $ è un
    ricoprimento per $ \M $.
  \item I cambiamenti di coordinate siano buoni, cioè date due carte
    $ \phi_1\colon A_1 \subseteq U_1 \to \RN{n} $, $ \phi_2 \colon A_2 \subseteq U_2 \to \RN{n} $ con
    $ A_1, A_2 $ intorni aperti $ A_1 \cap A_2 \not = 0 $ allora:
    \[
      \phi_2 \circ \phi_1^{-1} \colon \phi_1(A_1 \cap A_2) \to \phi_2(A_1 \cap A_2) \in \mathcal{C}^\infty(\phi_1(A_1 \cap A_2) \subseteq \RN{n})
    \]
  \end{enumerate}
  Inoltre si richiede per questioni tecniche che l'atlante sia massimale\footnote{Si può sempre
    estendere in modo univoco un atlante in modo che l'estensione sia massimale.}.
\end{definition}
\newmathsymb{formdiff}{\Omega^k(\M)}{Spazio delle $ k $-forme differenziali su
  $ \M $}
\begin{definition}
  Su $ \M $ si definisce una \textbf{$ k $-forma differenziale}\index{Forma differenziali} come
  una applicazione multilineare completamente antisimmetrica $ \omega $ tale che
  \[
    \omega_x \colon \underbrace{\mathcal{T}_x\M \times \dots \times \mathcal{T}_x\M}_{k} \to \RN{}
  \]
  Dove $ \mathcal{T}_x\M $ è lo \textbf{spazio tangente} a $ \M $ in $ x $, cioè lo spazio
  dei vettori in $ x $.
  In generale una $ k $-forma si scrive, usando la convenzione di Einstein, come:
  \[
    \omega_x = a_{i_1\dots i_k}(x) \d x^{i_1} \wedge \dots \wedge \d x^{i_k}
  \]
  dove $ \wedge $ indica il prodotto wedge, definito a breve. Lo spazio delle
  $ k $-forme su $ \M $ si denota con $ \Omega^k(\M) $ ed è uno spazio vettoriale
  con la naturale operazione di somma e prodotto per uno scalare.
\end{definition}
\newmathsymb{wedge}{\alpha\wedge\beta}{Prodotto wedge tra $ \alpha $ e $ \beta $}
\begin{definition}
  Lo spazio delle forme differenziali può essere reso un'algebra con il \textbf{prodotto wedge}\index{Prodotto wedge}
  che associa a una $ p $-forma e a una $ q $-forma una $ (p+q) $-forma:
  \begin{align*}
    \Omega^p(\M) \times \Omega^q(\M) & \to \Omega^{p+q}(\M) \\
    (\omega_1, \omega_2) & \mapsto \omega_1 \wedge \omega_2
  \end{align*}
  In componenti il prodotto wedge di due forme si ottiene usando le proprietà di anello dell'algebra
  considerando però che si richiede che:
  \[
    \forall i,j \; \d x^i \wedge \d x^j = - \d x^j \wedge \d x^i
  \]
\end{definition}
Il prodotto wedge è bilineare e associativo e non commutativo, ma vale che:
\[
  \omega_1 \wedge \omega_2 = (-)^{pq} \omega_2 \wedge \omega_1
\]
Con il quale si ritrova giustamente che $ \d x^i \wedge \d x^j = - \d x^j \wedge \d x^i $.
\begin{example}\hfill
  \begin{itemize}
  \item Le $ 0 $-forme sono funzioni ordinarie
  \item Le $ 1 $-forme sono variabili di Grassmann
  \end{itemize}
\end{example}
\newmathsymb{derest}{\d{\omega}}{Derivata esterna di $ \omega $}
\begin{definition}
  Si definisce la \textbf{derivata esterna}\index{Derivata esterna}:
  \begin{align*}
    \d{} \colon \Omega^k(\M) & \to \Omega^{k+1}(\M) \\
    \omega & \mapsto \d \omega
  \end{align*}
  Con:
  \[
    \d \omega = \frac{\partial a_{i_1\dots i_k}(x )}{\partial x^j} \d x^j \wedge \d x^{i_1} \wedge \dots \wedge \d x^{i_k}
  \]
\end{definition}
\begin{osservation}
  Si può verificare esplicitamente che $ \d{}^2 = 0 $.
\end{osservation}
In ciò che segue considero $ \M $ varietà differenziabile connessa (in caso non
sia connessa mi restringo alle componenti connesse), con base numerabile
(questo è una richiesta puramente tecnica), senza bordo e orientata.
\begin{definition}
  Una varietà differenziabile $ \M $ di dimensione $ n $ si dice
  \textbf{orientata}\index{Varietà differenziabile orientata} se esiste una
  $ n $-forma $ \omega $ tale che $ \omega(p) \not= 0 \; \forall p \in \M $. Una forma con tale proprietà
  è detta \textbf{forma di volume}\index{Forma di volume}. Equivalentemente si può
  dire che una varietà differenziabile è orientata se tutte i cambiamenti
  di coordinate hanno determinante Jacobiano positivo.
\end{definition}
\begin{osservation}
  La richiesta di orientazione serve affinché gli integrali di forme differenziabili
  siano ben definiti, infatti la forma di volume dà origine alla misura di integrazione
  alla Lebesgue.
\end{osservation}
Per le forme differenziali vale inoltre il teorema di Stokes:
\begin{theorem}[Teorema di Stokes\index{Teorema di Stokes}]
  Se $ \M $ è una varietà differenziabile di dimensione $ n $ con bordo $ \partial \M $
  e $ \omega $ una $ (n-1) $-forma differenziabile su $ \M $ con supporto compatto, allora:
  \[
    \int_\M \d \omega = \int_{\partial \M} \omega
  \]
\end{theorem}
\begin{definition}
  Si definisce il \textbf{complesso di de Rham}\index{Complesso di de Rham}
  il complesso $ (\Omega^\bullet, \d{}) $.
\end{definition}
\newmathsymb{rot}{\nabla\times}{Rotore}
\newmathsymb{rot}{\nabla\cdot}{Divergenza}
\begin{example}
  Considero $ \M = \RN{3} $, questa è una varietà differenziabile avente come
  carta la mappa identità. Considero il complesso di de Rham:
  \[
    \begin{tikzcd}[nodes = {column sep = 10 pt}]
      0 \rar & \Omega^0(\M) \rar & \Omega^1(\M) \rar & \Omega^2(\M) \rar & \Omega^3(\M) \rar & 0
    \end{tikzcd}
  \]
  La prima derivata esterna corrisponde ad un gradiente, in quanto le
  $ 0 $-forme sono funzioni ordinarie. Considero
  $ \omega = a \d x + b \d y + c \d z $ con $ (x,y,z) $ coordinate di
  $ \RN{3} $, e $ a, b, c \in \Omega^0(\M) $ allora:
  \begin{align*}
    \d \omega & =          \frac{\partial a}{\partial y} \d y \wedge \d x + \frac{\partial a}{\partial z} \d z \wedge \d x + \frac{\partial b}{\partial x} \d x \wedge \d y +
             \frac{\partial b}{\partial z} \d z \wedge \d y + \frac{\partial c}{\partial x} \d x \wedge \d z + \frac{\partial c}{\partial y} \d y \wedge \d z = \\
         & =      \left[ - \frac{\partial a}{\partial y} + \frac{\partial b}{\partial x} \right] \d x \wedge \d y +
                 \left[ - \frac{\partial a}{\partial z} + \frac{\partial c}{\partial x} \right] \d x \wedge \d z +
                 \left[ - \frac{\partial b}{\partial z} + \frac{\partial c}{\partial y} \right] \d y \wedge \d z =                      \\
         & =  \nabla \times (a \d x + b \d y + c \d y)
  \end{align*}
  Dove con $ \nabla \times $ si intende il rotore. Ma è noto che il rotore del gradiente di una funzione è nullo,
  cioè $ \nabla \times \nabla f = 0 $, cioè $ d^2 = 0 $ per $ k = 0 $ e $ k = 1 $.
  Faccio il passo successivo. Sia $ \eta \in \Omega^2(\M) $:
  \[
    \eta = p \d x \wedge \d y - q \d x \wedge \d z + r \d y \wedge \d z
  \]
  Il bordo è:
  \begin{align*}
    \d \eta & = \frac{\partial p}{\partial z} \d z \wedge \d x \wedge \d y - \frac{\partial q}{\partial y} \d y \wedge \d x \wedge \d z + \frac{\partial r}{\partial x} \d x \wedge \d y \wedge \d z = \\
    & = \left( \frac{\partial r}{\partial x} + \frac{\partial q}{\partial y} + \frac{\partial p}{\partial z} \right) \d x \wedge \d y \wedge \d z\\
    & = \nabla \cdot \eta \d x \wedge \d y \wedge \d z
  \end{align*}
  Dove con $ \nabla \cdot $ si intende la divergenza. Ma è noto che la divergenza di un rotore è nulla,
  cioè $ \nabla \cdot \nabla \times f = 0 $, cioè $ d^2 = 0 $ per $ k = 1 $ e $ k = 2 $.
\end{example}
\begin{definition}
  Si chiama \textbf{coomologia di de Rham}\index{Coomologia di de Rham}
  l'omologia del complesso di de Rham $ (\Omega^\bullet, \d{}) $:
  \[
    H^p_{dR}(\M) = \quot{\ker{ \d{} \colon \Omega^p(\M) \to \Omega^{p+1}(\M)}}{\im{\d{} \colon \Omega^{p-1}(\M) \to \Omega^p(\M)}}
  \]
  Indicando con:
  \begin{gather*}
    Z^p(\M) = \ker{ \d{} \colon \Omega^p(\M) \to \Omega^{p+1}(\M)} = \set{\omega \in \Omega^p(\M) | \d \omega = 0} \\
    B^p(\M) = \im{\d{} \colon \Omega^{p-1}(\M) \to \Omega^p(\M)} = \set{\gamma \in \Omega^p(\M) | \exists \rho \in \Omega^{p+1}(\M) | \gamma = \d \rho}
  \end{gather*}
  Si ha che $ Z^p(\M) $ sono le  \textbf{$ p $-forme chiuse}\index{Forme chiuse} e $ B^p(\M) $ sono
  le \textbf{ $ p $-forme esatte}\index{Forme esatte}.
\end{definition}
Il generico elemento di $ H^p_{dR}(\M) $ è $ [\omega] $ con $ \omega $ chiusa. Se
$ H^p_{dR}(\M) $ è banale significa che tutte le forme sono esatte, in quanto
non ci sono forme chiuse che non siano anche esatte (cioè elementi di
$ B^p(\M) $ che non sono in $ Z^p(\M) $).
\begin{osservation}
  La coomologia di de Rham può essere strutturata in modo naturale a spazio
  vettoriale sui reali, e se $ \M $ è anche compatto lo spazio è sui reali. Se
  $ \M $ non è compatto si costruisce la \textbf{coomologia a supporto
    compatto}\index{Coomologia a supporto compatto} $ H^p_c(\M) $ in cui si
  lavora con le forme differenziali a \textbf{supporto compatto}\index{Supporto
    compatto}, cioè tali che la chiusura dell'insieme su cui tali forme sono non
  nulle è un insieme compatto. Chiaramente se $ \M $ è compatta ogni forma
  differenziale è a supporto compatto.
\end{osservation}
\begin{lemma}
  Si dimostra che, a differenza della coomologia di de Rham, la coomologia a supporto
  compatto è covariante e non controvariante.
\end{lemma}

\subsection{Dualità di Poincaré}
\begin{osservation}
  Se $ b \colon V \times W \to \mathbb{F} $ è un funzionale bilineare su $ V,W $ spazi
  vettoriali e $ \mathbb{F} $ campo, allora questo induce un'applicazione:
  \begin{align*}
    B \colon V & \to W^* = \hom{W, \mathbb{F}} \\
    v & \mapsto B(v)
  \end{align*}
  Con:
  \begin{align*}
    B(v) \colon W & \to \mathbb{F} \\
    w & \mapsto b(v,w)
  \end{align*}
  Cioè $ B(v) = b(v, \cdot) $. Si dimostra che se $ b \colon V \times V \to \mathbb{F} $
  è non degenere (cioè se $ b(v,w) = 0 \; \forall w \in V $ implica che $ v = 0$) allora
  $ B \colon V \to V^* $ è un isomorfismo, e quindi esiste un accoppiamento canonico tra
  $ V $ e il suo duale.
\end{osservation}
Costruisco l'applicazione $ b $ per gli spazi $ \Omega $. Sia $ k \leq \dim \M $ definisco:
\begin{align*}
  I \colon \Omega^k(\M) \times \Omega^{n-k}(\M) & \to \RN{} \\
  (\alpha,\beta) & \mapsto \int_\M \alpha \wedge \beta
\end{align*}
Se $ \M $ è compatto l'integrale è ben definito, se $ \M $ non è compatto si
deve lavorare con forme differenziali a supporto compatto. Assumo $ \M $
compatto, definisco una mappa $ I $ sulla coomologia di de Rham, la quale è uno
spazio vettoriale sui reali e quindi suscettibile dell'osservazione precedente:
\begin{align*}
  I \colon H^k_{dR}(\M) \times H^{n-k}_{dR}(\M) & \to \RN{} \\
  ([\alpha], [\beta]) & \mapsto \int_\M \alpha \wedge \beta
\end{align*}
Questa mappa è ben definita, infatti considero altri due rappresentanti per le
classi $ [\alpha] $ e $ [\beta] $ $ \alpha + \d \alpha' $ e $ \beta + \d \beta' $. Ho che:
\[
  \int_\M (\alpha + \d \alpha') \wedge (\beta + \d \beta') = \int_\M \alpha \wedge \beta + \int_\M \d \alpha' \wedge \beta + \int_\M \d \alpha' \wedge \d \beta' + \int_\M \alpha \wedge \d \beta'
\]
Ma considerando che $ \alpha $ e $ \beta $ sono chiuse, cioè $ \d \alpha = \d \beta = 0 $:
\begin{gather*}
  \d (\alpha' \wedge \beta) = \d \alpha' \wedge \beta + \cancel{(-)^{k-1} \alpha' \wedge \d \beta} \Rightarrow  \d \alpha' \wedge \beta = \d (\alpha' \wedge \beta) \\
  \d (\alpha \wedge \beta') = \cancel{\d \alpha \wedge \beta'} + (-)^{k} \alpha \wedge \d \beta' \Rightarrow  \alpha \wedge \d \beta' = (-)^k  \d (\alpha \wedge \beta') \\
  \d (\alpha' \wedge \d \beta') = \d \alpha' \wedge \d \beta' + \cancel{\d \alpha' \wedge \d {\d \beta}} \Rightarrow \d \alpha' \wedge \d \beta' =  \d (\alpha' \wedge \d \beta')
\end{gather*}
Quindi:
\[
  \int_\M (\alpha + \d \alpha') \wedge (\beta + \d \beta') = \int_\M \alpha \wedge \beta + \int_\M \d {(\alpha' \wedge \beta)} + \int_\M \d {(\alpha' \wedge \d \beta')} + \int_\M \d {(\alpha \wedge \d \beta')}
\]
Per il teorema di Stokes le forme esatte integrate su $ \M $ sono nulle non
essendoci termini di bordo, quindi la mappa è ben definita in quanto:
\[
  \int_\M (\alpha + \d \alpha') \wedge (\beta + \d \beta') = \int_\M \alpha \wedge \beta
\]
\begin{theorem}[Teorema di isomorfismo di Poincaré\index{Teorema di isomorfismo di Poincaré}]
  Se $ \M $ è una varietà differenziabile senza bordo e orientata, allora la
  mappa:
  \begin{align*}
    D \colon H^k_{dR}(\M) & \to (H_c^{n-k}(\M))^* \\
    [\alpha] & \mapsto D([\alpha])
  \end{align*}
  con:
  \begin{align*}
    D([\alpha]) \colon H^{n-k}_c(\M) & \to \RN{} \\
    [\beta] & \mapsto \int_\M \alpha \wedge \beta
  \end{align*}
  è un isomorfismo di gruppi abeliani, cioè $ H^k_{dR}(\M) \cong (H^{n-k}_c(\M))^* $.
\end{theorem}
\begin{proof}\emph{Idee della dimostrazione.}
  La dimostrazione è piuttosto articolata e si svolge per passi:
  \begin{enumerate}
  \item Dimostrazione del teorema per $ \M = \RN{n} $
  \item Dimostrazione del teorema per $ U $ aperto in $ \M $ tale che sia diffeomorfo a $ \RN{n} $
    e con $ D $ ristretta a $ U $
  \item Dimostrazione del teorema per qualsiasi aperto di $ \RN{n} $
  \item Dimostrazione del teorema per qualsiasi aperto proprio di $ \M $
  \item Dimostrazione del teorema per $ \M $
  \end{enumerate}

  \subparagraph{Dimostrazione del punto uno} Bisogna dimostrare che
  $ D \colon H^k_{dR}(\RN{n}) \to (H^{n-k}(\RN{n}))^* $ è un isomorfismo. Siccome
  $ \RN{n} $ è semplicemente connesso tutte le forme chiuse sono esatte (lemma
  di Poincaré) e quindi tutti i gruppi di coomologia di de Rham per $ k > 0 $
  sono banali in quanto esiste solo la classe $ [0] $, l'unico gruppo non nullo
  è $ H^0_{dR}(\RN{n}) $. Ma
  \[
    H^0_{dR}(\RN{n}) = \quot{Z^0(\RN{n})}{B^0(\RN{n})} = Z^0(\RN{n}) =
    {\set{\text{funzioni costanti}}} \cong \RN{}
  \]
  In quanto $ B^0(\RN{n}) $ è banale, e $ Z^0(\RN{n}) $ è formato dalle
  $ 0 $-forme chiuse, cioè le funzioni il cui gradiente è nullo, ovvero le
  funzioni costanti. Quindi:
  \[
    H^0_{dR}(\RN{n}) \cong
    \begin{cases}
      \RN{} & \text{se } k = 0 \\
      0 & \text{altrimenti}
    \end{cases}
  \]
  Ma anche $ H^{n-k}_c(\RN{n}) $ ha gli stessi gruppi di coomologia:
  \[
    H^{n-k}_c(\RN{n}) \cong
    \begin{cases}
      \RN{} & \text{se } k = 0 \\
      0 & \text{altrimenti}
    \end{cases}
  \]
  infatti $ \RN{n} $ è semplicemente connesso, mentre
  $ H^{n}_c(\RN{n}) \cong \RN{} $ in quanto è generato da $ n $-forme a supporto compatto del tipo:
  \[
    \omega = \phi(x_1, \dots, x_n) \d x^1 \wedge \dots \wedge \d x^n
  \]
  con $ \phi \in \mathcal{C}^\infty $ a supporto compatto. Queste forma è esatta, infatti
  nel caso $ n = 1 $ ho $ \omega = \phi(t) \d t $, ponendo:
  \[
    \psi(x) = \int_{-\infty}^x \phi(t) \d t
  \]
  Ho che $ \psi $ è una $ 0 $-forma tale che
  $ \d \psi (x) = \psi'(x) \d x = \phi(x) \d x $, quindi $ \omega $ è esatta ed è perciò il
  generatore del gruppo di coomologia. Per $ n $ generico integro una alla volta
  tutte le variabili e ottengo il medesimo risultato.

  Per mostrare che $ D $ è isomorfismo è sufficiente mostrarlo per
  $ D \colon H^0_{dR}(\RN{n}) \to (H^n_c(\RN{n}))^* $, ma:
  \begin{align*}
    D \colon \RN{} & \to {\RN{}}^* \cong \RN{} \\
    1 & \mapsto D(1)
  \end{align*}
  Se dimostro che $ D(1) $ è un generatore di $ \RN{} $ ho finito. Per mostrare
  che $ D(1) $ è un generatore è sufficiente che controllo che non sia $ 0 $,
  ma il funzionale $ 0 $ è quella mappa che manda tutte le funzioni in $ 0 $:
  cioè è tale che $ D(1)(\phi) = 0 $ $ \forall \phi $, per mostrare che $ D(1) $ non è $ 0 $
  basta quindi trovare una funziona $ \phi $ tale che $ D(1)(\phi) \not = 0 $, ma
  \[
    D(1)(\phi) = \int_{\RN{n}} \phi \d x^1 \wedge \dots \wedge \d x^n
  \]
  Ma questa è facilmente costruibile, basta prendere una funzione tipo
  mollificatore.

  \subparagraph{Dimostrazione del punto due} Se $ U $ è un aperto in $ \M $
  diffeomorfo a $ \RN{n} $ siccome i gruppi di coomologia sono invarianti per
  diffeomorfismi allora la mappa
  $ D_U \colon H^k_{dR}(U) \to (H^{n-k}_c(U))^\star $ è un isomorfismo.

  \subparagraph{Dimostrazione del punto tre} Considero una base $ \mathcal{B} $
  della topologia usuale di $ \RN{n} $ tale che:
  \begin{enumerate}
  \item L'intersezione di due aperti in $ \mathcal{B} $ è ancora in $ \mathcal{B} $
  \item Il teorema vale per ogni aperto in $ \mathcal{B} $
  \end{enumerate}
  Una possibile scelta di questa base è quella dei polirettangoli aperti i quali
  essendo diffeomorfi a $ \RN{n} $ soddisfano il teorema di dualità di Poincaré,
  come si è dimostrato precedentemente.
  Si dimostrano i seguenti lemmi:
  \begin{lemma}
    Il teorema è valido per ogni unione finita di aperti di $ \mathcal{B} $.
  \end{lemma}
  \begin{lemma}
    Il teorema è valido per ogni unione non necessariamente finita di elementi di $ \mathcal{B} $.
  \end{lemma}
  Siccome ogni aperto è unione, al più infinita di elementi di $ \mathcal{B} $ essendo $ \mathcal{B} $
  una base il punto è dimostrato.

  \subparagraph{Dimostrazione del punto quattro}

  \subparagraph{Dimostrazione del punto cinque} Siano $ V_1, V_2 $ aperti propri
  in $ \M $ tali che $ \M = V_1 \cup V_2 $, introducendo l'abbreviazione
  $ V_{12} = V_1 \cap V_2 $ allora per il teorema di Mayer-Vietoris la successione corta
  di complessi:
  \[
    \begin{tikzcd}[nodes = {row sep = 1 pt}]
      0 \rar & \Omega^\bullet(\M) \rar & \Omega^\bullet(V_1) \oplus \Omega^\bullet(V_2) \rar & \Omega^\bullet(V_{12}) \rar & 0 \\
      {} & \omega \rar[mapsto] & \omega\lvert_{V_1} \oplus\; \omega\lvert_{V_2} & {} & {} \\
      {} & {} & (\eta_1, \eta_2) \rar[mapsto] & \eta_1\lvert_{V_{12}} - \;\eta_2\lvert_{V_{12}} & {}
    \end{tikzcd}
  \]
  Induce quella lunga in coomologia:
  \[
    \begin{tikzcd}[nodes = {column sep = 6 pt, inner sep = 1.5pt}]
      H_{dR}^{k-1}(V_1) \oplus H_{dR}^{k-1}(V_2) \rar{\alpha_1} \dar{D_{V_1} \oplus D_{V_2}} & H^{k-1}_{dR}(V_{12}) \dar{D_{V_{12}}} \rar{\alpha_2} &
      H^k_{dR}(\M) \rar{\alpha_3} \dar{D_{V_{12}}} & H_{dR}^{k}(V_1) \oplus H_{dR}^{k}(V_2) \rar{\alpha_4}  \dar{D_{V_1} \oplus D_{V_2}} & H^k_{dR}(V_{12}) \dar{D_{V_{12}}}  \\
      (H_{dR}^{k-1}(V_1))^* \oplus (H_{dR}^{k-1}(V_2))^* \rar{\beta_1} &
      (H^{k-1}_{dR}(V_{12}))^* \rar{\beta_2} &(H^k_{dR}(\M))^* \rar{\beta_3} &
      (H_{dR}^{k}(V_1))^* \oplus (H_{dR}^{k}(V_2))^* \rar{\beta_4} & (h^k_{dR}(V_{12}))^*
    \end{tikzcd}
  \]
  Nella seconda riga si è usato il fatto che il duale di una somma diretta di spazi finitamente
  generati è la somma dei duali.
  Per i punti dimostrati in precedenza tutte le mappe $ D $ sono isomorfismi, a parte
  quella centrale, se dimostro che i quadrati sono commutativi per il lemma dei cinque
  $ D $ deve essere un isomorfismo. Per comodità chiamo $ \phi_1 = D_{V_1} \oplus D_{V_2} $,
  $ \phi_2 = D_{V_{12}} $, $ \phi_3 = D $, $ \phi_5 = D_{V_1} \oplus D_{V_2} $ t $ \phi_5 = D_{V_{12}} $.
  \begin{osservation}
    L'esistenza di queste successioni è dovuta al fatto che ci sono delle mappe
    di inclusione $ \tau \colon V_i \to \M $, Nel caso della successione in coomologia di
    de Rham l'associazione è contravariante e quindi si scambia il verso, nel caso
    della coomologia a supporto compatto l'associazione è covariante, ma si scambia il verso
    in quanto si prende il duale.
  \end{osservation}
  Bisogna dimostrare che i quadrati sono commutativi. Quello in mezzo è semplice,
  considero $ [\alpha] \in H^k_{dR}(\M) $:
  \begin{gather*}
    [\alpha_1] = [\tau_1^\star(\alpha)] \in H^k_{dR}(V_1) \\
    [\alpha_2] = [\tau_2^\star(\alpha)] \in H^k_{dR}(V_2)
  \end{gather*}
  Cioè $ \alpha = \alpha_1 + \alpha_2 $, ma:
  \begin{gather*}
    D(\alpha) \colon \beta \to \int_\M \alpha \wedge \beta =  \int_\M (\alpha_1 + \alpha_2) \wedge \beta =  \int_{V_1} \alpha_1 \wedge \beta + \int_{V_2} \alpha_2 \wedge \beta \\
    (D(\alpha_1) + D(\alpha_2)) \colon \beta \to  \int_{V_1} \alpha_1 \wedge \beta +  \int_{V_2} \alpha_2 \wedge \beta
  \end{gather*}
  Quindi giustamente è commutativo.
  Poi ho:
  \[
    \begin{tikzcd}
      H^{k-1}_{dR}(V_{12}) \rar & H^k_{dR}(\M) \\
      (H^{n-k+1}_{c}(V_{12}))^* \rar{\Phi} & (H^{n-k}_{c}(\M))^*
    \end{tikzcd}
  \]
  Considero $ [\rho] \in H^{k-1}_{dR}(V_{12}) $ quindi con $ \d \rho = 0 $. Ho che
  $ \rho = (\rho_1 - \rho_2) \lvert_{V_{12}} $ con
  $ \rho_1 \in \Omega^{k-1}(V_1) $ e $ \rho_2 \in \Omega^{k-1}(V_2) $ infatti
  $ \rho $ è a supporto compatto quindi posso estenderla fuori dall'insieme.
  L'omomorfismo di connessione è $ \rho \mapsto \rho' \in \Omega^k(\M) $ tale che $ \rho'\lvert_{V_1} = \d \rho_1 $
  e $ \rho' \lvert_{V_2} = \d \rho_2 $. Quindi:
  \[
    \begin{tikzcd}
      \rho \rar \dar & \rho' \dar \\
      D([\rho]) & D([\rho'])
    \end{tikzcd}
  \]
  Devo mostrare che $ \Phi(D[\rho]) = D([\rho']) $, in questo modo il diagramma è commutativo.
  Ma $ \Phi = F^* $ con $ F \colon H^{n-k}_c(\M) \to H^{n-k=1}(V_{12}) $. Sia $ \tau \in \Omega_c^{n-k}(\M) $
  con $ \d \tau = 0 $ allora $ \tau = \tau_1 + \tau_2 $ con $ \tau_1 = \tau \lvert_{V_1} $ e $ \tau_2 = \tau_{V_2} $
  quindi:
  \[
    \d \tau = \d \tau_1 + \d \tau_2 \Rightarrow 0 = \d \tau_1 + \d \tau_2 \Rightarrow \d \tau_1 = - \d \tau_2
  \]
  Ma $ \tau_1 $ è definito su $ V_1 $ e $ \tau_2 $ su $ V_2 $, quindi devono necessariamente
  essere entrambi definiti su $ V_{12} $.
  Poi ho $ \rho \in H^k_{dR}(\M) $ e $ \tau \in H^{n-k}_c(\M) $:
  \[
    D([\rho'])([\tau]) = \int_\M \rho' \wedge \tau = \int_\M \rho' \wedge (\tau_1 + \tau_2) = \int_{V_1} \rho' \wedge \tau_1 + \int_{V_2} \rho' \wedge \tau_2
  \]
  Ma essendo $ \rho $ chiusa:
  \begin{gather*}
    \d {(\rho \wedge \tau_1)} = \cancel{\d \rho \wedge \tau_1} + (-)^k \rho \wedge \d \tau_1 \; \Rightarrow \; \d {(\rho \wedge \tau_1)} = (-)^k \rho \wedge \d \tau_1 \\
    \d {(\rho \wedge \tau_2)} = \cancel{\d \rho \wedge \tau_2} + (-)^k \rho \wedge \d \tau_2 \; \Rightarrow \; \d {(\rho \wedge \tau_2)} = (-)^k \rho \wedge \d \tau_2
  \end{gather*}
  E:
  \[
    \rho' \lvert_{V_1} = \d \rho_1 \;   \rho' \lvert_{V_2} = \d \rho_2 \; \Rightarrow \;  \int_{\M} \rho' \wedge \tau = \int_{V_1} \d \rho_1 \wedge \tau_1 + \int_{V_2} \d \rho_2 \wedge \tau_2
  \]
  Quindi:
  \begin{gather*}
    D([\rho'])([\tau]) = \int_{\M} \rho' \wedge \tau = \int_{V_1} \d \rho_1 \wedge \tau_1 + \int_{V_2} \d \rho_2 \wedge \tau_2 = \\
    = \cancel{\int_{V_1} \d{(\rho_1 \wedge \tau)}} + (-)^{k+1} \int_{V_1} \rho_1 \wedge \d \tau_1 + \cancel{\int_{V_2} \d {(\rho_2 \wedge \tau_2)}} + (-)^{k+1} \int_{V_2} \rho_2 \wedge \d \tau_2 =
  \end{gather*}
  Cioè:
  \[
    (-)^{k+1} \int_{\M} \rho' \wedge \tau = \int_{V_1} \rho_1 \wedge \d \tau_1 + \int_{V_2} \rho_2 \wedge \tau_2 = \int_{V_12} (\rho_1 - \rho_2) \wedge \d \tau_1 = \int_{V_12} \rho \wedge \d \tau
  \]
  Quindi usando Stokes:
  \[
    (-)^{k+1} \int_{\M} \rho' \wedge \tau = \int_{\M} \rho \wedge \d \tau
  \]
  Anche gli altri quadrati si dimostrano in maniera analoga, in modo ancora più laborioso.
\end{proof}

% lezione 21

\section{Teorema di de Rham}

Sia $ X $ varietà differenziabile di dimensione $ n $, sono ben definiti i
gruppi di coomologia singolare a valori in $ \RN{} $ $ H^p(X) $ e il gruppo di
coomologia di de Rham. Questi sono spazi vettoriali reali di dimensione finita
su $ \RN{} $. Questi due spazi sono strutturalmente differenti, il primo è
formato da classi di equivalenza di omomorfismi, mentre il secondo da classi di
equivalenza di forme differenziali. Nonostante la differenza esiste un importante
risultato che li collega:
\begin{theorem}[Teorema di de Rham\index{Teorema di de Rham}]
  Sia $ X $ varietà differenziale di dimensione $ n $, allora:
  \[
    \forall p \in \set{0, \dots, n} \; H^p_{dR}(X) \cong H^p(X;\RN{})
  \]
\end{theorem}
Questo teorema offre una rappresentazione esplicita dei gruppi di coomologia
singolare in termini di forme differenziali, per le quali è disponibile un corposo
set di tecniche.

Per dimostrare il teorema di de Rham bisogna prima costruire un omomorfismo di
gruppi abeliani $ \rho \colon H^p_{dR}(X) \to H^p(X;\RN{}) $ e dopodiché dimostrare che è
anche isomorfismo di gruppi abeliani. Per costruire $ \rho $ definisco un omomorfismo
$ \rho \colon \Omega^p(X) \to \hom{S_p(X), \RN{}} $ e poi passo a livello di omologia, infatti
per il teorema dei coefficienti universali l'omologia del complesso degli
omomorfismi è isomorfa alla coomologia. Quindi definirò $ \rho $ tramite:
\[
  \begin{tikzcd}
    Z^p(X) \rar{r} \dar{\pi_1} & \hom{S_p(X), \RN{}} \dar{\pi_2} \\
    H^p_{dR}(X) \rar{\rho} & H^p(X, \RN{})
  \end{tikzcd}
\]
Sia $ \omega \in \Omega^p(X) $, allora $ r(\omega) \in \hom{S_p(X), \RN{}} $, quindi $ r(\omega) $ agisce
sui $ p $-simplessi singolari $ \sigma \colon \Delta_p \to X $ e produce un numero reale.
Se $ \sigma^\star $ è il \textbf{pullback}\index{Pullback} di $ \sigma $, cioè la mappa:
\begin{align*}
  \sigma^\star \colon \Omega^p(X) & \to \Omega^p(\Delta_p) \\
  \omega & \mapsto \sigma^\star(\omega) = \omega \circ \sigma
\end{align*}
Si è tentati di definire:
\[
  r(\omega)(\sigma) = \int_{\Delta_p} \sigma^\star(\omega)
\]
Tuttavia questo non sarebbe sensato, in quanto $ \sigma $ è solo continua, ma per
avere una forma differenziale liscia con il pullback $ \sigma $ dovrebbe essere $ \mathcal{C}^\infty $.
Per questo motivo si introduce un gruppo intermedio.
\begin{definition}
  Un $ p $-simplesso singolare $ \sigma $ si dice \textbf{liscio}\index{Simplesso singolare liscio}
  se $ \sigma $ è $ \mathcal{C}^\infty $.
\end{definition}
\begin{osservation}
  I $ p $-simplessi singolari sono definiti su $ \Delta_p $ la quale è una varietà
  non differenziabile, per questo si dice che $ \sigma $ è
  $ \mathcal{C}^\infty $ se considerato un intorno aperto $ U $ di $ \Delta_p $ e una
  mappa $ F \colon U \to X $ tale che $ F\lvert_{\Delta_p} = \sigma $ risulta che
  $ F $ è $ \mathcal{C}^\infty $ su $ U $.
\end{osservation}
\begin{definition}
  Si definisce il \textbf{complesso delle $ p $-catene singolari
    lisce}\index{Complesso delle catene singolari} come il gruppo libero
  generato dai $ p $-simplessi singolari lisci su $ X $ con il bordo ottenuto
  restringendo l'operatore di bordo del complesso delle catene ai simplessi
  singolari lisci. Questo si indica con $ (S^\infty_p(X), \partial) $, e il generico
  elemento $ c \in S^\infty_p(X) $ si può scrivere come:
  \[
    c = \sum_\sigma n_\sigma \sigma \text{ con $ \sigma $ $ p $-simplesso singolare liscio}
  \]
  Esplicitamente l'azione dell'operatore di bordo è:
  \begin{align*}
    \partial = \partial \lvert_{S^\infty_p} \colon S_p^\infty(X) & \to S^\infty_{p-1}(X) \\
    \sigma & \mapsto \partial \sigma = \sum_{i=0}^p (-)^i \sigma \circ F_i^{\; p}
  \end{align*}
  Questa operazione è ben definita in quanto la composizione di applicazioni
  $ \mathcal{C}^\infty $ è ancora $ \mathcal{C}^\infty $.
\end{definition}
Si può calcolare la coomologia di questo complesso, che è:
\[
  H^p_\infty(X, \RN{}) = \frac{\ker{ \delta \colon \hom{S^\infty_p(X), \RN{}} \to \hom{S^\infty_{p+1}(X), \RN{}}}}{ \im{\delta : \hom{S^\infty_{p-1}(X), \RN{}} \to  \hom{S^\infty_p(X), \RN{}}}}
\]
Dove $ \delta $ è il cobordo.
\begin{osservation}
  Se $ F \colon X \to Y $ è una mappa liscia tra varietà differenziabili
  questa induce una mappa $ F^\star $:
  \begin{align*}
    F^\star \colon H^p_\infty(Y, \RN{}) & \to H^p_\infty(X, \RN{}) \\
    \llbracket \phi \rrbracket & \mapsto \llbracket \phi \circ F \rrbracket
  \end{align*}
  Questa è sensata in quanto $ \phi \circ F $ è $ \mathcal{C}^\infty $ in quanto
  composizione di funzioni $ \mathcal{C}^\infty $.
\end{osservation}
Con questo nuovo gruppo si può definire un omomorfismo $ r $ come ero tentato
di fare, ma solo sui simplessi singolari lisci, dopodiché si estende
la definizione alle catene di simplessi singolari lisci:
\begin{align*}
  r \colon \Omega^p(X) & \to \hom{S_p^\infty(X), \RN{}} \\
  \omega & \mapsto r(\omega)
\end{align*}
Con
\begin{align*}
  r(\omega) \colon S_p^\infty(X) & \to \RN{} \\
  \sigma & \mapsto \int_{\Delta_p} \sigma^\star(\omega)
\end{align*}
Come notazione si pone:
\[
  \int_{\Delta_p} \sigma^\star(\omega) = \int_\sigma \omega
\]
La notazione rende chiara la costruzione che si sta facendo: un simplesso
singolare può essere immaginato come la sua immagine in $ X $, quindi
$ \int_\sigma \omega $ non è altro che l'integrale di $ \omega $ sulla regione di spazio
$ \sigma(\Delta_p) $. Estendendo alle catene si ottiene l'integrale su spazi che
non sono immagini di simplessi standard:
\begin{align*}
  r(\omega) \colon S_p^\infty(X) & \to \RN{} \\
  c = \sum n_\sigma \sigma & \mapsto \int_c \omega = \sum n_\sigma \int_\sigma \omega
\end{align*}
Questo è un omomorfismo per la linearità dell'integrale infatti:
\begin{gather*}
  r(\lambda \omega_1 + \mu \omega_2)(c) = \int_c \lambda \omega_1 + \mu \omega_2 = \lambda \int_c \omega_1 + \mu \int_c \omega_2
\end{gather*}
Bisogna tuttavia fare attenzione che $ \Delta_p $ è una varietà topologica con bordo
che deve essere orientata. Questo si può fare induttivamente: si orienta il
punto, dopodiché si orienta il segmento in modo compatibile al punto, e così
via. La dimostrazione dettaglia è piuttosto tecnica.

A questo punto ho due complessi: $ (\Omega^\bullet, \d{}) $ e $ (\hom{(S^\infty_p(X), \RN{})}, \delta) $:
\[
  \begin{tikzcd}[nodes = {column sep = 10 pt}]
    \dots \rar{\d{}} & \Omega^p(X) \dar{r} \rar{\d{}} & \Omega^{p+1}(X)  \dar{r} \rar{\d{}} & \Omega^{p+2}(X)  \dar{r} \rar{\d{}} & \dots \\
    \dots \rar{\delta} & \hom{S^\infty_p(X), \RN{}} \rar{\delta} & \hom{S^\infty_{p+1}(X), \RN{}} \rar{\delta} &\hom{S^\infty_{p+2}(X), \RN{}} \rar{\delta} & \dots
  \end{tikzcd}
\]

% \begin{lemma}
%   La mappa
%   $ r \colon (\Omega^\bullet(X), \d{}) \to (\hom{(S^\infty_\bullet(X), \RN{})}, \delta) $ precedentemente definita
%   à una mappa di cocatene, cioè $ r \circ d = \delta \circ r $.
% \end{lemma}
\begin{theorem}[Teorema di Stokes\index{Teorema di Stokes}]
  Il \textbf{teorema di Stokes} è equivalente ad affermare che la mappa
  $ r \colon (\Omega^\bullet(X), \d{}) \to (\hom{(S^\infty_\bullet(X), \RN{})}, \delta) $ precedentemente definita
  è una mappa di cocatene, cioè $ r \circ d = \delta \circ r $ e il seguente diagramma è
  commutativo:
  \[
    \begin{tikzcd}
      \Omega^p(X) \rar{\d{}} \dar{r} & \Omega^{p+1}(X) \dar{r} \\
      \hom{S^\infty_p(X), \RN{}} \rar{\delta} & \hom{S^\infty_{p+1}(X), \RN{}}
    \end{tikzcd}
  \]
\end{theorem}
\begin{proof}
  Sia $ \omega \in \Omega^p(X) $ allora $ r(\omega) \in \hom{S^\infty_p(X), \RN{}} $, applicando $ \delta $
  e valutandolo su un simplesso singolare liscio $ \sigma $:
  \[
    \delta \circ r(\omega)(\sigma) = r(\omega)(\partial \sigma) = \int_{\partial \sigma} \omega
  \]
  Ma $ \partial \sigma $ è $ (p-1) $-catena e:
  \[
    \partial \sigma = \sum_{i=0}^p(-)^i\sigma \circ F_i^{\; p}
  \]
  Quindi:
  \[
    \int_{\partial \sigma} \omega = \sum_{i=0}^p \int_{\sigma \circ F_i^{\; p}} \omega = \sum_{i=0}^p(-)^i \int_{\Delta_{p-1}} (\sigma \circ F_i^{\; p})^\star (\omega) =
  \]
  Ma il pullback è controvariante cioè $ (f \circ g)^\star = g^\star \circ f^\star $ e quindi:
  \[
    =\sum_{i=0}^p (-)^p \int_{\Delta_{p-1}} (F_i^{\; p})^\star \circ \sigma^\star (\omega) =
  \]
  Dopo conti tediosi:
  \[
    = \int_{\bigcup_{i=0}^p F_i^{\; p}(\Delta_{p-1})} \sigma^\star(\omega) = \int_{\partial(\Delta_p)} \sigma^\star(\omega) =
  \]
  Per il teorema di Stokes:
  \[
    = \int_{\Delta_p} \d{(\sigma^\star(\omega))}
  \]
  Quindi:
  \[
    (\delta \circ r)(\omega)(\sigma) = \int_{\Delta_p} \d{(\sigma^\star(\omega))}
  \]
  Ma si dimostra che il pullback commuta con la derivata esterna e quindi:
  \[
    = \int_{\Delta_p} \sigma^\star(\d \omega) = \int_\sigma \d \omega = r (\d \omega) (\sigma) = (r \circ d)(\omega)(\sigma)
  \]
  Quindi il quadrato è commutativo. Al contrario, se il quadrato è commutativo
  segue immediatamente il teorema di Stokes.
\end{proof}
\eproof
Siccome $ r $ è una mappa di cocatene induce una mappa a livello di coomologia
e quindi esiste ben definita:
\begin{align*}
  \rho \colon H^p_{dR}(X) & \to H^p_\infty (X, \RN{}) \\
  \llbracket \omega \rrbracket & \mapsto \llbracket r(\omega) \rrbracket
\end{align*}
Sostanzialmente qui si sta utilizzando il teorema dell'omomorfismo di connessione.
\begin{lemma}
  Se $ F \colon X \to Y $ è un'applicazione liscia tra varietà differenziali allora
  il seguente diagramma commuta:
  \[
    \begin{tikzcd}
      H^p_{dR}(Y) \rar{F^\star} \dar{\rho_Y} & H^p_{dR}(X) \dar{\rho_X} \\
      H^p_\infty(Y, \RN{}) \rar{F^\star} & H^p_\infty(X, \RN{})
    \end{tikzcd}
  \]
\end{lemma}
\begin{proof}
  Considero il diagramma:
  \[
    \begin{tikzcd}
      Z^p(Y) \rar{F^\star} \dar{r_Y} & Z^{p+1}(X) \dar{r_x} \\
      S^p_\infty(Y, \RN{}) \rar{F^\star} & S^p_\infty(X, \RN{})
    \end{tikzcd}
  \]
  Devo mostrare che $ F^\star \circ r_Y = r_X \circ F^\star $ e che è ben definito il passaggio
  all'omologia, cioè  $ (F^\star \circ \rho_Y)([\omega]_{dR}) = (\rho_X \circ F^\star)([\omega]_{dR}) $.

  Sia $ \omega \in \Omega^p(Y) $, %cioè $ \omega $ forma differenziale chiusa ($ \d \omega = 0 $)
  e sia $ \sigma $ un $ p $-simplesso singolare liscio in $ X $. Devo
  mostrare che $ r_X \circ F^\star = F^\star \circ r_Y $. Li valuto:
  \begin{gather*}
    (r_X \circ F^\star)(\omega)(\sigma) = \int_\sigma F^\star(\omega) = \int_{\Delta_p} \sigma^\star \circ F^\star (\omega) = \int_{\Delta_p} (F \circ \sigma)^\star(\omega) = \\
    = \int_{F \circ \sigma} \omega = r_Y(\omega)(F \circ \sigma)
  \end{gather*}
  Quindi $ (r_X \circ F^\star)(\omega)(\sigma) = r_Y(\omega)(F \circ \sigma) $ ma per la definizione di pullback
  \begin{align*}
    F^\star \colon S^p_\infty(Y, \RN{}) & \to S^p_\infty(X, \RN{}) \\
    \phi & \mapsto F^*(\phi) = F \circ \phi
  \end{align*}
  E quindi:
  \[
    F^\star \left(\int_\bullet \omega \right) = \int_{F(\bullet)} \omega \; \text{ cioè }\; (F^\star \circ r_Y)(\omega)(\bullet) = r_Y(\omega)(F (\bullet))
  \]
  Cioe:
  \[
    r_Y(\omega)(F \circ \sigma) = (F^\star (r_Y(\omega))(\sigma) = (F^\star \circ r_Y)(\omega)(\sigma)
  \]
  Si può quindi passare alla coomologia senza problemi, infatti ho mostrato che
  la commutatività vale per tutte le forme, sia per quelle chiuse che per quelle
  esatte.
  % Ora mostro che $ (\rho_X \circ F^\star)([\omega]_{dR}) = (F^\star \circ \rho_Y)([\omega]_{dR}) $ con $ [\omega]_{dR} = \omega + \d \eta $.
  % \[
  %   (\rho_X \circ F^\star)(\omega + \d \eta) = (\rho_X \circ F^\star)(\omega) + (\rho_X \circ F^\star)(\d \eta)
  % \]
  % La prima parte ho già che è ok, la seconda:
  % \[
  %   \rho_X(F^\star \d \eta)(\sigma) = \int_\sigma F^\star \d \nu = \int_{F \circ \sigma} \d \eta
  % \]
  % Ma
  % \[
  %   (F^\star \circ \rho_Y)(\d \eta)(\sigma) = \rho_Y(\d \eta)(F \circ \sigma) = \int_{F \circ \sigma} \d \eta
  % \]
  % Quindi tutto va bene.
\end{proof}
\begin{definition}
  Sia $ X $ una varietà differenziale, questa si dice \textbf{de Rham}\index{Varietà differenziabile de Rham}
  se l'applicazione $ \rho \colon H^p_{dR}(X) \to H^p_\infty(X, \RN{}) $ è un isomorfismo.
\end{definition}
\begin{lemma}
  Essere de Rham è invariante per diffeomorfismi, cioè se $ X $ è una varietà
  differenziabile de Rham e $ F \colon X \to Y $ è diffeomorfismo di varietà
  differenziabili, allora anche $ Y $ è de Rham.
\end{lemma}
\begin{proof}
  Siccome $ F \colon X \to Y $ è diffeomorfismo allora esiste $ G \colon Y \to X $ diffeomorfismo
  inverso, quindi $ H^p_{dR}(X) \cong H^p_{dR}(Y) $ per la funtorialità del pullback,
  e anche $ F^\star_\infty $ è isomorfismo per lo stesso motivo.
  Quindi ho il diagramma:
  \[
    \begin{tikzcd}
      H^p_{dR}(Y) \dar{\rho_Y} \rar{F^\star_{dR}} & H^p_{dR}(X) \dar{\rho_X} \\
      H^p_\infty(Y, \RN{}) \rar{F^\star_\infty} & H^p_\infty(X, \RN{})
    \end{tikzcd}
  \]
  Siccome il diagramma commuta $ \rho_Y = (F^\star_\infty)^{-1} \circ \rho_X \circ F^\star_{dR} $ è
  isomorfismo essendo composizione di isomorfismi.
\end{proof}
\eproof
A questo punto il teorema di de Rham si può esprimere come:
\begin{theorem}[Teorema di de Rham\index{Teorema di de Rham}]
  Ogni varietà differenziale è de Rham e soddisfa la condizione
  \[
    H^p_\infty(X, \RN{}) \cong H^p(X, \RN{}) \; \forall p
  \]
  Cioè vale che:
  \[
    H^p(X, \RN{}) \cong H^p_{dR}(X) \; \forall p
  \]
\end{theorem}
\begin{proof} \emph{Idee della dimostrazione}.
  La dimostrazione è per passi. Bisogna dimostrare
  \begin{enumerate}
  \item Se $ \set{X_j} $ è una famiglia numerabile di varietà de Rham
    allora la loro unione disgiunta è una varietà de Rham.
  \item Ogni aperto convesso di $ \RN{n} $ è de Rham.
  \item Se $ X $ è una varietà differenziale che ammette un ricoprimento
    finito di aperti de Rham allora è de Rham.
  \item Se $ X $ ha una base di aperti de Rham allora è de Rham.
  \item Ogni aperto di $ \RN{n} $ è de Rham.
  \item Ogni varietà differenziale è de Rham.
  \item Ogni varietà soddisfa $  H^p(X, \RN{}) \cong H^p_{dR}(X) \; \forall p $.
  \end{enumerate}

  \subparagraph{Dimostrazione del punto sei} Siccome $ X $ è una varietà
  differenziabile ammette un atlante formato da insiemi $ U_\alpha $ (con relative
  carte) che sono diffeomorfi ad aperti $ V_\alpha $ in $ \RN{n} $. Per il punto
  cinque $ V_\alpha $ sono de Rham, quindi per il lemma precedente $ U_\alpha $ sono
  de Rham. Ora basta mostrare che $ X $ ammette una base di aperti costruiti
  a partire da questi $ U_\alpha $ in questo modo per il punto quattro $ X $ è de
  Rham. Per definizione di atlante le carte forniscono un ricoprimento di $ X $,
  bisogna mostrare che l'intersezione di due carte è de Rham.
\end{proof}

\subsection{Duale di Poincaré}

\begin{definition}
  Sia $ X $ una varietà differenziabile compatta orientata senza bordo di
  dimensione $ n $ e sia $ \sigma $ un $ p $-simplesso singolare liscio, allora si
  definisce il \textbf{duale di Poincaré}\index{Duale di Poincaré} la
  $ p $-forma $ \mathcal{P}(\sigma) $ tale che $ \forall \omega \in \Omega^{n-p}(X) $ valga:
  \[
    \int_\sigma \omega = \int_X \mathcal{P}(\sigma) \wedge \omega
  \]

\end{definition}
Se $ X $ è una varietà differenziabile compatta orientata senza bordo
allora ho il teorema di Poincaré e quello di de Rham. Quindi ho:
\[
  \begin{tikzcd}
    {} & (H^{n-p}_{dR}(X))^* \arrow{dl}{\cong} \\
    H^p_{dR}(X) \arrow{ur}{D} \arrow{dr}{\rho} & {} \\
    {} & H^p(X, \RN{}) \arrow{ul}{\cong}
  \end{tikzcd}
\]
Questo si costruisce prendendo un $ \llbracket \tilde{\sigma} \rrbracket \in H^p(X, \RN{}) $ poi
con de Rham viene mandato in $  \mathcal{P}(\sigma)  $ tale che $ \rho(\llbracket\mathcal{P}(\sigma) \rrbracket) = \llbracket\tilde{\sigma}\rrbracket $
e quindi con Poincaré $ D(\llbracket\mathcal{P}(\sigma) \rrbracket) = \int_X P(\sigma) \wedge \bullet $.
Cioè è quella forma differenziale che tramite la dualità di Poincaré
produce una forma che per de Rham finisce in $ \sigma $.


%%% Local Variables:
%%% ispell-local-dictionary: "italiano"
%%% mode: latex
%%% TeX-master: "notes"
%%% End:


\printindex


\end{document}


%%% Local Variables:
%%% ispell-local-dictionary: "italiano"
%%% mode: latex
%%% TeX-master: t
%%% End:
