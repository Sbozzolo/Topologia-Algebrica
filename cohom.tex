\chapter{Coomologia singolare}

% Lezione 15

\section{Prodotto tensore}

Ho trovato che per $ n $ pari:
\[
  H_i(\Pjr{n}) \cong
  \begin{cases}
    \Z & \text{se } i = 0 \\
    \Z_2 & \text{se $ i $ pari e $ i < n $} \\
    0 & \text{altrimenti}
  \end{cases}
\]
Mentre per $ n $ dispari:
\[
  H_i(\Pjr{n}) \cong
  \begin{cases}
    \Z & \text{se } i = 0,n \\
    \Z_2 & \text{se $ i $ pari e $ i < n $} \\
    0 & \text{altrimenti}
  \end{cases}
\]
Non mi piace. Voglio cambiare i coefficienti.

Sia $ A, B $ gruppi abeliani, è ben definito il prodotto
cartesiano:
\[
  A \times B = \set{(a,b) | a \in A, b \in B}
\]
Sia $ F(A,B) $ il gruppo libero generato dalle coppie $ (a, b) \in A \times B $.
% in notazione additiva.
% Il gruppo $ F(A,B) $ è abeliano in quanto
% $ A $ e $ B $ lo sono, e l'operazione di somma è:
% \[
%   (a_1, b_1 ) + (a_2, b_2) = (a_1 + a_2, b_1 + b_2)
% \]
\newmathsymb{tensprod}{\otimes}{Prodotto tensore}
\begin{definition}
  Se $ A, B $ sono $ \Z $-moduli si definisce il \textbf{prodotto
    tensore}\index{Prodotto tensore} tra $ A $ e $ B $, come:
  \[
    A \otimes B = \quot{F(A,B)}{R(A,B)}
  \]
  Dove $ F(A,B) $ è il gruppo libero generato da $ A \times B $
  % con operazione $ (a_1, b_1) + (a_2, b_2) = (a_1 + a_2, b_1 + b_2) $,
  e $ R(A,B) $ il gruppo generato in $ F(A,B) $ dalle espressioni:
  \begin{gather*}
    (a_1 + a_2, b) - (a_1, b) - (a_2, b) \\
    (a, b_1 + b_2) - (a, b_1) - (a, b_2) \\
    n (a, b) - (na, b) \\
    n (a, b) - (a, nb)
  \end{gather*}.
  Gli elementi di $ A \otimes B $ sono generati dai simboli
  $ a \otimes b $ con $ a \in A $ e $ b \in B $ imponendo le relazioni:
  \begin{gather*}
    (a_1 + a_2) \otimes b = a_1 \otimes b + a_2 \otimes b \\
    a \otimes (b_1 + b_2) = a \otimes b_1 + a \otimes b_2 \\
    n (a \otimes b) = (na) \otimes b \\
    n (a \otimes b) = a \otimes (nb)
  \end{gather*}
  Infatti il quoziente manda a zero le espressioni in $ R(A,B) $.
\end{definition}
\begin{osservation}
  Il generico elemento di $ A \otimes B $ non è della forma $ a \otimes b $, m
  a è una combinazione lineare di oggetti di questo tipo, detti
  tensori puri, i quali generano $ A \otimes B $ come modulo.
\end{osservation}
\begin{example}
  Si parla di cambiamento di coefficienti in questo senso: considero
  $ \RN{3} \otimes \mathbb{C} $, una base di questo spazio è data da:
  \[
    \set{e \otimes 1, e \otimes i, f \otimes 1, f \otimes i, g \otimes 1, g \otimes i}
  \]
  Il generico elemento di $ \RN{3} \otimes \mathbb{C} $ è:
  \begin{gather*}
    a_1(e \otimes 1) + a_2(e \otimes i) + a_3(f \otimes 1) + a_4(f \otimes i) + a_5(g \otimes 1) + a_6(g \otimes i) = \\
    = (a_1e + a_3f + a_5g) \otimes 1 + (a_2e + a_4f + a_6g) \otimes i = \\
    = e \otimes (a_1 + a_2i) + f \otimes (a_3 + a_4i) + g \otimes (a_5 + a_6i)
  \end{gather*}
  Nell'ultima riga si vede che ora si hanno dei vettori nella base di partenza
  ma con coefficienti complessi. Questa è la complessificazione di $ \RN{3} $.
\end{example}
\begin{proposition}[Proprietà universale\index{Proprietà universale! \vedi{Prodotto tensore}}]
  Sia $ G $ un gruppo abeliano e $ \psi \colon A \times B \to G $ un'applicazione bilineare continua,
  allora esiste un unico omomorfismo $ \phi \colon A \otimes B \to G $ tale che il diagramma:
  \[
    \begin{tikzcd}
      A \times B \rar{\psi} \dar{\pi} & G \\
      A \otimes B \arrow{ru}{\phi} & {}
    \end{tikzcd}
  \]
  è commutativo, con:
  \begin{align*}
    \pi \colon A \times B & \to A \oplus B \\
    (a,b) & \mapsto a \otimes b
  \end{align*}
  In pratica $ \psi $ fattorizza per il prodotto tensoriale
  ($ \psi = \phi \circ \pi $). La proprietà è detta universale perché esiste mostra che
  esiste un solo prodotto tensoriale, ed è equivalente a dire
  che:
  \[
    \hom{A,B} \cong A^* \otimes B
  \]
\end{proposition}
\begin{proof}
  La costruzione di $ \phi $ è banale, è tale che $ \phi(a \otimes b) = \phi(\pi(a,b)) = \psi(a,b) $,
  bisogna solo verificare che è ben definita.
  Considero un elemento $ c \otimes d $ equivalente a $ a \otimes b $, cioè tali
  che $ (a,b) - (c,d) \in R(A,B) $, devo mostrare che $ \phi(a \otimes b) = \phi(c \otimes d) $,
  cioè che $ \psi((a,b)) = \psi((c,d)) $, ovvero che $ \psi((a,b)) - \psi((c,d)) = 0 $, ma
  $ (a,b) - (c,d) \in R(A,B) $ e:
  \[
    \psi((c,d) - (a,b)) = \sum_\alpha \psi((r_\alpha, s_\alpha)) = \sum_\alpha \phi \left(\pi((r_\alpha, s_\alpha))\right) = 0
  \]
  con $ (r_\alpha,s_\alpha) $ base di $ R(A,B) $, che al quoziente vanno a zero, ma
  $ \phi $ è un omomorfismo per costruzione (dato che per ipotesi $ \psi $ lo è, e il
  prodotto tensoriale è bilineare) quindi $ \phi(\pi((r_\alpha, s_\alpha))) = 0 $.
\end{proof}
\begin{example}
  Siano $ V, W $ spazi vettoriali reali, gli spazi
  $ \tilde{V} = V \otimes \mathbb{C} $, $ \tilde{W} = W \otimes \mathbb{C} $ sono spazi
  vettoriali complessi. La proprietà universale permette di estendere in modo
  univoco le funzioni lineari $ f \colon V \to W $ a funzioni lineari
  $ \tilde{f} : \tilde{V} \to \tilde{W} $ e quindi lavorare con $ \tilde{V} $ e
  $ \tilde{W} $ esattamente come se fossero usuali spazi vettoriali complessi.
\end{example}
Un'altra importante proprietà del prodotto tensore è il suo comportamento
rispetto agli omomorfismi.

\begin{proposition}
  Siano $ f \colon A \to B $ e $ g \colon A' \to B' $ omomorfismi, posso definire l'azione di
  sui generatori di $ A \otimes A' $:
  \begin{align*}
    f \otimes g \colon A \otimes A' & \to B \otimes B' \\
    a \otimes a' & \to f(a) \otimes g(a')
  \end{align*}
  Estendendo per linearità si definisce $ f \otimes g $ su tutto $ A \otimes A' $, il quale è
  per definizione omomorfismo di gruppi abeliani.
\end{proposition}

\begin{proposition}
  Vale che $ A \otimes B \cong B \otimes A $, cioè il prodotto tensore è simmetrico.
\end{proposition}
\begin{proof}
  Se per la proprietà universale (con $ G = B \otimes A $) trovo una mappa bilineare continua
  $ \psi \colon A \times B \to A \otimes B $ allora esiste un omomorfismo
  $ \phi_1 \colon A \otimes B \to B \otimes A $, quindi posso scambiare $ A $ e
  $ B $ e trovare un secondo omomorfismo
  $ \phi_2 \colon B \otimes A \to A \otimes B $, e quindi mostrare che $ \phi_1 $ e $ \phi_2 $ sono inverse.
  Sia:
  \begin{align*}
    \psi \colon A \times B & \to B \otimes A \\
    (x,y) & \mapsto y \otimes x
  \end{align*}
  Questa applicazione è continua e bilineare, allora per l'universalità sono ben
  definite $ \phi_1 $ e $ \phi_2 $, e:
  \[
    \begin{tikzcd}[nodes = {row sep=5pt}]
      A \otimes B \rar{\phi_1} & B \otimes A \rar{\phi_2} & A \times B \\
      a \otimes b \arrow[mapsto]{r} & b \otimes a \arrow[mapsto]{r} & a \otimes b
    \end{tikzcd}
  \]
  Quindi $ \phi_1 \circ \phi_2 = \Id{A \otimes B} $, e analogamente  $ \phi_2 \circ \phi_1 = \Id{B \otimes A} $.
\end{proof}
\eproof
Un'ulteriore proprietà da analizzare è il comportamento rispetto alle successioni
esatte. Considero una successone esatta corta di $ \Z $-moduli:
\[
  \begin{tikzcd}
    0 \rar{} & R \rar{\alpha} & F \rar{\beta} & A \rar & 0
  \end{tikzcd}
\]
Considero $ G $ gruppo abeliano, allora ho:
\[
  \begin{tikzcd}
    R \otimes G \rar{\alpha'} & F \otimes G \rar{\beta'} & A \otimes G
  \end{tikzcd}
\]
Questa successione è esatta? Per verificarlo utilizzo un lemma:
\begin{lemma}
  Se $ A $ è uno $ \Z $-modulo allora $ A \otimes \Z \cong A $.
\end{lemma}
\begin{proof}
  Costruisco esplicitamente l'isomorfismo. Siano $ \tau $ e $ \sigma $ definiti da:
  \begin{align*}
    \tau \colon A & \to A \otimes \Z \\
    a & \mapsto a \otimes 1
  \end{align*}
  E:
  \begin{align*}
    \sigma \colon A \otimes \Z & \to A \\
    \tilde{a} \otimes n & \mapsto n \tilde{a}
  \end{align*}
  Mostro che sono omomorfismi:
  \[
    \tau(a + b) \otimes 1 = a \otimes 1 + b \otimes 1 = \tau(a) + \tau(b)
  \]
  \begin{gather*}
    \sigma(\tilde{a} \otimes n + \tilde{b} \otimes m ) = \sigma(n \tilde{a} \otimes 1 + m \tilde{b} \otimes
    1) = \sigma ((n \tilde{a} + m\tilde{b}) \otimes 1) = \\ = n \tilde{a} + m \tilde{b} =
    \sigma(\tilde{a} \otimes n) + \sigma(\tilde{b} \otimes m)
  \end{gather*}
  Poi $ \sigma $ e $ \tau $ sono inversi, infatti:
  \[
    \begin{tikzcd}[nodes={row sep = 5 pt}]
      A \rar{} & A \otimes \Z \rar & A \\
      a \arrow[mapsto]{r}{\tau} & a \otimes 1 \arrow[mapsto]{r}{\sigma} & a
    \end{tikzcd}
  \]
  E:
  \[
    \begin{tikzcd}[nodes={row sep = 5 pt}]
      A \otimes \Z \rar & A \rar & A \otimes Z \\
      a \otimes n \arrow[mapsto]{r}{\sigma} & n \tilde{a} \arrow[mapsto]{r}{\tau} & n \tilde{a} \otimes 1 = \tilde{a} \otimes n
    \end{tikzcd}
  \]
  Quindi $ \tau $ e $ \sigma $ costituiscono isomorfismi tra $ A \otimes \Z $ e $ A $.
\end{proof}
\begin{example}
  Considero la successione esatta corta:
  \[
    \begin{tikzcd}
      0 \rar{} & n \Z \rar{\alpha} & \Z \rar{\beta} & \quot{\Z}{n \Z} \rar & 0
    \end{tikzcd}
  \]
  In particolare per $ n = 6 $:
  \[
    \begin{tikzcd}
      0 \rar{} & 6 \Z \rar{\alpha} & \Z \rar{\beta} & \quot{\Z}{6 \Z} \rar & 0
    \end{tikzcd}
  \]
  Tensorizzo per $ \Z $:
  \[
    \begin{tikzcd}[nodes = {row sep = 3pt}]
      6 \Z \otimes \Z \rar{\alpha \otimes \Id{}} & \Z \otimes \Z \rar{\beta \otimes \Id{}} & \quot{\Z}{6 \Z} \otimes \Z  \\
      6x \otimes y \rar[mapsto] & x \otimes y \rar[mapsto] & \bar{x} \otimes y
    \end{tikzcd}
  \]
  Con $ \bar{x} $ classe modulo 6 di $ x $. La successione è
  esatta perché passando all'isomorfismo la successione è:
  \[
    \begin{tikzcd}[nodes = {row sep = 3pt}]
      0 \rar{} & 6 \Z \rar{\alpha} & \Z \rar{\beta} & \Z_6 \rar & 0 \\
    \end{tikzcd}
  \]
  La quale è esatta.
  % Usando il lemma precedente cerco il
  % nucleo di $ \alpha \otimes \Id{} $, quindi risolvo $ 6 xy = 0 $ da cui
  % $ x = 0 $ o $ y = 0 $, e in entrambi i casi $ x \otimes y = 0 $ è zero, quindi
  % siccome il nucleo è banale $ \alpha \otimes \Id{} $ è inieittiva. L'applicazione
  % $ \beta \otimes \Id{} $ è suriettiva, infatti se prendo $ \bar{k} \otimes u $ con
  % $ \bar{k} \in \Z_6 $ e $ u \in \Z $, la preimmagine è chiaramente
  % $ k \otimes u $. Il nucleo di $ \beta \otimes \Id{} $ è dato da
  % $ \bar{x} \otimes y = 0 $, cioè $ \bar{x} y = 0 $ in $ \Z_6 $ via
  % isomorfismo. Se $ m = 6t $ allora $ \bar{m} = 0 $ quindi
  % $ \ker{\beta \otimes \Id{}} $ contiene $ \Z \otimes \Z $, cioè
  % $ \Z \otimes \Z \subseteq \ker{\beta \otimes \Id{}} $. Ma
  % $ \bar{m} \otimes n = n \bar{m} \otimes \Id{} = \bar{nm} \otimes \Id{} $, quindi
  % $ mn = 6t $, ma $ n $ è intero, quindi $ m $ deve essere multiplo di 6, e
  % quindi è vera anche l'inclusione inversa. La successione è quindi esatta.
\end{example}
\begin{example}
  Considero la stessa successione di prima, ma ora tensorizzo per $ \quot{\Z}{4 \Z}  \cong \Z_4 $:
  \[
    \begin{tikzcd}[nodes = {row sep = 3pt}]
      0 \rar{} & 6 \Z \otimes \Z_4 \rar{\alpha \otimes \Id{}} & \Z \otimes \Z_4 \rar{\beta \otimes \Id{}} & \Z_6 \otimes \Z_4 \rar & 0 \\
      {}  & 6x \otimes \bar{y} \rar[mapsto] & x \otimes \bar{y} \rar[mapsto] & \bar{x} \otimes \bar{y}
    \end{tikzcd}
  \]
  Considero in particolare l'applicazione:
  \begin{align*}
    6 \Z \otimes \Z_4 & \to \Z \otimes \Z_4 \\
    6 x \otimes \bar{y} & \mapsto x \otimes \bar{y}
  \end{align*}
  Questa ha un nucleo non banale, usando il lemma
  precedente:
  \begin{align*}
    \Z \otimes \Z_4 & \to \Z_4 \\
     x \otimes \bar{y} & \mapsto \overline{xy}
  \end{align*}
  E l'elemento $ x = 6 $ e $ y = 2 $ viene mandato in $ \bar{12} $ che è $ 0 $ in $ \Z_4 $.

  % ad esempio l'elemento $ 6 \otimes \bar{2} $ finisce in
  % Cerco il nucleo in isomorfismo: $ 6 x \bar{k} = \bar{0} $, ma questo ha soluzioni non banali,
  % ad esempio $ 6 \otimes \bar{2} $, infatti va nella classe $ \bar{12} $ che è $ \bar{0} $, ma quindi
  % non è iniettiva.
\end{example}
Da questi esempi si nota che in generale successioni esatte non vanno in successioni esatte,
cioè $ R \otimes G \to F \otimes G \to A \otimes G $ non è sempre esatta. Per poter dire qualcosa di generale conviene
fare la seguente osservazione:
\begin{osservation}
Considero $ \alpha \otimes \Id{} \colon R \otimes G \to F \otimes G $ allora:
\[
  \quot{F \otimes G}{(\alpha \otimes \Id{})(R \otimes G)} \cong \quot{F}{\alpha(R)} \otimes G
\]
\end{osservation}
\begin{proof}
Costruisco esplicitamente l'isomorfismo:
\begin{align*}
  \eta \colon \quot{F}{\alpha(R)} \otimes G & \to \quot{F \otimes G}{(\alpha \otimes \Id{})(R \otimes G)} \\
  [\alpha] \otimes g & \mapsto [\alpha \otimes g]'
\end{align*}
Questa mappa è ben definita, infatti se $ b \sim a $, cioè se $ [b] = [a] $ allora
$ -b + a \in \alpha(R) $, quindi:
\begin{gather*}
  [b] \otimes g \mapsto [b \otimes g]' \\
  [a] \otimes g \mapsto [a \otimes g]'
\end{gather*}
Ma $ b = a + \alpha(r) $ con $ r \in R $ quindi $ b \otimes g = (a + \alpha(r)) \otimes g = a \otimes g + \alpha(r) \otimes g $
e quindi:
\[
  [b \otimes g]' = [a \otimes g + \alpha(r) \otimes g]' = [a \otimes g]' + [\alpha(r) \otimes g]'
\]
Ma;
\[
  [\alpha(r) \otimes g]' = [(\alpha \otimes \Id{})(r \otimes g)]' = 0
\]
In quanto $ [,]' $ è nello spazio quoziente rispetto $ (\alpha \otimes \Id{}) $.
L'applicazione è quindi ben definita e lineare, l'inversa è chiaramente la
mappa $ [a \otimes g]' \mapsto [a] \otimes g $, che è ben definita per il medesimo ragionamento.
\end{proof}


Ma a questo punto $ {F} \slash {\alpha(R)} \otimes G \cong A \otimes G $, infatti per il teorema fondamentale
degli omomorfismi:
\[
  \quot{F}{\im{\alpha}} = \quot{F}{\ker{\beta}} \cong \im{\beta} = A
\]
Quindi
$ A \otimes G \cong {F \otimes G} \slash {{(\alpha \otimes \Id{})(R \otimes G)}} $. In questo modo posso sempre
costruire una successione esatta tensorizzando, rinunciando all'iniettività di
$ \alpha \otimes \Id{} $, ma mantenendo
$ \ker{\beta \otimes \Id{}} = \im{\alpha \otimes \Id{}} $ e $ \beta \otimes \Id{} $ suriettiva:
\[
  \begin{tikzcd}
    0 \rar & \ker{\alpha \otimes \Id{}} \rar{i} & R \otimes G \rar{\alpha \otimes \Id{}} & F \otimes G \rar{\beta \otimes \Id{}} & A \otimes G \rar & 0
  \end{tikzcd}
\]
$ i $ è iniettiva perché è un inclusione, mentre $ \beta \otimes \Id{} $ è suriettiva
in quanto è una proiezione al quoziente. Si mantiene $ \ker{\beta \otimes \Id{}} = \im{\alpha \otimes \Id{}} $
in quanto tensorizzando si perde l'esattezza solo a sinistra.

\begin{definition}
  Se $ A $ è uno $ \Z $-modulo una successione esatta corta del tipo:
  \[
    \begin{tikzcd}
      0 \rar & R \rar{\alpha} & F \rar{\beta} & A \rar & 0
    \end{tikzcd}
  \]
  con $ R $ e $ F $ $ \Z $-moduli liberi è detta \textbf{risoluzione di
    $ A $}\index{Risoluzione di $ A $} oppure \textbf{presentazione di
    $ A $}\index{Presentazione di $ A $! \vedi{Risoluzione di $ A $}}.
\end{definition}

\begin{osservation}
  Esiste sempre almeno una risoluzione di $ A $ ottenuta prendendo $ F $ è il
  gruppo libero generato da $ A $ e $ R $ il gruppo delle relazioni da imporre
  per riottenere $ A $. Tensorizzando:
  \[
    \begin{tikzcd}
      0 \rar & \ker{\alpha \otimes \Id{}} \rar & R \otimes G \rar{\alpha \otimes \Id{}} & F \otimes G \rar{\beta \otimes \Id{}} & A \otimes G \rar & 0
    \end{tikzcd}
  \]
  Potrebbero comunque esserci altre successioni esatte:
  \[
    \begin{tikzcd}
      0 \rar & R' \rar{\alpha} & F' \rar{\beta} & A \rar & 0
    \end{tikzcd}
  \]
  Tensorizzando:
  \[
    \begin{tikzcd}
      0 \rar & \ker{\alpha' \otimes \Id{}} \rar & R' \otimes G \rar{\alpha \otimes \Id{}} & F' \otimes G \rar{\beta \otimes \Id{}} & A \otimes G \rar & 0
    \end{tikzcd}
  \]
\end{osservation}

\newmathsymb{torsion}{\tor{}}{Modulo di torsione}
\begin{definition}
  Si chiama \textbf{modulo di torsione}\index{Modulo di torsione} di $ A $ e di $ G $ il
  gruppo $ \ker{\alpha \otimes \Id{}} $, e lo si indica con $ \tor{A,G} $. Quindi vale che:
    \[
    \begin{tikzcd}
      0 \rar & \tor{A,G} \rar & R \otimes G \rar{\alpha \otimes \Id{}} & F \otimes G \rar{\beta \otimes \Id{}} & A \otimes G \rar & 0
    \end{tikzcd}
  \]
\end{definition}
\begin{lemma}
  Il modulo di torsione non dipende dalla scelta della risoluzione di $ A $, cioè con risoluzioni
  differenti si ottengono moduli di torsione isomorfi.
\end{lemma}

\begin{lemma}
  Se $ F_1 $ è un gruppo libero allora $ \tor{A,F_1} \cong 0 $, e quindi il modulo
  di torsione è dovuto alla parte di torsione di $ G $.
\end{lemma}
\begin{proof}
  Considero una presentazione di $ A $:
  \[
    \begin{tikzcd}
      0 \rar & R \rar & F \rar & A \rar & 0
    \end{tikzcd}
  \]
  Tensorizzo per $ F_1 $:
  \[
    \begin{tikzcd}
      0 \rar & \tor{A,F_1} \rar & R \otimes F_1 \rar{\phi} & F \otimes F_1 \rar & A \otimes F_1 \rar & 0
    \end{tikzcd}
  \]
  La mappa $ \phi = \alpha \otimes \Id{} $ è iniettiva, infatti $ R \cong \Z^r $, $ F \cong \Z^n $ e $ F_1 \cong \Z^{n_1} $,
  quindi $ \phi \colon \Z^r \otimes \Z^{n_1} \to \Z^n \otimes \Z^{n_1} $, cioè:
  \begin{align*}
    \Z^n \otimes \Z^{n_1} & \to  \Z^n \otimes \Z^{n_1} \\
    \vec{v} \otimes \vec{w} & \mapsto \alpha(\vec{v}) \otimes \vec{w}
  \end{align*}
  \begin{exercise}
    Mostrare che $ \Z^s \otimes \Z^r \cong \Z^{sr} $.
    Hint:  $ \set{e_1 \otimes f_j} $ è una base di $ \Z^s \otimes \Z^r $ se $ \set{e_1} $
    e $ \set{f_j} $ lo sono per $ \Z^s $ e $ \Z^r $, mostrarlo.
  \end{exercise}
  Quindi:
  \begin{align*}
    \phi \colon \Z^{rn_1} & \to \Z^{n n_1} \\
    \vec{v} \otimes \vec{w} & \mapsto \alpha(\vec{v})\otimes \vec{w}
  \end{align*}
  [TERMINARE QUESTA DIMOSTRAZIONE, (MA COME?)]
  Essendo $ \phi $ iniettiva per l'esattezza della successione deve essere $ \tor{A, F_1} \cong 0 $.
\end{proof}

\begin{proposition}
  Se $ A $ e $ B $ sono $ \Z $-moduli allora $ \tor{A,B} \cong \tor{B,A} $.
\end{proposition}
\begin{proof}
  La dimostrazione è un diagram chase.
  Considero una risoluzione di $ B $ e di $ A $:
  \[
    \begin{tikzcd}
      0 \rar{} & R_B \rar{\alpha} & F_B \rar{\beta} & B \rar & 0
    \end{tikzcd}
  \]
  \[
    \begin{tikzcd}
      0 \rar{} & R_A \rar{\alpha} & F_A \rar{\beta} & A \rar & 0
    \end{tikzcd}
  \]
  Tensorizzo questa per $ B $:
  \[
    \begin{tikzcd}
      0 \rar{} & \tor{A,B} \rar & R_A \otimes B \rar{\alpha} & F_A \otimes B \rar{\beta} & A \otimes B \rar & 0
    \end{tikzcd}
  \]
  Tensorizzo altre cose e le metto in verticale, usando la simmetria:
  \[
    \begin{tikzcd}[nodes={column sep= 10 pt}]
      {} & {} & {} & {} & 0 \dar & {} \\
      {} & {} & \dots \dar & \dots \dar & \tor{B,A} \dar & {} \\
      {} & 0 \rar & R_A \otimes R_B \rar \dar & F_A \otimes R_B \dar \rar & A \otimes R_B \dar  \rar & 0 \\
      {} & 0 \rar & R_A \otimes F_B \rar \dar & F_A \otimes F_B \dar \rar & A \otimes F_B \dar  \rar & 0 \\
      0 \rar & \tor{A,B} \rar & R_A \otimes B \rar \dar & F_A \otimes B \rar \dar & A \otimes B  \rar \dar & 0 \\
      {} & {} & 0 & 0 & 0 & {} \\
    \end{tikzcd}
  \]
  Bisogna risalire da $ \tor{A,B} $ a $ \tor{B,A} $ e viceversa. Questa operazione è piuttosto
  noiosa. [MANCA]
\end{proof}

\begin{lemma}
  Siano $ A, B, C $ gruppi abeliani, vale che $  \tor{A, B} \oplus \tor{A, C} \cong \tor{A, B \oplus C} $.
\end{lemma}
\begin{proof}
  Infatti considero una presentazione di $ A $:
  \[
    \begin{tikzcd}
      0 \rar & R \rar & F \rar & A \rar & 0
    \end{tikzcd}
  \]
  Tensorizzo per $ B \otimes C $:
  \[
    \begin{tikzcd}[nodes = {column sep = 15pt}]
      0 \rar & \tor{A, B \oplus C} \rar & R \otimes (B \oplus C) \rar & F \otimes (B \oplus C)\rar & A \otimes (B \oplus C) \rar & 0
    \end{tikzcd}
  \]
  Ma posso anche tensorizzare separatamente per $ B $ e $ C $:
  \[
    \begin{tikzcd}[nodes = {column sep = 15pt, row sep = 10 pt}]
      0 \rar & \tor{A, B } \rar & R \otimes B \rar & F \otimes B \rar & A \otimes B \rar & 0 \\
      0 \rar & \tor{A, C} \rar & R \otimes C \rar & F \otimes C \rar & A \otimes C \rar & 0
    \end{tikzcd}
  \]
  Sommandole:
  \[
    \begin{tikzcd}[nodes = {column sep = 8pt, inner sep = 2pt, outer sep = 0.5pt}]
      0 \rar & \tor{A, B} \oplus \tor{A, C} \rar & R \otimes B \oplus R \oplus C \rar & F \otimes B \oplus F \oplus C \rar & A \otimes B \oplus A \otimes C  \rar & 0
    \end{tikzcd}
  \]
  Ma quindi:
  \[
    \tor{A, B} \oplus \tor{A, C} \cong \tor{A, B \oplus C}
  \]
  Essendo il modulo di torsione unico a meno di isomorfi.
 \end{proof}

% \begin{corollary}
%   Vale che $ \tor{A, \Z} = 0 $.
% \end{corollary}
% \begin{proof}
%   Infatti $ \tor{A, \Z} \cong \tor{\Z, A} $.
% \end{proof}

\begin{example}
  Considero $ \Disk{2} $, ci attacco una $ \Sph{1} $ con la mappa:
  \begin{align*}
    f_n \colon \partial \Disk{2} = \Sph{1} & \to \Sph{1} \\
    z & \mapsto z^n
  \end{align*}
  sia $ X_n = \Disk{2} \cup_{f_n} \Sph{1} $ lo spazio topologico preso in considerazione.
  Usando l'omologia cellulare trovo che:
  \[
    H_k(X_n) \cong
    \begin{cases}
      \Z & \text{se } k = 0 \\
      ? & \text{se } k \in \set{1,2} \\
      0 & \text{se } k \geq 3
    \end{cases}
  \]
  Infatti $ X_n $ è un CW complesso con una $ 0 $-cella, una $ 1 $-cella e una $ 2 $-cella,
  quindi la successione è:
  \[
    \begin{tikzcd}
      0 \rar{d_3} & \Z \rar{d_2} & \Z \rar{d_1} & \Z \rar{d_0} & 0
    \end{tikzcd}
  \]
  Ho che $ \ker{d_0} = \Z $ e $ \im{d_1} = 0 $ in quanto $ H_0(X_n) = \quot{\ker{d_0}}{\im{d_1}} $
  ma so che $ \ker{d_0} = \Z $ e $ H_0(X_n) = \Z $ quindi $ \im{d_1} = 0 $.

  Ora calcolo $ H_1(X_n) = \quot{\ker{d_1}}{\im{d_2}} $. Siccome $ \im{d_1} = 0 $ allora $ d_1 \colon \Z \to 0 $,
  quindi $ \ker{d_1} = \Z $, mi rimane da calcolare $ \im{d_2} $, ma:
  $ d_2 \colon S_2^{CW}(X_n) \to S_1^{CW}(X_n) $, per calcolarla:
  \[
    \begin{tikzcd}
      \partial \Disk{2} \rar{f_n} \arrow{rd}{\phi} & X^{(1)} \dar \\
      {} & \quot{X^{(1)}}{X^{(0)}} = X^{(1)}
    \end{tikzcd}
  \]
  Quindi $ \deg{\phi} = \deg{f} = n $ data la definizione di $ f $, per questo $ d_2 $ è la moltiplicazione
  per $ n $:
  \begin{align*}
    d_2 \colon \Z \to \Z \\
    x & \mapsto n x
  \end{align*}
  E quindi $ \ker{d_2} = 0 $ e $ \im{d_2} = \ Z $, da cui: $ H_1(X_n) = \quot{\Z}{n \Z} = \Z_n $.
  Invece $ H_2(X_n) = \quot{\ker{d_2}}{\im{d_3}} = 0 $, per cui:
  \[
    H_k(X_n) =
    \begin{cases}
      \Z & \text{se } k = 0 \\
      \Z_n & \text{se } k = 1 \\
      0 & \text{se } k = 2 \\
      0 & \text{se } k \geq 3
    \end{cases}
  \]

\end{example}

Ora vorrei cambiare coefficienti.

\section{Cambiamento di coefficienti}

Sia $ G $ un gruppo abeliano e $ X $ uno spazio topologico, considero il complesso
$ (S_\bullet(X) \otimes G, \partial \otimes \Id{G}) $:
\[
  \begin{tikzcd}
    \dots \rar & S_{p+1}(X) \otimes G \rar{\partial \otimes \Id{G}} & S_p(X) \otimes G \rar{\partial \otimes \Id{G}} & S_{p-1}(X) \otimes G \rar & \dots
  \end{tikzcd}
\]
Un modo compatto per scrivere il complesso è $ (S_\bullet(X;G), \partial) $. Ora i coefficienti non
sono più in $ \Z $, ma in $ G $. Definisco l'omologia singolare a coefficienti in $ G $
come l'omologia singolare di questo complesso. Se $ G = \Z $ si torna alla consueta omologia
singolare.

Mi pongo questa domanda: se $ X $ è uno spazio topologico e $ G $ un gruppo abeliano, che relazione
c'è tra $ H_k(X) \oplus G $ e $ H_k(X;G) $? Vale che $ H_k(X) \oplus G \cong H_k(X;G) $?

\begin{example}
  Considero $ X_9 $, so che $ H_1(X_9) \cong \Z_9 $, quindi $ H_1(X_9) \otimes \Z_6 = \Z_9 \otimes \Z_6 $.
  Gli elementi di $ \Z_9 \otimes \Z_6 $ sono del tipo $ [n]_9 \otimes [m]_6 $, questi sono 54 elementi,
  ma molti possono essere zero. In effetti vale che:
  \begin{lemma}
    $ \Z_n \otimes \Z_m \cong \Z_d $ dove $ d $ è il massimo comune divisore tra $ n $ e $ m $.
  \end{lemma}
  \begin{exercise}
    Verificare il precedente lemma. Un modo per farlo è costruire esplicitamente
    l'isomorfismo:
    \begin{align*}
      \Z_n \otimes \Z_m & \to \Z_d \\
      [a]_n \otimes [b]_m & \mapsto [ab]_d
    \end{align*}
  \end{exercise}
  Considero le successioni:
  \[
    \begin{tikzcd}
      0 \rar & Z_1(X_9) \rar & S_1(X_9) \rar & B_0(X_9) \rar & 0
    \end{tikzcd}
  \]
  E
  \[
    \begin{tikzcd}
      0 \rar & B_1(X_9) \rar & Z_1(X_9) \rar & H_1(X_9) \rar & 0
    \end{tikzcd}
  \]
  Questa non è esatta, ma anzi definisce l'omologia, e in questo caso non spezza
  perché $ H_1(X_9) $ ha torsione.
\end{example}
\begin{theorem}[Teorema dei coefficienti universali\index{Teorema dei coefficienti universali}]
  La successione:
  \[
    \begin{tikzcd}[nodes = {column sep = 10pt}]
      0 \rar & H_k(S_\bullet(X)) \otimes G \rar & H_k(S_\bullet(X) \otimes G) \rar & \tor{H_{k-1}(S_\bullet(X)), G} \rar & 0
    \end{tikzcd}
  \]
  spezza in modo non naturale, cioè non esiste un'unica sezione. Si ha quindi che
  $ H_k(S_\bullet(X) \otimes G) \not \cong H_k(S_\bullet(X)) \otimes G $ ma c'è un pezzo
  di torsione, cioè vale che:
  \begin{gather*}
    H_k(X;G) = H_k(S_\bullet \otimes G) \cong \\
    \cong H_k(S_\bullet(X)) \otimes G \oplus \tor{H_k(S_\bullet), G} = H_k(X) \otimes G \oplus \tor{H_k(X), G}
  \end{gather*}
  Ho le successioni:
  \[
    \begin{tikzcd}[nodes = {column sep = 10pt}]
      {} & 0 \dar & \dots \dar & {} & {} \\
      {} & B_p \dar & S_{p+1} \dar & \dots \dar &  \\
      0 \rar & Z_p \dar \rar & S_p \dar \rar & B_{p-1} \rar \dar  & 0 \\
      {} & H_p \dar & S_p \dar & Z_{p-1} \dar & {} \\
      {} & 0 & \dots & \dots & {}
    \end{tikzcd}
  \]
  Quando tensorizzo escono fuori delle torsioni.
  \[
    \begin{tikzcd}[nodes = {column sep = 10pt}]
      {}     & \dots  \dar               & \dots \dar            & 0   \dar              & {} \\
      {}     & B_p \otimes G \dar      & S_{p+1} \otimes G\dar   & \tor{H_{p-1},G} \dar            &    \\
      0 \rar & Z_p \otimes G \dar \rar & S_p \otimes G \dar \rar & B_{p-1} \otimes G \rar \dar & 0  \\
      {}     & H_p \otimes G \dar      & S_p \otimes G \dar      & Z_{p-1} \otimes G \dar      & {} \\
      {}     & 0                 & \dots                 & \dots                 & {}
    \end{tikzcd}
  \]
  La successione orizzontale è esatta in quanto $ B_{p-1} $ è libero e quindi
  $ \tor{B_{p-1}, G} \cong \tor{G, B_{p-1}} \cong 0 $.
\end{theorem}

In particolare nell'esempio:
\[
  H_k(X_9; \Z_6) \cong
  \begin{cases}
    H_0(X) \otimes G \oplus \tor{H_1,G} & \text{se } k = 0 \\
    H_1(X) \otimes G \oplus \tor{H_0,G} & \text{se } k = 1 \\
    H_2(X) \otimes G \oplus \tor{H_1,G} & \text{se } k = 2
  \end{cases}
\]
Ma $ \tor{H_{-1}, G} \cong 0 $ in quanto $ H_{-1} \cong 0 $, quindi
$ H_0(X_9, \Z_6) \cong \Z_6 $. Poi $ \tor{H_0, G} = \tor{\Z, \Z_6} = 0 $ in quanto
$ \Z $ è libero, quindi $ H_1(X_9, \Z_6) \cong \Z_9 \otimes \Z_6 \cong \Z_3 $. Infine
$ \tor{H_1, G} \cong \tor{\Z_9, \Z_6} \cong \Z-3 $. Quindi:
\[
  H_k(X_9, \Z_6) \cong
  \begin{cases}
    \Z_6 & \text{se } k = 0 \\
    \Z_3 & \text{se } k = 1 \\
    \Z_3 & \text{se } k = 2
  \end{cases}
\]

\begin{osservation}
  Esempi di gruppi di coefficienti che si possono utilizzare sono $ \Z $, $ \Z_n $,
  $ \mathbb{Q} $, $ \RN{} $, $ \mathbb{C} $, $ \mathbb{F} $. In questi casi si ha che:
  \[
    H_k(X, \mathbb{F}) \cong H_k(X) \otimes \mathbb{F}
  \]
  Infatti questi sono moduli liberi e quindi non hanno torsione.
\end{osservation}

\begin{osservation}
  In generale se $ G $ è un gruppo abeliano finitamente generato c'è il teorema di struttura
  per cui $ G \cong \Z^n \oplus T $, per cui dal teorema dei coefficienti universali:
  \begin{gather*}
    H_k(X;G) \cong H_k(X) \otimes G \oplus \tor{H_{n-1}(X), G} \cong \\
    \cong H_n(X) \otimes (\Z^n \oplus T) \oplus \tor{H_{n-1}(X), \Z^n \oplus T}
  \end{gather*}
  Usando la bilinearità del prodotto tensore:
  \[
    H_n(X) \otimes (\Z^n \oplus T) \cong H_n(X) \otimes \Z^r \oplus H_n(X) \otimes T
  \]
  Questo in generale dipende da $ X $, ma se in particolare $ X $ è un CW
  complesso finito, allora anche $ H_n(X) $ è finitamente generato, quindi
  $ H_n(X) \cong \Z^{s_n} \oplus T' $ per cui vale che:
  \begin{gather*}
    H_n(X) \otimes \Z^{r_n} \cong (\Z^{s_n} \oplus T') \otimes \Z^r \cong \Z^{s_n r_n} \oplus \Z^{r_n} \otimes T' \cong \Z^{sr} \\
    H_n(X) \otimes T = (\Z^{s_n} \oplus T') \otimes T \cong T' \otimes T
  \end{gather*}
  infatti
  $ \Z^k \otimes T' = (\Z \otimes \Z \dots ) \otimes T' = (\Z \otimes T')^k = 0 $ in quanto
  $ T' $ è di torsione. [PERCHÉ'????].

  Poi ho
  $ \tor{H_{n-1}(X), \Z^{r_n} \oplus T} = \tor{H_{n-1}(X), \Z^n} \oplus \tor{H_{n-1}(X), T} $
  per un lemma precedente, quindi in questo caso, siccome $ \Z $ è libero quindi
  $ \tor{H_{n-1}(X), \Z^{r_n}} = (\tor{H_{n-1}(X), \Z})^{r_n} = 0 $, allora:
   \begin{gather*}
     \tor{H_{n-1}(X), \Z^{r_n} \oplus T} \cong \tor{H_{n-1}(X), \Z^{r_n}} \oplus \tor{H_{n-1}(X), T} = \\
     = \tor{H_{n-1}(X), T} = \tor{\Z^{s_{n-1}} \oplus T'_{n-1}, T}
   \end{gather*}
   Quindi:
   \[
     H_n(X; G) \cong \Z^{s_n r_n} \oplus T_n' \oplus T \oplus \tor{T_{n-1}', T}
   \]
   Dove $ H_k(X) \cong \Z^{s_k} \oplus T_k' $ e $ G \cong \Z^r \oplus T $.
   $ H_n(X;G) $ ha quindi una parte libera e delle parti di torsione che si calcolano
   sapendo fare $ \Z_h \otimes \Z_k $ (infatti $ T $ e $ T' $ sono fatte così).
 \end{osservation}

 \begin{exercise}
   Considerare la successione:
   \[
     \begin{tikzcd}
       0 \rar & h \Z \rar & \Z \rar & \quot{\Z}{h \Z} \rar & 0
     \end{tikzcd}
   \]
   Calcolare il modulo di torsione.
 \end{exercise}

\section{Coomologia singolare}

\newmathsymb{hom}{\hom{A,B}}{Spazio degli omomorfismi da $ A $ a $ B $}
Dato uno spazio topologico $ X $ e un gruppo abeliano $ G $ ho costruito le
catene in $ X $ a coefficienti in $ G $ e ho definito l'omologia singolare a
coefficienti in $ G $ come l'omologia di questo complesso. Posso fare anche un'altra
costruzione, considero lo spazio degli omomorfismi da $ S_k(X) $ a $ G $
$ \hom{S_k(X), G} $. A questo punto posso considerare il duale del complesso delle catene:
\[
  \begin{tikzcd}
    \dots \rar & S_{p+1}(X) \rar{\partial} & S_p(X) \rar{\partial} & S_{p-1}(X) \rar & \dots
  \end{tikzcd}
\]
Un elemento di $ \hom{S_p(X), G} $ è un omomorfismo $ \phi \colon S_p(X) \to G $, componendo $ \phi $
con $ \partial \colon S_{p+1}(X) \to S_p(X) $ ottengo  $ \phi' = \phi \circ \partial \colon S_{p+1}(X) \to G $, quindi la composizione
per il bordo è un'operazione controvariante perché inverte il verso. Ho il complesso degli spazi
di omomorfismi:
\[
  \begin{tikzcd}[nodes = {column sep = 14 pt}]
    \dots \rar & \hom{S_{p-1}(X),G} \rar{\delta} & \hom{S_p(X),G} \rar{\delta} & \hom{S_{p+1}(X),G} \rar & \dots
  \end{tikzcd}
\]
Come notazione si pone $ \hom{S_p(X),G} = S^p(X;G) $. $ \delta $ è il \textbf{cobordo}\index{Cobordo},
che non è nient'altro che la composizione per il bordo:
\begin{align*}
  \delta \colon S^p(X;G) & \to S^{p+1}(X;G) \\
  \phi & \mapsto \phi \circ \partial = \delta(\phi)
\end{align*}
Questo è un operatore di bordo, cioè $ \delta^2 = 0 $, infatti:
\[
  \delta^2(\phi) = \delta(\delta(\phi)) = \delta (\phi \circ \partial) = \phi \circ \partial^2 = 0
\]
Questo è un complesso.
\begin{definition}
  Si chiama \textbf{coomologia singolare}\index{Coomologia singolare} di uno
  spazio topologico $ X $ con coefficienti in $ G $, e si indica con
  $ H^p(X; G) $ l'omologia del complesso degli omomorfismi $ S^\bullet(X;G) $.
\end{definition}
Quindi per definizione la coomologia singolare è $ H^p(X;G) = H_p(\hom{S_\bullet(X), G}, \delta) $.

A questo punto ho due possibilità: costruire i gruppi di omologia singolare
$ H_p(X) $ e considerare gli omomorfismi tra tali gruppi e $ G $, oppure
costruire il gruppo di coomologia, cioè prima considerare gli omomorfismi, e
quindi costruire l'omologia. Quello che si trova è che in generale queste
due costruzioni sono differenti, cioè:
\[
  \hom{H_p(X), G} \not \cong H^p(X ; G)
\]

\begin{example}
  Considero la successione esatta corta:
  \[
    \begin{tikzcd}
      0 \rar & 4 \Z \rar & \Z \rar & \quot{\Z}{4\Z} \rar & 0
    \end{tikzcd}
  \]
  E scelgo come gruppo $ G = \Z_6 $.
  Quando prendo il duale la successione si inverte essendo controvariante,
  e rimane esatta solo a sinistra. Per renderla esatta anche a destra bisogna
  aggiungere un termine analogo al modulo di torsione, in modo che la successione
  sia:
  \[
    \begin{tikzcd}[nodes = {column sep = 18 pt}]
      0 \rar & \hom{\Z_4, \Z_6} \rar & \hom{\Z, \Z_6} \rar & \hom{4\Z, \Z_6} \rar & \ext{\Z_4, \Z_6} \rar & 0
    \end{tikzcd}
  \]
  La presenza di questi moduli è responsabile della non uguaglianza tra i gruppi
  $ \hom{H_p(X), G} $ e $ H^p(X ; G) $, come formalizza il teorema dei
  coefficienti universali.
\end{example}
\begin{definition}
  Siano $ A, B $ $ \Z $-moduli, considero una risoluzione di $ A $:
  \[
    \begin{tikzcd}
      0 \rar & R \rar & F \rar & A \rar & 0
    \end{tikzcd}
  \]
  Passando agli omomorfismi la successione si gira e si aggiunge il \textbf{conucleo}\index{Conucleo}
  \[
    \begin{tikzcd}
      0 \rar & \hom{A,B} \rar & \hom{F,B} \rar{\beta} & \hom{R,B} \rar{\gamma} & \coker{\beta} \rar & 0
    \end{tikzcd}
  \]
  Il conucleo è esattamente quel gruppo che rende esatta la successione, cioè
  $ \im{\gamma} $, ma per il primo teorema degli isomorfismi:
  \[
    \coker{\beta} := \im{\gamma} \cong \quot{\hom{R,B}}{\ker{\gamma}} \cong \quot{\hom{R,B}}{\im{\beta}}
  \]
  Quindi il conucleo rende esatta a destra la successione.
  Esistono anche altre presentazioni, ma si dimostra che tutti i conuclei sono isomorfi,
  questo gruppo è proprio il \textbf{modulo di estensione di $ A $ e $ B $}\index{Modulo di estensione}.
\end{definition}
\begin{lemma}
  Se $ F $ è libero allora $ \ext{F, G} \cong 0 $ con $ G $ gruppo abeliano generico.
\end{lemma}
\begin{proof}
  Considero la presentazione:
  \[
    \begin{tikzcd}
      0 \rar & 0 \rar & F \rar & F \rar & 0
    \end{tikzcd}
  \]
  Passando agli omomorfismi ho che il conucleo è zero infatti:
  \[
    \begin{tikzcd}
      0 \rar & \hom{F,G} \rar & \hom{F,G} \rar & 0 \rar{\gamma} & \ext{F,G} \rar & 0
    \end{tikzcd}
  \]
  Quindi $ \ext{F,G} = \im{\gamma} = 0 $.
\end{proof}
\eproof
Il teorema dei coefficienti universali quindi si riformula anche per la coomologia:
\begin{theorem}[Teorema dei coefficienti universali\index{Teorema dei coefficienti universali}]
  Le successioni esatte corte:
  \[
    \begin{tikzcd}[nodes = {column sep = 12 pt}]
      0 \rar & \ext{H_{n-1}(X), G} \rar & H^n(X;G) \arrow[bend left]{l}{} \rar & \hom{H_n(X), G} \rar & 0
    \end{tikzcd}
  \]
  E:
  \[
    \begin{tikzcd}[nodes = {column sep = 18 pt}]
      0 \rar & H_n(X) \oplus G \rar & H_n(X; G) \rar & \arrow[bend left]{l}{} \tor{H_{n-1}(X), G} \rar & 0
    \end{tikzcd}
  \]
  Spezzano in modo non naturale (cioè non esiste una sola sezione), e quindi:
  \begin{gather*}
    H_n(X; G) \cong H_n(X) \oplus G \oplus \tor{H_{n-1}(X), G} \\
    H^n(X; G) \cong \hom{H_n(X), G} \oplus \ext{H_{n-1}(X), G}
  \end{gather*}
\end{theorem}
%
% lezione 17
%
% Sia $ X $ spazio topologico e $ G $ gruppo abeliano, ho
% \[
%   \begin{tikzcd}[nodes={row sep = 5 pt}]
%     H_\bullet \oplus G \rar & \lar H^\bullet(X;G) \\
%     \hom{H_\bullet(X); G} \rar & \lar H_\bullet(X;G)
%   \end{tikzcd}
% \]
% Quale è esattamente la relazione tra questi gruppi?
% La risposta è data dal teorema dei coefficienti universali.
% \begin{theorem}
%   Esiste una successione esatta spezzante in modo non naturale, cioè possono
%   esistere più sezioni in omologia singolare e in coomologia, queste sono:
%   \[
%     \begin{tikzcd}[nodes = {column sep = 12 pt}]
%       0 \rar & \ext{H_{p-1}(X), G} \rar & \arrow[bend left]{l}{} H^p(X;G) \rar & \hom{H_p(X), G} \rar & 0
%     \end{tikzcd}
%   \]
%   E:
%   \[
%     \begin{tikzcd}[nodes = {column sep = 18 pt}]
%       0 \rar & H_n(X) \oplus G \rar & H_n(X; G) \rar & \arrow[bend left]{l}{} \tor{H_{n-1}(X), G} \rar & 0
%     \end{tikzcd}
%   \]
% \end{theorem}
\begin{proof}
  La dimostrazione per le due successioni è praticamente identica, dimostro quella in coomologia.
  Voglio costruire la successione:
  \[
    \begin{tikzcd}[nodes = {column sep = 12 pt}]
      0 \rar & \ext{H_{p-1}(X), G} \rar & \arrow[bend left]{l}{} H^p(X;G) \rar & \hom{H_p(X), G} \rar & 0
    \end{tikzcd}
  \]
  Per definizione $ H^p(X;G) $ è l'omologia del complesso delle cocatene $ S^p $ con il cobordo,
  dove $ S^p(X;G) = \hom{S_p(X), G} $ e il cobordo è:
  \begin{align*}
    \delta \colon S^p(X;G) & \to S^{p+1}(X;G) \\
    \phi & \mapsto \phi \circ \partial
  \end{align*}
  La dimostrazione richiede che si costruisca un diagramma, quindi elenco alcune
  successioni esatte, omettendo per concisione l'esplicita dipendenza dallo spazio
  topologico, che si intende essere $ X $.

  \noindent
  Per definizione $ H_p(X) = Z_p \slash B_p $ (cicli modulo i bordi), quindi ho:
  \[
    \begin{tikzcd}
      0 \rar & B_p \rar{i} & Z_p \rar{\pi} & H_p \rar & 0
    \end{tikzcd}
  \]
  Non necessariamente questa spezza perché $ H_p $ può essere di torsione.
  Poi ho:
  \[
    \begin{tikzcd}
      0 \rar & Z_p \rar & S_p \rar{\partial} & \arrow[bend left]{l}{i} B_{p-1} \rar & 0
    \end{tikzcd}
  \]
  Questa spezza perché tra le catene singolari ci sono quelle che si esprimono
  come bordo e quindi c'è una sezione, che sui generatori (sono entrambi gruppi
  liberi) agisce come:
  \begin{align*}
    i \colon B_{p-1} \to S_p \\
    \partial c & \mapsto c
  \end{align*}
  In questo modo$ \partial \circ i = \Id{B_{p-1}} $.
  Poi ho a partire da:
  \[
    \begin{tikzcd}
      0 \rar & B_p \rar & Z_p \rar & H_p \rar & 0
    \end{tikzcd}
  \]
  Passando agli omomorfismi:
  \[
    \begin{tikzcd}
      0 \rar & \hom{H_p, G} \rar & \hom{Z_p, G} \rar{t_p} & \hom{B_p, G} \rar & \ext{H_p, G} \rar & 0
    \end{tikzcd}
  \]
  Per definizione ho che:
  \[
    \ext{H_p,G} \cong \quot{\hom{B_p,G}}{\im{t_p}}
  \]
  Oltre a ciò ho la successione:
  \[
    \begin{tikzcd}
      0 \rar & Z_{p+1} \rar & S_{p+1} \rar & B_p \rar & 0
    \end{tikzcd}
  \]
  Passando agli omomorfismi:
  \[
    \begin{tikzcd}
      0 \rar & \hom{B_p, G } \rar & \hom{S_{p+1}, G} \rar & \dots
    \end{tikzcd}
  \]
  Poi ho la successione:
  \[
    \begin{tikzcd}
      0 \rar & Z_{p-1} \rar & S_{p-1} \rar & B_{p-2} \rar & 0
    \end{tikzcd}
  \]
  Prendendo gli omomorfismi (non c'è il modulo di estensione in quanto i gruppi sono liberi):
  \[
    \begin{tikzcd}
      0 \rar & \hom{B_{p-2}, G} \rar & \hom{S_{p-1}, G} \rar & \hom{Z_{p-1}, G} \rar & 0
    \end{tikzcd}
  \]
  Infine, siccome ho la successione spezzante:
  \[
    \begin{tikzcd}
      0 \rar & Z_p \rar & S_p \arrow[bend left]{l}{\phi} \rar & B_{p-1} \rar & 0
    \end{tikzcd}
  \]
  È ben definita la sezione, e quindi posso definire la mappa:
  \begin{align*}
    \Phi \colon \hom{Z_p, G} & \to \hom{S_p, G} \\
    \alpha \colon Z_p \to G & \mapsto \alpha \circ \phi \colon S_p \to G
  \end{align*}
  cioè $ \Phi = \alpha \circ \phi $.
  Il mio obiettivo è trovare la successione esatta:
  \[
    \begin{tikzcd}
      0 \rar & \ext{H_{p-1}, G} \rar{\beta_1} & H^p(X; G) \rar{\beta_2} & \hom{H_p, G} \rar & 0
    \end{tikzcd}
  \]
  Mettendo insieme le successioni costruite ottengo un diagramma su cui posso
  fare diagram chase:
  \[
    \begin{tikzcd}
      {} & 0 & \dots & \dots & {} \\
      {} & \ext{H_{p-1},G} \uar & \hom{S_{p+1},G} \uar & \lar{\sigma} \hom{B_p,G} \uar & 0 \lar \\
      0 \rar & \hom{B_{p-1}, G} \uar{\lambda_2} \rar{\alpha_1} & \rar{\alpha_2} \hom{S_p,G} \uar{\delta} & \hom{Z_p,G} \rar \uar{\tau_2} \arrow[bend left]{l}{\Phi}  & 0 \\
      0 & \lar \hom{Z_{p-1}, G} \uar{\lambda_1} & \lar{\Delta} \hom{S_{p-1},G} \uar{\delta} & \hom{H_p,G} \uar{\tau_1}  & {} \\
      {} & \dots \uar & \dots \uar & 0 \uar{} & {}
    \end{tikzcd}
  \]
  Costruisco $ \beta_2 $.
  Per definizione:
  \[
    H^p(X;G) = \quot{\ker{\delta\colon S^p(X;G) \to S^{p+1}(X;G)}}{\im{\delta \colon S^{p-1}(X;G) \to S^p(X;G)}}
  \]
  Se $ \llbracket f \rrbracket \in H^p(X; G) $ allora $ f \in S^p $ e $ \delta(f) = 0 $. Applicando
  $ \sigma \circ \tau_2 \circ \alpha_2 $ a $ f $ e usando la commutatività:
  \[
    \sigma \circ \tau_2 \circ \alpha_2 (f) = \delta(f) = 0
  \]
  Quindi $ \sigma \circ \tau_2 \circ \alpha_2 (f) = 0 $, ma $ \sigma $ è iniettiva e quindi
  $ \tau_2(\alpha_2(f)) = 0 $, cioè
  $ \alpha_2(f) \in \ker{\tau_2} = \im{\tau_1} $ per l'esattezza della successione e quindi
  $ \exists g \in \hom{H_p, G} $ tale che $ \tau_1(g) = \alpha_2(f) $, e quindi ho trovato un
  elemento $ g $ a partire da $ f $. Pongo $ \beta_2(f) = g_f $. Per verificare che
  l'applicazione sia ben definita devo controllare che cambiando rappresentante
  della classe di equivalenza $ \llbracket f \rrbracket $ si ottenga il medesimo
  $ g_f $, equivalentemente posso verificare che l'associazione che ho definito
  mandi tutto il modulo $ \im{\delta \colon S^{p-1}(X;G) \to S^p(X;G)} $ in zero. Sia
  $ \delta (h) \in \im{\delta \colon S^{p-1}(X;G) \to S^p(X;G)} \subseteq S^p $ devo verificare che
  $ \beta_2(\delta(h)) = 0 $. Per trovare l'elemento $ g_{\delta(h)} $ applico $ \sigma \circ \tau_2 \circ \alpha_2 $
  e uso la commutatività:
  \[
    \sigma \circ \tau_2 \circ \alpha_2 (\delta(h)) = \delta \circ \delta (h) = 0
  \]
  Quindi per l'iniettivita di $ \sigma $ $ \tau_2(\alpha_2(\delta(h))) = 0 $ perciò
  $ \alpha_2(\delta(h)) \in \ker{\tau_2} = \im{\tau_1} $ e quindi esiste
  $ v \in \hom{H_p,G} $ tale che $ \alpha_2(\delta(h)) = \tau_1(v) $ e quindi si definisce
  $ \beta_2(\delta(h)) = v $. Devo mostrare che $ v = 0 $ per mostrare che
  $ \beta_2 $ è ben definita, ma $ \tau_1 $ è iniettiva, quindi posso mostrare che
  $ \alpha_2 \circ \delta (h) = 0 $ per mostrare che $ v = 0 $. Ma ho:
  \[
    \begin{tikzcd}
      S_p \rar{\partial} & S_{p-1} \rar{h} & G
    \end{tikzcd}
  \]
  \[
    \begin{tikzcd}
      Z_p \rar{i} & S_{p} \rar{h \circ \partial} & G
    \end{tikzcd}
  \]
  Quindi $ \alpha_2(\delta(h)) = \alpha_2(h \circ \partial) $ e $ \alpha_2(h \circ \partial) = h \circ \partial \circ i \colon Z_p \to G $.
  Ma in $ Z_p $ ci sono solo quelli di bordo nullo, cioè se $ c \in Z_p $:
  \[
    (h \circ \partial \circ i)(c) = h \circ \partial(c) = h(0) = 0
  \]
  Quindi:
  \[
    \alpha_2 \circ \delta(h) = 0 \quad \Rightarrow \quad \tau_1(v) = 0
  \]
  Ma quindi $ v = 0 $ in quando $ \tau_1 $ è iniettiva. Ma questo significa
  che $ \beta_2 $ è ben definita:
  \begin{align*}
    \beta_2 \colon H^p(X;G) & \to \hom{H_p,G} \\
    \llbracket f \rrbracket & \mapsto g_f \; | \; \tau_1(g_f) = \alpha_2(f)
  \end{align*}

  Ora costruisco $ \beta_1 \colon \ext{H_{p-1}, G} \to H^p(X;G) $. Parto da $ u \in \ext{H_{p-1}, G} $,
  $ \lambda_2 $ è suriettiva, quindi esiste $ \tilde{u} \in \hom{B_{p-1}.G} $ tale che
  $ \lambda_2(\tilde{u}) = u $, poi ho che $ \alpha_1(\tilde{u}) \in \hom{S_p, G} = S^p $,
  quindi posso definire:
  \begin{align*}
    \beta_1 \colon \ext{H_{p-1}, G} & \to H^p(X,G) \\
    u & \mapsto \llbracket\alpha_1(\tilde{u})\rrbracket \; | \; \lambda_2(\tilde{u}) = u
  \end{align*}
  Per poter scendere a livello di omologia devo mostrare che se
  $ u \in \ext{H_{p-1},G} $ allora $ \alpha_1(\tilde{u}) \in S^p $ è un cociclo, cioè
  $ \delta(\alpha_1(\tilde{u})) = 0 $, ma per la commutatività:
  \[
    \delta \circ \alpha_1 = (\sigma \circ \tau_2 \circ \alpha_2)(\alpha_1) = \sigma \circ \tau_2 \circ \alpha_2 \circ \alpha_1 = 0
  \]
  In quanto $ \alpha_2 \circ \alpha_1 = 0 $ perché la successione è esatta, e quindi
  $ \delta(\alpha_1(\tilde{u})) = 0 $. Bisogna mostrare che $ \beta_1 $ è ben definita, cioè
  comunque scelga la preimmagine $ \tilde{u} $ ottengo sempre la medesima classe
  di equivalenza in $ H^p(X,G) $. Se $ \tilde{u} $ non fosse unico, ma se
  esistessero $ \tilde{u}_1, \tilde{u}_2 $ tali che
  $ \lambda_2(\tilde{u}_1) = \lambda_2(\tilde{u}_1) = u $, siccome $ \lambda_2 $ è un omomorfismo
  $ \lambda_2(\tilde{u}_1 - \tilde{u}_2) = 0 $, quindi
  $ \tilde{u}_1 - \tilde{u}_2 \in \ker{\lambda_2} = \im{\lambda_1} $ per l'esattezza della
  successione, quindi esiste $ V \in \hom{Z_{p-1}, G} $ tale che
  $ \lambda_1(V) = \tilde{u}_1 - \tilde{u}_2 $. Ma $ \Delta $ è suriettiva, quindi esiste
  $ w \in \hom{S_{p-1}, G} $ tale che $ \Delta(w) = V $. Quindi per la commutatività:
  \[
    \delta(w) = \alpha_1 \circ \lambda_1 \circ \Delta (w) = \alpha_1 \circ \lambda_1 (V) = \alpha_1 ( \tilde{u}_1 - \tilde{u}_2 )
  \]
  Quindi:
  \[
    \alpha_1(\tilde{u}_1) - \alpha_2(\tilde{u}_2) = \delta(w)
  \]
  Le immagini differiscono per un cobordo quindi danno origine alla stessa
  classe di equivalenza e perciò $ \beta_1 $ è ben definita.
  \\ \\
  Così ho costruito le due applicazione che mi servivano, ma non ho ancora
  finito, devo mostrare che $ \beta_2 $ è suriettiva, $ \beta_1 $ iniettiva, che
  $ \im{\beta_2} = \ker{\beta_1} $ e che la successione spezza.

  Dimostro che $ \beta_1 $ è iniettiva mostrando che il suo nucleo è banale.
  Considero $ u \in \ker{\beta_1} $, quindi tale che $ \beta_1(u) = 0 $, allora
  $ \llbracket\alpha_1(\tilde{u})\rrbracket = 0 $. Questo è vero se
  $ \alpha_1(\tilde{u}) $ è un cobordo, cioè esiste $ z $ tale che
  $ \alpha_1(\tilde{u}) = \delta z $. Applicando $ \alpha_1 $ e usando la commutatività:
  \[
    \alpha_1 \circ \alpha_1 (\tilde{u}) = \delta \circ \tau_2 \circ \alpha_2 \circ \alpha_1 (\tilde{u}) = 0
  \]
  Ma $ \alpha_1 $ è iniettiva quindi $ \tilde{u} = 0 $, ma
  $ u = \lambda_2(\tilde{u})$ quindi $ u = 0 $ siccome $ \lambda_2 $ è omomorfismo, e perciò
  $ \ker{\beta_1} = 0 $, e quindi $ \beta_1 $ è iniettiva.

  Per dimostrare che $ \beta_2 $ è suriettiva considero $ v \in \hom{H_p, G} $, allora
  $ \Phi(\tau_1(v)) \in \hom{S_p,G} $ applicando $ \delta $ e usando la commutatività:
  \[
    \delta \circ \Phi(\tau_1(v)) = (\sigma \circ \tau_2 \circ \alpha_2) \circ \Phi \circ \tau_1 (v) = \sigma \circ \tau_2 \circ (\alpha_2 \circ \Phi) \circ \tau_1 (v)
  \]
  Per trovare l'azione di $ \alpha_2 $ considero la successione spezzante:
  \[
    \begin{tikzcd}
      0 \rar & Z_p \rar{\psi} & S_p \arrow[bend left]{l}{\phi} \rar & B_{p-1} \rar & 0
    \end{tikzcd}
  \]
  Quindi ho:
  \begin{align*}
    \alpha_2 \colon \hom{S_p, G} & \to \hom{Z_p, G} \\
    \omega \colon S_p \to G & \mapsto \omega \circ \psi \colon Z_p \to G
  \end{align*}
  E quindi:
  \[
    (\alpha_2 \circ \Phi)(\omega) = \alpha_2(\omega \circ \phi) = \omega \circ \phi \circ \psi = \omega
  \]
  Quindi $ \alpha_2 \circ \Phi = \Id{\hom{Z_p,G}} $ e quindi
  $ \sigma \circ \tau_2 \circ \tau_2 (v) = 0 $ in quanto
  $ \tau_2 \circ \tau_1 = 0 $, dato che la colonna è esatta, quindi
  $ \delta(\Phi(\tau_1(v))) = 0 $, cioè $ \Phi(\tau_1(v) $ è un cociclo in
  $ S_p $ ed è tale che $ \beta_2(\llbracket\Phi(\tau_1(v))\rrbracket) = v $. Faccio
  agire $ \beta_2 $:
  \[
    \beta_2 \colon \llbracket \Phi(\tau_1(v)) \rrbracket \mapsto x \; | \; \tau_1(x) = \alpha_2 \circ \Phi(\tau_1(v)))= \tau_1(v)
  \]
  Essendo $ \tau_1 $ iniettivo ho che $ x = v $ quindi $ \beta_2(\llbracket\Phi(\tau_1(v))\rrbracket) = v $.

  Ora devo mostrare che $ \im{\beta_1} = \ker{\beta_2} $. Mostro che
  $ \im{\beta_1} \subseteq \ker{\beta_2} $. Sia $ u \in \ext{H_{p-1},G} $ allora
  $ \beta_1(u) = \llbracket\alpha_1(\tilde{u})\rrbracket $, applicando $ \beta_2 $:
  \[
    \beta_2 \colon \llbracket \alpha_1(\tilde{u}) \rrbracket \mapsto k \; | \; \tau_1(k) = \alpha_2 \circ \alpha_1(k) = 0
  \]
  Ma $ \tau_1 $ è iniettivo quindi $ \beta_2(\llbracket \alpha_1(\tilde{u}) \rrbracket) = 0 $
  e quindi $ \beta_1(u) = \llbracket \alpha_1(\tilde{u}) \rrbracket \in \ker{\beta_2} $.

  Ora mostro che $ \ker{\beta_2} \subseteq \im{\beta_1} $, sia
  $ \llbracket f \rrbracket \in H^p(X;G) $, se
  $ \beta_2(\llbracket f \rrbracket) = 0 $ allora $ \alpha_2(f) = \tau_1(0) $ quindi
  $ \alpha_2(f) = 0 $, quindi $ f \in \ker{\alpha_2} = \im{\alpha_1} $ per l'esattezza della
  successione, e quindi esiste $ f' \in \hom{B_{p-1}, G} $ tale che
  $ \alpha_1(f') = f $. Definendo $ u = \lambda_2(f') $ si ha:
  \[
    \beta_1 \colon u \mapsto \llbracket \alpha_1(\tilde{u}) \rrbracket \; | \; \lambda_2(\tilde{u}) = u = \lambda_2(f')
  \]
  Quindi $ \tilde{u} - f' \in \ker{\lambda_2} = \im{\lambda_1} $, quindi esiste $ \eta \in \hom{Z_{p-1}, G} $
  tale che $ \lambda_1(\eta) = \tilde{u} - f' $, ma $ \Delta $ è suriettivo, quindi esiste
  $ \chi \in \hom{S_{p-1},G} $ tale che $ \Delta(\chi) = \eta $, quindi per la commutatività:
  \[
    \alpha_1(\tilde{u}) = \alpha_1(f' + \lambda_1(\eta)) = \alpha_1 \circ f' + \alpha_1 \circ \lambda_1 \circ \Delta (\chi) = \alpha_1 \circ f' + \delta \chi
  \]
  Passando alla classe di equivalenza in coomologia:
  \[
    \llbracket \alpha_1(\tilde{u}) \rrbracket = \llbracket \alpha_1 \circ f' \rrbracket
  \]
  E quindi $ \beta_1(u) = \llbracket \alpha_1 \circ f' \rrbracket = \llbracket f \rrbracket $, cioè $ \llbracket f \rrbracket \in \im{\beta_1} $.
  \\ \\ \noindent
  La successione è quindi esatta, ma bisogna ancora verificare che spezza:
  \[
    \begin{tikzcd}
      0 \rar & \ext{H_{p-1}(X), G} \rar{\beta_1} & H^p(X;G) \rar{\beta_2} & \hom{H_p(X), G} \rar \arrow[bend left]{l}{\rho} & 0
    \end{tikzcd}
  \]
  Sia $ y \in \hom{H_p, G} $, definisco $ \rho(y) $ come
  $ \rho(y) = \llbracket \Phi(\tau_1(y)) \rrbracket$, in questo modo per costruzione
  $ \beta_2 \circ \rho = \Id{\hom{H_p,G}} $, infatti $ \rho $ percorre
  il diagramma in modo inverso a $ \beta_2 $.
\end{proof}

\begin{example}[Coomologia dello spazio proiettivo reale]
  So che l'omologia dello spazio proiettivo reale con $ n = 3 $ è:
  \[
    H_p(\Pjr{3}, \Z_2) \cong
    \begin{cases}
      \Z & \text{se } p = 0 \\
      \Z_2 & \text{se } p = 1 \\
      0 & \text{se } p = 2 \\
      \Z & \text{se } p = 3
    \end{cases}
  \]
  Applico il teorema dei coefficienti universali, per ogni $ p \in \mathbb{N} $:
  \[
    H^p(\Pjr{3}, \Z_2) \cong \hom{H_p(\Pjr{3}, \Z_2)} \oplus \ext{H_{p-1}(\Pjr{3}, \Z_2)}
  \]
  Quindi:
  \begin{gather*}
    H^0(\Pjr{3}, \Z_2) \cong \hom{\Z, \Z_2} \\
    H^1(\Pjr{3}, \Z_2) \cong \hom{\Z_2, \Z_2} \oplus \ext{\Z, \Z_2} \\
    H^2(\Pjr{3}, \Z_2) \cong \hom{, \Z_2} \oplus \ext{\Z_2, \Z_2} \cong \ext{\Z_2, \Z_2} \\
    H^3(\Pjr{3}, \Z_2) \cong \hom{\Z, \Z_2} \oplus \ext{0, \Z_2} \cong \hom{\Z, \Z_2}
  \end{gather*}
  Calcolo i gruppi che mi mancano:
  \[
    \hom{\Z, \Z_2} = \set{\phi \colon \Z \to \Z_2 } \cong \Z_2
  \]
  Infatti considero l'azione sui generatori, devo decidere dove mandare il generatore di $ \Z $,
  che è $ 1 $, lo posso mandare in $ 0 $ o in $ 1 $, quindi ho due possibili applicazioni, e quindi
  lo spazio degli omomorfismi è isomorfo a $ \Z_2 $. Considerazioni analoghe valgono per
  \[
    \hom{\Z_2, \Z_2} = \set{\phi \colon \Z_2 \to \Z_2 } \cong \Z_2
  \]
  Infatti $ 0 $ deve andare in $ 0 $ essendo un omomorfismo.
  Per calcolare $ \ext{\Z, \Z_2} $ considero la risoluzione:
  \[
    \begin{tikzcd}
      0 \rar & 0 \rar & \Z \rar & \Z \rar & 0
    \end{tikzcd}
  \]
  Passando agli omomorfismi:
  \[
    \begin{tikzcd}
      0 \rar & \hom{\Z, \Z_2} \rar & \hom{\Z, \Z_2} \rar & 0 \rar & 0
    \end{tikzcd}
  \]
  Quindi $ \ext{\Z, \Z_2} \cong 0 $.
  Invece per $ \ext{\Z, \Z_2} $ considero la risoluzione:
  \[
    \begin{tikzcd}
      0 \rar & 2 \Z \rar{i} & \Z \rar{\pi} & \Z_2 \rar & 0
    \end{tikzcd}
  \]
  Passando agli omomorfismi:
  \[
    \begin{tikzcd}
      0 \rar & \Z_2 \rar & \Z_2 \rar & \hom{2\Z, \Z_2} \rar & \ext{\Z_2, \Z_2} \rar & 0
    \end{tikzcd}
  \]
  Tra $ \Z_2 $ e $ \Z_2 $ l'unica possibile mappa iniettiva è l'isomorfismo, quindi
  la successione spezza e $ \ext{\Z_2, \Z_2} \cong \hom{\Z, \Z_2} \cong \Z_2 $.
  Nel complesso quindi:
  \[
    H^k(\Pjr{3}, \Z_2) \cong \Z_2
  \]
  In realtà si dimostra che $ \forall n \in \mathbb{N} $:
  \[
    H^k(\Pjr{n}, \Z_2) \cong \Z_2
  \]
  Mentre:
  \[
    H^k(\Pjc{n}, \Z) \cong \Z
  \]
\end{example}

% lezione 18

\section{Prodotto cup}

\begin{example}
  Sia $ X = \Pjc{2} $ e $ Y = \Sph{2} \vee \Sph{4} $, è vero che $ X $ e $ Y $ sono
  omotopicamente equivalenti? Mi aspetto che non lo siano in quanto $ X $ è una
  varietà topologica, mentre $ Y $ no, dato che possiede un punto (quello a cui
  le due sfere sono incollate, che non possiede un intorno omeomorfo a
  $ \RN{n} $). Per verificarlo posso usare gli invarianti topologici che
  conosco.
  \subsubsection{Gruppo fondamentale}
  Con Seifert-Van Kampen si trova che $ \pi_1(X) \cong \set{1} $ e
  $ \pi_1(Y) \cong \set{1} $, e quindi i gruppi fondamentali sono isomorfi.
  \subsubsection{Gruppi di omologia}
  Per calcolare i gruppi di omologia utilizzo la struttura di CW complesso,
  sia $ X $ che $ Y $ sono formati da una $ 0 $-cella, una $ 2 $-cella e una
  $ 4 $-cella, quindi il complesso delle catene è:
  \[
    \begin{tikzcd}
      0 \rar & S_4^{CW} \rar & S_3^{CW} \rar & S_2^{CW} \rar & S_1^{CW} \rar & S_0^{CW} \rar & 0
    \end{tikzcd}
  \]
  E in entrambi i casi questa si riduce a:
  \[
    \begin{tikzcd}
      0 \rar & \Z \rar & 0 \rar & \Z \rar & \Z \rar & \Z \rar & 0
    \end{tikzcd}
  \]
  Quindi entrambi gli spazi hanno come gruppi di omologia
  $ H_k(X) \cong H_k(Y) \cong \Z $ per $ k \in \set{0,2,4} $.
  \subsubsection{Gruppi di coomologia}
  Con il teorema di coefficienti universali $ H^k(\bullet;G) \cong \hom{H_k(\bullet), G} \oplus \ext{H_{k-1}(\bullet), G} $,
  quindi essendo uguali i gruppi di omologia:
  \[
    H^k(X;G) \cong H^k(Y;G) \cong
    \begin{cases}
      G & \text{se } k = 0 \\
      0 & \text{se } k = 1 \\
      G & \text{se } k = 2 \\
      0 & \text{se } k = 3 \\
      G & \text{se } k = 4 \\
    \end{cases}
  \]
  (Infatti $ H_k(X) \cong \Z $ e quindi $ \hom{\Z, G} \cong G $).

  Ho quindi bisogno di strumenti più fini, per questo e per altri motivi rendo i
  gruppi di coomologia un anello.

\end{example}

\subsection{Richiami di algebra degli anelli}

\begin{definition}
  Un anello commutativo $ \R $ si dice \textbf{dominio di integrità}\index{Dominio di integrità}
  se il prodotto tra qualsiasi coppia di elementi non nulli è un elemento non nullo, cioè
  vale che se $ ab = 0 $ allora o $ a = 0 $ o $ b = 0 $ $ \forall a, b \in \R $.
\end{definition}

\begin{proposition}
  In un dominio di integrità $ \R $ valgono le leggi di cancellazione del prodotto, cioè:
  \[
    \forall a,x,y \in \R \quad ax = ay \Rightarrow x = y
  \]
\end{proposition}

\begin{definition}
  Un \textbf{ideale}\index{Ideale} $ I $ di un anello commutativo $ \R $ è un
  sottoinsieme di $ \R $ tale che $ \forall a,b \in \R $ e $ \forall x,y \in I $ sia
  $ a x + b y \in I $.
\end{definition}

\newmathsymb{idgen}{(i)}{Ideale generato da $ i $}
\begin{definition}
  Un \textbf{dominio a ideali principali}\index{Dominio a ideali principali}
  (PID, \emph{principal ideal domain}) è un dominio di integrità in cui ogni
  ideale è principale\index{Ideale principale}, cioè generato da un solo
  elemento, cioè $ \forall I $ ideale esista $ i \in A $ tale che
  $ I = (i) = \set{a i | a \in A} $. Con la scrittura $ (i) $ si indica l'ideale
  generato.
\end{definition}

\newmathsymb{pid}{PID}{Dominio a ideali principali}
\begin{example}
  Esempi di PID sono $ \Z, \RN{}, \mathbb{F}, \mathbb{K}[x] $.
\end{example}

\subsection{Prodotto cup}

\newmathsymb{cuppr}{\cup}{Prodotto cup}
Sia $ H^\star (X,\R) := \bigoplus_k H^k(X,\R) $, il prodotto
cup $ \cup $ rende $ (H^\star (X,\R), +, \cup) $ un anello.

Sia $ X $ uno spazio spazio topologico, e $ \R $ un PID,
$ S^k(X, \R) $ e $ S^l(X, \R) $ sono gli insiemi delle cocatene,
voglio costruire una mappa:
\begin{align*}
  \cup \colon S^k(X, \R) \times S^l(X, \R) & \to S^{k+l}(X, \R) \\
  (\phi, \psi) & \mapsto \phi \cup \psi
\end{align*}
E quindi passare a livello di coomologia in modo da fornire la struttura
ad anello.

Se $ \phi \cup \psi \in S^{k+l}(X, \R) $ significa che $ \phi \cup \psi \in \hom{S_{k+1}(X), \R} $
e quindi $ \phi \cup \psi \colon S_{k+l}(X) \to \R $, e l'azione di questa mappa può essere
definita solo sui simplessi singolari e quindi estesa per linearità su tutto
lo spazio delle catene. Sia $ \sigma \colon \Delta_{k+l} \to X $ un simplesso singolare, si
può anche vedere il simplesso standard come inviluppo convesso di punti:
\[
  \Delta_{k+l} = [v_0, \dots, v_k, v_{k+1}, \dots, v_{k+l}]
\]
E quindi si può restringere il simplesso singolare sulla parte generata
dai primi $ k $ punti e su quella generata dagli ultimi $ l $:
\[
  \sigma \lvert_{[v_0, \dots, v_k]} \colon \Delta_k \to X \quad \sigma \lvert_{[v_{k+1}, \dots, v_{k+l}]} \colon \Delta_l \to X
\]
A questo punto la definizione dell'azione di $ \phi \cup \psi $ su $ \sigma $ risulta naturale:
\[
  (\phi \cup \psi)(\sigma) = \phi \left(\sigma \lvert_{[v_0, \dots, v_k]} \right) \cdot \psi \left(\sigma \lvert_{[v_{k+1}, \dots, v_{k+l}]}\right)
\]
Questa definizione è ben posta, il prodotto tra i due termini è infatti il prodotto in $ \R $.
Per passare a livello di coomologia (indicando con abuso di notazione $ \cup^\star = \cup $):
\begin{align*}
  \cup \colon H^k(X, \R) \times H^l(X, \R) & \to H^{k+l}(X, \R) \\
  (\llbracket \phi \rrbracket, \llbracket \psi \rrbracket) & \mapsto \llbracket \phi \circ \psi \rrbracket
\end{align*}
Verifico che questa applicazione è ben definita. Si ha che $ \phi $ e $ \psi $ sono
cocicli, cioè $ \delta \phi = \delta \psi = 0 $, e tutti gii altri elementi della classe differiscono
per un cobordo da $ \phi $ e $ \psi $, cioè sono della forma $ \phi + \delta\phi_1 $ e $ \psi + \delta\psi_1 $.
L'applicazione è ben definita se:
\begin{enumerate}
\item $ \phi \cup \psi $ è un bordo
\item Elementi omologhi in $ H^k(X, \R) \times H^l(X, \R) $ vengono
  mandati in elementi omologhi in $ H^{k+l}(X, \R) $.
\end{enumerate}
Per verificare la prima di queste si utilizza il seguente lemma:
\begin{lemma}
  vale che $ \delta(\phi \cup \psi) = \delta \phi \cup \psi + (-)^k \phi \cup \delta \psi $, quindi se $ \phi $ e $ \psi $ sono cocicli,
  anche $ \phi \cup \psi $ lo è.
\end{lemma}
\begin{exercise}
  Verificare il lemma.
\end{exercise}
Per verificare la seconda richiesta mostro che esiste $ \eta \in S^{k+l-1}(X) $ tale che:
\[
  (\phi + \delta\phi_1) \cup (\psi + \delta\psi_1) = \phi \cup \psi + \delta \eta
\]
Utilizzando il precedente lemma si ha che:
\begin{gather*}
  \delta(\phi \cup \psi_1) = \cancel{\delta \phi \cup \psi_1} + (-)^k (\phi \cup \delta\psi_1) \; \Rightarrow \; \phi \cup \delta \psi_1 = (-)^k \delta(\phi \cup \psi_1) = \delta ((-)^k \phi \cup \psi_1) \\
  \delta(\phi_1 \cup \psi) = \delta \phi_1 \cup \psi + \cancel{(-)^{k-1}(\phi_1 \cup \delta \psi)} \; \Rightarrow \; \delta \phi_1 \cup \psi = \delta(\phi_1 \cup \psi) \\
  \delta(\phi_1 \cup \delta \psi_1) = \delta \phi_1 \cup \delta \psi_1 + \cancel{(-)^{k-1}\phi_1 \cup \delta^2 \psi_1} \; \Rightarrow \; \delta \phi_1 \cup \delta \psi_1  =  \delta(\phi_1 \cup \delta \psi_1)
\end{gather*}
Ma quindi definendo $ \eta = (-)^k \phi \cup \delta \psi_1 + \phi_1 \cup \psi + \phi_1 \cup \delta \psi_1 $:
\[
  (\phi + \delta \phi_1) \cup (\psi + \delta \psi_1) = \phi \cup \psi + \phi \cup \delta \Psi_1 + \delta \phi_1 \cup \psi + \delta \phi_1 \cup \delta \psi_1 = \phi \cup \psi + \delta \eta
\]
La mappa è quindi ben definita a livello di coomologia e quindi si può dare la struttura
ad anello a $ H^\star(X,\R) $.

Se in particolare, come da qui in avanti assumo, $ X $ è connesso per archi:
\[
  H^0(X, \R) \cong \hom{H_0(X), \R} \cong \hom{\Z, \R} \cong \R
\]
Dove $ \hom{\Z, \R} \cong \R $ in quanto per specificare un omomorfismo da
$ \Z $ a $ \R $ mi basta dire quale è l'immagine di $ 1 $, la quale può essere
un qualunque elemento di $ \R $. Ma $ \R $ è unitario, quindi possiede un
elemento unità, e quindi si definisce l'unità in $ H^0(X, \R) $ e quindi in
tutto $ H^\star0(X, \R) $ come l'elemento che corrisponde a $ \Id{\R} $ e che quindi
corrisponde anche a $ \Id{\hom{H_0(X), \R}} $, cioè $ \Id{} \colon \llbracket \phi \rrbracket \mapsto \llbracket \phi \rrbracket $.
Osservo che in $ H^\star(X, \R) $:
\[
  \llbracket \phi \rrbracket \cup \llbracket \Id{} \rrbracket = \llbracket \phi \cup \Id{} \rrbracket = \llbracket \phi \rrbracket = \llbracket \Id{} \cup \phi \rrbracket = \llbracket \Id{} \rrbracket \cup \llbracket \phi \rrbracket
\]
Quindi $ H^\star(X, \R) $ è un anello unitario, ma in generale non commutativo.

\begin{lemma}
  Siano $ X $ e $ Y $ spazi topologici omotopicamente equivalenti allora gli
  anelli di coomologia sono isomorfi (come anelli).
\end{lemma}
\begin{proof}
  Se $ X $ è equivalente a $ Y $ allora i gruppi di omologia sono isomorfi, cioè
  $ H_\star(X) \cong H_\star(Y) $, per il teorema dei coefficienti universali anche i gruppi
  di coomologia sono isomorfi come $ \Z $-moduli, cioè $ H^\star(X) \cong H^\star(Y) $,
  devo mostrare che l'isomorfismo è anche di anelli.
  Se $ X \sim_H Y $ significa che esiste una mappa continua $ f \colon X \to Y $ e una
  $ g \colon Y \to X $ tali che $ f \circ g \sim_H \Id{Y} $ e $ g \circ f \sim_H \Id{X} $.
  Essendo $ f $ continua è ben definita
  \begin{align*}
    f_\sharp \colon S_k(X) & \to S_k(Y) \\
    \sigma & \mapsto f \circ \sigma
  \end{align*}
  Ma anche:
  \begin{align*}
    f^\sharp \colon S^k(Y) & \to S^X(X) \\
    \phi & \mapsto f^\sharp(\phi) = \phi(f_\sharp)
  \end{align*}
  Quindi si può passare alla coomologia:
  \begin{align*}
    f^\star \colon H^k(Y) & \to H^k(X) \\
    \llbracket \phi \rrbracket & \mapsto \llbracket f^\sharp \circ \phi \rrbracket
  \end{align*}
  Questa mappa è un omomorfismo di anelli, cioè:
  \begin{gather*}
    f^\star(\llbracket\phi\rrbracket + \llbracket\psi\rrbracket) = f^\star(\llbracket\phi\rrbracket) + f^\star(\llbracket\psi\rrbracket) \\
    f^\star(\llbracket\phi\rrbracket \cup \llbracket\psi\rrbracket) = f^\star(\llbracket\phi\rrbracket) \cup f^\star(\llbracket\psi\rrbracket)
  \end{gather*}
  Infatti, il comportamento rispetto alla somma è vero perché è vero anche come $ \Z $-moduli,
  mentre per il prodotto:
  Considero $ \sigma \colon \Delta_{k+l} \to X $ simplesso singolare:
  \begin{gather*}
    (f^\sharp(\phi) \cup f^\sharp(\psi))(\sigma) = (f^\sharp(\phi)(\sigma\lvert_{[v_0, \dots, v_k]})) (f^\sharp(\psi)(\sigma\lvert_{[v_{k+1}, \dots, v_{k+l}]})) = \\
    = \phi(f_\sharp(\sigma\lvert_{[v_o, \dots, v_k]}))\psi(f_\sharp(\sigma\lvert_{[v_{k+1}, \dots, v_{k+l}]})) = \phi \cup \psi (f_\sharp (\sigma)) = (f^\sharp(\phi \cup \psi))(\sigma)
  \end{gather*}
  Si può applicare il medesimo ragionamento anche per $ g $ e per l'assioma omotopico
  $ (f \circ g)^\star = (\Id{Y})^\star $ e $ (g \circ f)^\star = (\Id{X})^\star $, ma quindi $ f^\star $ e $ g^\star $ sono
  una l'inversa dell'altra ed essendo anche omomorfismi sono isomorfismi.
\end{proof}

\begin{example}
  Se $ X = \Sph{n} $, so che:
  \[
    H_k(\Sph{n}) \cong H^k(\Sph{n}) \cong
    \begin{cases}
      \Z & \text{se } k \in \set{0,n} \\
      0 & \text{altrimenti}
    \end{cases}
  \]
  Ho che $ H^0(\Sph{n}) = \langle\Id{}\rangle $ e $ H^n(\Sph{n}) = \langle\alpha\rangle $ con $ \alpha $ opportuno
  generatore. La tabella di moltiplicazione tra questi generatori è:
  \[
    \begin{array}{c|cc}
      {} & \Id{} & \alpha \\ \hline
      \Id{} & \Id{} & \alpha \\
      \alpha  & \alpha & 0 \\
    \end{array}
  \]
  Dove $ \alpha^2 = 0 $ in quanto $ \alpha^2 $ è in $ H^{2n}(X, G) `= 0 $.
  Quindi $ H^\star(\Sph{n}) = \Z[\Id{}] \oplus \Z[\alpha] $, e il generico elemento è
  della forma $ a + b \alpha $ con $ a,b \in \Z $ e $ \alpha^2 = 0 $, quindi:
  \[
    H^\star(\Sph{n}) \cong \quot{\Z[\alpha]}{(\alpha^2)}
  \]
\end{example}
\begin{example}
  A questo punto si posseggono gli strumenti necessari per risolvere il problema della
  distinzione tra $ \Pjc{2} $ e $ \Sph{2} \vee \Sph{4} $.
  Per Mayer-Vietoris $ H^\star(\Sph{2} \vee \Sph{4}) \cong H^\star(\Sph{2}) \oplus H^\star(\Sph{4}) $, quindi
  \[
    H^\star(\Sph{2} \vee \Sph{4}) \cong \quot{\Z[\alpha]}{(\alpha^2)} \oplus \quot{\Z[\beta]}{(\beta^2)}
  \]
  Successivamente dimostrerò che:
  \[
    H^\star(\Pjc{n}) \cong \quot{Z[x]}{(x^{n+1})}
  \]
  Dove $ x $ è un generatore di $ H^{2}(\Pjc{2}) $.
  Ora mostro quindi che $ \Sph{2} \vee \Sph{4} \not \sim_H \Pjc{2} $.
  \begin{gather*}
    H^\star(\Pjc{2}) = \set{a_0 + a_1 x + a_2 x^2 | x^3 = 0} \\
    H^\star(\Pjc{2}) = \set{(b_0 + b_1 \alpha, a_0 + a_1 \beta) | \alpha^2 = 0, \; \beta^2 = 0}
  \end{gather*}
  Se questi gruppi fossero isomorfi ci sarebbe una corrispondenza:
  \[
    x \leftrightarrow (b_0 + b_1 \alpha, a_0 + a_1 \beta)
  \]
  Ma quindi anche:
  \[
    x^3 = 0 \leftrightarrow (b_0^3 + 3 b_0^2 b_1 \alpha, a_0^3 + 3 a_0^2 a_1 \alpha)
  \]
  Ma se fosse un isomorfismo $ 0 $ dovrebbe andare in $ 0 $, cioè:
  \[
    \begin{cases}
      b_0^3 + 3 b_0^2 b_1 \alpha = 0 \Rightarrow b_0 = 0  \\
      a_0^3 + 3 a_0^2 a_1 \beta = 0 \Rightarrow a_0 = 0
    \end{cases}
  \]
  Cioè:
  \[
    x \leftrightarrow  (b_1 \alpha, a_1 \beta)
  \]
  Ma prendendo il quadrato avrei che:
  \[
    x^2 \leftrightarrow  (0, 0)
  \]
  Che è assurdo.
\end{example}

\begin{theorem}
  Siano $ x, y $ i generatori rispettivamente di $ H^1(\Pjr{n}; \Z_2) $ e $ H^2(\Pjc{n}; \Z) $,
  cioè:
  \[
    \langle x \rangle = H^1(\Pjr{n}, \Z_2)  \quad \langle y \rangle = H^2(\Pjc{n}, \Z)
  \]
  allora vale che:
  \begin{gather*}
    H^\star(\Pjr{n}; \Z_2) \cong \quot{\Z_2[x]}{(x^{n+1})} \\
    H^\star(\Pjc{n}; \Z) \cong \quot{\Z[y]}{(y^{n+1})}
  \end{gather*}
\end{theorem}
\begin{proof}
  La dimostrazione per i due risultati è la stessa, lo dimostro per il caso
  reale. La dimostrazione è per induzione, e in ciò che segue è sottinteso che
  il gruppo di coefficienti è $ \Z_2 $.
  Per $ n = 1 $ è noto che $ \Pjr{1} \cong \Sph{1} $, e quindi ho già calcolato
  l'anello di coomologia:
  \[
    H^\star(\Pjr{1}) \cong \quot{\Z[x]}{(x^2)}
  \]
  Per $ n > 1 $ considero due indici $ i, j $. Mostro che posso restringermi al
  caso in cui $ i + j = n $. Se $ i + j < n $ considero
  $ u \colon \Pjr{k} \to \Pjr{n} $, ho che $ u^\star \colon H^l(\Pjr{n}) \homoto H^l(\Pjr{l}) $
  con $ l \leq j $, ma quindi:
  \begin{gather*}
    0 + \alpha_i \in H^i(\Pjr{n}) \mapsto u^\star(\alpha_i) \not = 0 \\
    0 + \alpha_j \in H^j(\Pjr{n}) \mapsto u^\star(\alpha_j) \not = 0
  \end{gather*}
  Ma $ u^\star(\alpha_i \cup \alpha+j) \not = 0 $ e $ \alpha_i \cup \alpha_j \in H^{i+j}(\Pjr{k}) $. Se $
  u^\star(\alpha_i \cup \alpha_j) = 0 $ quindi $ u^\star(\alpha_i) \cup u^\star(\alpha_j) = 0 $, ma $ u^\star(\alpha_i) = 0 $
  e $ u^\star(\alpha_j) \not = 0 $ e ili prodotto cup non manda in zero. In altri termini
  se $ i + j < n $ allora $ \alpha_i \cup \alpha_j $ è generatore di $ H^{i+j}(\Pjr{n}) $ e
  mi riconduco al caso precedente. Posso quindi fissare $ i, j $ tali che $ i +
  j = n $, e prendo $ \alpha_i, \alpha_j $ generatori tali che $ \langle\alpha_i\rangle = H^i(\Pjr{n}) $ e
  $ \langle\alpha_j\rangle = H^j(\Pjr{n}) $.

  Per definizione $ \Sph{n} = \set{(x_0, \dots, x_n) \in \RN{n+1} | \sum x_i^2 = 1} $,
  considero:
  \begin{gather*}
    \Sph{i} = \set{(x_0, \dots, x_{i}, 0, 0, \dots, 0) \in \Sph{n}} \\
    \Sph{j} = \set{(0, 0, \dots, x_{n-j}, \dots, x_n) \in \Sph{n}}
  \end{gather*}
  Se $ i + j = n $ queste due sottosfere si intersecano in due punti:
  \[
    \Sph{i} \cap \Sph{j} = \set{(0,\dots,\pm1,\dots,0)}
  \]
  \begin{figure}[htbp]
    \centering
    \begin{tikzpicture}
      \draw (0,0) circle (1);
      \draw[name path=line1]
      (0,1) to[out=-120, in=150] (0,-1);
      \draw[name path=line2]
      (-1,0) to[out=-50, in=-150] (1,0);
      \draw[dashed, name path=line3]
      (0,1) to[out=-40, in=60] (0,-1);
      \draw[dashed, name path=line4]
      (-1,0) to[out=30, in=130] (1,0);
      \path [name intersections={of=line1 and line2,by=E}];
      \node [fill=black,inner sep=1pt] at (E) {};
      \path [name intersections={of=line3 and line4,by=F}];
      \node [fill=black,inner sep=1pt] at (F) {};
    \end{tikzpicture}
    \caption{Intersezione tra $ \Sph{i} $ e $ \Sph{j} $}
    \label{fig:lez18:intersection}
  \end{figure}
  So che $ \Pjr{n} = \Sph{n} \slash \sim = \Pjr{n-1} \cup_\pi \Disk{n} $, dove $ \sim $ è la
  relazione antipodale. Quindi $ \Pjr{n} - \set{p} $ è retratto di deformazione
  di $ \Pjr{n-1} $.
  Costruisco il seguente diagramma commutativo:
  \[
    \begin{tikzcd}[nodes = {column sep = 10pt, outer sep = 0pt, inner sep = 1.5pt}]
      H^i(\Pjr{n}) \times H^j(\Pjr{n}) \rar{\cup} & H^n(\Pjr{n}) \\
      H^i(\Pjr{n}, \Pjr{n} \setminus \Pjr{j}) \times H^j(\Pjr{n}, \Pjr{n} \setminus \Pjr{i}) \uar{\cong} \rar{\cup} \dar{\cong} & H^n(\Pjr{n}, \Pjr{n} \setminus \set{p}) \dar{\cong} \uar{\cong} \\
      H^i(\RN{n}, \RN{n} \setminus \RN{j}) \times H^j(\RN{n}, \RN{n} \setminus \RN{i})  \rar{\cup} & H^n(\RN{n}, \RN{n} \setminus \set{\vec{0}})
    \end{tikzcd}
  \]
  Infatti ho $ \cup \colon H^i(\Pjr{n}) \times H^j(\Pjr{n}) \to H^n(\Pjr{n}) $. Poi ho la successione
  esatta lunga in omologia:
  \[
    \begin{tikzcd}[nodes = {column sep = 5pt, inner sep = 1.5pt}]
      H^{n-1}(\Pjr{n}) \rar & H^{n-1}(\Pjr{n} \setminus \set{p}) \rar & H^{n-1}(\Pjr{n}, \Pjr{n} \setminus \set{p}) \rar & H^n(\Pjr{n}) \rar & H^n(\Pjr{n} \setminus \set{p})
    \end{tikzcd}
  \]
  Ma $ H^{n-1}(\Pjr{n}) \cong H^{n-1}(\Pjr{n-1}) $ quindi $ H^{n-1}(\Pjr{n} \setminus \set{p}) \cong H^{n-1}(\Pjr{n}) $
  quindi $ H^n(\Pjr{n}, \Pjr{n} \setminus \set{p}) \cong H^n(\Pjr{n}) $.
  Poi ho $ \Pjr{i} \cong \Sph{i} \setminus \sim $ e $ \Pjr{j} \cong \Sph{j} \setminus \sim $, quindi
  $ H^i(\Pjr{n}, \Pjr{n} \setminus \Pjr{j}) \times H^j(\Pjr{n}, \Pjr{n} \setminus \Pjr{i}) $ vanno
  in $ H^i(\Pjr{n}) \times H^i(\Pjr{n}) $.

  Poi $ \Pjr{n} \setminus (\Pjr{n} \setminus \Pjr{j} \cup \Pjr{n} \setminus \Pjr{i}) = \Pjr{n} \setminus (\Pjr{n} \setminus (\Pjr{j}) \cap \Pjr{i}) = \Pjr{n} \setminus \set{p} $
  con $ p = [0,\dots,1,0,\dots,0] $.

  Faccio escissione con $ U_i = \set{[x_0, \dots, x_n] | x_i \not = 0} \cong \RN{n} $,
  ma $ U_i = \left(\frac{x_0}{x_i}, \dots, \frac{x_n}{x_i} \right) $ è contraibile.

  Prendo $ \Pjr{n} \setminus U_i $ e faccio l'escissione in omologia e poi prendo il duale, così
  ottengo:
  \begin{gather*}
    H^n(\Pjr{n}, \Pjr{n} \setminus \set{p}) \cong H^n(\Pjr{n} \setminus (\Pjr{n} \setminus U_i)), (\Pjr{n} \setminus \set{p}) \setminus (\Pjr{n} \setminus U_i) \cong \\
    \cong H^n(U_i, U_i \setminus \set{p}) = H^n(\RN{n}, \RN{n} \setminus \set{\vec{0}})
  \end{gather*}

  Poi c'è il prodotto cup in basso in quanto ho la successione esatta:
  \[
    \begin{tikzcd}
      H^{n-1}(\RN{n}) \rar & H^{n-1}(\RN{i} \setminus \RN{j}) \rar & H^n(\RN{n}, \RN{n} \setminus \RN{j}) \rar & H^n(\RN{n})
    \end{tikzcd}
  \]
  Ma $ H^n(\RN{n}) \cong 0 $, quindi la successione è:
  \[
    \begin{tikzcd}
      0 \rar & H^{n-1}(\RN{i} \setminus \RN{j}) \rar & H^n(\RN{n}, \RN{n} \setminus \RN{j}) \rar & 0
    \end{tikzcd}
  \]
  E quindi:
  \[
     H^{n-1}(\RN{i} \setminus \RN{j}) \cong  H^n(\RN{n}, \RN{n} \setminus \RN{j}) \cong H^{n-1}(\Sph{n-j-1}) \cong H^{i-1}(\Sph{i-1})
  \]

  Con l'ipotesi induttiva costruisco gli isomorfismi in alto a sinistra.
  Mancano da dimostrare delle cose.

\end{proof}

\section{Coomologia di de Rham}

\begin{definition}
  Uno spazio topologico $ \M $ è una \textbf{varietà differenziabile}\index{Varietà differenziabile}
  di dimensione $ n $ se ogni punto $ p \in \M $ ammette un intorno aperto $ A $ omeomorfo a $ \RN{n} $
  con omeomorfismo realizzato da una \textbf{carta}\index{Carta di varietà differenziabile}
  $ \phi \colon A \to \RN{n} $ con $ \phi \in \mathcal{C}^\infty $, e tale che i cambiamenti di coordinate siano buoni, cioè
  date due carte $ \phi \colon A_1 \to \RN{n} $, $ \phi_2 \colon A_2 \to \RN{n} $ con $ A_1 \cap A_2 \not = 0 $ allora:
  \[
    \phi_2 \circ \phi_1^{-1} \colon \phi_1(A_1 \cap A_2) \to \phi_2(A_1 \cap A_2) \in \mathcal{C}^\infty(\phi_1(A_1 \cap A_2) \subseteq \RN{n})
  \]
\end{definition}
\newmathsymb{formdiff}{\Omega^k(\M)}{Spazio delle $ k $-forme differenziali su
  $ \M $}
\begin{definition}
  Su $ \M $ si definisce una \textbf{$ k $-forma differenziale}\index{Forma differenziali} come
  una applicazione multilineare antisimmetrica $ \omega $ tale che
  \[
    \omega(x) \colon \underbrace{\mathcal{T}_x\M \times \dots \times \mathcal{T}_x\M}_{k} \to \RN{}
  \]
  Dove $ \mathcal{T}_x\M $ è lo \textbf{spazio tangente} a $ \M $ in $ x $.
  In generale una $ k $-forma si scrive, usando la convenzione di Einstein, come:
  \[
    \omega = a_{i_1\dots i_k}(x) \d x^{i_1} \wedge \dots \wedge \d x^{i_k}
  \]
  dove $ \wedge $ indica il prodotto wedge, definito a breve. Lo spazio delle
  $ k $-forme su $ \M $ si denota con $ \Omega^k(\M) $ ed è uno spazio vettoriale.
\end{definition}
\newmathsymb{wedge}{\alpha\wedge\beta}{Prodotto wedge tra $ \alpha $ e $ \beta $}
\begin{definition}
  Lo spazio delle forme differenziali può essere reso un'algebra con il \textbf{prodotto wedge}\index{Prodotto wedge}
  che associa a una $ p $-forma e a una $ q $-forma una $ (p+q) $-forma:
  \begin{align*}
    \Omega^p(\M) \times \Omega^q(\M) & \to \Omega^{p+q}(\M) \\
    (\omega_1, \omega_2) & \mapsto \omega_1 \wedge \omega_2
  \end{align*}
  In componenti il prodotto wedge di due forme si ottiene usando le proprietà di anello dell'algebra
  considerando però che si richiede che:
  \[
    \forall i,j \; \d x^i \wedge \d x^j = - \d x^j \wedge \d x^i
  \]
\end{definition}
Il prodotto wedge è bilineare e associativo.
\begin{example}\hfill
  \begin{itemize}
  \item Le $ 0 $-forme sono funzioni ordinarie
  \item Le $ 1 $-forme sono variabili di Grassmann
  \end{itemize}
\end{example}
\newmathsymb{derest}{\d{\omega}}{Derivata esterna di $ \omega $}
\begin{definition}
  Si definisce la \textbf{derivata esterna}\index{Derivata esterna}:
  \begin{align*}
    \d{} \colon \Omega^k(\M) & \to \Omega^{k+1}(\M) \\
    \omega & \mapsto \d \omega
  \end{align*}
  Con:
  \[
    \d \omega = \frac{\partial a_{i_1\dots i_k}(x )}{\partial x^j} \d x^j \wedge \d x^{i_1} \wedge \dots \wedge \d x^{i_k}
  \]
\end{definition}
\begin{osservation}
  Si può verificare esplicitamente che $ \d{}^2 = 0 $.
\end{osservation}
In ciò che segue considero $ \M $ varietà differenziabile connessa (in caso non
sia connessa mi restringo alle componenti connesse), con base numerabile
(questo è una richiesta puramente tecnica), senza bordo e orientata.
\begin{definition}
  Una varietà differenziabile $ \M $ di dimensione $ n $ si dice
  \textbf{orientata}\index{Varietà differenziabile orientata} se esiste una
  $ n $-forma $ \omega $ tale che $ \omega(p) \not= 0 \; \forall p \in \M $. Una forma con tale proprietà
  è detta \textbf{forma di volume}\index{Forma di volume}. Equivalentemente si può
  dire che una varietà differenziabile è orientata se tutte i cambiamenti
  di coordinate hanno determinante Jacobiano positivo.
\end{definition}
\begin{osservation}
  La richiesta di orientazione serve affinché gli integrali di forme differenziabili
  siano ben definiti, infatti la forma di volume dà origine alla misura di integrazione
  alla Lebesgue.
\end{osservation}
Per le forme differenziali vale inoltre il teorema di Stokes:
\begin{theorem}[Teorema di Stokes\index{Teorema di Stokes}]
  Se $ \M $ è una varietà differenziabile di dimensione $ n $ con bordo $ \partial \M $
  e $ \omega $ una $ (n-1) $-forma differenziabile su $ \M $ con supporto compatto, allora:
  \[
    \int_\M \d \omega = \int_{\partial \M} \omega
  \]
\end{theorem}
\begin{definition}
  Si definisce il \textbf{complesso di de Rham}\index{Complesso di de Rham}
  il complesso $ (\Omega^\bullet, \d{}) $.
\end{definition}
\newmathsymb{rot}{\nabla\times}{Rotore}
\newmathsymb{rot}{\nabla\cdot}{Divergenza}
\begin{example}
  Considero $ \M = \RN{3} $, questa è una varietà differenziabile avente come
  carta la mappa identità. Considero il complesso di de Rham:
  \[
    \begin{tikzcd}[nodes = {column sep = 10 pt}]
      0 \rar & \Omega^0(\M) \rar & \Omega^1(\M) \rar & \Omega^2(\M) \rar & \Omega^3(\M) \rar & 0
    \end{tikzcd}
  \]
  La prima derivata esterna corrisponde ad un gradiente, in quanto le
  $ 0 $-forme sono funzioni ordinarie. Considero
  $ \omega = a \d x + b \d y + c \d z $ con $ (x,y,z) $ coordinate di
  $ \RN{3} $, e $ a, b, c \in \Omega^0(\M) $ allora:
  \begin{align*}
    \d \omega & =          \frac{\partial a}{\partial y} \d y \wedge \d x + \frac{\partial a}{\partial z} \d z \wedge \d x + \frac{\partial b}{\partial x} \d x \wedge \d y +
             \frac{\partial b}{\partial z} \d z \wedge \d y + \frac{\partial c}{\partial x} \d x \wedge \d z + \frac{\partial c}{\partial y} \d y \wedge \d z = \\
         & =      \left[ - \frac{\partial a}{\partial y} + \frac{\partial b}{\partial x} \right] \d x \wedge \d y +
                 \left[ - \frac{\partial a}{\partial z} + \frac{\partial c}{\partial x} \right] \d x \wedge \d z +
                 \left[ - \frac{\partial b}{\partial z} + \frac{\partial c}{\partial y} \right] \d y \wedge \d z =                      \\
         & =  \nabla \times (a \d x + b \d y + c \d y)
  \end{align*}
  Dove con $ \nabla \times $ si intende il rotore. Ma è noto che il rotore del gradiente di una funzione è nullo,
  cioè $ \nabla \times \nabla f = 0 $, cioè $ d^2 = 0 $ per $ k = 0 $ e $ k = 1 $.
  Faccio il passo successivo. Sia $ \eta \in \Omega^2(\M) $:
  \[
    \eta = p \d x \wedge \d y - q \d x \wedge \d z + r \d y \wedge \d z
  \]
  Il bordo è:
  \begin{align*}
    \d \eta & = \frac{\partial p}{\partial z} \d z \wedge \d x \wedge \d y - \frac{\partial q}{\partial y} \d y \wedge \d x \wedge \d z + \frac{\partial r}{\partial x} \d x \wedge \d y \wedge \d z = \\
    & = \left( \frac{\partial r}{\partial x} + \frac{\partial q}{\partial y} + \frac{\partial p}{\partial z} \right) \d x \wedge \d y \wedge \d z\\
    & = \nabla \cdot \eta \d x \wedge \d y \wedge \d z
  \end{align*}
  Dove con $ \nabla \cdot $ si intende la divergenza. Ma è noto che la divergenza di un rotore è nulla,
  cioè $ \nabla \cdot \nabla \times f = 0 $, cioè $ d^2 = 0 $ per $ k = 1 $ e $ k = 2 $.
\end{example}
\begin{definition}
  Si chiama \textbf{coomologia di de Rham}\index{Coomologia di de Rham}
  l'omologia del complesso di de Rham $ (\Omega^\bullet, \d{}) $:
  \[
    H^p_{dR}(\M) = \quot{\ker{ \d{} \colon \Omega^p(\M) \to \Omega^{p+1}(\M)}}{\im{\d{} \colon \Omega^{p-1}(\M) \to \Omega^p(\M)}}
  \]
  Indicando con:
  \begin{gather*}
    Z^p(\M) = \ker{ \d{} \colon \Omega^p(\M) \to \Omega^{p+1}(\M)} = \set{\omega \in \Omega^p(\M) | \d \omega = 0} \\
    B^p(\M) = \im{\d{} \colon \Omega^{p-1}(\M) \to \Omega^p(\M)} = \set{\gamma \in \Omega^p(\M) | \exists \rho \in \Omega^{p+1}(\M) | \gamma = \d \rho}
  \end{gather*}
  Si ha che $ Z^p(\M) $ sono le  \textbf{$ p $-forme chiuse}\index{Forme chiuse} e $ B^p(\M) $ sono
  le \textbf{ $ p $-forme esatte}\index{Forme esatte}.
\end{definition}
Il generico elemento di $ H^p_{dR}(\M) $ è $ [\omega] $ con $ \omega $ chiusa. Se
$ H^p_{dR}(\M) $ è banale significa che tutte le forme sono esatte, in quanto
non ci sono forme chiuse che non siano anche esatte (cioè elementi di
$ B^p(\M) $ che non sono in $ Z^p(\M) $).
\begin{osservation}
  Se $ \M $ è anche compatto la coomologia di de Rham è uno spazio vettoriale
  sui reali finitamente generato. Se $ \M $ non è compatto si costruisce la
  \textbf{coomologia a supporto compatto}\index{Coomologia a supporto compatto}
  $ H^p_c(\M) $ in cui si lavora con le forme differenziali a \textbf{supporto
    compatto}\index{Supporto compatto}, cioè tali che la chiusura dell'insieme
  su cui tali forme sono non nulle è un insieme compatto. Chiaramente se $ \M $ è
  compatta ogni forma differenziale è a supporto compatto.
\end{osservation}
\begin{lemma}
  Si dimostra che, a differenza della coomologia di de Rham, la coomologia a supporto
  compatto è covariante e non controvariante.
\end{lemma}

\subsection{Dualità di Poincaré}
\begin{osservation}
  Se $ b \colon V \times W \to \mathbb{F} $ è un funzionale bilineare su $ V,W $ spazi
  vettoriali e $ \mathbb{F} $ campo, allora questo induce un'applicazione:
  \begin{align*}
    B \colon V & \to W^* = \hom{W, \mathbb{F}} \\
    v & \mapsto B(v)
  \end{align*}
  Con:
  \begin{align*}
    B(v) \colon W & \to \mathbb{F} \\
    w & \mapsto b(v,w)
  \end{align*}
  Cioè $ B(v) = b(v, \cdot) $. Si dimostra che se $ b \colon V \times V \to \mathbb{F} $
  è non degenere (cioè se $ b(v,w) = 0 \; \forall w \in V $ implica che $ v = 0$) allora
  $ B \colon V \to V^* $ è un isomorfismo, e quindi esiste un accoppiamento canonico tra
  $ V $ e il suo duale.
\end{osservation}
Costruisco l'applicazione $ b $ per gli spazi $ \Omega $. Sia $ k \leq \dim \M $ definisco:
\begin{align*}
  I \colon \Omega^k(\M) \times \Omega^{n-k}(\M) & \to \RN{} \\
  (\alpha,\beta) & \mapsto \int_\M \alpha \wedge \beta
\end{align*}
Se $ \M $ è compatto l'integrale è ben definito, se $ \M $ non è compatto si
deve lavorare con forme differenziali a supporto compatto. Assumo $ \M $
compatto, definisco una mappa $ I $ sulla coomologia di de Rham:
\begin{align*}
  I \colon H^k_{dR}(\M) \times H^{n-k}_{dR}(\M) & \to \RN{} \\
  ([\alpha], [\beta]) & \mapsto \int_\M \alpha \wedge \beta
\end{align*}
Questa mappa è ben definita, infatti considero altri due rappresentanti per le
classi $ [\alpha] $ e $ [\beta] $ $ \alpha + \d \alpha' $ e $ \beta + \d \beta' $. Ho che:
\[
  \int_\M (\alpha + \d \alpha') \wedge (\beta + \d \beta') = \int_\M \alpha \wedge \beta + \int_\M \d \alpha' \wedge \beta + \int_\M \d \alpha' \wedge \d \beta' + \int_\M \alpha \wedge \d \beta'
\]
Ma considerando che $ \alpha $ e $ \beta $ sono chiuse, cioè $ \d \alpha = \d \beta = 0 $:
\begin{gather*}
  \d (\alpha' \wedge \beta) = \d \alpha' \wedge \beta + \cancel{(-)^{k-1} \alpha' \wedge \d \beta} \Rightarrow  \d \alpha' \wedge \beta = \d (\alpha' \wedge \beta) \\
  \d (\alpha \wedge \beta') = \cancel{\d \alpha \wedge \beta'} + (-)^{k} \alpha \wedge \d \beta' \Rightarrow  \alpha \wedge \d \beta' = (-)^k  \d (\alpha \wedge \beta') \\
  \d (\alpha' \wedge \d \beta') = \d \alpha' \wedge \d \beta' + \cancel{\d \alpha' \wedge \d {\d \beta}} \Rightarrow \d \alpha' \wedge \d \beta' =  \d (\alpha' \wedge \d \beta')
\end{gather*}
Quindi:
\[
  \int_\M (\alpha + \d \alpha') \wedge (\beta + \d \beta') = \int_\M \alpha \wedge \beta + \int_\M \d {(\alpha' \wedge \beta)} + \int_\M \d {(\alpha' \wedge \d \beta')} + \int_\M \d {(\alpha \wedge \d \beta')}
\]
Per il teorema di Stokes le forme esatte integrate su $ \M $ sono nulle non essendoci termini di bordo,
quindi la mappa è ben definita in quanto:
\[
  \int_\M (\alpha + \d \alpha') \wedge (\beta + \d \beta') = \int_\M \alpha \wedge \beta
\]
\begin{theorem}[Teorema di isomorfismo di Poincaré\index{Teorema di isomorfismo di Poincaré}]
  Se $ \M $ è una varietà differenziabile senza bordo e orientata, allora la
  mappa:
  \begin{align*}
    D \colon H^k_{dR}(\M) & \to (H_c^{n-k}(\M))^* \\
    [\alpha] & \mapsto D([\alpha])
  \end{align*}
  con:
  \begin{align*}
    D([\alpha]) \colon H^{n-k}_c(\M) & \to \RN{} \\
    [\beta] & \mapsto \int_\M \alpha \wedge \beta
  \end{align*}
  è un isomorfismo di gruppi abeliani, cioè $ H^k_{dR}(\M) \cong (H^{n-k}_c(\M))^* $.
\end{theorem}
\begin{proof}
  La dimostrazione è piuttosto articolata e si svolge per passi:
  \begin{enumerate}
  \item Dimostrazione del teorema per $ \M = \RN{n} $
  \item Dimostrazione del teorema per $ U $ aperto in $ \M $ tale che sia diffeomorfo a $ \RN{n} $
    e con $ D $ ristretta a $ U $
  \item Dimostrazione del teorema per qualsiasi aperto di $ \RN{n} $
  \item Dimostrazione del teorema per qualsiasi aperto proprio di $ \M $
  \item Dimostrazione del teorema per $ \M $
  \end{enumerate}

  \subparagraph{Dimostrazione del punto uno} Bisogna dimostrare che
  $ D \colon H^k_{dR}(\RN{n}) \to (H^{n-k}(\RN{n}))^* $ è un isomorfismo. Siccome
  $ \RN{n} $ è semplicemente connesso tutte le forme chiuse sono esatte (lemma
  di Poincaré) e quindi tutti i gruppi di coomologia di de Rham per $ k > 0 $
  sono banali in quanto esiste solo la classe $ [0] $, l'unico gruppo non nullo
  è $ H^0_{dR}(\RN{n}) $. Ma
  \[
    H^0_{dR}(\RN{n}) = \quot{Z^0(\RN{n})}{B^0(\RN{n})} = Z^0(\RN{n}) =
    {\set{\text{funzioni costanti}}} \cong \RN{}
  \]
  In quanto $ B^0(\RN{n}) $ è banale, e $ Z^0(\RN{n}) $ è formato dalle
  $ 0 $-forme chiuse, cioè le funzioni il cui gradiente è nullo, ovvero le
  funzioni costanti. Quindi:
  \[
    H^0_{dR}(\RN{n}) \cong
    \begin{cases}
      \RN{} & \text{se } k = 0 \\
      0 & \text{altrimenti}
    \end{cases}
  \]
  Ma anche $ H^{n-k}_c(\RN{n}) $ ha gli stessi gruppi di coomologia:
  \[
    H^{n-k}_c(\RN{n}) \cong
    \begin{cases}
      \RN{} & \text{se } k = 0 \\
      0 & \text{altrimenti}
    \end{cases}
  \]
  infatti $ \RN{n} $ è semplicemente connesso, mentre
  $ H^{n}_c(\RN{n}) \cong \RN{} $ in quanto è generato da $ n $-forme a supporto compatto del tipo:
  \[
    \omega = \phi(x_1, \dots, x_n) \d x^1 \wedge \dots \wedge \d x^n
  \]
  con $ \phi \in \mathcal{C}^\infty $ a supporto compatto. Queste forma è esatta, infatti
  nel caso $ n = 1 $ ho $ \omega = \phi(t) \d t $, ponendo:
  \[
    \psi(x) = \int_{-\infty}^x \phi(t) \d t
  \]
  Ho che $ \psi $ è una $ 0 $-forma tale che
  $ \d \psi (x) = \psi'(x) \d x = \phi(x) \d x $, quindi $ \omega $ è esatta ed è perciò il
  generatore del gruppo di coomologia. Per $ n $ generico integro una alla volta
  tutte le variabili e ottengo il medesimo risultato.

  Per mostrare che $ D $ è isomorfismo è sufficiente mostrarlo per
  $ D \colon H^0_{dR}(\RN{n}) \to (H^n_c(\RN{n}))^* $, ma:
  \begin{align*}
    D \colon \RN{} & \to {\RN{}}^* \cong \RN{} \\
    1 & \mapsto D(1)
  \end{align*}
  Se dimostro che $ D(1) $ è un generatore di $ \RN{} $ ho finito. Per mostrare
  che $ D(1) $ è un generatore è sufficiente che controllo che non sia $ 0 $,
  ma il funzionale $ 0 $ è quella mappa che manda tutte le funzioni in $ 0 $:
  cioè è tale che $ D(1)(\phi) = 0 $ $ \forall \phi $, per mostrare che $ D(1) $ non è $ 0 $
  basta quindi trovare una funziona $ \phi $ tale che $ D(1)(\phi) \not = 0 $, ma
  \[
    D(1)(\phi) = \int_{\RN{n}} \phi \d x^1 \wedge \dots \wedge \d x^n
  \]
  Ma questa è facilmente costruibile, basta prendere una funzione tipo
  mollificatore.

  \subparagraph{Dimostrazione del punto due} Se $ U $ è un aperto in $ \M $
  diffeomorfo a $ \RN{n} $ siccome i gruppi di coomologia sono invarianti per
  diffeomorfismi allora la mappa
  $ D_U \colon H^k_{dR}(U) \to (H^{n-k}_c(U))^\star $ è un isomorfismo.

  \subparagraph{Dimostrazione del punto tre} Considero una base $ \mathcal{B} $
  della topologia usuale di $ \RN{n} $ tale che:
  \begin{enumerate}
  \item L'intersezione di due aperti in $ \mathcal{B} $ è ancora in $ \mathcal{B} $
  \item Il teorema vale per ogni aperto in $ \mathcal{B} $
  \end{enumerate}
  Una possibile scelta di questa base è quella dei polirettangoli aperti i quali
  essendo diffeomorfi a $ \RN{n} $ soddisfano il teorema di dualità di Poincaré,
  come si è dimostrato precedentemente.
  Si dimostrano i seguenti lemmi:
  \begin{lemma}
    Il teorema è valido per ogni unione finita di aperti di $ \mathcal{B} $.
  \end{lemma}
  \begin{lemma}
    Il teorema è valido per ogni unione non necessariamente finita di elementi di $ \mathcal{B} $.
  \end{lemma}
  Siccome ogni aperto è unione, al più infinita di elementi di $ \mathcal{B} $ essendo $ \mathcal{B} $
  una base il punto è dimostrato.

  \subparagraph{Dimostrazione del punto quattro}

  \subparagraph{Dimostrazione del punto cinque} Siano $ V_1, V_2 $ aperti propri
  in $ \M $ tali che $ \M = V_1 \cup V_2 $, introducendo l'abbreviazione
  $ V_{12} = V_1 \cap V_2 $ allora per il teorema di Mayer-Vietoris la successione corta
  di complessi:
  \[
    \begin{tikzcd}[nodes = {row sep = 1 pt}]
      0 \rar & \Omega^\bullet(\M) \rar & \Omega^\bullet(V_1) \oplus \Omega^\bullet(V_2) \rar & \Omega^\bullet(V_{12}) \rar & 0 \\
      {} & \omega \rar[mapsto] & \omega\lvert_{V_1} \oplus\; \omega\lvert_{V_2} & {} & {} \\
      {} & {} & (\eta_1, \eta_2) \rar[mapsto] & \eta_1\lvert_{V_{12}} - \;\eta_2\lvert_{V_{12}} & {}
    \end{tikzcd}
  \]
  Induce quella lunga in coomologia:
  \[
    \begin{tikzcd}[nodes = {column sep = 6 pt, inner sep = 1.5pt}]
      H_{dR}^{k-1}(V_1) \oplus H_{dR}^{k-1}(V_2) \rar{\alpha_1} \dar{D_{V_1} \oplus D_{V_2}} & H^{k-1}_{dR}(V_{12}) \dar{D_{V_{12}}} \rar{\alpha_2} &
      H^k_{dR}(\M) \rar{\alpha_3} \dar{D_{V_{12}}} & H_{dR}^{k}(V_1) \oplus H_{dR}^{k}(V_2) \rar{\alpha_4}  \dar{D_{V_1} \oplus D_{V_2}} & H^k_{dR}(V_{12}) \dar{D_{V_{12}}}  \\
      (H_{dR}^{k-1}(V_1))^* \oplus (H_{dR}^{k-1}(V_2))^* \rar{\beta_1} &
      (H^{k-1}_{dR}(V_{12}))^* \rar{\beta_2} &(H^k_{dR}(\M))^* \rar{\beta_3} &
      (H_{dR}^{k}(V_1))^* \oplus (H_{dR}^{k}(V_2))^* \rar{\beta_4} & (h^k_{dR}(V_{12}))^*
    \end{tikzcd}
  \]
  Nella seconda riga si è usato il fatto che il duale di una somma diretta di spazi finitamente
  generati è la somma dei duali.
  Per i punti dimostrati in precedenza tutte le mappe $ D $ sono isomorfismi, a parte
  quella centrale, se dimostro che i quadrati sono commutativi per il lemma dei cinque
  $ D $ deve essere un isomorfismo. Per comodità chiamo $ \phi_1 = D_{V_1} \oplus D_{V_2} $,
  $ \phi_2 = D_{V_{12}} $, $ \phi_3 = D $, $ \phi_5 = D_{V_1} \oplus D_{V_2} $ t $ \phi_5 = D_{V_{12}} $.
  \begin{osservation}
    L'esistenza di queste successioni è dovuta al fatto che ci sono delle mappe
    di inclusione $ \tau \colon V_i \to \M $, Nel caso della successione in coomologia di
    de Rham l'associazione è contravariante e quindi si scambia il verso, nel caso
    della coomologia a supporto compatto l'associazione è covariante, ma si scambia il verso
    in quanto si prende il duale.
  \end{osservation}
  Bisogna dimostrare che i quadrati sono commutativi. Quello in mezzo è semplice,
  considero $ [\alpha] \in H^k_{dR}(\M) $:
  \begin{gather*}
    [\alpha_1] = [\tau_1^\star(\alpha)] \in H^k_{dR}(V_1) \\
    [\alpha_2] = [\tau_2^\star(\alpha)] \in H^k_{dR}(V_2)
  \end{gather*}
  Cioè $ \alpha = \alpha_1 + \alpha_2 $, ma:
  \begin{gather*}
    D(\alpha) \colon \beta \to \int_\M \alpha \wedge \beta =  \int_\M (\alpha_1 + \alpha_2) \wedge \beta =  \int_{V_1} \alpha_1 \wedge \beta + \int_{V_2} \alpha_2 \wedge \beta \\
    (D(\alpha_1) + D(\alpha_2)) \colon \beta \to  \int_{V_1} \alpha_1 \wedge \beta +  \int_{V_2} \alpha_2 \wedge \beta
  \end{gather*}
  Quindi giustamente è commutativo.
  Poi ho:
  \[
    \begin{tikzcd}
      H^{k-1}_{dR}(V_{12}) \rar & H^k_{dR}(\M) \\
      (H^{n-k+1}_{c}(V_{12}))^* \rar{\Phi} & (H^{n-k}_{c}(\M))^*
    \end{tikzcd}
  \]
  Considero $ [\rho] \in H^{k-1}_{dR}(V_{12}) $ quindi con $ \d \rho = 0 $. Ho che
  $ \rho = (\rho_1 - \rho_2) \lvert_{V_{12}} $ con
  $ \rho_1 \in \Omega^{k-1}(V_1) $ e $ \rho_2 \in \Omega^{k-1}(V_2) $ infatti
  $ \rho $ è a supporto compatto quindi posso estenderla fuori dall'insieme.
  L'omomorfismo di connessione è $ \rho \mapsto \rho' \in \Omega^k(\M) $ tale che $ \rho'\lvert_{V_1} = \d \rho_1 $
  e $ \rho' \lvert_{V_2} = \d \rho_2 $. Quindi:
  \[
    \begin{tikzcd}
      \rho \rar \dar & \rho' \dar \\
      D([\rho]) & D([\rho'])
    \end{tikzcd}
  \]
  Devo mostrare che $ \Phi(D[\rho]) = D([\rho']) $, in questo modo il diagramma è commutativo.
  Ma $ \Phi = F^* $ con $ F \colon H^{n-k}_c(\M) \to H^{n-k=1}(V_{12}) $. Sia $ \tau \in \Omega_c^{n-k}(\M) $
  con $ \d \tau = 0 $ allora $ \tau = \tau_1 + \tau_2 $ con $ \tau_1 = \tau \lvert_{V_1} $ e $ \tau_2 = \tau_{V_2} $
  quindi:
  \[
    \d \tau = \d \tau_1 + \d \tau_2 \Rightarrow 0 = \d \tau_1 + \d \tau_2 \Rightarrow \d \tau_1 = - \d \tau_2
  \]
  Ma $ \tau_1 $ è definito su $ V_1 $ e $ \tau_2 $ su $ V_2 $, quindi devono necessariamente
  essere entrambi definiti su $ V_{12} $.
  Poi ho $ \rho \in H^k_{dR}(\M) $ e $ \tau \in H^{n-k}_c(\M) $:
  \[
    D([\rho'])([\tau]) = \int_\M \rho' \wedge \tau = \int_\M \rho' \wedge (\tau_1 + \tau_2) = \int_{V_1} \rho' \wedge \tau_1 + \int_{V_2} \rho' \wedge \tau_2
  \]
  Ma essendo $ \rho $ chiusa:
  \begin{gather*}
    \d {(\rho \wedge \tau_1)} = \cancel{\d \rho \wedge \tau_1} + (-)^k \rho \wedge \d \tau_1 \; \Rightarrow \; \d {(\rho \wedge \tau_1)} = (-)^k \rho \wedge \d \tau_1 \\
    \d {(\rho \wedge \tau_2)} = \cancel{\d \rho \wedge \tau_2} + (-)^k \rho \wedge \d \tau_2 \; \Rightarrow \; \d {(\rho \wedge \tau_2)} = (-)^k \rho \wedge \d \tau_2
  \end{gather*}
  E:
  \[
    \rho' \lvert_{V_1} = \d \rho_1 \;   \rho' \lvert_{V_2} = \d \rho_2 \; \Rightarrow \;  \int_{\M} \rho' \wedge \tau = \int_{V_1} \d \rho_1 \wedge \tau_1 + \int_{V_2} \d \rho_2 \wedge \tau_2
  \]
  Quindi:
  \begin{gather*}
    D([\rho'])([\tau]) = \int_{\M} \rho' \wedge \tau = \int_{V_1} \d \rho_1 \wedge \tau_1 + \int_{V_2} \d \rho_2 \wedge \tau_2 = \\
    = \cancel{\int_{V_1} \d{(\rho_1 \wedge \tau)}} + (-)^{k+1} \int_{V_1} \rho_1 \wedge \d \tau_1 + \cancel{\int_{V_2} \d {(\rho_2 \wedge \tau_2)}} + (-)^{k+1} \int_{V_2} \rho_2 \wedge \d \tau_2 =
  \end{gather*}
  Cioè:
  \[
    (-)^{k+1} \int_{\M} \rho' \wedge \tau = \int_{V_1} \rho_1 \wedge \d \tau_1 + \int_{V_2} \rho_2 \wedge \tau_2 = \int_{V_12} (\rho_1 - \rho_2) \wedge \d \tau_1 = \int_{V_12} \rho \wedge \d \tau
  \]
  Quindi usando Stokes:
  \[
    (-)^{k+1} \int_{\M} \rho' \wedge \tau = \int_{\M} \rho \wedge \d \tau
  \]
  Anche gli altri quadrati si dimostrano in maniera analoga, in modo ancora più laborioso.
\end{proof}

% lezione 21

\section{Teorema di de Rham}

Sia $ X $ varietà differenziabile di dimensione $ n $, sono ben definiti i
gruppi di coomologia singolare a valori in $ \RN{} $ $ H^p(X) $ e il gruppo di
coomologia di de Rham. Questi sono spazi vettoriali reali di dimensione finita
su $ \RN{} $. Questi due spazi sono strutturalmente differenti, il primo è
formato da classi di equivalenza di omomorfismi, mentre il secondo da classi di
equivalenza di forme differenziali. Nonostante la differenza esiste un importante
risultato che li collega:
\begin{theorem}[Teorema di de Rham\index{Teorema di de Rham}]
  Sia $ X $ varietà differenziale di dimensione $ n $, allora:
  \[
    \forall p \in \set{0, \dots, n} \; H^p_{dR}(X) \cong H^p(X;\RN{})
  \]
\end{theorem}
Questo teorema offre una rappresentazione esplicita dei gruppi di coomologia
singolare in termini di forme differenziali, per le quali è disponibile un corposo
set di tecniche.

Per dimostrare il teorema di de Rham bisogna prima costruire un omomorfismo
$ \rho \colon H^p_{dR}(X) \to H^p(X;\RN{}) $ e dopodiché dimostrare che è un isomorfismo.
Per costruire $ \rho $ definisco un omomorfismo $ \rho \colon \Omega^p(X) \to \hom{S_p(X), \RN{}} $
e poi passo a livello di omologia, infatti per il teorema dei coefficienti
universali l'omologia del complesso degli omomorfismi è isomorfa alla coomologia.
Quindi definirò $ \rho $ tramite:
\[
  \begin{tikzcd}
    Z^p(X) \rar{r} \dar{\pi_1} & \hom{S_p(X), \RN{}} \dar{\pi_2} \\
    H^p_{dR}(X) \rar{\rho} & H^p(X, \RN{})
  \end{tikzcd}
\]
Sia $ \omega \in \Omega^p(X) $, allora $ r(\omega) \in \hom{S_p(X), \RN{}} $, quindi $ r(\omega) $ agisce
sui $ p $-simplessi singolari $ \sigma \colon \Delta_p \to X $ e produce un numero reale.
Se $ \sigma^\star $ è il \textbf{pullback}\index{Pullback} di $ \sigma $, cioè la mappa:
\begin{align*}
  \sigma^\star \colon \Omega^p(X) & \to \Omega^p(\Delta_p) \\
  \omega & \mapsto \sigma^\star = \omega \circ \sigma
\end{align*}
Si è tentati di definire:
\[
  r(\omega)(\sigma) = \int_{\Delta_p} \sigma^\star(\omega)
\]
Tuttavia questo non sarebbe sensato, in quanto $ \sigma $ è solo continua, ma per
avere una forma differenziale liscia con il pullback $ \sigma $ dovrebbe essere $ \mathcal{C}^\infty $.
Per questo motivo si introduce un gruppo intermedio.
\begin{definition}
  Un $ p $-simplesso singolare $ \sigma $ si dice \textbf{liscio}\index{Simplesso singolare liscio}
  se $ \sigma $ è $ \mathcal{C}^\infty $.
\end{definition}
\begin{osservation}
  I $ p $-simplessi singolari sono definiti su $ \Delta_p $ la quale è una varietà
  non differenziabile, per questo si dice che $ \sigma $ è
  $ \mathcal{C}^\infty $ se considerato un intorno aperto $ U $ di $ \Delta_p $ e una
  mappa $ F \colon U \to X $ tale che $ F\lvert_{\Delta_p} = \sigma $ risulta che
  $ F $ è $ \mathcal{C}^\infty $ su $ U $.
\end{osservation}
\begin{definition}
  Si definisce il \textbf{complesso delle $ p $-catene singolari
    lisce}\index{Complesso delle catene singolari} come il gruppo libero
  generato dai $ p $-simplessi singolari lisci su $ X $ con il bordo ottenuto
  restringendo l'operatore di bordo del complesso delle catene ai simplessi
  singolari lisci. Questo sin indica con $ (S^\infty_p(X), \partial) $, e il generico
  elemento $ c \in S^\infty_p(X) $ si può scrivere come:
  \[
    c = \sum_\sigma n_\sigma \sigma \text{ con $ \sigma $ $ p $-simplesso singolare liscio}
  \]
  Esplicitamente l'azione dell'operatore di bordo è:
  \begin{align*}
    \partial = \partial \lvert_{S^\infty_p} \colon S_p^\infty(X) & \to S^\infty_{p-1}(X) \\
    \sigma & \mapsto \partial \sigma = \sum_{i=0}^p (-)^i \sigma \circ F_i^{\; p}
  \end{align*}
  Questa operazione è ben definita in quanto la composizione di applicazioni
  $ \mathcal{C}^\infty $ è ancora $ \mathcal{C}^\infty $.
\end{definition}
Si può calcolare la coomologia di questo complesso, che è:
\[
  H^p_\infty(X, \RN{}) = \quot{\ker{ \delta \colon \hom{S^\infty_p(X), \RN{}} \to \hom{S^\infty_{p+1}(X), \RN{}}}}{\im{\hom{S^\infty_{p-1}(X), \RN{}} \to  \hom{S^\infty_p(X), \RN{}}}}
\]
Dove $ \delta $ è il cobordo.
\begin{osservation}
  Se $ F \colon X \to Y $ è una mappa liscia tra varietà differenziabili
  questa induce una mappa $ F^\star $:
  \begin{align*}
    F^\star \colon H^p_\infty(Y, \RN{}) & \to H^p_\infty(X, \RN{}) \\
    \llbracket c \rrbracket & \mapsto \llbracket F \circ c \rrbracket
  \end{align*}
  Questa è sensata in quanto $ F \circ c $ è $ \mathcal{C}^\infty $ in quanto
  composizione di funzioni $ \mathcal{C}^\infty $.
\end{osservation}
Con questo nuovo gruppo si può definire un omomorfismo $ r $ come ero tentato
di fare, ma solo sui simplessi singolari lisci, dopodiché si estende
la definizione alle catene di simplessi singolari lisci:
\begin{align*}
  r \colon \Omega^p(X) & \to \hom{S_p^\infty(X), \RN{}} \\
  \omega & \mapsto r(\omega)
\end{align*}
Con
\begin{align*}
  r(\omega) \colon S_p^\infty(X) & \to \RN{} \\
  \sigma & \mapsto \int_{\Delta_p} \sigma^\star(\omega)
\end{align*}
Come notazione si pone:
\[
  \int_{\Delta_p} \sigma^\star(\omega) = \int_\sigma \omega
\]
Estendendo alle catene:
\begin{align*}
  r(\omega) \colon S_p^\infty(X) & \to \RN{} \\
  c = \sum n_\sigma \sigma & \mapsto \int_c \omega = \sum n_\sigma \int_\sigma \omega
\end{align*}
Questo è un omomorfismo per la linearità dell'integrale infatti:
\begin{gather*}
  r(\lambda \omega_1 + \mu \omega_2)(c) = \int_c \lambda \omega_1 + \mu \omega_2 = \lambda \int_c \omega_1 + \mu \int_c \omega_2
\end{gather*}
Bisogna tuttavia fare attenzione che $ \Delta_p $ è una varietà topologica con bordo
che deve essere orientata. Questo si può fare induttivamente: si orienta il
punto, dopodiché si orienta il segmento in modo compatibile al punto, e così
via. La dimostrazione dettaglia è piuttosto tecnica.

A questo punto ho due complessi: $ (\Omega^\bullet, \d{}) $ e $ (\hom{(S^\infty_p(X), \RN{})}, \delta) $:
\[
  \begin{tikzcd}[nodes = {column sep = 10 pt}]
    \dots \rar{\d{}} & \Omega^p(X) \dar{r} \rar{\d{}} & \Omega^{p+1}(X)  \dar{r} \rar{\d{}} & \Omega^{p+2}(X)  \dar{r} \rar{\d{}} & \dots \\
    \dots \rar{\delta} & \hom{S^\infty_p(X), \RN{}} \rar{\delta} & \hom{S^\infty_{p+1}(X), \RN{}} \rar{\delta} &\hom{S^\infty_{p+2}(X), \RN{}} \rar{\delta} & \dots
  \end{tikzcd}
\]

\begin{lemma}
  La mappa
  $ r \colon (\Omega^\bullet(X), \d{}) \to (\hom{(S^\infty_\bullet(X), \RN{})}, \delta) $ precedentemente definita
  à una mappa di cocatene, cioè $ r \circ d = \delta \circ r $.
\end{lemma}
\begin{proof}
  Sia $ \omega \in \Omega^p(X) $ allora $ r(\omega) \in \hom{S^\infty_p(X), \RN{}} $, applicando $ \delta $
  e valutandolo su un simplesso singolare liscio $ \sigma $:
  \[
    \delta \circ r(\omega)(\sigma) = r(\omega)(\partial \sigma) = \int_{\partial \sigma} \omega
  \]
  Ma $ \partial \sigma $ è $ (p-1) $-catena e:
  \[
    \partial \sigma = \sum_{i=0}^p(-)^i\sigma \circ F_i^{\; p}
  \]
  Quindi:
  \[
    \int_{\partial \sigma} \omega = \sum_{i=0}^p \int_{\sigma \circ F_i^{\; p}} \omega = \sum_{i=0}^p(-)^i \int_{\Delta_{p-1}} (\sigma \circ F_i^{\; p})^\star (\omega) =
  \]
  Ma il pullback è controvariante cioè $ (f \circ g)^\star = g^\star \circ f^\star $ e quini:
  \[
    =\sum_{i=0}^p (-)^p \int_{A_{p-1}} (F_i^{\; p})^\star \circ \sigma^\star (\omega) =
  \]
  Dopo numerosi conti tediosi:
  \[
    = \int_{\bigcup_{i=0}^p F_i^{\; p}(\Delta_{p-1})} \sigma^\star(\omega) = \int_{\partial(\Delta_p)} \sigma^\star(\omega) =
  \]
  Per il teorema di Stokes:
  \[
    = \int_{\Delta_p} \d{(\sigma^\star(\omega))}
  \]
  Quindi:
  \[
    (\delta \circ r)(\omega)(\sigma) = \int_{\Delta_p} \d{(\sigma^\star(\omega))}
  \]
  Ma si dimostra che il pullback commuta con la derivata esterna e quindi:
  \[
    = \int_{\Delta_p} \sigma^\star(\d \omega) = \int_\sigma \d \omega = r (\d \omega) (\sigma) = (r \circ d)(\omega)(\sigma)
  \]
  Quindi il quadrato è commutativo.
\end{proof}
Siccome $ r $ è una mappa di cocatene induce una mappa a livello di coomologia
e quindi esiste ben definita:
\begin{align*}
  \rho \colon H^p_{dR}(X) & \to H^p_\infty (X, \RN{}) \\
  \llbracket \omega \rrbracket & \mapsto \llbracket r(\omega) \rrbracket
\end{align*}
\begin{lemma}
  Se $ F \colon X \to Y $ è un'applicazione liscia tra varietà differenziali allora
  il seguente diagramma commuta:
  \[
    \begin{tikzcd}
      H^p_{dR}(Y) \rar{F^\star} \dar{\rho_Y} & H^p_{dR}(X) \dar{\rho_X} \\
      H^p_\infty(Y, \RN{}) \rar{F^\star} & H^p_\infty(X, \RN{})
    \end{tikzcd}
  \]
\end{lemma}
\begin{proof}
  Sia $ \omega \in Z^p(Y) $, cioè $ \omega $ forma differenziale chiusa ($ \d \omega = 0 $)
  e sia $ \sigma $ un $ p $-simplesso singolare liscio in $ X $. Devo
  mostrare che $ \rho_X \circ F^\star = F^\star \circ \rho_Y $. Li valuto:
  \begin{gather*}
    (\rho_X \circ F^\star)(\omega)(\sigma) = \int_\sigma F^\star(\omega) = \int_{\Delta_p} \sigma^\star \circ F^\star (\omega) = \int_{\Delta_p} (F \circ \sigma)^\star(\omega) = \\
    = \int_{F \circ \sigma} \omega = \rho_Y(\omega)(F \circ \sigma)
  \end{gather*}
  Quindi $ (\rho_X \circ F^\star)(\omega)(\sigma) = \rho_Y(\omega)(F \circ \sigma) $ ma $ \rho_Y(\omega)(F \circ \sigma) = (F^\star \circ \rho_Y)(\omega)(\sigma) $
  per la definizione di pullback.
  Ora mostro che $ (\rho_X \circ F^\star)([\omega]_{dR}) = (F^\star \circ \rho_Y)([\omega]_{dR}) $ con $ [w]_{dR} = \omega + \d \eta $.
  \[
    (\rho_X \circ F^\star)(\omega + \d \eta) = (\rho \circ F^\star)(\omega) + (\rho_X \circ F^\star)(\d \eta)
  \]
  La prima parte ho già che è ok, la seconda:
  \[
    \rho_X(F^\star \d \eta)(\sigma) = \int_\sigma F^\star \d \nu = \int_{F \circ \sigma} \d \eta
  \]
  Ma
  \[
    (F^\star \circ \rho_Y)(\d \eta)(\sigma) = \rho_Y(\d \eta)(F \circ \sigma) = \int_{F \circ \sigma} \d \eta
  \]
  Quindi tutto va bene.
\end{proof}
\begin{definition}
  Sia $ X $ una varietà differenziale, questa si dice \textbf{de Rham}\index{Varietà differenziabile de Rham}
  se l'applicazione $ \rho $ è un isomorfismo.
\end{definition}
\begin{lemma}
  Essere de Rham è invariante per diffeomorfismi, cioè se $ X $ è de Rham e $ F \colon X \to Y $
  è diffeomorfismo, allora anche $ Y $ è de Rham.
\end{lemma}
\begin{proof}
  Siccome $ F \colon X \to Y $ è diffeomorfismo allora esiste $ G \colon Y \to X $ diffeomorfismo
  inverso, quindi $ H^p_{dR}(X) \cong H^p_{dR}(Y) $ per la funtorialità del pullback,
  e anche $ F^\star_\infty $ lo è per motivo analogo.
  Quindi ho il diagramma:
  \[
    \begin{tikzcd}
      H^p_{dR}(Y) \dar{\rho_Y} \rar{F^\star_{dR}} & H^p_{dR}(X) \dar{\rho_X} \\
      H^p_\infty(Y, \RN{}) \rar{F^\star_\infty} & H^p_\infty(X, \RN{})
    \end{tikzcd}
  \]
  Siccome il diagramma commuta $ \rho_Y = (F^\star_\infty)^{-1} \circ \rho_X \circ F^\star_{dR} $ è
  isomorfismo essendo composizione di isomorfismi.
\end{proof}
\eproof
A questo punto il teorema di de Rham si può scrivere esprimere come:
\begin{theorem}[Teorema di de Rham\index{Teorema di de Rham}]
  Ogni varietà differenziale è de Rham inoltre soddisfa la condizione
  \[
    H^p(X, \RN{}) \cong H^p_{dR}(X) \; \forall p
  \]
\end{theorem}
\begin{proof}
  La dimostrazione è per passi. Bisogna dimostrare
  \begin{itemize}
  \item Se $ \set{X_j} $ è una famiglia numerabile di varietà de Rham
    allora la loro unione disgiunta è una varietà de Rham.
  \item Ogni aperto convesso di $ \RN{n} $ è de Rham.
  \item Se $ X $ è una varietà differenziale che ammette un ricoprimento
    finito di aperti de Rham allora è de Rham.
  \item Se $ X $ ha una base di aperti de Rham allora è de Rham.
  \item Ogni aperto di $ \RN{n} $ è de Rham.
  \item Ogni varietà differenziale è de Rham.
  \item Ogni varietà soddisfa $  H^p(X, \RN{}) \cong H^p_{dR}(X) \; \forall p $.
  \end{itemize}

  \subparagraph{Dimostrazione del punto sei} Siccome $ X $ è una varietà
  differenziabile ammette un atlante formato da insiemi $ U_\alpha $ (con relative
  carte) che sono diffeomorfi ad aperti $ V_\alpha $ in $ \RN{n} $. Per il punto
  cinque $ V_\alpha $ sono de Rham, quindi per il lemma precedente $ U_\alpha $ sono
  de Rham. Ora basta mostrare che $ X $ammette una base di aperti costruiti
  a partire da questi $ U_\alpha $ in questo modo per il punto quattro $ X $ è de
  Rham.
  [MANCA QUALCOSA]
\end{proof}

\subsection{Duale di Poincaré}


Se $ X $ è una varietà differenziabile compatta orientata senza bordo
allora ho il teorema di Poincaré e quello di de Rham. Quindi ho:
\[
  \begin{tikzcd}
    {} & (H^{n-p}_{dR}(X))^* \\
    H^p_{dR}(X) \arrow{ur}{D} \arrow{dr}{\rho} & {} \\
    {} & H^p(X, \RN{})
  \end{tikzcd}
\]
\begin{definition}
  Dato $ \llbracket \sigma \rrbracket \in H_p(X) $ si definisce il \textbf{duale di Poincaré}\index{Duale di Poincaré}
  $ \llbracket\mathcal{P}(\sigma)\rrbracket \in H^p_{dR}(X) $ tale che:
  \[
    \int_\sigma \omega = \int_X \mathcal{P}(\sigma) \wedge \omega
  \]
  Cioè è quella forma differenziale che tramite la dualità di Poincaré
  produce una forma che per de Rham finisce in $ \sigma $.
\end{definition}
Questo si costruisce prendendo un $ \llbracket \tilde{\sigma} \rrbracket \in H^p(X, \RN{}) $ poi
con de Rham viene mandato in $  \mathcal{P}(\sigma)  $ tale che $ \rho(\llbracket\mathcal{P}(\sigma) \rrbracket) = \tilde{\sigma} $
e quindi con Poincaré $ D(\llbracket\mathcal{P}(\sigma) \rrbracket) = \int_X \omega_{\tilde{\sigma}} \wedge \eta $


%%% Local Variables:
%%% ispell-local-dictionary: "italiano"
%%% mode: latex
%%% TeX-master: "notes"
%%% End:
